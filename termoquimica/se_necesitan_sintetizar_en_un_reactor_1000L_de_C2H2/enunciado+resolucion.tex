\begin{frame}
	\frametitle{\ejerciciocmd}
	\framesubtitle{Enunciado}
	\textbf{
			Dadas las siguientes reacciones:
\begin{itemize}
    \item \ce{I2(g) + H2(g) -> 2 HI(g)}~~~$\Delta H_1 = \SI{-0,8}{\kilo\calorie}$
    \item \ce{I2(s) + H2(g) -> 2 HI(g)}~~~$\Delta H_2 = \SI{12}{\kilo\calorie}$
    \item \ce{I2(g) + H2(g) -> 2 HI(ac)}~~~$\Delta H_3 = \SI{-26,8}{\kilo\calorie}$
\end{itemize}
Calcular los parámetros que se indican a continuación:
\begin{description}%[label={\alph*)},font={\color{red!50!black}\bfseries}]
    \item[\texttt{a)}] Calor molar latente de sublimación del yodo.
    \item[\texttt{b)}] Calor molar de disolución del ácido yodhídrico.
    \item[\texttt{c)}] Número de calorías que hay que aportar para disociar en sus componentes el yoduro de hidrógeno gas contenido en un matraz de \SI{750}{\cubic\centi\meter} a \SI{25}{\celsius} y \SI{800}{\torr} de presión.
\end{description}
\resultadocmd{\SI{12,8}{\kilo\calorie}; \SI{-13,0}{\kilo\calorie}; \SI{12,9}{\calorie}}

		}
\end{frame}

\begin{frame}
	\frametitle{\ejerciciocmd}
	\framesubtitle{Datos del problema}
	\textbf{\begin{center}
		{\Large ¿$V(\ce{C2H2})$?}
	\end{center}}
%%%%% ACETILENO
	$$
		\tcbhighmath[boxrule=0.4pt,arc=4pt,colframe=green,drop fuzzy shadow=red]{V(\ce{C2H2}) = \SI{1000}{\liter}}\quad
		\tcbhighmath[boxrule=0.4pt,arc=4pt,colframe=green,drop fuzzy shadow=red]{\Delta H_f^0(\ce{C2H2})= \SI{226,8}{\kilo\joule\per\mol}}
	$$
%%%%% AGUA
	$$
		\tcbhighmath[boxrule=0.4pt,arc=4pt,colframe=blue,drop fuzzy shadow=red]{P(\ce{H2O(g)}) = \SI{3,2}{\atm}}\quad
		\tcbhighmath[boxrule=0.4pt,arc=4pt,colframe=blue,drop fuzzy shadow=red]{V(\ce{H2O(g)}) = \SI{950}{\liter}}\quad
		\tcbhighmath[boxrule=0.4pt,arc=4pt,colframe=blue,drop fuzzy shadow=red]{T(\ce{H2O(g)}) = \SI{117,2}{\celsius}}
	$$
	$$
		\tcbhighmath[boxrule=0.4pt,arc=4pt,colframe=blue,drop fuzzy shadow=red]{c_e(\ce{H2O(g)}) = c_e(\ce{H2O(s)}) = \SI{,5}{\calorie\per\gram\per\celsius}}\quad
		\tcbhighmath[boxrule=0.4pt,arc=4pt,colframe=blue,drop fuzzy shadow=red]{c_e(\ce{H2O(l)}) = \SI{4,18}{\joule\per\gram\per\kelvin}}\quad
	$$
	$$
		\tcbhighmath[boxrule=0.4pt,arc=4pt,colframe=blue,drop fuzzy shadow=red]{\Delta H_\text{vap}^0(\ce{H2O})=\SI{40,7}{\kilo\joule\per\mol}}\quad
		\tcbhighmath[boxrule=0.4pt,arc=4pt,colframe=blue,drop fuzzy shadow=red]{\Delta H_f^0(\ce{H2O}) = \SI{-285,8}{\kilo\joule\per\mol}}
	$$
%%%
%%%%% PROPANO
	$$
		\tcbhighmath[boxrule=0.4pt,arc=4pt,colframe=orange,drop fuzzy shadow=green]{P(\ce{C3H8})=\SI{2,3}{\atm}}\quad
		\tcbhighmath[boxrule=0.4pt,arc=4pt,colframe=orange,drop fuzzy shadow=green]{\Delta H_f^0(\ce{C3H8})=\SI{-104,6}{\kilo\joule\per\mol}}
	$$
%%%% CO2
	$$
		\tcbhighmath[boxrule=0.4pt,arc=4pt,colframe=black,drop fuzzy shadow=blue]{\Delta H_f^0(\ce{CO2})=\SI{-393,51}{\kilo\joule\per\mol}}
	$$
\end{frame}

\begin{frame}
	\frametitle{\ejerciciocmd}
	\framesubtitle{Resolución (\rom{1}): calor absorbido por el \ce{C2H2}}
	\structure{Nº de moles de \ce{C2H2} usando \underline{ecuación de los gases ideales}:} $P\vdot V=n\vdot R\vdot T\Rightarrow n=\frac{P\vdot V}{R\vdot T}$
	$$
		n(\ce{C2H2})=\frac{\SI{1}{\cancel\atm}\vdot\SI{1000}{\cancel\liter}}{\SI{,082}{\cancel\atm\cancel\liter\per\mol\per\cancel\kelvin}\vdot\SI{298,15}{\cancel\kelvin}}=\SI{40,90}{\mol}
	$$
	\visible<2->{
		\structure{Reacción de formación del acetileno a partir de sus elementos:}
		$$
			\ce{2C(s) + H2(g) ->[\Delta] C2H2(g)}\quad\Delta H_f^0(\ce{C2H2})=\SI{226,8}{\kilo\joule\per\mol}
		$$
				}
	\visible<3->{
		\structure{Calor absorbido (\underline{\textbf{MAYOR QUE CERO}}) por el \ce{C2H2}:}
		$$
			Q_{\text{absorbido}}(\ce{C2H2}) = \SI{226,8}{\kilo\joule\per\cancel\mol}\vdot\SI{40,90}{\cancel\mol}=\SI{9276,72}{\kilo\joule}
		$$
				}
\end{frame}

\begin{frame}
	\frametitle{\ejerciciocmd}
	\framesubtitle{Resolución (\rom{2}): calor emitido o cedido por el \ce{H2O}}
	\begin{overprint}
		\onslide<1>
			\structure{Estado inicial del agua es vapor, de nuevo \underline{ecuación de los gases ideales}}:
			$$
				n(\ce{H2O}) = 		\frac{\SI{3,2}{\cancel\atm}\vdot\SI{950}{\cancel\liter}}{\SI{,082}{\cancel\atm\cancel\liter\per\mol\per\cancel\kelvin}\vdot\SI{390,35}{\cancel\kelvin}}=\SI{94,97}{\mol}
			$$
		\onslide<2->
			\structure{Masa de agua:} $n=\rfrac{m}{Mm}\Rightarrow m=n\vdot Mm$
			$$
				m(\ce{H2O}) = \SI{94,97}{\cancel\mol}\vdot\SI{18}{\gram\per\cancel\mol}=\SI{1709,54}{\gram}
			$$
	\end{overprint}
	\visible<3->{
		\structure{El calor que puede ceder el agua se resume en el siguiente esquema:}\footnote{Recordemos que $\Delta T(\si{\celsius})=\Delta T(\si{\kelvin})$}
		\begin{overprint}
			\onslide<3>
				$$
					\ce{
						$\underset{\SI{390,35}{\kelvin}}{\ce{H2O(g)}}$
						->[$\textcolor{green}{Q_1}$][{$\Delta T_1=\SI{-17,2}{\kelvin}$}]
						$\underset{\SI{373,15}{\kelvin}}{\ce{H2O(g)}}$
					}
				$$
			\onslide<4>
				$$
					\ce{
						$\underset{\SI{390,35}{\kelvin}}{\ce{H2O(g)}}$
						->[$\textcolor{green}{Q_1}$][{$\Delta T_1=\SI{-17,2}{\kelvin}$}]
						$\underset{\SI{373,15}{\kelvin}}{\ce{H2O(g)}}$
						->[$\textcolor{red}{Q_2}$][$\Delta$fase]
						$\underset{\SI{373,15}{\celsius}}{\ce{H2O(l)}}$
					}
				$$
			\onslide<5->
				$$
					\ce{
						$\underset{\SI{390,35}{\kelvin}}{\ce{H2O(g)}}$
						->[$\textcolor{green}{Q_1}$][{$\Delta T_1=\SI{-17,2}{\kelvin}$}]
						$\underset{\SI{373,15}{\kelvin}}{\ce{H2O(g)}}$
						->[$\textcolor{red}{Q_2}$][$\Delta$fase]
						$\underset{\SI{373,15}{\celsius}}{\ce{H2O(l)}}$
						->[$\textcolor{blue}{Q_3}$][{$\Delta T_3=\SI{-75}{\kelvin}$}]
						$\underset{\SI{298,15}{\kelvin}}{\ce{H2O(l)}}$
					}
				$$
		\end{overprint}
		$$
			\textcolor{green}{Q_1} = \SI{1709,54}{\cancel\gram}\vdot\SI{2,09}{\joule\per\cancel\gram\per\cancel\kelvin}\vdot(\SI{-17,2}{\cancel\kelvin}) = \SI{-61454,37}{\joule} = \SI{-61,45}{\kilo\joule}
		$$
				}
	\visible<4->{
		\alert{\textbf{RECORDAD}, función de estado del proceso \ce{gas(g) -> l{í}quido(l)} es igual pero de signo contrario a \ce{l -> g} (condensación, $cond$), entonces $\Delta H_{cond}=-\Delta H_v$}
		$$
			\textcolor{red}{Q_2}=\underbrace{\SI{-40,7}{\kilo\joule\per\cancel\mol}}_{\Delta H_{cond}=-\Delta H_v}\vdot\SI{94,97}{\cancel\mol} = \SI{-3865,45}{\kilo\joule}
		$$
				}
	\visible<5->{
		$$
			\textcolor{blue}{Q_3} = \SI{1709,54}{\cancel\gram}\vdot\SI{4,18}{\joule\per\cancel\gram\per\cancel\kelvin}\vdot(\SI{-75}{\cancel\kelvin}) = \SI{-535939,29}{\joule} = \SI{-535,94}{\kilo\joule}
		$$
		$$
			Q_{\text{cedido}}(\ce{H2O})=\textcolor{green}{Q_1}+\textcolor{red}{Q_2}+\textcolor{blue}{Q_3}
		$$
				}
	\visible<6->{
		\structure{El calor cedido del agua es (\underline{\textbf{MENOR QUE CERO}}):}
		$$
			Q_{\text{cedido}}(\ce{H2O})=\SI{-4462,84}{\kilo\joule}
		$$
				}
\end{frame}

\begin{frame}
	\frametitle{\ejerciciocmd}
	\framesubtitle{Resolución (\rom{3}): calor de \ce{C3H8} necesario}
	\centering\textbf{Suponemos que el proceso transcurre en un sistema aislado.}
	\structure{Según el 1"er principio de la Termodinámica:} $\Delta U=0=Q_{\text{total}}+W$
	\visible<2->{
		\structure{No se realiza ningún trabajo ($\Delta V=0$):} $Q_{\text{total}}+\cancelto{0}{W}=0$
	}
	\visible<3->{
		\begin{overprint}
			\onslide<3>
				\structure{Transferencia de calor total tiene que ser cero:}
				$$
					Q_{\text{total}}=Q_{\text{cedido}}(\ce{H2O})+Q_{\text{absorbido}}(\ce{C2H2})+Q_{\text{cedido}}(\ce{C3H8})=0
				$$
			\onslide<4>
				\structure{Despejando el calor que nos interesa:}
				$$
					\underbrace{Q_{\text{cedido}}(\ce{C3H8})}_{<0}=\overbrace{-\underbrace{Q_{\text{cedido}}(\ce{H2O})}_{\SI{-4462,84}{\kilo\joule}}}^{\SI{4462,84}{\kilo\joule}}\underbrace{-\overbrace{Q_{\text{absorbido}}(\ce{C2H2})}^{\SI{9276,72}{\kilo\joule}}}_{\SI{-9276,72}{\kilo\joule}}
				$$
			\onslide<5>
				$$
					\underbrace{Q_{\text{cedido}}(\ce{C3H8})}_{<0}=-(\SI{-4462,84}{\kilo\joule})-(\SI{+9276,72}{\kilo\joule})
				$$
			\onslide<6->
				$$
					\underbrace{Q_{\text{cedido}}(\ce{C3H8})}_{<0}=\SI{+4462,84}{\kilo\joule}-\SI{9276,72}{\kilo\joule} = \SI{-4813,88}{\kilo\joule}
				$$
		\end{overprint}
	}
\end{frame}

\begin{frame}
	\frametitle{\ejerciciocmd}
	\framesubtitle{Resolución (\rom{4}): variación de entalpía de combustión del \ce{C3H8}}
	\structure{Aplicamos la \textbf{ley de Hess} a la siguiente reacción:}
	$$
		\ce{C3H8(g) + 5O2(g) -> 3CO2(g) + 4H2O(l)}
	$$
	$$
		\Delta H^0_{\text{reacción}}=\sum_{i=1}^{\text{nº productos}} n_i\vdot\Delta H^0_{f,i} - \sum_{m=1}^{\text{nº reactivos}} n_j\vdot\Delta H^0_{f,j}
	$$
	\visible<2->{
		$$
			\Delta H_C^0(\ce{C3H8}) = 3\times(\SI{-393,51}{\kilo\joule\per\mol})+4\times(\SI{-285,8}{\kilo\joule\per\mol})-1\times(\SI{-104,6}{\kilo\joule\per\mol})
		$$
		\structure{Recordad:}
		\begin{itemize}
			\item $\Delta H^0_f(\ce{O2})=0$, por eso no está incluido.
			\item Hay que comprobar el coeficiente estequiométrico de \ce{C3H8} y dividir por él si es distinto de \num{1}.
		\end{itemize}
				}
	\visible<3->{
		\structure{Resultado:}
		\begin{center}
			\myovalbox{\textcolor{yellow}{$\Delta H_C^0(\ce{C3H8}) = \SI{-2219,13}{\kilo\joule\per\mol}$}}
		\end{center}
				}
\end{frame}

\begin{frame}
	\frametitle{\ejerciciocmd}
	\framesubtitle{Resolución (\rom{5}): volumen de \ce{C3H8}}
	\structure{El número de moles se obtiene al dividir el calor cedido del \ce{C3H8} entre su variación de entalpía de combustión:}
	$$
		n(\ce{C3H8})=\frac{Q_{\text{cedido}}(\ce{C3H8})}{\Delta H_C^0(\ce{C3H8})}\Rightarrow n(\ce{C3H8})=\frac{\SI{-4813,88}{\cancel\kilo\joule}}{\SI{-2219,13}{\cancel\kilo\joule\per\mol}}=\SI{2,17}{\mol}
	$$
	\structure{Con la ecuación de los gases ideales obtenemos el volumen:}
	$$
		P\vdot V=n\vdot R\vdot T\Rightarrow V=\frac{n\vdot R\vdot T}{P}\Rightarrow V(\ce{C3H8})=\frac{\SI{2,17}{\cancel\mol}\vdot\SI{,082}{\cancel\atm\liter\per\cancel\mol\per\cancel\kelvin}\vdot\SI{298,15}{\cancel\kelvin}}{\SI{2,3}{\cancel\atm}}
	$$
	$$
		\tcbhighmath[boxrule=0.4pt,arc=4pt,colframe=orange,drop fuzzy shadow=green]{V(\ce{C3H8})=\SI{23,06}{\liter}}
	$$
\end{frame}
