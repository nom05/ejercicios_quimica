Se necesitan sintetizar en un reactor \SI{1000}{\liter} de acetileno (\ce{C2H2}) a partir de sus elementos fundamentales (\ce{C(s)} y \ce{H2(g)}) a una presión de \SI{1}{\atm} y \SI{25}{\celsius}. El reactor tiene dos vías que suministran la energía necesaria. Por un lado, una fuente geotermal de vapor de agua a \SI{117,2}{\celsius} y presión de \SI{3,2}{\atm} que ha suministrado para la reacción \SI{950}{\liter} (recordad que el agua saldrá a la misma temperatura que el acetileno del reactor). Por otro lado la combustión de propano (\ce{C3H8}). ¿Cuántos litros de propano necesitaremos a \SI{25}{\celsius} y \SI{2,3}{\atm}?\\[.3cm]
DATOS: $c_e(\ce{H2O(l)})=\SI{4,18}{\joule\per\gram\per\celsius}$, $c_e(\ce{H2O(g)})=c_e(\ce{H2O(s)})=\SI{,5}{\calorie\per\gram\per\celsius}$. $\Delta H_v^0(\ce{H2O})=\SI{40,7}{\kilo\joule\per\mol}$. $Mm(\ce{H2O})=\SI{18}{\gram\per\mol}$. $\Delta H_f^0(\ce{C2H2})= \SI{226,8}{\kilo\joule\per\mol}$. $\Delta H_f^0(\ce{C3H8}) = \SI{-104,6}{\kilo\joule\per\mol}$. $\Delta H_f^0(\ce{H2O}) = \SI{-285,8}{\kilo\joule\per\mol}$.\\$\Delta H_f^0(\ce{CO2}) = \SI{-393,51}{\kilo\joule\per\mol}$
\resultadocmd{\SI{23,06}{\liter}}
