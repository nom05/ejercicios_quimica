\begin{frame}
    \frametitle{\ejerciciocmd}
    \framesubtitle{Enunciado}
    \textbf{
		Dadas las siguientes reacciones:
\begin{itemize}
    \item \ce{I2(g) + H2(g) -> 2 HI(g)}~~~$\Delta H_1 = \SI{-0,8}{\kilo\calorie}$
    \item \ce{I2(s) + H2(g) -> 2 HI(g)}~~~$\Delta H_2 = \SI{12}{\kilo\calorie}$
    \item \ce{I2(g) + H2(g) -> 2 HI(ac)}~~~$\Delta H_3 = \SI{-26,8}{\kilo\calorie}$
\end{itemize}
Calcular los parámetros que se indican a continuación:
\begin{description}%[label={\alph*)},font={\color{red!50!black}\bfseries}]
    \item[\texttt{a)}] Calor molar latente de sublimación del yodo.
    \item[\texttt{b)}] Calor molar de disolución del ácido yodhídrico.
    \item[\texttt{c)}] Número de calorías que hay que aportar para disociar en sus componentes el yoduro de hidrógeno gas contenido en un matraz de \SI{750}{\cubic\centi\meter} a \SI{25}{\celsius} y \SI{800}{\torr} de presión.
\end{description}
\resultadocmd{\SI{12,8}{\kilo\calorie}; \SI{-13,0}{\kilo\calorie}; \SI{12,9}{\calorie}}

	}
\end{frame}

\begin{frame}
    \frametitle{\ejerciciocmd}
    \framesubtitle{Datos del problema}
    \begin{enumerate}[label={\alph*)},font={\color{red!50!black}\bfseries}]
        \item ¿$V_{\max}$ con equilibrio líquido-vapor?
        \item Si $V = \SI{3,0}{\liter}$, ¿$P(\ce{CS2(g)})$?
        \item Si $V = \SI{6,0}{\liter}$, ¿$P(\ce{CS2(g)})$?
    \end{enumerate}
    $$
        P_v(\ce{CS2}) = \frac{298}{760}~\si{\atm}
    $$
    $$
        T(\ce{CS2}) = 20+273,15~\si{\kelvin}
    $$
    $$
        m(\ce{CS2})=\SI{6,00}{\gram}
    $$
    $$
        Mm(\ce{CS2}) = \SI{76,14}{\gram\per\mol}
    $$
\end{frame}

\begin{frame}
    \frametitle{\ejerciciocmd}
    \framesubtitle{Resolución (\rom{1}): Volumen máximo con equilibrio líquido-vapor}
    \structure{$V_{\max}$ corresponderá a partir de cuándo el líquido es gas:}
    \begin{overprint}
        \onslide<1>
            $$
                P\vdot V = n\vdot R\vdot T
            $$
        \onslide<2>
            $$
                V = \frac{\overbrace{n}^{n=\frac{m}{Mm}}\vdot R\vdot T}{P}
            $$
        \onslide<3>
            $$
                V_{\max}(\ce{CS2}) = \frac{m(\ce{CS2})\vdot R\vdot T(\ce{CS2})}{P(\ce{CS2})\vdot Mm(\ce{CS2})}
            $$
        \onslide<4>
            $$
                \tcbhighmath[boxrule=0.4pt,arc=4pt,colframe=green,drop fuzzy shadow=blue]{
                V_{\max}(\ce{CS2}) =
                \frac{\SI{6}{\cancel\gram}\vdot\SI{,082}{\cancel\atm\liter\per\cancel\mol\per\cancel\kelvin}\vdot(20+273,15~\si{\cancel\kelvin})}{\frac{298}{760}~\si{\cancel\atm}\vdot\SI{76,14}{\cancel\gram\per\cancel\mol}} =
                \SI{4,83}{\liter}
                }
            $$
    \end{overprint}
\end{frame}

\begin{frame}
    \frametitle{\ejerciciocmd}
    \framesubtitle{Resolución (\rom{2}): Si $V = \SI{3,0}{\liter}$, presión del gas}
    \structure{Según el apartado anterior:}
    $$
        V_{\max}(\ce{CS2}) = \SI{4,83}{\liter} > V(\ce{CS2}) = \SI{3}{\liter}
    $$
    \visible<2->{
        \structure{Entonces:} todo \textbf{\underline{volumen}} de recipiente \textbf{\underline{inferior}} al $V_{\max}$ en las mismas condiciones provoca la \textbf{\underline{condensación}}. Por tanto, hay líquido y su $P(\ce{CS2(g)})$ será
        $$
            \tcbhighmath[boxrule=0.4pt,arc=4pt,colframe=green,drop fuzzy shadow=blue]{
                P_v(\ce{CS2}) = \frac{298}{760}~\si{\atm} = \SI{,392}{\atm}
            }
        $$
                }
\end{frame}

\begin{frame}
    \frametitle{\ejerciciocmd}
    \framesubtitle{Resolución (\rom{3}): Si $V = \SI{6,0}{\liter}$, presión del gas}
    \structure{Habíamos visto:}
    $$
        V_{\max}(\ce{CS2}) = \SI{4,83}{\liter} < V(\ce{CS2}) = \SI{6}{\liter}
    $$
    \visible<2->{
        \structure{Entonces:} todo \textbf{\underline{volumen}} de recipiente \textbf{\underline{superior}} al $V_{\max}$ en las mismas condiciones provoca la \textbf{\underline{evaporación}}. Por tanto, todo el \ce{CS2} es gas:
        \begin{overprint}
            \onslide<3>
                $$
                    P\vdot V = n\vdot R\vdot T
                $$
            \onslide<4>
                $$
                    P = \frac{\overbrace{n}^{n=\frac{m}{Mm}}\vdot R\vdot T}{V}
                $$
            \onslide<5>
                $$
                    P = \frac{m\vdot R\vdot T}{Mm\vdot V}
                $$
            \onslide<6->
                $$
                    P(\ce{CS2(g)}) = \frac{\SI{6}{\cancel\gram}\vdot\SI{,082}{\atm\cancel\liter\per\cancel\mol\per\cancel\kelvin}\vdot\SI{293,15}{\cancel\kelvin}}{\SI{76,14}{\cancel\gram\per\cancel\mol}\vdot\SI{6}{\cancel\liter}}
                $$
        \end{overprint}
        \visible<6->{
            $$
                \tcbhighmath[boxrule=0.4pt,arc=4pt,colframe=green,drop fuzzy shadow=blue]{
                    P(\ce{CS2(g)}) = \SI{,316}{\atm} = \SI{240}{\torr}
                }
            $$
                    }
    }
\end{frame}

