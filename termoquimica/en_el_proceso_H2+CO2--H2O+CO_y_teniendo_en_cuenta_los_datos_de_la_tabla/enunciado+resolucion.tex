\begin{frame}
    \frametitle{\ejerciciocmd}
    \framesubtitle{Enunciado}
    \textbf{
		Una reacción tiene una constante de velocidad de \SI{,017}{\per\second} a \SI{298}{\kelvin} y una energía libre de activación del \SI{27,235}{\kilo\joule\per\mol}. La adición de un catalizador disminuye dicha energía de activación hasta un \SI{33}{\percent} de su valor inicial. Calcule la nueva constante de velocidad.
\resultadocmd{ \SI{26,86}{\per\second} }

	}
\end{frame}

\begin{frame}
    \frametitle{\ejerciciocmd}
    \framesubtitle{Resolución (\rom{1}): $\Delta G^o$ a \SI{25}{\celsius}}
    \structure{Reacción en equilibrio:} \ce{H2(g) + CO2(g) <=> H2O(g) + CO(g)}
    \structure{Calculamos $\Delta S_R$:}
    \begin{overprint}
        \onslide<1>
            $$
                \Delta S^o_{\text{Reacción}} = \sum S^o_{\text{productos}} - S^o_{\text{reactivos}}
            $$
        \onslide<2>
            $$
                \Delta S^o_R = \sum S^o_p - S^o_r
            $$
        \onslide<3>
            $$
                \textstyle\Delta S^o_R = \scriptstyle\SI{188,72}{\joule\per\mol\per\kelvin} 
                           + \SI{197,56}{\joule\per\mol\per\kelvin} 
                           - (\SI{-130,57}{\joule\per\mol\per\kelvin} + \SI{213,68}{\joule\per\mol\per\kelvin})
            $$
        \onslide<4->
            $$
                \Delta S^o_R = \SI{303,17}{\joule\per\mol\per\kelvin}
            $$
    \end{overprint}
    \visible<5->{
        \structure{Calculamos $\Delta H^o_R$:}
        \begin{overprint}
            \onslide<5>
                $$
                    \Delta H^o_{\text{Reacción}} = \sum\Delta H^o_{\text{productos}} - \Delta H^o_{\text{reactivos}}
                $$
            \onslide<6>
                $$
                    \Delta H^o_R = \sum\Delta H^o_p - \Delta H^o_r
                $$
            \onslide<7>
                $$
                    \Delta H^o_R = \SI{-228,58}{\kilo\joule\per\mol} \SI{-137,15}{\kilo\joule\per\mol} - (\SI{0}{\kilo\joule\per\mol} \SI{-393,50}{\kilo\joule\per\mol})
                $$
            \onslide<8->
                $$
                    \Delta H^o_R = \SI{27,77}{\kilo\joule\per\mol}
                $$
        \end{overprint}
                }
    \visible<9->{
        \structure{Calculamos $\Delta G^o$:}
        \begin{overprint}
            \onslide<9>
                $$
                    \Delta G^o = \Delta H^o_R - T\vdot\Delta S^o_R
                $$
            \onslide<10>
                $$
                    \Delta G^o = \SI{27,77e3}{\joule\per\mol} - \SI{298,15}{\cancel\kelvin}\vdot\SI{303,17}{\joule\per\mol\per\cancel\kelvin}
                $$
            \onslide<11->
                $$
                    \tcbhighmath[boxrule=0.4pt,arc=4pt,colframe=red,drop fuzzy shadow=green]{\Delta G^o = \SI{-62620}{\joule\per\mol} = \SI{-62,62}{\kilo\joule\per\mol}}
                $$
        \end{overprint}
                }
\end{frame}

\begin{frame}
    \frametitle{\ejerciciocmd}
    \framesubtitle{Resolución (\rom{2}): $\Delta G$ en equilibrio}
    \visible<2->{
        \structure{Por definición de equilibrio:}
        $$
            \tcbhighmath[boxrule=0.4pt,arc=4pt,colframe=red,drop fuzzy shadow=green]{\Delta G = 0}
        $$
                }
\end{frame}

\begin{frame}
    \frametitle{\ejerciciocmd}
    \framesubtitle{Resolución (\rom{3}): $T$ de equilibrio}
    \visible<1->{
        \structure{Por definición de equilibrio:}
        $$
            \Delta G = 0
        $$
                }
    \visible<2->{
        \structure{Por definición de $\Delta G$:}
        \begin{overprint}
            \onslide<2>
                $$
                    \overbrace{\Delta G}^{\Delta G_{\text{equilibrio}} = 0} =  \Delta H - T\vdot\Delta S
                $$
            \onslide<3>
                $$
                    0 = \Delta H - T\vdot\Delta S = 0
                $$
            \onslide<4>
                $$
                    T\vdot\Delta S = \Delta H
                $$
            \onslide<5>
                $$
                    T = \frac{\Delta H}{\Delta S}
                $$
            \onslide<6->
                $$
                    T = \frac{\SI{27,77e3}{\cancel\joule\per\cancel\mol}}{\SI{303,17}{\cancel\joule\per\cancel\mol\per\kelvin}}
                $$
        \end{overprint}
                }
    \visible<7>{
        $$
            \tcbhighmath[boxrule=0.4pt,arc=4pt,colframe=red,drop fuzzy shadow=green]{T = \SI{91,60}{\kelvin}}
        $$
                }
\end{frame}

