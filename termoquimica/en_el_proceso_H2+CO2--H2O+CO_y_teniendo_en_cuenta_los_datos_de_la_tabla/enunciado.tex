En el proceso \ce{H2(g) + CO2(g) <=> H2O(g) + CO(g)}, y teniendo en cuenta los datos de la tabla a \SI{25}{\celsius}:
    \begin{center}
        \begin{tabular}{lSSSS}
            \toprule
                                                      & \ce{H2(g)} & \ce{CO2(g)} & \ce{H2O(g)} & \ce{CO(g)}\\
            \midrule
            $S_f^o$ (\si{\joule\per\kelvin\per\mol})  & -130,57    &  213,68     &  188,72     &  197,56   \\
            $\Delta H_f^o$ \si{\kilo\joule\per\mol}   & 0          & -393,50     & -228,58     & -137,15   \\   
            \bottomrule
        \end{tabular}
    \end{center}
    Calcular:
    \begin{enumerate}[label={\alph*)},font={\color{red!50!black}\bfseries}]
        \item $\Delta G$ a \SI{25}{\celsius}.
        \item ¿Cuál es el valor de $\Delta G$ en el equilibrio a $T$ y $P$ constante?
        \item ¿A qué temperatura se alcanza el equilibrio? (Suponer que $\Delta H$ y $\Delta S$ no varían con la temperatura y con la $P$).
    \end{enumerate}
\resultadocmd{
            \SI{-62,62}{\kilo\joule\per\mol};
            \num{0};
            \SI{91,60}{\kelvin}
}
