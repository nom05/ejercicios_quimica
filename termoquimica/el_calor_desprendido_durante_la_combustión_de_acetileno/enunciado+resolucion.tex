\begin{frame}
	\frametitle{\ejerciciocmd}
	\framesubtitle{Enunciado}
	\textbf{
		Una reacción tiene una constante de velocidad de \SI{,017}{\per\second} a \SI{298}{\kelvin} y una energía libre de activación del \SI{27,235}{\kilo\joule\per\mol}. La adición de un catalizador disminuye dicha energía de activación hasta un \SI{33}{\percent} de su valor inicial. Calcule la nueva constante de velocidad.
\resultadocmd{ \SI{26,86}{\per\second} }

		}
\end{frame}

\begin{frame}
	\frametitle{\ejerciciocmd}
	\framesubtitle{Datos del problema}
	\centering{\huge $\Delta H^o_{\text{formación}}(\ce{C2H2})=\Delta H^o_f(\ce{C2H2})$?}
	\begin{center}
		\tcbhighmath[boxrule=0.4pt,arc=4pt,colframe=yellow,drop fuzzy shadow=black]{\Delta H^o_{\text{combustión}}(\ce{C2H2})=\Delta H^o_C(\ce{C2H2})=\SI{-1299,1}{\kilo\joule\per\mol}}
	\end{center}
	\begin{center}
		\tcbhighmath[boxrule=0.4pt,arc=4pt,colframe=green,drop fuzzy shadow=orange]{\Delta H^o_f(\ce{O2})=\SI{0}{\kilo\joule\per\mol}}
		\tcbhighmath[boxrule=0.4pt,arc=4pt,colframe=blue,drop fuzzy shadow=green]{\Delta H^o_f(\ce{CO2})=\SI{-393,5}{\kilo\joule\per\mol}}
		\tcbhighmath[boxrule=0.4pt,arc=4pt,colframe=red,drop fuzzy shadow=blue]{\Delta H^o_f(\ce{H2O})=\SI{-285,8}{\kilo\joule\per\mol}}
	\end{center}
	\visible<2->{
		\myovalbox{\textcolor{white}{NOTA IMPORTANTE:}} El texto del enunciado pone ``calor desprendido''. Por tanto, según el criterio de signos de la Termodinámica, la \textbf{\underline{entalpía de combustión}} del \ce{C2H2} será \textbf{\underline{negativa}}.
				}
\end{frame}

\begin{frame}
	\frametitle{\ejerciciocmd}
	\framesubtitle{Resolución (\rom{1}): reacción de combustión y ley de Hess}
	\structure{Reacción de combustión de \ce{C2H2}:}\\[.4cm]
	\begin{overprint}
		\onslide<1>
			\centering\ce{C2H2(g) + O2(g) -> CO2(g) + H2O(l)}
		\onslide<2>
			\centering
			\tcbhighmath[boxrule=0.4pt,arc=4pt,colframe=green,drop fuzzy shadow=blue]{\ce{\textbf{C2}H2(g)}}
			\ce{ + O2(g) -> }
			\tcbhighmath[boxrule=0.4pt,arc=4pt,colframe=green,drop fuzzy shadow=blue]{\ce{\textbf{C}O2(g)}}
			\ce{ + H2O(l)} (\textbf{\ce{C}} sin ajustar)
		\onslide<3>
			\centering
			\tcbhighmath[boxrule=0.4pt,arc=4pt,colframe=green,drop fuzzy shadow=blue]{\ce{\textbf{C2}H2(g)}}
			\ce{ + O2(g) -> }
			\tcbhighmath[boxrule=0.4pt,arc=4pt,colframe=green,drop fuzzy shadow=blue]{\ce{\textbf{2C}O2(g)}}
			\ce{ + H2O(l)} (\textbf{\ce{C}} ajustado)
		\onslide<4>
			\centering
			\tcbhighmath[boxrule=0.4pt,arc=4pt,colframe=red,drop fuzzy shadow=green]{\ce{C2\textbf{H2}(g)}}
			\ce{ + O2(g) -> }
			\ce{2CO2(g) + }
			\tcbhighmath[boxrule=0.4pt,arc=4pt,colframe=red,drop fuzzy shadow=green]{\ce{\textbf{H2}O(l)}} (\textbf{\ce{H}} ajustado)
		\onslide<5>
			\centering
			\ce{C2H2(g) + }
			\tcbhighmath[boxrule=0.4pt,arc=4pt,colframe=blue,drop fuzzy shadow=red]{\ce{O2(g)}}
			\ce{ -> }
			\tcbhighmath[boxrule=0.4pt,arc=4pt,colframe=blue,drop fuzzy shadow=red]{\ce{\textbf{2}C\textbf{O2}(g)}}
			\ce{ + }
			\tcbhighmath[boxrule=0.4pt,arc=4pt,colframe=blue,drop fuzzy shadow=red]{\ce{H2\textbf{O}(l)}} (\textbf{\ce{O}} sin ajustar)
		\onslide<6>
			\centering
			\ce{C2H2(g) + }
			\tcbhighmath[boxrule=0.4pt,arc=4pt,colframe=blue,drop fuzzy shadow=red]{\ce{\textbf{$\mathbf{\rfrac{5}{2}}$O2}(g)}}
			\ce{ -> }
			\tcbhighmath[boxrule=0.4pt,arc=4pt,colframe=blue,drop fuzzy shadow=red]{\ce{\textbf{2}C\textbf{O2}(g)}}
			\ce{ + }
			\tcbhighmath[boxrule=0.4pt,arc=4pt,colframe=blue,drop fuzzy shadow=red]{\ce{H2\textbf{O}(l)}} (\textbf{\ce{O}} ajustado)
        \onslide<7->
            \centering\structure<8>{\textbf{Ajuste 1}:} \ce{C2H2(g) + 5/2O2(g) -> 2CO2(g) + H2O(l)}\\
            \centering\structure<9>{\textbf{Ajuste 2:}} \ce{2C2H2(g) + 5O2(g) -> 4CO2(g) + 2H2O(l)}
	\end{overprint}
    \visible<7->{
        \structure{Por la ley de Hess:} $n(\ce{C2H2})\times\Delta H^o_C(\ce{C2H2}) = \sum_{i=1}^{n\text{ prods}}{n_i\cdot\Delta H^o_{f,i}} - \sum_{j=1}^{n\text{ reacts}}{n_j\cdot\Delta H^o_{f,j}}$
                }
    \visible<8->{
        \structure<8>{\textbf{Ajuste 1:}}
        \begin{align*}
            \SI{1}{\mol}\times\Delta H^o_C(\ce{C2H2}) &= \overbrace{2}^{n(\ce{CO2})}\times\Delta H^o_f(\ce{CO2}) 
            + \overbrace{1}^{n(\ce{H2O})}\times\Delta H^o_f(\ce{H2O})\\
            &- \underbrace{1}_{n(\ce{C2H2})}\times\Delta H^o_f(\ce{C2H2})
            - \underbrace{\rfrac{5}{2}}_{n(\ce{O2})}\times\cancelto{0}{\Delta H^o_f(\ce{O2})}
        \end{align*}
                }
    \visible<9->{
        \structure<9>{\textbf{Ajuste 2:}}
        \begin{align*}
            \SI{2}{\mol}\times\Delta H^o_C(\ce{C2H2}) &= \overbrace{4}^{n(\ce{CO2})}\times\Delta H^o_f(\ce{CO2}) 
            + \overbrace{2}^{n(\ce{H2O})}\times\Delta H^o_f(\ce{H2O})\\
            &- \underbrace{2}_{n(\ce{C2H2})}\times\Delta H^o_f(\ce{C2H2})
            - \underbrace{5}_{n(\ce{O2})}\times\cancelto{0}{\Delta H^o_f(\ce{O2})}
        \end{align*}
               }
\end{frame}

\begin{frame}
    \frametitle{\ejerciciocmd}
    \framesubtitle{Resolución (\rom{2}): cálculo de la entalpía de formación del \ce{C2H2}}
    \structure{Partiendo del \textbf{ajuste 1} de la anterior diapositiva y despejando:}\\[.4cm]
    \begin{overprint}
        \onslide<1>
            $$
                \Cline[blue]{\Delta H^o_C(\ce{C2H2})} = 2\cdot\Delta H^o_f(\ce{CO2}) + \Delta H^o_f(\ce{H2O}) - \Cline[green]{\Delta H^o_f(\ce{C2H2})}
            $$
        \onslide<2>
            $$
                \Cline[green]{\Delta H^o_f(\ce{C2H2})} = 2\cdot\Delta H^o_f(\ce{CO2}) + \Delta H^o_f(\ce{H2O}) - \Cline[blue]{\Delta H^o_C(\ce{C2H2})}
            $$
    \end{overprint}
    \visible<2->{
        \structure{Sustituimos por los correspondientes valores:}
        $$
            \Delta H^o_f(\ce{C2H2}) = 2\cdot\left(\SI{-393,5}{\kilo\joule\per\mol}\right) + \left(\SI{-285,80}{\kilo\joule\per\mol}\right) - \left(\SI{-1299,1}{\kilo\joule\per\mol}\right)
        $$
        \centering\tcbhighmath[boxrule=0.4pt,arc=4pt,colframe=yellow,drop fuzzy shadow=black]{\Delta H^o_f(\ce{C2H2})=\SI{226,3}{\kilo\joule\per\mol}}
                }
\end{frame}
