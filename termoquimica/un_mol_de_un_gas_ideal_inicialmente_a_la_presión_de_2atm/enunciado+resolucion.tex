\begin{frame}
	\frametitle{\ejerciciocmd}
	\framesubtitle{Enunciado}
	\textbf{
		Una reacción tiene una constante de velocidad de \SI{,017}{\per\second} a \SI{298}{\kelvin} y una energía libre de activación del \SI{27,235}{\kilo\joule\per\mol}. La adición de un catalizador disminuye dicha energía de activación hasta un \SI{33}{\percent} de su valor inicial. Calcule la nueva constante de velocidad.
\resultadocmd{ \SI{26,86}{\per\second} }

		}
\end{frame}

\begin{frame}
	\frametitle{\ejerciciocmd}
	\framesubtitle{Datos del problema (\rom{1})}
	\textbf{¿$\Delta U\text{ (Energía interna)}$, $Q$, $W$?}
	$$
		\tcbhighmath[boxrule=0.4pt,arc=4pt,colframe=black,drop fuzzy shadow=red]{n = \SI{1}{\mol}}\quad
		\tcbhighmath[boxrule=0.4pt,arc=4pt,colframe=black,drop fuzzy shadow=red]{C_p = \frac{5}{2}R}\quad
		\tcbhighmath[boxrule=0.4pt,arc=4pt,colframe=black,drop fuzzy shadow=red]{C_v = \frac{3}{2}R}
	$$
	\structure{Estado 1:}
		$$
			\tcbhighmath[boxrule=0.4pt,arc=4pt,colframe=green,drop fuzzy shadow=red]{P_1=\SI{2}{\atm}}\quad
			\tcbhighmath[boxrule=0.4pt,arc=4pt,colframe=green,drop fuzzy shadow=red]{V_1=\SI{25}{\liter}}
		$$
	\structure{Estado 2:}
		$$
			\tcbhighmath[boxrule=0.4pt,arc=4pt,colframe=blue,drop fuzzy shadow=green]{P_2=\SI{2}{\atm}}\quad
			\tcbhighmath[boxrule=0.4pt,arc=4pt,colframe=blue,drop fuzzy shadow=green]{V_2=\SI{50}{\liter}}
		$$
	\structure{Estado 3:}
		$$
			\tcbhighmath[boxrule=0.4pt,arc=4pt,colframe=red,drop fuzzy shadow=blue]{P_3=\SI{1}{\atm}}\quad
			\tcbhighmath[boxrule=0.4pt,arc=4pt,colframe=red,drop fuzzy shadow=blue]{V_3=\SI{25}{\liter}}
		$$
	\structure{Estado 4:}
		$$
			\tcbhighmath[boxrule=0.4pt,arc=4pt,colframe=orange,drop fuzzy shadow=black]{P_4=\SI{1}{\atm}}\quad
			\tcbhighmath[boxrule=0.4pt,arc=4pt,colframe=orange,drop fuzzy shadow=black]{V_4=\SI{50}{\liter}}
		$$
\end{frame}

\begin{frame}
	\frametitle{\ejerciciocmd}
	\framesubtitle{Datos del problema (\rom{2})}
	\textbf{¿$\Delta U\text{ (Energía interna)}$, $Q$, $W$?}
	\structure{Proceso \rom{1}:} \colorbox{green}{\color{white}\textbf{1}}\ce{->}\colorbox{blue}{\color{white}\textbf{2}}
		$$
			\tcbhighmath[boxrule=0.4pt,arc=4pt,colframe=green,drop fuzzy shadow=red]{\Delta P_{\rom{1}}=\SI{0}{\atm}}\quad
			\tcbhighmath[boxrule=0.4pt,arc=4pt,colframe=green,drop fuzzy shadow=red]{\Delta V_{\rom{1}}=\SI{25}{\liter}}
		$$
	\structure{Proceso \rom{2}:} \colorbox{blue}{\color{white}\textbf{2}}\ce{->}\colorbox{orange}{\color{white}\textbf{4}}
		$$
			\tcbhighmath[boxrule=0.4pt,arc=4pt,colframe=blue,drop fuzzy shadow=green]{\Delta P_{\rom{2}}=\SI{-1}{\atm}}\quad
			\tcbhighmath[boxrule=0.4pt,arc=4pt,colframe=blue,drop fuzzy shadow=green]{\Delta V_{\rom{2}}=\SI{0}{\liter}}
		$$
	\structure{Proceso \rom{3}:} \colorbox{green}{\color{white}\textbf{1}}\ce{->}\colorbox{red}{\color{white}\textbf{3}}
		$$
			\tcbhighmath[boxrule=0.4pt,arc=4pt,colframe=red,drop fuzzy shadow=blue]{\Delta P_{\rom{3}}=\SI{-1}{\atm}}\quad
			\tcbhighmath[boxrule=0.4pt,arc=4pt,colframe=red,drop fuzzy shadow=blue]{\Delta V_{\rom{3}}=\SI{0}{\liter}}
		$$
	\structure{Proceso \rom{4}:} \colorbox{red}{\color{white}\textbf{3}}\ce{->}\colorbox{orange}{\color{white}\textbf{4}}
		$$
			\tcbhighmath[boxrule=0.4pt,arc=4pt,colframe=orange,drop fuzzy shadow=black]{\Delta P_{\rom{4}}=\SI{0}{\atm}}\quad
			\tcbhighmath[boxrule=0.4pt,arc=4pt,colframe=orange,drop fuzzy shadow=black]{\Delta V_{\rom{4}}=\SI{25}{\liter}}
		$$
\end{frame}

\begin{frame}
	\frametitle{\ejerciciocmd}
	\framesubtitle{Resolución (\rom{1}): determinación de las temperaturas de los estados}
	\structure{Usando la ecuación de los gases ideales:} $P\cdot V = n\cdot R\cdot T\Rightarrow T=\rfrac{P\cdot V}{n\cdot R}$
	\begin{columns}
		\column{.45\textwidth}
			\structure{Estado 1}
				$$
					T_1=\frac{\overbrace{P_1}^{\SI{2}{\atm}}\cdot\overbrace{V_1}^{\SI{25}{\liter}}}{\underbrace{n}_{\SI{1}{\mol}}\cdot R}=\frac{50}{R}~\si{\kelvin}
				$$
			\structure{Estado 2}
				$$
					T_2=\frac{\overbrace{P_2}^{\SI{2}{\atm}}\cdot\overbrace{V_2}^{\SI{50}{\liter}}}{\underbrace{n}_{\SI{1}{\mol}}\cdot R}=\frac{100}{R}~\si{\kelvin}
				$$
		\column{.45\textwidth}
			\structure{Estado 3}
				$$
					T_3=\frac{\overbrace{P_3}^{\SI{1}{\atm}}\cdot\overbrace{V_3}^{\SI{25}{\liter}}}{\underbrace{n}_{\SI{1}{\mol}}\cdot R}=\frac{25}{R}~\si{\kelvin}
				$$
			\structure{Estado 4}
				$$
					T_4=\frac{\overbrace{P_4}^{\SI{1}{\atm}}\cdot\overbrace{V_4}^{\SI{50}{\liter}}}{\underbrace{n}_{\SI{1}{\mol}}\cdot R}=\frac{50}{R}~\si{\kelvin}
				$$
	\end{columns}
\end{frame}

\begin{frame}
	\frametitle{\ejerciciocmd}
	\framesubtitle{Resolución (\rom{2}): energía interna, calor y trabajo de cada proceso}
	\structure{Partiendo de la expresión:} $\Delta U=Q+W$
	\visible<1->{
		\structure{Proceso \rom{1}:} (presión constante)
			$$
				\Delta U_{\rom{1}} = Q_P + W_P = \overbrace{n\cdot C_P\cdot\Delta T}^{Q_P}  - \overbrace{P\cdot\Delta V_{\rom{1}}}^{W_P}
			$$
			\begin{overprint}
				\onslide<1>
					$$
						\Delta U_{\rom{1}} = \SI{1}{\cancel\mol}\cdot\frac{5}{2}\cancel{R}~\si{\atm\liter\per\cancel\mol\per\cancel\kelvin}\cdot\left(\frac{100}{\cancel{R}}-\frac{50}{\cancel{R}}\right)\si{\cancel\kelvin}
						-\SI{2}{\atm}\cdot\SI{25}{\liter}
					$$
				\onslide<2->
					$$
						\Delta U_{\rom{1}} = \overbrace{\SI{125}{\atm\liter}}^{Q_{\rom{1}}}-\overbrace{\SI{50}{\atm\liter}}^{W_{\rom{1}}} = \SI{75}{\cancel\atm\cancel\liter}\cdot\frac{\SI{8,314}{\joule\per\cancel\mol\per\cancel\kelvin}}{\SI{,082}{\cancel\atm\cancel\liter\per\cancel\mol\per\cancel\kelvin}}
					$$
			\end{overprint}
				}
	\visible<3->{
		\structure{Proceso \rom{2}:} (volumen constante)
			$$
				\Delta U_{\rom{2}} = Q_V + W_V = \overbrace{n\cdot C_V\cdot\Delta T}^{Q_V}  - \cancelto{W_V=0}{P\cdot\Delta V_{\rom{2}}}
			$$
				\begin{overprint}
					\onslide<3>
						$$
							\Delta U_{\rom{2}} = \SI{1}{\cancel\mol}\cdot\frac{3}{2}\cancel{R}~\si{\atm\liter\per\cancel\mol\per\cancel\kelvin}\cdot\left(\frac{50}{\cancel{R}}-\frac{100}{\cancel{R}}\right)\si{\cancel\kelvin}
						$$
					\onslide<4>
						$$
							\Delta U_{\rom{2}} = \overbrace{\SI{-75}{\atm\liter}}^{Q_{\rom{2}}} = \SI{-75}{\cancel\atm\cancel\liter}\cdot\frac{\SI{8,314}{\joule\per\cancel\mol\per\cancel\kelvin}}{\SI{,082}{\cancel\atm\cancel\liter\per\cancel\mol\per\cancel\kelvin}}
						$$
				\end{overprint}
				}
\end{frame}

\begin{frame}
	\frametitle{\ejerciciocmd}
	\framesubtitle{Resolución (\rom{3}): energía interna, calor y trabajo de cada proceso}
	\structure{Partiendo de la expresión:} $\Delta U=Q+W$
	\visible<1->{
		\structure{Proceso \rom{3}:} (volumen constante)
			$$
				\Delta U_{\rom{3}} = Q_V + W_V = \overbrace{n\cdot C_V\cdot\Delta T}^{Q_V}  - \cancelto{W_V=0}{P\cdot\Delta V_{\rom{2}}}
			$$
			\begin{overprint}
				\onslide<1>
					$$
						\Delta U_{\rom{3}} = \SI{1}{\cancel\mol}\cdot\frac{3}{2}\cancel{R}~\si{\atm\liter\per\cancel\mol\per\cancel\kelvin}\cdot\left(\frac{25}{\cancel{R}}-\frac{50}{\cancel{R}}\right)\si{\cancel\kelvin}
					$$
				\onslide<2->
					$$
						\Delta U_{\rom{3}} = \overbrace{\frac{3}{2}\times\SI{-25}{\atm\liter}}^{Q_{\rom{3}}} = -\frac{75}{2}\si{\cancel\atm\cancel\liter}\cdot\frac{\SI{8,314}{\joule\per\cancel\mol\per\cancel\kelvin}}{\SI{,082}{\cancel\atm\cancel\liter\per\cancel\mol\per\cancel\kelvin}}
					$$
			\end{overprint}
				}
	\visible<3->{
		\structure{Proceso \rom{4}:} (presión constante)
			$$
				\Delta U_{\rom{4}} = Q_P + W_P = \overbrace{n\cdot C_P\cdot\Delta T}^{Q_P}  - - \overbrace{P\cdot\Delta V_{\rom{4}}}^{W_P}
			$$
			\begin{overprint}
				\onslide<3>
					$$
						\Delta U_{\rom{4}} = \SI{1}{\cancel\mol}\cdot\frac{5}{2}\cancel{R}~\si{\atm\liter\per\cancel\mol\per\cancel\kelvin}\cdot\left(\frac{50}{\cancel{R}}-\frac{25}{\cancel{R}}\right)\si{\cancel\kelvin}
-\SI{1}{\atm}\cdot\SI{25}{\liter}
					$$
				\onslide<4>
				$$
						\Delta U_{\rom{4}} = \overbrace{\frac{125}{2}\si{\atm\liter}}^{Q_{\rom{4}}}-\overbrace{\SI{25}{\atm\liter}}^{W_{\rom{4}}} = \frac{75}{2}\si{\cancel\atm\cancel\liter}\cdot\frac{\SI{8,314}{\joule\per\cancel\mol\per\cancel\kelvin}}{\SI{,082}{\cancel\atm\cancel\liter\per\cancel\mol\per\cancel\kelvin}}
				$$
			\end{overprint}
				}
\end{frame}

\begin{frame}
	\frametitle{\ejerciciocmd}
	\framesubtitle{Resolución (\rom{4}): energía interna, calor y trabajo de los apartados a y b}
	\structure{Apartado a:} Suma de procesos \rom{1} y \rom{2}
		$$
			\tcbhighmath[boxrule=0.4pt,arc=4pt,colframe=green,drop fuzzy shadow=red]{\Delta U_{\text{apartado a}}} = \Delta U_{\rom{1}} + \Delta U_{\rom{2}}\Rightarrow\Delta U_{\text{apartado a}}=75\cdot\frac{8,314}{0,082}\si{\joule}-75\cdot\frac{8,314}{0,082}\si{\joule}=\tcbhighmath[boxrule=0.4pt,arc=4pt,colframe=green,drop fuzzy shadow=red]{\SI{0}{\joule}}
		$$
		$$
			\tcbhighmath[boxrule=0.4pt,arc=4pt,colframe=green,drop fuzzy shadow=red]{Q_{\text{apartado a}}} = (125-75)\cdot\frac{8,314}{0,082}\si{\joule} = \tcbhighmath[boxrule=0.4pt,arc=4pt,colframe=green,drop fuzzy shadow=red]{\SI{5069,5}{\joule}}
		$$
		$$
			\tcbhighmath[boxrule=0.4pt,arc=4pt,colframe=green,drop fuzzy shadow=red]{W_{\text{apartado a}}} = -50\cdot\frac{8,314}{0,082}\si{\joule} = \tcbhighmath[boxrule=0.4pt,arc=4pt,colframe=green,drop fuzzy shadow=red]{\SI{-5069,5}{\joule}}
		$$
	\visible{
		\structure{Apartado b:} Suma de procesos \rom{3} y \rom{4}
			$$
				\tcbhighmath[boxrule=0.4pt,arc=4pt,colframe=blue,drop fuzzy shadow=green]{\Delta U_{\text{apartado b}}} = \Delta U_{\rom{3}} + \Delta U_{\rom{4}}\Rightarrow\Delta U_{\text{apartado b}}=-\frac{75}{2}\cdot\frac{8,314}{0,082}\si{\joule}+\frac{75}{2}\cdot\frac{8,314}{0,082}\si{\joule}=\tcbhighmath[boxrule=0.4pt,arc=4pt,colframe=blue,drop fuzzy shadow=green]{\SI{0}{\joule}}
			$$
			$$
				\tcbhighmath[boxrule=0.4pt,arc=4pt,colframe=blue,drop fuzzy shadow=green]{Q_{\text{apartado a}}} = \left(-\frac{75}{2}+\frac{125}{2}\right)\cdot\frac{8,314}{0,082}\si{\joule} = \tcbhighmath[boxrule=0.4pt,arc=4pt,colframe=blue,drop fuzzy shadow=green]{\SI{2534,8}{\joule}}
			$$
			$$
				\tcbhighmath[boxrule=0.4pt,arc=4pt,colframe=blue,drop fuzzy shadow=green]{W_{\text{apartado a}}} = -25\cdot\frac{8,314}{0,082}\si{\joule} = \tcbhighmath[boxrule=0.4pt,arc=4pt,colframe=blue,drop fuzzy shadow=green]{\SI{-2534,8}{\joule}}
			$$
			}
\end{frame}

\begin{frame}
	\frametitle{\ejerciciocmd}
	\framesubtitle{Resolución (\rom{4}): energía interna, calor y trabajo del apartado c}
	\structure{Condiciones de este apartado:} $\Delta T=0$ y $\Delta P=0$
    $$
        PV=nRT\Rightarrow P\mathrm{d}V=nR\mathrm{d}T
    $$
    Si $\Delta T = 0$ entonces también $\mathrm{d}T=0$ si el proceso es reversible:
	\begin{block}<2->{Proceso reversible}
        $$
            \mathrm{d}U = n\cdot c_V\mathrm{d}T - P\mathrm{d}V = \underbrace{\cancelto{0}{n\cdot c_V\mathrm{d}T}}_{Q=0} - \underbrace{\cancelto{0}{n\cdot R\mathrm{d}T}}_{W=0}\Rightarrow\Delta U=\SI{0}{\joule}
        $$
	\end{block}
    \begin{alertblock}<3->{Proceso irreversible}
        $$
            \mathrm{d}U=\cancelto{Q=0}{n\cdot c_V\mathrm{d}T} - P\mathrm{d}V\Rightarrow\Delta U=-P\int\mathrm{d}V\Rightarrow \Delta U=W=\SI{-1}{\atm}\cdot\left(50-25\right)\si{\liter}=\SI{-2534,8}{\joule}
        $$
    \end{alertblock}
\end{frame}
