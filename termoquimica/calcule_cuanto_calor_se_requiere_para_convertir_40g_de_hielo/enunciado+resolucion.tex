\begin{frame}
	\frametitle{\ejerciciocmd}
	\framesubtitle{Enunciado}
	\textbf{
		Dadas las siguientes reacciones:
\begin{itemize}
    \item \ce{I2(g) + H2(g) -> 2 HI(g)}~~~$\Delta H_1 = \SI{-0,8}{\kilo\calorie}$
    \item \ce{I2(s) + H2(g) -> 2 HI(g)}~~~$\Delta H_2 = \SI{12}{\kilo\calorie}$
    \item \ce{I2(g) + H2(g) -> 2 HI(ac)}~~~$\Delta H_3 = \SI{-26,8}{\kilo\calorie}$
\end{itemize}
Calcular los parámetros que se indican a continuación:
\begin{description}%[label={\alph*)},font={\color{red!50!black}\bfseries}]
    \item[\texttt{a)}] Calor molar latente de sublimación del yodo.
    \item[\texttt{b)}] Calor molar de disolución del ácido yodhídrico.
    \item[\texttt{c)}] Número de calorías que hay que aportar para disociar en sus componentes el yoduro de hidrógeno gas contenido en un matraz de \SI{750}{\cubic\centi\meter} a \SI{25}{\celsius} y \SI{800}{\torr} de presión.
\end{description}
\resultadocmd{\SI{12,8}{\kilo\calorie}; \SI{-13,0}{\kilo\calorie}; \SI{12,9}{\calorie}}

		}
\end{frame}

\begin{frame}
	\frametitle{\ejerciciocmd}
	\framesubtitle{Datos del problema}
	\centering{\huge $Q_{\text{absorbido}}$?}
	\begin{center}
		\tcbhighmath[boxrule=0.4pt,arc=4pt,colframe=yellow,drop fuzzy shadow=black]{m(\ce{H2O})=\SI{40}{\gram}}
	\end{center}
	\begin{center}
		\tcbhighmath[boxrule=0.4pt,arc=4pt,colframe=orange,drop fuzzy shadow=black]{c_e(\ce{H2O(s)})=c_e(\ce{H2O(g)})=\SI{,5}{\calorie\per\gram\per\kelvin}}
		\tcbhighmath[boxrule=0.4pt,arc=4pt,colframe=orange,drop fuzzy shadow=black]{c_e(\ce{H2O(l)})=\SI{1,0}{\calorie\per\gram\per\kelvin}}
	\end{center}
	\begin{center}
		\tcbhighmath[boxrule=0.4pt,arc=4pt,colframe=red,drop fuzzy shadow=green]{T_{\text{inicial}}=\SI{-10}{\celsius}}
		\tcbhighmath[boxrule=0.4pt,arc=4pt,colframe=red,drop fuzzy shadow=black]{T_{\text{fus}}=\SI{0}{\celsius}}
		\tcbhighmath[boxrule=0.4pt,arc=4pt,colframe=red,drop fuzzy shadow=black]{T_{\text{vap}}=\SI{100}{\celsius}}
		\tcbhighmath[boxrule=0.4pt,arc=4pt,colframe=red,drop fuzzy shadow=green]{T_{\text{final}}=\SI{120}{\celsius}}
	\end{center}
	\begin{center}
		\tcbhighmath[boxrule=0.4pt,arc=4pt,colframe=blue,drop fuzzy shadow=black]{\Delta H^o_{\text{fus}}=\SI{80}{\calorie\per\gram}}
		\tcbhighmath[boxrule=0.4pt,arc=4pt,colframe=blue,drop fuzzy shadow=black]{\Delta H^o_{\text{vap}}=\SI{540}{\calorie\per\gram}}
	\end{center}
\end{frame}

\begin{frame}
	\frametitle{\ejerciciocmd}
	\framesubtitle{Resolución (\rom{1}): esquema del proceso termodinámico y calor total}
	\begin{overprint}
		\onslide<1>
			\structure{Partimos del siguiente estado:}
			$$
				\ce{
						$\underset{\SI{-10}{\celsius}}{\ce{H2O(s)}}$
					}
			$$
		\onslide<2>
			\structure{Calentamos hasta la temperatura de fusión:}
			$$
				\ce{
						$\underset{\SI{-10}{\celsius}}{\ce{H2O(s)}}$
						->[$\textcolor{red}{Q_1}$][{$\Delta T_1=\SI{10}{\celsius}$}]
						$\underset{\SI{0}{\celsius}}{\ce{H2O(s)}}$
					}
			$$
		\onslide<3>
			\structure{Proceso de fusión:}
			$$
				\ce{
						$\underset{\SI{-10}{\celsius}}{\ce{H2O(s)}}$
						->[$\textcolor{red}{Q_1}$][{$\Delta T_1=\SI{10}{\celsius}$}]
						$\underset{\SI{0}{\celsius}}{\ce{H2O(s)}}$
						->[$\textcolor{blue}{Q_2}$]
						$\underset{\SI{0}{\celsius}}{\ce{H2O(l)}}$
					}
			$$
		\onslide<4>
			\structure{Calentamos hasta la vaporización:}
				$$
					\ce{
						$\underset{\SI{-10}{\celsius}}{\ce{H2O(s)}}$
						->[$\textcolor{red}{Q_1}$][{$\Delta T_1=\SI{10}{\celsius}$}]
						$\underset{\SI{0}{\celsius}}{\ce{H2O(s)}}$
						->[$\textcolor{blue}{Q_2}$]
						$\underset{\SI{0}{\celsius}}{\ce{H2O(l)}}$
						->[$\textcolor{green}{Q_3}$][{$\Delta T_3=\SI{100}{\celsius}$}]
						$\underset{\SI{100}{\celsius}}{\ce{H2O(l)}}$
					}
				$$
		\onslide<5>
			\structure{Proceso de vaporización:}
				$$
					\ce{
						$\underset{\SI{-10}{\celsius}}{\ce{H2O(s)}}$
						->[$\textcolor{red}{Q_1}$][{$\Delta T_1=\SI{10}{\celsius}$}]
						$\underset{\SI{0}{\celsius}}{\ce{H2O(s)}}$
						->[$\textcolor{blue}{Q_2}$]
						$\underset{\SI{0}{\celsius}}{\ce{H2O(l)}}$
						->[$\textcolor{green}{Q_3}$][{$\Delta T_3=\SI{100}{\celsius}$}]
						$\underset{\SI{100}{\celsius}}{\ce{H2O(l)}}$
						->[$\textcolor{orange}{Q_4}$]
						$\underset{\SI{100}{\celsius}}{\ce{H2O(g)}}$
					}
				$$
		\onslide<6>
			\structure{Calentamos hasta la temperatura final:}
				$$
					\ce{
						$\underset{\SI{-10}{\celsius}}{\ce{H2O(s)}}$
						->[$\textcolor{red}{Q_1}$][{$\Delta T_1=\SI{10}{\celsius}$}]
						$\underset{\SI{0}{\celsius}}{\ce{H2O(s)}}$
						->[$\textcolor{blue}{Q_2}$]
						$\underset{\SI{0}{\celsius}}{\ce{H2O(l)}}$
						->[$\textcolor{green}{Q_3}$][{$\Delta T_3=\SI{100}{\celsius}$}]
						$\underset{\SI{100}{\celsius}}{\ce{H2O(l)}}$
						->[$\textcolor{orange}{Q_4}$]
%						$\underset{\SI{100}{\celsius}}{\ce{H2O(g)}}$
%						->[$Q_5$][{$\Delta T_5=\SI{20}{\celsius}$}]
%						$\underset{\SI{120}{\celsius}}{\ce{H2O(g)}}$
					}
				$$
				$$
					\ce{
						->[$\textcolor{orange}{Q_4}$]
						$\underset{\SI{100}{\celsius}}{\ce{H2O(g)}}$
						->[$Q_5$][{$\Delta T_5=\SI{20}{\celsius}$}]
						$\underset{\SI{120}{\celsius}}{\ce{H2O(g)}}$
					}
				$$
	\end{overprint}
	\visible<2->{
		$$
			\textcolor{red}{Q_1} = m\cdot c_e(\ce{H2O(s)})\cdot\Delta T_1\Rightarrow \textcolor{red}{Q_1} = \SI{40}{\cancel\gram}\cdot\SI{,5}{\calorie\per\cancel\gram\per\cancel\celsius}\cdot\SI{10}{\cancel\celsius} = \SI{200}{\calorie}
		$$
				}
	\visible<3->{
		$$
			\textcolor{blue}{Q_2} =  \SI{80}{\calorie\per\cancel\gram}\cdot\SI{40}{\cancel\gram} = \SI{3200}{\calorie}
		$$
				}
	\visible<4->{
		$$
			\textcolor{green}{Q_3} = m\cdot c_e(\ce{H2O(l)})\cdot\Delta T_3\Rightarrow \textcolor{green}{Q_3} = \SI{40}{\cancel\gram}\cdot\SI{1,0}{\calorie\per\cancel\gram\per\cancel\celsius}\cdot\SI{100}{\cancel\celsius} = \SI{4000}{\calorie}
		$$
				}
	\visible<5->{
		$$
			\textcolor{orange}{Q_4} =  \SI{540}{\calorie\per\cancel\gram}\cdot\SI{40}{\cancel\gram} = \SI{21600}{\calorie}
		$$
				}
	\visible<6->{
		$$
			Q_5 = m\cdot c_e(\ce{H2O(g)})\cdot\Delta T_5\Rightarrow Q_5 = \SI{40}{\cancel\gram}\cdot\SI{,5}{\calorie\per\cancel\gram\per\cancel\celsius}\cdot\SI{20}{\cancel\celsius} = \SI{400}{\calorie}
		$$
		$$
			Q_{\text{total}} = \sum_{i=1}^5 Q_i = \SI{200}{\calorie}+\SI{3200}{\calorie}+\SI{4000}{\calorie}+\SI{21600}{\calorie}+\SI{400}{\calorie}
		$$
		\centering\tcbhighmath[boxrule=0.4pt,arc=4pt,colframe=orange,drop fuzzy shadow=black]{Q_{\text{total}} = \SI{29400}{\calorie} = \SI{29,40}{\kilo\calorie}}
				}
\end{frame}
