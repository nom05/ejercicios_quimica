La combustión del propano desprende \SI{2200}{\kilo\joule\per\mol}. Suponiendo que se aprovecha el \SI{80}{\percent} de esta energía ¿Cuántos \si{\gram} de agua a \SI{15}{\celsius} se pueden convertir en vapor de agua a \SI{100}{\celsius} cuando se queman \SI{1000}{\liter} de propano medidos a \SI{1}{\atm} y \SI{0}{\celsius}?\\
Datos del \ce{H2O}: $\Delta H_v= \SI{40,7}{\kilo\joule\per\mol}$; $C_e = \SI{1}{\calorie\per\gram\per\celsius}$.
\resultadocmd{\SI{30061,111}{\gram}}
