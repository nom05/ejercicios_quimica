\begin{frame}
    \frametitle{\ejerciciocmd}
    \framesubtitle{Enunciado}
    \textbf{
	Dadas las siguientes reacciones:
\begin{itemize}
    \item \ce{I2(g) + H2(g) -> 2 HI(g)}~~~$\Delta H_1 = \SI{-0,8}{\kilo\calorie}$
    \item \ce{I2(s) + H2(g) -> 2 HI(g)}~~~$\Delta H_2 = \SI{12}{\kilo\calorie}$
    \item \ce{I2(g) + H2(g) -> 2 HI(ac)}~~~$\Delta H_3 = \SI{-26,8}{\kilo\calorie}$
\end{itemize}
Calcular los parámetros que se indican a continuación:
\begin{description}%[label={\alph*)},font={\color{red!50!black}\bfseries}]
    \item[\texttt{a)}] Calor molar latente de sublimación del yodo.
    \item[\texttt{b)}] Calor molar de disolución del ácido yodhídrico.
    \item[\texttt{c)}] Número de calorías que hay que aportar para disociar en sus componentes el yoduro de hidrógeno gas contenido en un matraz de \SI{750}{\cubic\centi\meter} a \SI{25}{\celsius} y \SI{800}{\torr} de presión.
\end{description}
\resultadocmd{\SI{12,8}{\kilo\calorie}; \SI{-13,0}{\kilo\calorie}; \SI{12,9}{\calorie}}

	}
\end{frame}

\begin{frame}
    \frametitle{\ejerciciocmd}
    \framesubtitle{Datos del problema}
    \structure{Reacción:} \ce{CH3CH2CH3 + 5O2 ->[$\Delta H_C = \SI{-2200}{\kilo\joule\per\mol}$] 3CO2 + 4H2O}
    \begin{center}
        \textbf{\Large ¿m(\ce{H2O(l)}) en gramos $\Rightarrow$ en vapor?}
    \end{center}
    $$
        \tcbhighmath[boxrule=0.4pt,arc=4pt,colframe=green,drop fuzzy shadow=yellow]{\Delta H_C(\ce{CH3CH2CH3}) = \SI{-2200}{\kilo\joule\per\mol}}\quad
        \tcbhighmath[boxrule=0.4pt,arc=4pt,colframe=green,drop fuzzy shadow=yellow]{V(\ce{CH3CH2CH3}) = \SI{1000}{\liter}}
    $$
    $$
        \tcbhighmath[boxrule=0.4pt,arc=4pt,colframe=green,drop fuzzy shadow=yellow]{P(\ce{CH3CH2CH3}) = \SI{1}{\atm}}\quad
        \tcbhighmath[boxrule=0.4pt,arc=4pt,colframe=green,drop fuzzy shadow=yellow]{T(\ce{CH3CH2CH3}) = \SI{0}{\celsius}}
    $$
    $$
        \tcbhighmath[boxrule=0.4pt,arc=4pt,colframe=green,drop fuzzy shadow=yellow]{\text{Aprovechamiento de energía} = \SI{80}{\percent}}
    $$
    $$
        \tcbhighmath[boxrule=0.4pt,arc=4pt,colframe=blue,drop fuzzy shadow=red]{T_{\text{inicial}}(\ce{H2O(l)}) = \SI{15}{\celsius}}\quad
        \tcbhighmath[boxrule=0.4pt,arc=4pt,colframe=blue,drop fuzzy shadow=red]{T_{\text{final}}(\ce{H2O(g)}) = \SI{100}{\celsius}}
    $$
    $$
        \tcbhighmath[boxrule=0.4pt,arc=4pt,colframe=blue,drop fuzzy shadow=red]{\Delta H_v(\ce{H2O})= \SI{40,7}{\kilo\joule\per\mol}}\quad
        \tcbhighmath[boxrule=0.4pt,arc=4pt,colframe=blue,drop fuzzy shadow=red]{C_e(\ce{H2O}) = \SI{1}{\calorie\per\gram\per\celsius}}
    $$
    $$
        \tcbhighmath[boxrule=0.4pt,arc=4pt,colframe=blue,drop fuzzy shadow=red]{Mm(\ce{H2O}) = \SI{18,02}{\gram\per\mol}}
    $$
\end{frame}

\begin{frame}
    \frametitle{\ejerciciocmd}
    \framesubtitle{Resolución (1): calor emitido por el propano}
    \structure{\ce{CH3CH2CH3} es un gas:}
    \begin{overprint}
        \onslide<1>
            $$
                P\cdot V = n\cdot R\cdot T
            $$
        \onslide<2>
            $$
                n = \frac{P\cdot V}{R\cdot T}
            $$
        \onslide<3->
            $$
                n(\ce{CH3CH2CH3}) = \frac{\SI{1}{\cancel\atm}\cdot\SI{1000}{\cancel\liter}}{\SI{,082}{\cancel\atm\liter\per\mol\per\cancel\kelvin}\cdot\SI{273,15}{\cancel\kelvin}}
            $$
    \end{overprint}
    \visible<3->{
        $$
            n(\ce{CH3CH2CH3}) = \SI{44,65}{\mol}
        $$
                }
    \visible<4->{
        \structure{Calor emitido por el propano:}
        $$
            Q_{\text{emitido}}(\ce{CH3CH2CH3}) = \SI{-2200}{\kilo\joule\per\cancel\mol}\cdot\SI{44,65}{\cancel\mol} = \SI{-98221,74}{\kilo\joule}
        $$
                }
    \visible<5->{
    \structure{Aprovechamiento de la energía de un \SI{80}{\percent}:}
        $$
            Q_{\text{emitido}}(\ce{CH3CH2CH3}) = \SI{-98221,74}{\kilo\joule}\times\frac{8\cancel{0}}{10\cancel{0}} = \SI{-78576,80}{\kilo\joule}
        $$
            }
\end{frame}

\begin{frame}
    \frametitle{\ejerciciocmd}
    \framesubtitle{Resolución (2): calor recibido por el agua}
    \structure{El proceso termodinámico es el siguiente:}
    $$
        \ce{           
                $\underset{\SI{15}{\celsius}}{\ce{H2O(l)}}$
                ->[Q_1]
                $\underset{\SI{100}{\celsius}}{\ce{H2O(l)}}$
                ->[Q_2]
                $\underset{\SI{100}{\celsius}}{\ce{H2O(g)}}$
            }
    $$
    \visible<2->{
        \structure{Si tenemos un sistema aislado:}
        \begin{overprint}
            \onslide<2>
                $$
                    -Q_{\text{emitido}}(\ce{CH3CH2CH3}) = Q_{\text{recibido}}(\ce{H2O})
                $$
            \onslide<3>
                $$
                    -Q_{\text{emitido}}(\ce{CH3CH2CH3}) = \overbrace{Q_1(\ce{H2O(l)})}^{\text{calentar el agua de \SI{15}{\celsius} hasta \SI{100}{\celsius}}} + \underbrace{Q_2(\ce{H2O(g)})}_{\text{hervir el agua}}
                $$
            \onslide<4>
                $$
                    -Q_{\text{emitido}}(\ce{CH3CH2CH3}) = \overbrace{Q_1(\ce{H2O(l)})}^{Q_1=m\cdot C_e\cdot\Delta T} + \underbrace{Q_2(\ce{H2O(g)})}_{n\cdot\Delta H_v}
                $$
            \onslide<5>
                $$
                    -Q_{\text{emitido}}(\ce{CH3CH2CH3}) = m(\ce{H2O})\cdot C_e(\ce{H2O(l)})\cdot\Delta T(\ce{H2O}) + \overbrace{n(\ce{H2O})}^{n=\frac{m}{Mm}}\cdot\Delta H_v(\ce{H2O})
                $$
            \onslide<6>
                $$
                    -Q_{\text{emitido}}(\ce{CH3CH2CH3}) = m(\ce{H2O})\cdot C_e(\ce{H2O(l)})\cdot\Delta T(\ce{H2O}) + \frac{m(\ce{H2O})}{Mm(\ce{H2O})}\cdot\Delta H_v(\ce{H2O})
                $$
            \onslide<7>
                $$
                    -Q_{\text{emitido}}(\ce{CH3CH2CH3}) = m(\ce{H2O})\cdot\left(C_e(\ce{H2O(l)})\cdot\Delta T(\ce{H2O}) + \frac{\Delta H_v(\ce{H2O})}{Mm(\ce{H2O})}\right)
                $$
            \onslide<8->
                $$
                    m(\ce{H2O})
                    =
                    -\frac{Q_{\text{emitido}}(\ce{CH3CH2CH3})}{\left(C_e(\ce{H2O(l)})\cdot\Delta T(\ce{H2O}) + \frac{\Delta H_v(\ce{H2O})}{Mm(\ce{H2O})}\right)}
                $$
        \end{overprint}
                }
    \visible<9->{
        \structure{Sustituyendo por los correspondientes valores:}
        $$
            m(\ce{H2O})
            =
            -          
            \frac{\SI{-78576,80e3}{\cancel\joule}}{\left(\underbrace{\SI{4,18}{\cancel\joule\per\gram\per\cancel\celsius}}_{\SI{1}{\calorie\per\gram\per\celsius}}\cdot\left(100-15~\si{\cancel\celsius}\right) + \frac{\SI{40,7e3}{\cancel\joule\per\cancel\mol}}{\SI{18,02}{\gram\per\cancel\mol}}\right)}
        $$
                }
    \visible<10->{
        $$
            \tcbhighmath[boxrule=0.4pt,arc=4pt,colframe=blue,drop fuzzy shadow=red]{
                m(\ce{H2O})
                =
                \SI{30061,111}{\gram} = \SI{30}{\kilogram}
            }
        $$
                }
\end{frame}
