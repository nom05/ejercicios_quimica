Dentro de un sistema aislado tenemos los siguientes procesos:
\begin{itemize}
	\item Se queman \SI{20}{\liter} de propano (\ce{C3H8}) medidos a \SI{27}{\celsius} y \SI{874}{\torr}. Calcula el calor desprendido. DATOS: $\Delta H_f (\si{\kilo\joule\per\mol})$: \ce{C3H8(g)} \num{-103,8}; \ce{H2O(l)} \num{-285,8}; \ce{CO2(g)} \num{-393,5}. $R = \SI{,082}{\atm\liter\per\mol\per\kelvin}$.
	\item Se descomponen \SI{1500}{\gram} de carbonato de calcio según la reacción: \ce{CaCO3(s) -> CaO(s) + CO2(g)}. Calcular el calor absorbido. DATOS: $\Delta H_f (\si{\kilo\joule\per\mol})$: \ce{CO2(g)} \num{-393,5}; \ce{CaO(s)} \num{-635,5}; \ce{CaCO3} \num{-1207,0}.
	\item Averigua si \SI{20}{\liter} de agua a \SI{50}{\celsius} se enfriarán o se calentarán y su temperatura final. DATOS del agua: $c_e(\ce{l}) = \SI{4,18}{\joule\per\gram\per\celsius}$; $c_e(\text{v}) = c_e(\ce{s}) = \SI{2,09}{\joule\per\gram\per\celsius}$; $\Delta H_{\text{vap}} = \SI{540}{\cal\per\gram}$; $\Delta H_{\text{fus}} = \SI{80}{\cal\per\gram}$; $d=\SI{e3}{\kilogram\per\cubic\meter}$.
\end{itemize}
\resultadocmd{
				\SI{-2074,48}{\kilo\joule};
				\SI{2667,68}{\kilo\joule};
				\SI{42,90}{\celsius}
			}
