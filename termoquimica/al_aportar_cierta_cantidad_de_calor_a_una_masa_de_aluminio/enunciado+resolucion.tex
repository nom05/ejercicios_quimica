\begin{frame}
    \frametitle{\ejerciciocmd}
    \framesubtitle{Enunciado}
    \textbf{
	Dadas las siguientes reacciones:
\begin{itemize}
    \item \ce{I2(g) + H2(g) -> 2 HI(g)}~~~$\Delta H_1 = \SI{-0,8}{\kilo\calorie}$
    \item \ce{I2(s) + H2(g) -> 2 HI(g)}~~~$\Delta H_2 = \SI{12}{\kilo\calorie}$
    \item \ce{I2(g) + H2(g) -> 2 HI(ac)}~~~$\Delta H_3 = \SI{-26,8}{\kilo\calorie}$
\end{itemize}
Calcular los parámetros que se indican a continuación:
\begin{description}%[label={\alph*)},font={\color{red!50!black}\bfseries}]
    \item[\texttt{a)}] Calor molar latente de sublimación del yodo.
    \item[\texttt{b)}] Calor molar de disolución del ácido yodhídrico.
    \item[\texttt{c)}] Número de calorías que hay que aportar para disociar en sus componentes el yoduro de hidrógeno gas contenido en un matraz de \SI{750}{\cubic\centi\meter} a \SI{25}{\celsius} y \SI{800}{\torr} de presión.
\end{description}
\resultadocmd{\SI{12,8}{\kilo\calorie}; \SI{-13,0}{\kilo\calorie}; \SI{12,9}{\calorie}}

	   }
\end{frame}

\begin{frame}
    \frametitle{\ejerciciocmd}
    \framesubtitle{Datos del problema}
    $$
        \text{¿}\Delta T(\ce{Cu})?
    $$
    $$
        \tcbhighmath[boxrule=0.4pt,arc=4pt,colframe=blue,drop fuzzy shadow=red]{C_e(\ce{Al}) = \SI{0,21}{\calorie\per\gram\per\celsius}}\quad
        \tcbhighmath[boxrule=0.4pt,arc=4pt,colframe=blue,drop fuzzy shadow=red]{\Delta T(\ce{Al}) = \SI{57}{\celsius}}\quad
    $$
    $$
        \tcbhighmath[boxrule=0.4pt,arc=4pt,colframe=green,drop fuzzy shadow=yellow]{C_e(\ce{Cu}) = \SI{0,093}{\calorie\per\gram\per\celsius}}\quad
    $$
    $$
        \tcbhighmath[boxrule=0.4pt,arc=4pt,colframe=blue,drop fuzzy shadow=red]{Q(\ce{Al})}
        =
        \tcbhighmath[boxrule=0.4pt,arc=4pt,colframe=green,drop fuzzy shadow=yellow]{Q(\ce{Cu})}
        = Q
        \quad
        \tcbhighmath[boxrule=0.4pt,arc=4pt,colframe=blue,drop fuzzy shadow=red]{m(\ce{Al})}
        =
        \tcbhighmath[boxrule=0.4pt,arc=4pt,colframe=green,drop fuzzy shadow=yellow]{m(\ce{Cu})}
        = m
    $$
\end{frame}

\begin{frame}
    \frametitle{\ejerciciocmd}
    \framesubtitle{Resolución (1): incremento de T de cobre}
    \structure{Partimos de la igualdad:}
    \begin{overprint}
        \onslide<1>
            $$
                Q(\ce{Al}) = Q(\ce{Cu})
            $$
        \onslide<2>
            $$
                \overbrace{Q(\ce{Al})}^{Q=m\cdot C_e\cdot\Delta T} = \underbrace{Q(\ce{Cu})}_{Q=m\cdot C_e\cdot\Delta T}
            $$
        \onslide<3>
            $$
                \overbrace{m(\ce{Al})}^{m(\ce{Al})=m}\cdot C_e(\ce{Al})\cdot\Delta T(\ce{Al}) = \underbrace{m(\ce{Cu})}_{m(\ce{Cu})=m}\cdot C_e(\ce{Cu})\cdot\Delta T(\ce{Cu})
            $$
        \onslide<4>
            $$
                \cancel{m}\cdot C_e(\ce{Al})\cdot\Delta T(\ce{Al}) = \cancel{m}\cdot C_e(\ce{Cu})\cdot\Delta T(\ce{Cu})
            $$
        \onslide<5>
            $$
                C_e(\ce{Al})\cdot\Delta T(\ce{Al}) = C_e(\ce{Cu})\cdot\Delta T(\ce{Cu})
            $$
        \onslide<6->
            $$
                \Delta T(\ce{Cu}) = \frac{C_e(\ce{Al})}{C_e(\ce{Cu})}\cdot\Delta T(\ce{Al})
            $$
    \end{overprint}
    \visible<6->{
        \structure{Sustituimos por los correspondientes valores:}
        $$
                \Delta T(\ce{Cu}) = \frac{\SI{0,21}{\cancel\calorie\per\cancel\gram\per\cancel\celsius}}{\SI{0,093}{\cancel\calorie\per\cancel\gram\per\cancel\celsius}}\cdot\SI{57}{\celsius}
        $$
                }
    \visible<7->{
        $$
            \tcbhighmath[boxrule=0.4pt,arc=4pt,colframe=green,drop fuzzy shadow=yellow]{\Delta T(\ce{Cu}) = \SI{129}{\celsius}}
        $$
                }
\end{frame}
