Al aportar cierta cantidad de calor a una masa de aluminio ($C_e = \SI{0,21}{\calorie\per\gram\per\celsius}$) su temperatura se incrementa en \SI{57}{\celsius}. Suponiendo que la misma cantidad de calor se suministra a una masa igual de cobre ($C_e = \SI{0,093}{\calorie\per\gram\per\celsius}$), ¿cuál es el incremento de temperatura experimentado por el cobre?
\resultadocmd{\SI{129}{\celsius}}
