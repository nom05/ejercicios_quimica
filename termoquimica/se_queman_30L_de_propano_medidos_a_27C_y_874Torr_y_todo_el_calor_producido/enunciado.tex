Se queman \SI{30}{\liter} de propano medidos a \SI{27}{\celsius} y \SI{874}{\torr} y todo el calor producido se aprovecha para hacer posible la descomposición del carbonato cálcico [reacción: \ce{CaCO3(s) -> CO2(g) + CaO(s)}]. Calcular:
\begin{enumerate}[label={\alph*)},font={\color{red!50!black}\bfseries}]
	\item Cantidad de carbonato que puede descomponerse.
	\item Volumen de gas medido a \SI{1}{\atm} y \SI{0}{\celsius} que se obtiene a partir del carbonato descompuesto en el apartado anterior si el Rto del proceso es de \SI{93}{\percent}.
\end{enumerate}
Datos:
$\Delta H_f(\ce{C3H8(g)} )   =  \SI{-103,8}{\kilo\joule\per\mol}$;
$\Delta H_f(\ce{CaO(s)}  )   =  \SI{-635,5}{\kilo\joule\per\mol}$; 
$\Delta H_f(\ce{CaCO3(s)})   = \SI{-1207,0}{\kilo\joule\per\mol}$;
$\Delta H_f(\ce{H2O(l)}  )   =  \SI{-285,8}{\kilo\joule\per\mol}$;
$\Delta H_f(\ce{CO2(g)}  )   =  \SI{-393,5}{\kilo\joule\per\mol}$
\resultadocmd{
            \SI{1748,3}{\gram};
            \SI{364}{\liter}
		}
