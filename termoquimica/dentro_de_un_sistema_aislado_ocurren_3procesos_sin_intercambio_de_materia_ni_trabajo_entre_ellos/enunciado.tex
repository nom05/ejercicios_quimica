Dentro de un sistema aislado ocurren tres procesos sin intercambio de materia ni trabajo entre ellos:
\begin{enumerate}
	\item Síntesis directa de amoníaco (\ce{NH3(g)}) a partir de \ce{N2(g)} y \ce{H2(g)} empleando un catalizador de hierro (este método es conocido como proceso Haber-Bosch). Datos de esta reacción: $\Delta H_f^0(\ce{NH3(g)}) = \SI{-46,2}{\kilo\joule\per\mol}$, $Mm(\ce{N2}) = \SI{28,01}{\gram\per\mol}$.
	\item Calentamiento de \SI{2,5}{\liter} de agua desde los \SI{25}{\celsius} a los \SI{103}{\celsius}. Datos de este proceso: $d(\ce{H2O(l)}) = \SI{1000}{\kilogram\per\cubic\meter}$, $Mm(\ce{H2O}) = \SI{18}{\gram\per\mol}$, $c_e(\ce{H2O(l)}) = \SI{4,18}{\joule\per\gram\per\kelvin}$, $c_e(\ce{H2O(g)}) = \SI{,5}{\calorie\per\gram\per\celsius}$, $\Delta H_{\text{vap}}^0(\ce{H2O}) = \SI{40,7}{\kilo\joule\per\mol}$.
	\item Combustión de \SI{10}{\milli\liter} de etano (\ce{C2H6}) a una presión de \SI{2,3}{\atm} y una temperatura de \SI{25}{\degreeCelsius}. Datos para la combustión: $\Delta H_f^0 (\unit{\kilo\joule\per\mol})$ de etano: \num{-84,7}, de \ce{CO2}: \num{-393,5}, de agua \num{-285,8}; $R = \SI{,082}{\atm\liter\per\mol\per\kelvin}$.
\end{enumerate}
¿Qué masa de nitrógeno molecular (\ce{N2(g)}) en gramos se necesita como mínimo para sintetizar \ce{NH3} cumpliendo el 1"er principio de la Termodinámica?
\resultadocmd{
				\SI{1955,47}{\gram}
			}
