Se quemaron \SI{3}{\gram} de carbono hasta dióxido de carbono, \ce{CO2}, en un calorímetro de cobre. La masa del calorímetro es de \SI{1500}{\gram} y la masa de agua en el calorímetro \SI{2000}{\gram}. La temperatura inicial fue de \SI{20}{\celsius} y la final de \SI{31,3}{\celsius}. Calcule el poder calorífico del carbono. ($C_p(\ce{Cu})=\SI{,389}{\joule\per\gram\per\kelvin}$, $C_p(\ce{H2O})=\SI{4,18}{\joule\per\gram\per\kelvin}$).
\resultadocmd{\SI{1,010e5}{\joule\per\gram}}
