\begin{frame}
    \frametitle{\ejerciciocmd}
    \framesubtitle{Enunciado}
    \textbf{
		Dadas las siguientes reacciones:
\begin{itemize}
    \item \ce{I2(g) + H2(g) -> 2 HI(g)}~~~$\Delta H_1 = \SI{-0,8}{\kilo\calorie}$
    \item \ce{I2(s) + H2(g) -> 2 HI(g)}~~~$\Delta H_2 = \SI{12}{\kilo\calorie}$
    \item \ce{I2(g) + H2(g) -> 2 HI(ac)}~~~$\Delta H_3 = \SI{-26,8}{\kilo\calorie}$
\end{itemize}
Calcular los parámetros que se indican a continuación:
\begin{description}%[label={\alph*)},font={\color{red!50!black}\bfseries}]
    \item[\texttt{a)}] Calor molar latente de sublimación del yodo.
    \item[\texttt{b)}] Calor molar de disolución del ácido yodhídrico.
    \item[\texttt{c)}] Número de calorías que hay que aportar para disociar en sus componentes el yoduro de hidrógeno gas contenido en un matraz de \SI{750}{\cubic\centi\meter} a \SI{25}{\celsius} y \SI{800}{\torr} de presión.
\end{description}
\resultadocmd{\SI{12,8}{\kilo\calorie}; \SI{-13,0}{\kilo\calorie}; \SI{12,9}{\calorie}}

	}
\end{frame}

\begin{frame}
    \frametitle{\ejerciciocmd}
    \framesubtitle{Resolución (\rom{1})}
    $$
        T(\text{inicial}) = 20-30~\si{\celsius}\Rightarrow\text{ \textbf{No espontánea}}
    $$
    $$
        T(\text{final}) = \SI{1000}{\celsius}\Rightarrow\text{ \textbf{Espontánea}}
    $$
    \visible<2->{
        \structure{Recordemos cómo sabemos cuando una reacción es espontánea o no:}
        $$
            \Delta G = \Delta H - T\vdot\Delta S
        $$
        \begin{overprint}
            \onslide<2>
                \begin{center}
                    \begin{tabular}{cc}
                        \toprule
                        Situación    & Reacción     \\
                        \midrule
                        $\Delta G>0$ & No espontánea\\
                        $\Delta G=0$ & Equilibrio   \\
                        $\Delta G<0$ & Espontánea   \\
                        \bottomrule
                    \end{tabular}
                \end{center}
            \onslide<3>
                \alert{Reacción endotérmica:} $\Delta H > 0$
                \begin{center}
                    \begin{tabular}{ccc}
                        \toprule
                            Situación    & Reacción      & Ecuación                                 \\
                        \midrule
                            $\Delta G>0$ & No espontánea & $\Delta G = \Delta H - T\vdot\Delta S>0$ \\
                            $\Delta G=0$ & Equilibrio    & $\Delta G = \Delta H - T\vdot\Delta S=0$ \\
                            $\Delta G<0$ & Espontánea    & $\Delta G = \Delta H - T\vdot\Delta S<0$ \\
                        \bottomrule
                    \end{tabular}
                \end{center}
            \onslide<4>
                \alert{Reacción endotérmica:} $\Delta H > 0$
                \begin{center}
                    \begin{tabular}{ccc}
                        \toprule
                        Situación    & Reacción      & Ecuación                    \\
                        \midrule
                        $\Delta G>0$ & No espontánea & $\Delta H > T\vdot\Delta S$ \\
                        $\Delta G=0$ & Equilibrio    & $\Delta H = T\vdot\Delta S$ \\
                        $\Delta G<0$ & Espontánea    & $\Delta H < T\vdot\Delta S$ \\
                        \bottomrule
                    \end{tabular}
                \end{center}
            \onslide<5>
                \alert{Reacción endotérmica:} $\Delta H > 0$
                \begin{center}
                    \begin{tabular}{cccc}
                        \toprule
                        Situación    & Reacción               & Ecuación                    & T                                    \\
                        \midrule
                        $\Delta G>0$ & No espontánea          & $\Delta H > T\vdot\Delta S$ & Bajas(ambiente)                      \\
                        $\Delta G=0$ & Equilibrio             & $\Delta H = T\vdot\Delta S$ & Intermedias                          \\
                        $\Delta G<0$ & \textbf{Espontánea}    & $\Delta H < T\vdot\Delta S$ & \textbf{Altas (\SI{1000}{\celsius})} \\
                        \bottomrule
                    \end{tabular}
                \end{center}
        \end{overprint}
                }
\end{frame}
