Se pone \SI{1}{\liter} de aire, a \SI{1}{\atm} y \SI{86}{\celsius}, en contacto con \SI{1}{\milli\liter} de agua líquida a la misma temperatura. El volumen de la fase gaseosa permanece constante durante el experimento. La densidad del agua en estas condiciones es de \SI{0,97}{\gram\per\cubic\centi\meter} y su presión de vapor es \SI{0,593}{\atm}. Cuando se ha alcanzado el equilibrio, indicar:
\begin{enumerate}[label={\alph*)},font={\color{red!50!black}\bfseries}]
	\item La presión parcial del aire en la vasija;
	\item La presión parcial del vapor de agua;
	\item La presión total;
	\item La cantidad de agua que se habrá evaporado;
	\item El volumen de agua líquida que quedará.
\end{enumerate}
\resultadocmd{
		\SI{1}{\atm};
		\SI{0,593}{\atm};
		\SI{1,593}{\atm};
		\SI{,36}{\gram};
		\SI{,63}{\milli\liter}
}
