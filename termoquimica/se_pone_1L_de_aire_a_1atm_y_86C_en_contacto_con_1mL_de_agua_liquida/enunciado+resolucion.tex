\begin{frame}
    \frametitle{\ejerciciocmd}
    \framesubtitle{Enunciado}
    \textbf{
	Dadas las siguientes reacciones:
\begin{itemize}
    \item \ce{I2(g) + H2(g) -> 2 HI(g)}~~~$\Delta H_1 = \SI{-0,8}{\kilo\calorie}$
    \item \ce{I2(s) + H2(g) -> 2 HI(g)}~~~$\Delta H_2 = \SI{12}{\kilo\calorie}$
    \item \ce{I2(g) + H2(g) -> 2 HI(ac)}~~~$\Delta H_3 = \SI{-26,8}{\kilo\calorie}$
\end{itemize}
Calcular los parámetros que se indican a continuación:
\begin{description}%[label={\alph*)},font={\color{red!50!black}\bfseries}]
    \item[\texttt{a)}] Calor molar latente de sublimación del yodo.
    \item[\texttt{b)}] Calor molar de disolución del ácido yodhídrico.
    \item[\texttt{c)}] Número de calorías que hay que aportar para disociar en sus componentes el yoduro de hidrógeno gas contenido en un matraz de \SI{750}{\cubic\centi\meter} a \SI{25}{\celsius} y \SI{800}{\torr} de presión.
\end{description}
\resultadocmd{\SI{12,8}{\kilo\calorie}; \SI{-13,0}{\kilo\calorie}; \SI{12,9}{\calorie}}

	}
\end{frame}

\begin{frame}
\frametitle{\ejerciciocmd}
\framesubtitle{Datos del problema}
    \begin{enumerate}[label={\alph*)},font={\color{red!50!black}\bfseries}]
        \item ¿$P(\text{aire})$?
        \item ¿$P_v(\ce{H2O})$?
        \item ¿$P_T$?
        \item ¿$m(\ce{H2O})$ evaporada?
        \item ¿$V(\ce{H2O})$ restante?
    \end{enumerate}
    $$
        \tcbhighmath[boxrule=0.4pt,arc=4pt,colframe=blue,drop fuzzy shadow=red]{V(\text{aire}) = \SI{1}{\liter}}\quad
        \tcbhighmath[boxrule=0.4pt,arc=4pt,colframe=blue,drop fuzzy shadow=red]{P(\text{aire}) = \SI{1}{\atm}}\quad
        \tcbhighmath[boxrule=0.4pt,arc=4pt,colframe=blue,drop fuzzy shadow=red]{T(\text{aire}) = \SI{86}{\celsius}}
    $$
    $$
        \tcbhighmath[boxrule=0.4pt,arc=4pt,colframe=green,drop fuzzy shadow=blue]{V(\ce{H2O(l)}) = \SI{1}{\milli\liter}}\quad
        \tcbhighmath[boxrule=0.4pt,arc=4pt,colframe=green,drop fuzzy shadow=blue]{T(\ce{H2O}) = \SI{86}{\celsius}}\quad
        \tcbhighmath[boxrule=0.4pt,arc=4pt,colframe=green,drop fuzzy shadow=blue]{Mm(\ce{H2O}) = \SI{18,02}{\gram\per\mol}}\quad
    $$
    $$
        \tcbhighmath[boxrule=0.4pt,arc=4pt,colframe=green,drop fuzzy shadow=blue]{d(\ce{H2O(l)}) =\SI{0,97}{\gram\per\cubic\centi\meter}}\quad
        \tcbhighmath[boxrule=0.4pt,arc=4pt,colframe=green,drop fuzzy shadow=blue]{P_v(\ce{H2O(g)}) = \SI{0,593}{\atm}}
    $$
    $$
        V(\text{aire}+\ce{H2O}) = \text{constante}
    $$
\end{frame}

\begin{frame}
    \frametitle{\ejerciciocmd}
    \framesubtitle{Resolución (\rom{1}): Presión parcial del aire}
    \structure{Gases ideales:} Habíamos visto que, en las mezclas de gases ideales, estos no interaccionan entre sí. Entonces:
    $$
        P_{inicial}(\text{aire}) = P_{final}(\text{aire})
    $$
    \visible<2->{
        De los datos del problema:
        $$
            \tcbhighmath[boxrule=0.4pt,arc=4pt,colframe=blue,drop fuzzy shadow=red]{P(\text{aire}) = \SI{1}{\atm}}
        $$
                }
\end{frame}

\begin{frame}
    \frametitle{\ejerciciocmd}
    \framesubtitle{Resolución (\rom{2}): Presión parcial del vapor de \ce{H2O}}
    \structure{Equilibrio Líquido-Gas:} El agua se evapora hasta alcanzar la presión de vapor. De los datos del problema:
    $$
        \tcbhighmath[boxrule=0.4pt,arc=4pt,colframe=green,drop fuzzy shadow=blue]{P_v(\ce{H2O}) = \SI{0,593}{\atm}}
    $$
\end{frame}

\begin{frame}
    \frametitle{\ejerciciocmd}
    \framesubtitle{Resolución (\rom{3}): Presión total de la mezcla}
    \structure{Ley de Dalton:} $P_T = \sum_{i=1}^{n}P_i$
    $$
        \tcbhighmath[boxrule=0.4pt,arc=4pt,colframe=blue,drop fuzzy shadow=green]{P_T = P(\text{aire}) + P_v(\ce{H2O}) = \SI{1,593}{\atm}}
    $$
\end{frame}

\begin{frame}
    \frametitle{\ejerciciocmd}
    \framesubtitle{Resolución (\rom{4}): Agua evaporada}
    \structure{Ecuación de los gases ideales:}
    \begin{overprint}
        \onslide<1>
            $$
                P\vdot V = n\vdot R\vdot T
            $$
        \onslide<2>
            $$
                n = \frac{P\vdot V}{R\vdot T}
            $$
        \onslide<3>
            $$
                \overbrace{n}^{\frac{m}{Mm}} = \frac{P\vdot V}{R\vdot T}
            $$
        \onslide<4>
            $$
                \frac{m}{Mm} = \frac{P\vdot V}{R\vdot T}
            $$
        \onslide<5>
            $$
                m = \frac{P\vdot V\vdot Mm}{R\vdot T}
            $$
    \end{overprint}
        \visible<6>{
            $$
                \tcbhighmath[boxrule=0.4pt,arc=4pt,colframe=green,drop fuzzy shadow=blue]{m(\ce{H2O(g)}) = \frac{\SI{,593}{\cancel\atm}\vdot\SI{1}{\cancel\liter}\vdot\SI{18,02}{\gram\per\cancel\mol}}{\SI{,082}{\cancel\atm\cancel\liter\per\cancel\mol\per\cancel\kelvin}\vdot(273,15+86~\si{\cancel\kelvin})} = \SI{,36}{\gram}}
            $$
                    }
\end{frame}

\begin{frame}
    \frametitle{\ejerciciocmd}
    \framesubtitle{Resolución (\rom{5}): Volumen de agua líquida restante}
    \structure{Inicialmente teníamos:}
        $$
            V_{\text{inicial}}(\ce{H2O(l)}) = \SI{1}{\milli\liter}\quad\quad d(\ce{H2O(l)}) = \SI{,97}{\gram\per\milli\liter}
        $$
    \visible<2->{
        \structure{Cuya masa es:}
        \begin{overprint}
            \onslide<2>
                $$
                    d = \frac{m}{V}
                $$
            \onslide<3>
                $$
                    m = d\vdot V
                $$
            \onslide<4->
                $$
                    m_{\text{inicial}}(\ce{H2O(l)}) = \SI{,97}{\gram\per\cancel\milli\liter}\vdot\SI{1}{\cancel\milli\liter} = \SI{,97}{\gram}
                $$
        \end{overprint}
                }
    \visible<5->{
        \structure{En el apartado anterior:}
            $$
                m(\ce{H2O(g)}) = \SI{,36}{\gram}
            $$
                }
    \visible<6->{
        \structure{Volumen sin evaporar:}
        \begin{overprint}
            \onslide<6>
                $$
                    m_{\text{final}}(\ce{H2O(l)}) = \SI{,97}{\gram}-\SI{,36}{\gram} = \SI{,61}{\gram}
                $$
            \onslide<7>
                $$
                    V = \frac{m}{d}\text{ (ya despejado)}
                $$
            \onslide<8>
                $$
                    V_{\text{final}}(\ce{H2O(l)}) = \frac{\overbrace{\SI{,61}{\cancel\gram}}^{\SI{,97}{\gram}-\SI{,36}{\gram}}}{\SI{,97}{\cancel\gram\per\milli\liter}}
                $$
            \onslide<9->
                $$
                    \tcbhighmath[boxrule=0.4pt,arc=4pt,colframe=green,drop fuzzy shadow=blue]{V_{\text{final}}(\ce{H2O(l)}) = \SI{,63}{\milli\liter}}
                $$
        \end{overprint}
                }
\end{frame}
