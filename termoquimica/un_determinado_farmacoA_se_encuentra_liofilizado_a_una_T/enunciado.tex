Un determinado fármaco $A$ ($Mm = \SI{36}{\gram\per\mol}$) se encuentra liofilizado (fase sólida) a una temperatura de \SI{7}{\celsius}. Como solo es activo a temperaturas iguales o superiores a \SI{37}{\celsius} es necesario calentarlo hasta dicha temperatura. La cantidad de energía necesaria se puede obtener condensando vapor de agua (\SI{120}{\celsius}, $Mm = \SI{18}{\gram\per\mol}$) hasta \SI{37}{\celsius} y posterior cesión de esa energía al fármaco $A$. Calcule la masa en gramos de $A$ que se puede calentar hasta la temperatura citada si se dispone de \SI{100}{\gram} de agua caliente.

Algunos datos: densidad del agua líquida \SI{1}{\gram\per\milli\liter}; punto de ebullición del agua \SI{373}{\kelvin}; entalpía de vaporización del agua \SI{241,8}{\kilo\joule\per\mol}; calores específicos del agua: gas \SI{2,08}{\joule\per\gram\per\celsius}, líquida \SI{4,18}{\joule\per\gram\per\celsius}. Punto de fusión del $A$ a \SI{32}{\celsius}, entalpía de normal de formación en fase líquida de $A$ \SI{-70}{\kilo\joule\per\mol}, entalpía normal de formación en fase sólida de $A$ \SI{-216}{\kilo\joule\per\mol}, densidad de $A$ en fase líquida no disponible; calores específicos: sólido \SI{2,0}{\joule\per\mol\per\kelvin}; líquido \SI{3,1}{\joule\per\mol\per\kelvin}. $R = \SI{8,314}{\joule\per\mol\per\kelvin}$.

Suponga que todos los procesos tienen lugar en un sistema aislado y a la presión de \SI{1}{\atm}.
\resultadocmd{\SI{338,60}{\gram}}