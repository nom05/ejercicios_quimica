\begin{frame}
	\frametitle{\ejerciciocmd}
	\framesubtitle{Enunciado}
	\textbf{
			Dadas las siguientes reacciones:
\begin{itemize}
    \item \ce{I2(g) + H2(g) -> 2 HI(g)}~~~$\Delta H_1 = \SI{-0,8}{\kilo\calorie}$
    \item \ce{I2(s) + H2(g) -> 2 HI(g)}~~~$\Delta H_2 = \SI{12}{\kilo\calorie}$
    \item \ce{I2(g) + H2(g) -> 2 HI(ac)}~~~$\Delta H_3 = \SI{-26,8}{\kilo\calorie}$
\end{itemize}
Calcular los parámetros que se indican a continuación:
\begin{description}%[label={\alph*)},font={\color{red!50!black}\bfseries}]
    \item[\texttt{a)}] Calor molar latente de sublimación del yodo.
    \item[\texttt{b)}] Calor molar de disolución del ácido yodhídrico.
    \item[\texttt{c)}] Número de calorías que hay que aportar para disociar en sus componentes el yoduro de hidrógeno gas contenido en un matraz de \SI{750}{\cubic\centi\meter} a \SI{25}{\celsius} y \SI{800}{\torr} de presión.
\end{description}
\resultadocmd{\SI{12,8}{\kilo\calorie}; \SI{-13,0}{\kilo\calorie}; \SI{12,9}{\calorie}}

		}
\end{frame}

\begin{frame}
	\frametitle{\ejerciciocmd}
	\framesubtitle{Datos del problema}
	{\huge
		$$
			c_e(\text{aleación})=c_e(\text{al})?
		$$
	}
	\begin{center}
		\tcbhighmath[boxrule=0.4pt,arc=4pt,colframe=green,drop fuzzy shadow=blue]{m(\ce{al})=\SI{25}{\gram}}
		\tcbhighmath[boxrule=0.4pt,arc=4pt,colframe=blue,drop fuzzy shadow=red]{m(\ce{H2O})=\SI{90}{\gram}}
	\end{center}
	\begin{center}
		\tcbhighmath[boxrule=0.4pt,arc=4pt,colframe=green,drop fuzzy shadow=blue]{T_{\text{inicial}}(\ce{al})=\SI{100}{\celsius}=\SI{373,15}{\kelvin}}
		\tcbhighmath[boxrule=0.4pt,arc=4pt,colframe=blue,drop fuzzy shadow=red]{T_{\text{inicial}}(\ce{H2O})=\SI{25,32}{\celsius}=\SI{298,47}{\kelvin}}
	\end{center}
	\begin{center}
		\tcbhighmath[boxrule=0.4pt,arc=4pt,colframe=blue,drop fuzzy shadow=red]{c_e(\ce{H2O})=\SI{1}{\calorie\per\gram\per\kelvin}=\SI{4,18}{\joule\per\gram\per\celsius}}
	\end{center}
	\begin{center}
		\tcbhighmath[boxrule=0.4pt,arc=4pt,colframe=red,drop fuzzy shadow=black]{T_{\text{final}}=T_{\text{f}}=\SI{27,18}{\celsius}=\SI{298,47}{\kelvin}}
	\end{center}
\end{frame}

\begin{frame}
	\frametitle{\ejerciciocmd}
	\framesubtitle{Resolución (\rom{1}): determinación del calor específico}
	\centering\textbf{Suponemos que el proceso transcurre en un sistema aislado.}
	\structure{Según el 1"er principio de la Termodinámica:} $\Delta U=0=Q_{\text{total}}+W$
	\visible<2->{
		\structure{No se realiza ningún trabajo ($\Delta V=0$):} $Q_{\text{total}}+\cancelto{0}{W}=0$
	}
	\visible<3->{
		\structure{Transferencia de calor total tiene que ser cero:} $Q_{\text{total}}=Q(\ce{al})+Q(\ce{H2O})=0$
	}
	\visible<4->{
		\structure{Cumpliéndose:} $Q(\ce{al})=-Q(\ce{H2O})$
	}
	\visible<5->{
		\begin{overprint}
			\onslide<5>
				\centering\structure{Aplicando:} $Q=m\cdot c_e\Delta T$
				$$
					m(\ce{al})\cdot c_e(\ce{al})\cdot\overbrace{\Delta T(\ce{al})}^{T_{\text{f}}-T_{\text{i}}(\ce{al})} = -m(\ce{H2O})\cdot c_e(\ce{H2O})\cdot\underbrace{\Delta T(\ce{H2O})}_{T_{\text{f}}-T_{\text{i}}(\ce{H2O})}
				$$
			\onslide<6>
				\centering\structure{Despejamos $c_e{\ce{al}}$:}
				$$
					m(\ce{al})\cdot c_e(\ce{al})\cdot(T_{\text{f}}-T_{\text{i}}(\ce{al})) = -m(\ce{H2O})\cdot c_e(\ce{H2O})\cdot(T_{\text{f}}-T_{\text{i}}(\ce{H2O}))
				$$
			\onslide<7->
				\centering\structure{Despejamos $c_e{\ce{al}}$:}
				$$
					c_e(\ce{al}) = -\frac{m(\ce{H2O})\cdot c_e(\ce{H2O})\cdot(T_{\text{f}}-T_{\text{i}}(\ce{H2O}))}{m(\ce{al})\cdot(T_{\text{f}}-T_{\text{i}}(\ce{al}))}
				$$
		\end{overprint}
	}
	\visible<7->{
		\centering\structure{Sustituimos por los valores del enunciado:}
		$$
			c_e(\ce{al}) = -\frac{\SI{90}{\cancel\gram}\cdot\SI{1}{\calorie\per\cancel\gram\per\cancel\celsius}\cdot(\SI{27,18}{\cancel\celsius}-\SI{25,32}{\cancel\celsius}}{\SI{25}{\gram}\cdot(\SI{27,18}{\celsius}-\SI{100,0}{\celsius})}
		$$
		\centering\tcbhighmath[boxrule=0.4pt,arc=4pt,colframe=orange,drop fuzzy shadow=green]{c_e(\ce{al})=\SI{,091}{\calorie\per\gram\per\celsius}=\SI{,091}{\calorie\per\gram\per\kelvin}=\SI{,022}{\joule\per\gram\per\celsius}}
				}
\end{frame}
