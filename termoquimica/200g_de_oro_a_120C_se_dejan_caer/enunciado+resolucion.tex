\begin{frame}
	\frametitle{\ejerciciocmd}
	\framesubtitle{Enunciado}
	\textbf{
		Una reacción tiene una constante de velocidad de \SI{,017}{\per\second} a \SI{298}{\kelvin} y una energía libre de activación del \SI{27,235}{\kilo\joule\per\mol}. La adición de un catalizador disminuye dicha energía de activación hasta un \SI{33}{\percent} de su valor inicial. Calcule la nueva constante de velocidad.
\resultadocmd{ \SI{26,86}{\per\second} }

			}
\end{frame}

\begin{frame}
	\frametitle{\ejerciciocmd}
	\framesubtitle{Datos del problema}
	{\huge
		$$
			T_{\text{final}}?
		$$
	}
	\begin{center}
		\tcbhighmath[boxrule=0.4pt,arc=4pt,colframe=green,drop fuzzy shadow=blue]{m(\ce{Au})=\SI{200}{\gram}}
		\tcbhighmath[boxrule=0.4pt,arc=4pt,colframe=blue,drop fuzzy shadow=red]{m(\ce{H2O})=\SI{25}{\gram}}
	\end{center}
	\begin{center}
		\tcbhighmath[boxrule=0.4pt,arc=4pt,colframe=green,drop fuzzy shadow=blue]{T_{\text{inicial}}(\ce{Au})=\SI{120}{\celsius}=\SI{393,15}{\kelvin}}
		\tcbhighmath[boxrule=0.4pt,arc=4pt,colframe=blue,drop fuzzy shadow=red]{T_{\text{inicial}}(\ce{H2O})=\SI{10}{\celsius}=\SI{283,15}{\kelvin}}
	\end{center}
	\begin{center}
		\tcbhighmath[boxrule=0.4pt,arc=4pt,colframe=green,drop fuzzy shadow=blue]{c_e(\ce{Au})=\SI{,1308}{\joule\per\gram\per\kelvin}=\SI{,1308}{\joule\per\gram\per\celsius}}\\[.3cm]
		\tcbhighmath[boxrule=0.4pt,arc=4pt,colframe=blue,drop fuzzy shadow=red]{c_e(\ce{H2O})=\SI{1}{\calorie\per\gram\per\kelvin}=\SI{4,18}{\joule\per\gram\per\celsius}}
	\end{center}
\end{frame}

\begin{frame}
	\frametitle{\ejerciciocmd}
	\framesubtitle{Resolución (\rom{1}): determinación de la temperatura final}
	\centering\textbf{Suponemos que el proceso transcurre en un sistema aislado.}
	\structure{Según el 1"er principio de la Termodinámica:} $\Delta U=0=Q_{\text{total}}+W$
	\visible<2->{
			\structure{No se realiza ningún trabajo ($\Delta V=0$):} $Q_{\text{total}}+\cancelto{0}{W}=0$
				}
	\visible<3->{
		\structure{Transferencia de calor total tiene que ser cero:} $Q_{\text{total}}=Q(\ce{Au})+Q(\ce{H2O})=0$
				}
	\visible<4->{
		\structure{Cumpliéndose:} $Q(\ce{Au})=-Q(\ce{H2O})$
				}
	\visible<5->{
		\begin{overprint}
			\onslide<5>
				\centering\structure{Aplicando:} $Q=m\cdot c_e\Delta T$
				$$
					m(\ce{Au})\cdot c_e(\ce{Au})\cdot\overbrace{\Delta T(\ce{Au})}^{T_{\text{final}}-T_{\text{inicial}}(\ce{Au})} = -m(\ce{H2O})\cdot c_e(\ce{H2O})\cdot\underbrace{\Delta T(\ce{H2O})}_{T_{\text{final}}-T_{\text{inicial}}(\ce{H2O})}
				$$
			\onslide<6>
				\centering\structure{Despejamos $T_{\text{final}}=T_{\text{f}}$:} $T_{\text{inicial}} = T_{\text{i}}$
				$$
					m(\ce{Au})\cdot c_e(\ce{Au})\cdot(T_{\text{final}}-T_{\text{i}}(\ce{Au})) = -m(\ce{H2O})\cdot c_e(\ce{H2O})\cdot(T_{\text{f}}-T_{\text{i}}(\ce{H2O}))
				$$
			\onslide<7>
				\centering\structure{Despejamos $T_{\text{f}}:$}
				$$
					m(\ce{Au})\cdot c_e(\ce{Au})\cdot T_{\text{f}}-m(\ce{Au})\cdot c_e(\ce{Au})\cdot T_{\text{i}}(\ce{Au}) = -m(\ce{H2O})\cdot c_e(\ce{H2O})\cdot T_{\text{f}}+ m(\ce{H2O})\cdot c_e(\ce{H2O})\cdot T_{\text{i}}(\ce{H2O})
				$$
			\onslide<8>
				\centering\structure{Despejamos $T_{\text{f}}$:}
				$$
					m(\ce{Au})\cdot c_e(\ce{Au})\cdot T_{\text{f}} + m(\ce{H2O})\cdot c_e(\ce{H2O})\cdot T_{\text{f}} = m(\ce{Au})\cdot c_e(\ce{Au})\cdot T_{\text{i}}(\ce{Au}) + m(\ce{H2O})\cdot c_e(\ce{H2O})\cdot T_{\text{i}}(\ce{H2O})
				$$
			\onslide<9>
				\centering\structure{Despejamos $T_{\text{f}}$:}
				$$
					T_{\text{f}}\cdot\left[m(\ce{Au})\cdot c_e(\ce{Au}) + m(\ce{H2O})\cdot c_e(\ce{H2O})\right] = m(\ce{Au})\cdot c_e(\ce{Au})\cdot T_{\text{i}}(\ce{Au}) + m(\ce{H2O})\cdot c_e(\ce{H2O})\cdot T_{\text{i}}(\ce{H2O})
				$$
			\onslide<10->
				\centering\structure{Despejamos $T_{\text{f}}$:}
				$$
					T_{\text{f}} = \frac{m(\ce{Au})\cdot c_e(\ce{Au})\cdot T_{\text{i}}(\ce{Au}) + m(\ce{H2O})\cdot c_e(\ce{H2O})\cdot T_{\text{i}}(\ce{H2O})}{m(\ce{Au})\cdot c_e(\ce{Au}) + m(\ce{H2O})\cdot c_e(\ce{H2O})}
				$$
		\end{overprint}
				}
		\visible<10->{
			\centering\structure{Sustituimos por los valores del enunciado:}
			$$
				T_{\text{f}} = \frac{\SI{200}{\cancel\gram}\cdot\SI{,1308}{\cancel\joule\per\cancel\gram\per\cancel\kelvin}\cdot\SI{393,15}{\cancel\kelvin} + \SI{25}{\cancel\gram}\cdot \SI{4,18}{\cancel\joule\per\cancel\gram\per\cancel\kelvin}\cdot\SI{283,15}{\cancel\kelvin}}{\SI{200}{\cancel\gram}\cdot\SI{,1308}{\cancel\joule\per\cancel\gram\per\kelvin} + \SI{25}{\cancel\gram}\cdot\SI{4,18}{\cancel\joule\per\cancel\gram\per\kelvin}}
			$$
			\centering\tcbhighmath[boxrule=0.4pt,arc=4pt,colframe=orange,drop fuzzy shadow=green]{T_{\text{final}}=\SI{305,17}{\kelvin}=\SI{32}{\celsius}}
					}
\end{frame}
