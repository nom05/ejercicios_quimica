Se desea descomponer carbonato de calcio (\ce{CaCO3(s) -> CaO(s) + CO2(g)}, reacción endotérmica) utilizando la energía desprendida en la combustión de butano (\ce{C4H10}). Durante el proceso el \SI{72}{\percent} del calor producido se emplea en realizar la reacción, en tanto que la cantidad restante se utiliza para calentar agua. Calcule:\\[.3cm]
\begin{enumerate*}[label={\alph*)},font=\bfseries]
	\item ¿cuántos gramos de butano son necesarios para descomponer \SI{250}{\gram} del carbonato?,
	\item la cantidad de agua que se puede calentar desde \SI{25,0}{\celsius} a \SI{103,0}{\celsius} con el calor no utilizado en la reacción anterior.
\end{enumerate*}\\[.3cm]
Datos: $\Delta H_{\text{f}}$ (\si{\kilo\joule\per\mol}): butano \num{-124,7}; óxido de calcio: \num{-635,5}; carbonato de calcio \num{-1207,0}; agua \num{-285,8}, dióxido de carbono \num{-393,5}. $\Delta H_{\text{vap}}$ del agua a \SI{1}{\atm}: \SI{2257}{\joule\per\gram}; calores específicos del agua: líquida \SI{4,18}{\joule\per\gram\per\celsius}, sólida / gas \SI{2,09}{\joule\per\gram\celsius}.
\resultadocmd{\SI{12,48}{\gram}; \SI{67,15}{\gram}}
