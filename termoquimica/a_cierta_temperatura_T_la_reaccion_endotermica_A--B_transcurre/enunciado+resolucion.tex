\begin{frame}
    \frametitle{\ejerciciocmd}
    \framesubtitle{Enunciado}
    \textbf{
		Una reacción tiene una constante de velocidad de \SI{,017}{\per\second} a \SI{298}{\kelvin} y una energía libre de activación del \SI{27,235}{\kilo\joule\per\mol}. La adición de un catalizador disminuye dicha energía de activación hasta un \SI{33}{\percent} de su valor inicial. Calcule la nueva constante de velocidad.
\resultadocmd{ \SI{26,86}{\per\second} }

	}
\end{frame}

\begin{frame}
    \frametitle{\ejerciciocmd}
    \framesubtitle{Resolución (\rom{1}): signo de $\Delta S$}
    \structure{Reacción:} \ce{A -> B}
    \structure{Razonaremos utilizando la ecuación:} $\Delta G = \Delta H -T\vdot\Delta S$
    \visible<2->{
        \structure{Reacción endotérmica:} $\Delta H>0$
                }
    \visible<3->{
        \structure{Reacción transcurre hasta el final:} $\Delta G<0$
                }
    \visible<4->{
        \begin{overprint}
            \onslide<4>
                \structure{Por tanto:} $\overbrace{\Delta H}^{\Delta H>0} -T\vdot\Delta S<0$
            \onslide<5>
                \structure{Por tanto:} $\overbrace{\Delta H}^{\Delta H>0} <T\vdot\Delta S$
            \onslide<6->
                \structure{Por tanto:} $\overbrace{\Delta H}^{\Delta H>0} <\underbrace{T\vdot\Delta S}_{>0}$

                $T\vdot\Delta S$ tiene que ser mayor que cero porque es imposible que sea mayor que cero y negativo ya que $\Delta H>0$
        \end{overprint}
                }
    \visible<7>{
        \structure{Por el tercer principio de la Termodinámica:} $\tcbhighmath[boxrule=0.4pt,arc=4pt,colframe=red,drop fuzzy shadow=green]{\Delta S>0}$
                }
\end{frame}

\begin{frame}
    \frametitle{\ejerciciocmd}
    \framesubtitle{Resolución (\rom{2}): signo de $\Delta G$ para \ce{B -> A}}
    \structure{Reacción:} \ce{B -> A}
    \structure{Como $\Delta G$ es una función de estado y esta reacción es la inversa a la anterior:}
        $$
            \tcbhighmath[boxrule=0.4pt,arc=4pt,colframe=red,drop fuzzy shadow=green]{\Delta G > 0}
        $$
\end{frame}

\begin{frame}
    \frametitle{\ejerciciocmd}
    \framesubtitle{Resolución (\rom{3}): \ce{B -> A} a baja T}
    \structure{Reacción:} \ce{B -> A}
    \structure{Razonaremos utilizando la ecuación:} $\Delta G = \Delta H -T\vdot\Delta S$
    \visible<2->{
        \structure{Reacción exotérmica:} $\Delta H<0$
    }
    \visible<3->{
        \structure{Signo de la Entropía también se invierte:} $\Delta S<0$
    }
    \visible<4->{
        \begin{overprint}
            \onslide<4>
                \structure{Si $T$ baja (imaginaos casi cero):} $\Delta H < T\vdot\Delta S$
                
                Un número negativo es mayor que otro cuanto más cerca esté de cero.
            \onslide<5>
                \structure{Por tanto:} $\overbrace{\Delta H - T\vdot\Delta S}^{\Delta G} < 0$
            \onslide<6->
                \structure{Por tanto:} $\tcbhighmath[boxrule=0.4pt,arc=4pt,colframe=red,drop fuzzy shadow=green]{\Delta G < 0}$
        \end{overprint}
    }
\end{frame}
