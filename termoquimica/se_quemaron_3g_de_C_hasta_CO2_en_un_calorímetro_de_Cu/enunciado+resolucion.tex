\begin{frame}
	\frametitle{\ejerciciocmd}
	\framesubtitle{Enunciado}
	\textbf{
			Una reacción tiene una constante de velocidad de \SI{,017}{\per\second} a \SI{298}{\kelvin} y una energía libre de activación del \SI{27,235}{\kilo\joule\per\mol}. La adición de un catalizador disminuye dicha energía de activación hasta un \SI{33}{\percent} de su valor inicial. Calcule la nueva constante de velocidad.
\resultadocmd{ \SI{26,86}{\per\second} }

		}
\end{frame}

\begin{frame}
	\frametitle{\ejerciciocmd}
	\framesubtitle{Datos del problema}
	{\huge
		\centering Determinar el poder calorífico ($PC$)
		$$
			PC(\ce{C})?
		$$
	}
	\centering\ce{C(s) + O2(g) -> CO2(g)}
	\begin{center}
		\tcbhighmath[boxrule=0.4pt,arc=4pt,colframe=black,drop fuzzy shadow=yellow]{m(\ce{C})=\SI{3}{\gram}}
		\tcbhighmath[boxrule=0.4pt,arc=4pt,colframe=green,drop fuzzy shadow=blue]{m(\ce{Cu})=\SI{1500}{\gram}}
		\tcbhighmath[boxrule=0.4pt,arc=4pt,colframe=blue,drop fuzzy shadow=red]{m(\ce{H2O})=\SI{2000}{\gram}}
	\end{center}
	\begin{center}
		\tcbhighmath[boxrule=0.4pt,arc=4pt,colframe=green,drop fuzzy shadow=red]{T_{\text{inicial}}(\ce{Cu + H2O})=\SI{20,0}{\celsius}=\SI{293,15}{\kelvin}}
		\tcbhighmath[boxrule=0.4pt,arc=4pt,colframe=green,drop fuzzy shadow=red]{T_{\text{final}}(\ce{Cu + H2O})=\SI{31,3}{\celsius}=\SI{304,45}{\kelvin}}
	\end{center}
	\begin{center}
		\tcbhighmath[boxrule=0.4pt,arc=4pt,colframe=green,drop fuzzy shadow=blue]{C_p(\ce{Cu})=\SI{,389}{\joule\per\gram\per\kelvin}}
		\tcbhighmath[boxrule=0.4pt,arc=4pt,colframe=blue,drop fuzzy shadow=red]{C_p(\ce{H2O})=\SI{4,18}{\joule\per\gram\per\kelvin}}
	\end{center}
\end{frame}

\begin{frame}
	\frametitle{\ejerciciocmd}
	\framesubtitle{Resolución (\rom{1}): determinación del calor cedido del carbono}
	\centering\textbf{Se supone que dentro del calorímetro hay un sistema aislado.}
	\structure{Según el 1"er principio de la Termodinámica:} $\Delta U=0=Q_{\text{total}}+W$
	\visible<2->{
		\structure{No se realiza ningún trabajo porque el calorímetro no cambia de volumen ($\Delta V=0$):} $Q_{\text{total}}+\cancelto{0}{W}=0$
	}
	\visible<3->{
		\structure{Transferencia de calor total tiene que ser cero:} $Q_{\text{total}}=Q(\ce{Cu})+Q(\ce{H2O})+Q(\ce{C})=0$
				}
	\visible<4->{
		\structure{Según el criterio de signos termodinámico:}
		\begin{itemize}
			\item $Q(\ce{Cu})>0$ (absorbe calor)
			\item $Q(\ce{H2O})>0$ (absorbe calor)
			\item $Q(\ce{C})<0$ (cede calor)
		\end{itemize}
		\centering\structure{Quedándonos:} $Q(\ce{Cu})+Q(\ce{H2O})=-Q(C)$
				}
	\visible<5->{
		\begin{overprint}
			\onslide<5>
				\centering\structure{Aplicando:} $Q=m\cdot C_p\cdot\Delta T$
				$$
					-Q(\ce{C}) = \overbrace{m(\ce{Cu})\cdot C_p(\ce{Cu})\Delta T}^{Q(\ce{Cu})}+\underbrace{m(\ce{H2O})\cdot C_p(\ce{H2O})\cdot\Delta T}_{Q(\ce{H2O})}
				$$
			\onslide<6->
				\centering\structure{Operando:}
				$$
					Q(\ce{C}) = -\left[m(\ce{Cu})\cdot C_p(\ce{Cu}) + m(\ce{H2O})\cdot C_p(\ce{H2O})\right]\cdot\Delta T
				$$
		\end{overprint}
	}
	\visible<7->{
		\centering\structure{Sustituimos por los valores del enunciado:}
		$$
			Q(\ce{C}) = -\left[\SI{1500}{\gram}\cdot\SI{,389}{\joule\per\gram\per\kelvin} + \SI{2000}{\gram}\cdot\SI{4,18}{\joule\per\gram\per\kelvin}\right]\cdot\overbrace{\SI{11,30}{\kelvin}}^{\SI{304,45}{\kelvin}-\SI{293,15}{\kelvin}}
		$$
		\centering\centering\myovalbox{\textcolor{yellow}{$Q(\ce{C})=\SI{-1,010e5}{\joule}=\SI{-101,06}{\kilo\joule}$}}
				}
\end{frame}

\begin{frame}
	\frametitle{\ejerciciocmd}
	\framesubtitle{Resolución (\rom{2}): calcular el poder calorífico del carbono}
	$$
		\overbrace{PC(\ce{C})}^{\text{poder calorífico}} = \frac{\SI{-1,010e5}{\joule}}{\SI{3}{\gram}}
	$$
	\centering\tcbhighmath[boxrule=0.4pt,arc=4pt,colframe=black,drop fuzzy shadow=yellow]{PC(\ce{C}) = \SI{-33687,18}{\joule\per\gram} = \SI{-33,69}{\kilo\joule\per\gram}}
\end{frame}
