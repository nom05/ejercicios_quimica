\begin{frame}
    \frametitle{\ejerciciocmd}
    \framesubtitle{Enunciado}
    \textbf{
		Dadas las siguientes reacciones:
\begin{itemize}
    \item \ce{I2(g) + H2(g) -> 2 HI(g)}~~~$\Delta H_1 = \SI{-0,8}{\kilo\calorie}$
    \item \ce{I2(s) + H2(g) -> 2 HI(g)}~~~$\Delta H_2 = \SI{12}{\kilo\calorie}$
    \item \ce{I2(g) + H2(g) -> 2 HI(ac)}~~~$\Delta H_3 = \SI{-26,8}{\kilo\calorie}$
\end{itemize}
Calcular los parámetros que se indican a continuación:
\begin{description}%[label={\alph*)},font={\color{red!50!black}\bfseries}]
    \item[\texttt{a)}] Calor molar latente de sublimación del yodo.
    \item[\texttt{b)}] Calor molar de disolución del ácido yodhídrico.
    \item[\texttt{c)}] Número de calorías que hay que aportar para disociar en sus componentes el yoduro de hidrógeno gas contenido en un matraz de \SI{750}{\cubic\centi\meter} a \SI{25}{\celsius} y \SI{800}{\torr} de presión.
\end{description}
\resultadocmd{\SI{12,8}{\kilo\calorie}; \SI{-13,0}{\kilo\calorie}; \SI{12,9}{\calorie}}

	}
\end{frame}

\begin{frame}
    \frametitle{\ejerciciocmd}
    \framesubtitle{Resolución (\rom{1}): calor molar latente de sublimación del yodo}
    \structure{Identificamos el proceso:} sublimación
    \begin{center}
        \ce{I2(s) -> I2(g)}
    \end{center}
    \visible<2->{
        \structure{Buscamos los compuestos implicados en los estados correspondientes en las reacciones dadas:}
        \begin{center}
            \tcbhighmath[boxrule=0.4pt,arc=4pt,colframe=blue,drop fuzzy shadow=green]{\ce{I2(g)}} \ce{ + H2(g) -> 2 HI(g)}~~~$\Delta H_1 = \SI{-0,8}{\kilo\calorie}$\\
            \tcbhighmath[boxrule=0.4pt,arc=4pt,colframe=blue,drop fuzzy shadow=green]{\ce{I2(s)}} \ce{ + H2(g) -> 2 HI(g)}~~~$\Delta H_2 = \SI{12}{\kilo\calorie}$
        \end{center}
                }
    \visible<3->{
            \structure{El yodo gas debería estar en productos:}
            \begin{overprint}
                \onslide<3>
                    \begin{center}
                        \tcbhighmath[boxrule=0.4pt,arc=4pt,colframe=blue,drop fuzzy shadow=green]{\ce{I2(g)}} \ce{ + H2(g) -> 2 HI(g)}~~~$\Delta H_1 = \SI{-0,8}{\kilo\calorie}$\\
                        \tcbhighmath[boxrule=0.4pt,arc=4pt,colframe=blue,drop fuzzy shadow=green]{\ce{I2(s)}} \ce{ + H2(g) -> 2 HI(g)}~~~$\Delta H_2 = \SI{12}{\kilo\calorie}$
                    \end{center}
                \onslide<4->
                    \begin{center}
                        2
                        \cancel{\ce{HI(g)}}
                        \ce{->}
                        \tcbhighmath[boxrule=0.4pt,arc=4pt,colframe=blue,drop fuzzy shadow=green]{\ce{I2(g)}}
                        \ce{+}
                        \cancel{\ce{H2(g)}}
                        ~~~$\Delta H_1 = \SI{0,8}{\kilo\calorie}$\\
                        \tcbhighmath[boxrule=0.4pt,arc=4pt,colframe=blue,drop fuzzy shadow=green]{\ce{I2(s)}}
                        \ce{+}
                        \cancel{\ce{H2(g)}}
                        \ce{->}
                        \cancel{\ce{2 HI(g)}}
                        ~~~$\Delta H_2 = \SI{12}{\kilo\calorie}$
                    \end{center}
            \end{overprint}
                }
    \visible<5->{
        \structure{Sumamos ahora las reacciones y las entalpías:}
        \begin{center}
            \tcbhighmath[boxrule=0.4pt,arc=4pt,colframe=blue,drop fuzzy shadow=green]{\ce{I2(g)}}
            \ce{->}
            \tcbhighmath[boxrule=0.4pt,arc=4pt,colframe=blue,drop fuzzy shadow=green]{\ce{I2(s)}}
            ~~~$\tcbhighmath[boxrule=0.4pt,arc=4pt,colframe=blue,drop fuzzy shadow=green]{\Delta H_1 = \SI{12,8}{\kilo\calorie}}$
        \end{center}
                }
\end{frame}

\begin{frame}
    \frametitle{\ejerciciocmd}
    \framesubtitle{Resolución (\rom{2}): Calor molar de disolución del ácido yodhídrico}
    \structure{Identificamos el proceso:} disolución
    \begin{center}
        \ce{HI(g) -> HI(ac)}
    \end{center}
    \visible<2->{
        \structure{Buscamos los compuestos implicados en los estados correspondientes en las reacciones dadas:}
        \begin{center}
                \ce{I2(g) + H2(g) -> }\tcbhighmath[boxrule=0.4pt,arc=4pt,colframe=red,drop fuzzy shadow=green]{\ce{2 HI(g)}}~~~$\Delta H_1 = \SI{-0,8}{\kilo\calorie}$\\
                \ce{I2(g) + H2(g) -> }\tcbhighmath[boxrule=0.4pt,arc=4pt,colframe=red,drop fuzzy shadow=green]{\ce{2 HI(ac)}}~~~$\Delta H_3 = \SI{-26,8}{\kilo\calorie}$
        \end{center}
    }
    \visible<3->{
        \structure{El yoduro de hidrógeno gas debería estar en reactivos:}
        \begin{overprint}
            \onslide<3>
                \begin{center}
                    \ce{I2(g)}
                    \ce{+}
                    \ce{H2(g)}
                    \ce{->}
                    \tcbhighmath[boxrule=0.4pt,arc=4pt,colframe=red,drop fuzzy shadow=green]{\ce{2HI(g)}}
                    ~~~$\Delta H_1 = \SI{-0,8}{\kilo\calorie}$\\
                    \ce{I2(g)}
                    \ce{+}
                    \ce{H2(g)}
                    \ce{->}
                    \tcbhighmath[boxrule=0.4pt,arc=4pt,colframe=red,drop fuzzy shadow=green]{\ce{2 HI(ac)}}
                    ~~~$\Delta H_3 = \SI{-26,8}{\kilo\calorie}$
                \end{center}
            \onslide<4>
                \begin{center}
                    \tcbhighmath[boxrule=0.4pt,arc=4pt,colframe=red,drop fuzzy shadow=green]{\ce{2HI(g)}}
                    \ce{->}
                    \ce{I2(g)}
                    \ce{+}
                    \ce{H2(g)}
                    ~~~$\Delta H_1 = \SI{0,8}{\kilo\calorie}$\\
                    \ce{I2(g)}
                    \ce{+}
                    \ce{H2(g)}
                    \ce{->}
                    \tcbhighmath[boxrule=0.4pt,arc=4pt,colframe=red,drop fuzzy shadow=green]{\ce{2 HI(ac)}}
                    ~~~$\Delta H_3 = \SI{-26,8}{\kilo\calorie}$
                \end{center}
            \onslide<5->
                \begin{center}
                    \tcbhighmath[boxrule=0.4pt,arc=4pt,colframe=red,drop fuzzy shadow=green]{\ce{2HI(g)}}
                    \ce{->}
                    \cancel{\ce{I2(g)}}
                    \ce{+}
                    \cancel{\ce{H2(g)}}
                    ~~~$\Delta H_1 = \SI{0,8}{\kilo\calorie}$\\
                    .\cancel{\ce{I2(g)}}
                    \ce{+}
                    \cancel{\ce{H2(g)}}
                    \ce{->}
                    \tcbhighmath[boxrule=0.4pt,arc=4pt,colframe=red,drop fuzzy shadow=green]{\ce{2 HI(ac)}}
                    ~~~$\Delta H_3 = \SI{-26,8}{\kilo\calorie}$
                \end{center}
        \end{overprint}
                }
    \visible<6->{
        \structure{La reacción tiene que ser mol a mol:}
        \begin{overprint}
            \onslide<6>
                \begin{center}
                    \tcbhighmath[boxrule=0.4pt,arc=4pt,colframe=red,drop fuzzy shadow=green]{\ce{2HI(g)}}
                    \ce{->}
                    \tcbhighmath[boxrule=0.4pt,arc=4pt,colframe=red,drop fuzzy shadow=green]{\ce{2HI(ac)}}
                    ~~~$\Delta H = \SI{-26,0}{\kilo\calorie}$
                \end{center}
            \onslide<7>
                \begin{center}
                    \tcbhighmath[boxrule=0.4pt,arc=4pt,colframe=red,drop fuzzy shadow=green]{\ce{HI(g)}}
                    \ce{->}
                    \tcbhighmath[boxrule=0.4pt,arc=4pt,colframe=red,drop fuzzy shadow=green]{\ce{HI(ac)}}
                    ~~~$\Delta H = \frac{-26,0}{2}~\si{\kilo\calorie}$
                \end{center}
            \onslide<8>
                \begin{center}
                    \tcbhighmath[boxrule=0.4pt,arc=4pt,colframe=red,drop fuzzy shadow=green]{\ce{HI(g)}}
                    \ce{->}
                    \tcbhighmath[boxrule=0.4pt,arc=4pt,colframe=red,drop fuzzy shadow=green]{\ce{HI(ac)}}
                    ~~~$\tcbhighmath[boxrule=0.4pt,arc=4pt,colframe=red,drop fuzzy shadow=green]{\Delta H = \SI{-13,0}{\kilo\calorie}}$
                \end{center}
        \end{overprint}
                }
\end{frame}

\begin{frame}
    \frametitle{\ejerciciocmd}
    \framesubtitle{Resolución (\rom{3}): calor para disociar en sus componentes el yoduro de hidrógeno gas}
    \structure{Reacción:} \ce{2HI(g) -> I2(g) + H2(g)}~~~$\Delta H(\ce{HI}) = \frac{0,8}{2}\si{\kilo\calorie\per\mol}$
    \visible<2->{
        \structure{Calor de reacción al disociar $n$ moles de \ce{HI}}
        \begin{overprint}
            \onslide<2>
                $$
                    Q(\ce{HI}) = n(\ce{HI})\cdot\Delta H(\ce{HI})
                $$
            \onslide<3>
                $$
                    Q(\ce{HI}) = \overbrace{n(\ce{HI})}^{PV=nRT\Rightarrow n=\frac{PV}{RT}}\cdot\Delta H(\ce{HI})
                $$
            \onslide<4->
                $$
                    Q(\ce{HI}) = \frac{P(\ce{HI})\cdot V(\ce{HI})}{R\cdot T(\ce{HI})}\cdot\Delta H(\ce{HI})
                $$
        \end{overprint}
                }
    \visible<5>{
        \structure{Sustituimos:}
        $$
            Q(\ce{HI}) = \frac{\frac{80\cancel{0}}{76\cancel{0}}~\si{\cancel\atm}\cdot\SI{,750}{\cancel\liter}}{\SI{,082}{\cancel\atm\cancel\liter\per\mol\per\cancel\kelvin}\cdot\SI{298,15}{\cancel\kelvin}}\cdot\SI{,4e3}{\calorie\per\cancel\mol}
        $$
        $$
            \tcbhighmath[boxrule=0.4pt,arc=4pt,colframe=red,drop fuzzy shadow=green]{Q(\ce{HI}) =\SI{12,9}{\calorie}}
        $$
                }
\end{frame}
