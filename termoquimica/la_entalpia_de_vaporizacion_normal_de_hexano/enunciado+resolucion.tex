\begin{frame}
	\frametitle{\ejerciciocmd}
	\framesubtitle{Enunciado}
    	\textbf{
		Una reacción tiene una constante de velocidad de \SI{,017}{\per\second} a \SI{298}{\kelvin} y una energía libre de activación del \SI{27,235}{\kilo\joule\per\mol}. La adición de un catalizador disminuye dicha energía de activación hasta un \SI{33}{\percent} de su valor inicial. Calcule la nueva constante de velocidad.
\resultadocmd{ \SI{26,86}{\per\second} }

		}
\end{frame}

\begin{frame}
    \frametitle{\ejerciciocmd}
    \framesubtitle{Datos del problema}
    \begin{center}
      	{\huge\textbf{¿$P$, fases?} del hexano (\ce{C6H14})}\\[.4cm]
        Condiciones normales: \tcbhighmath[boxrule=0.4pt,arc=4pt,colframe=black,drop fuzzy shadow=red]{P=\SI{1}{\atm}}\ce{->}
        \tcbhighmath[boxrule=0.4pt,arc=4pt,colframe=black,drop fuzzy shadow=red]{T_{\text{ebullición}}=\SI{341,6}{\kelvin}}\\[.4cm]
        \tcbhighmath[boxrule=0.4pt,arc=4pt,colframe=black,drop fuzzy shadow=red]{\Delta H_{vap}=\SI{7,611}{\kilo\calorie\per\mol}}\quad
        \tcbhighmath[boxrule=0.4pt,arc=4pt,colframe=black,drop fuzzy shadow=red]{d=\SI{,6548}{\gram\per\milli\liter}}\quad
        \tcbhighmath[boxrule=0.4pt,arc=4pt,colframe=black,drop fuzzy shadow=red]{n=\SI{1}{\mol}}\\[.4cm]
        Dos recipientes de diferente volumen:
        \tcbhighmath[boxrule=0.4pt,arc=4pt,colframe=green,drop fuzzy shadow=blue]{V_1=\SI{6}{\liter}}\quad\quad
        \tcbhighmath[boxrule=0.4pt,arc=4pt,colframe=blue,drop fuzzy shadow=green]{V_2=\SI{100}{\liter}}
    \end{center}
    \visible<2->{
        Pero también:\\
        \centering\tcbhighmath[boxrule=0.4pt,arc=4pt,colframe=black,drop fuzzy shadow=red]{Mm=\SI{86,175}{\gram\per\mol}}
                }
\end{frame}

\begin{frame}
    \frametitle{\ejerciciocmd}
    \framesubtitle{Resolución (\rom{1}): determinación de la presión de vapor a \SI{313}{\kelvin}}
    \structure{Presión de vapor tenemos que utilizar la ecuación de Clausius-Clapeyron:}
    \begin{overprint}
        \onslide<1>
            $$
                \ln\left(\frac{P_2}{P_1}\right) = \frac{\Delta H}{R}\left(\frac{1}{T_1}-\frac{1}{T_2}\right)
            $$
            Con $T_1=T_{\text{ebullición}}=\SI{341,6}{\kelvin}$ a $P_1=\SI{1}{\atm}$
        \onslide<2->
            $$
				\ln\left(\frac{P_2}{\SI{1}{\atm}}\right) = \frac{\overbrace{\Delta H}^{\SI{7611}{\calorie\per\mol}}}{\underbrace{R}_{\SI{1,987}{\calorie\per\mol\per\kelvin}}}\left(\frac{1}{\underbrace{T_{\text{ebullición}}}_{\SI{341,6}{\kelvin}}}-\frac{1}{\underbrace{T_2}_{\SI{313}{\kelvin}}}\right)=\num{-1,025}
			$$
			\centering\myovalbox{\textcolor{yellow}{$P_2=\SI{,359}{\atm}=P_v(\text{hexano})$}}

    \end{overprint}
\end{frame}

\begin{frame}
	\frametitle{\ejerciciocmd}
	\framesubtitle{Resolución (\rom{2}): determinación de las fases en \SI{6}{\liter}}
	\begin{block}{Si todo el hexano es líquido}
		\centering$n=\rfrac{m}{Mm}$; $d=\rfrac{m}{V}$\ce{->}$V=\rfrac{n\cdot Mm}{d}$
		$$
			V(\text{líquido})=\frac{\SI{1}{\cancel\mol}\cdot\SI{86,175}{\cancel\gram\per\cancel\mol}}{\SI{,6548}{\cancel\gram\per\milli\liter}}=\SI{131,61}{\milli\liter}
		$$
		\centering\textbf{No ocupa los \SI{6}{\liter}}
	\end{block}
	\begin{alertblock}{Si todo el hexano es gas}<2->
		$$
			P\cdot V=n\cdot R\cdot T\Rightarrow P=\frac{\SI{1}{\cancel\mol}\cdot\SI{,082}{\atm\cancel\liter\per\cancel\mol\per\cancel\kelvin}\cdot\SI{313}{\cancel\kelvin}}{\SI{6}{\cancel\liter}}=\SI{4,28}{\atm}
		$$
		\centering$P>P_v$ ($\SI{4,28}{\atm}>\SI{,359}{\atm}$) por lo que el \textbf{gas se condensa}.
	\end{alertblock}
	\visible<2->{
		\centering\myovalbox{\textcolor{yellow}{Vamos a encontrarnos con dos fases: líquido+vapor}}
				}
	\visible<3->{
		\structure{Estimación de fases:} si suponemos que el $V(\text{líquido}) <<< V(\text{recipiente})$
		$$
			n(\text{gas})=\frac{P_v\cdot V(\text{recipiente})}{R\cdot T}\Rightarrow n(\text{gas})=\frac{\SI{,359}{\cancel\atm}\cdot\SI{6}{\cancel\liter}}{\SI{,082}{\cancel\atm\cancel\liter\per\mol\per\cancel\kelvin}\cdot\SI{313}{\cancel\kelvin}}=\SI{,0839}{\mol}
		$$
				}
		\visible<4->{
			$$
				m=n\cdot Mm\Rightarrow m(\text{gas})=\SI{,0839}{\cancel\mol}\cdot\SI{86,175}{\gram\per\cancel\mol}=\SI{7,23}{\gram}
			$$
			$$
				m(\text{líquido})=\overbrace{\SI{86,175}{\gram}}^{\SI{1}{\mol}}-\SI{7,23}{\gram}=\SI{78,95}{\gram}\Rightarrow V(\text{líquido})=\frac{\SI{78,95}{\cancel\gram}}{\SI{,6548}{\cancel\gram\per\milli\liter}}=\SI{120,6}{\milli\liter}
			$$
					}
\end{frame}

\begin{frame}
	\frametitle{\ejerciciocmd}
	\framesubtitle{Resolución (\rom{3}): determinación de las fases en \SI{100}{\liter}}
	$$
		P\cdot V=n\cdot R\cdot T\Rightarrow P=\frac{\SI{1}{\cancel\mol}\cdot\SI{,082}{\atm\cancel\liter\per\cancel\mol\per\cancel\kelvin}\cdot\SI{313}{\cancel\kelvin}}{\SI{100}{\cancel\liter}}=\SI{,26}{\atm}
	$$
	\centering$P<P_v$ ($\SI{,26}{\atm}<\SI{,359}{\atm}$) por lo que \textbf{únicamente} hay \textbf{fase gas}.
\end{frame}
