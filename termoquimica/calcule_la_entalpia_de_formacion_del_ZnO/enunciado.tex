Calcule la entalpía de formación del óxido de cinc, \ce{ZnO}, a partir de los datos siguientes de calores de reacción:\\[.3cm]
    \begin{center}
        \ce{H2SO4(ac) + Zn(s) -> ZnSO4(ac) + H2(g)}\quad\quad$\Delta H=\SI{80,1}{\kilo\calorie\per\mol}$\\
        \ce{2H2(g) + O2(g) -> 2H2O(l)}\quad\quad\quad\quad\quad$\Delta H=\SI{136,6}{\kilo\calorie\per\mol}$\\
        \ce{H2SO4(ac) + ZnO(s) -> ZnSO4(ac) + H2O(l)}\quad\quad$\Delta H=\SI{50,52}{\kilo\calorie\per\mol}$
    \end{center}
\resultadocmd{\SI{97,88}{\kilo\calorie\per\mol}}
