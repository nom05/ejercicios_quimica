\begin{frame}
	\frametitle{\ejerciciocmd}
	\framesubtitle{Enunciado}
	\textbf{
		Una reacción tiene una constante de velocidad de \SI{,017}{\per\second} a \SI{298}{\kelvin} y una energía libre de activación del \SI{27,235}{\kilo\joule\per\mol}. La adición de un catalizador disminuye dicha energía de activación hasta un \SI{33}{\percent} de su valor inicial. Calcule la nueva constante de velocidad.
\resultadocmd{ \SI{26,86}{\per\second} }

		}
\end{frame}

\begin{frame}
	\frametitle{\ejerciciocmd}
	\framesubtitle{Resolución (\rom{1}): determinación de $\Delta H_{\text{formación}}(\ce{ZnO})$}
	\structure{Combinando las 3 reacciones del enunciado buscamos la reacción:}
	$$
		\ce{Zn(s) + 1/2O2(g) -> ZnO}
	$$
	Reacciones de partida:
	\begin{align}
	    \ce{H2SO4(ac) + Zn(s) -> ZnSO4(ac) + H2(g)}\quad\Delta H=\SI{80,1}{\kilo\calorie\per\mol}\label{reac:ZnSO4}\\
		\ce{2H2(g) + O2(g) -> 2H2O(l)}\quad\Delta H=\SI{136,6}{\kilo\calorie\per\mol}\label{reac:H2O}\\
		\ce{H2SO4(ac) + ZnO(s) -> ZnSO4(ac) + H2O(l)}\quad\Delta H=\SI{50,52}{\kilo\calorie\per\mol}\label{reac:H2SO4}
	\end{align}
	\visible<2->{
		\structure{Aplicando la \underline{ley de Hess}, nos interesa que el \ce{H2SO4} y \ce{ZnSO4} aparezcan en reactivos y productos:}
		\begin{overprint}
			\onslide<2>
				\centering A la inversa de la Reacción~\ref{reac:H2SO4} le sumamos la Reacción~\ref{reac:ZnSO4}:
					\[\renewcommand\arraystretch{1.4}
						\arraycolsep=1.4pt
						\begin{array}{rl}\toprule[1pt]
								\ce{ZnSO4(ac) + H2O(l) -> H2SO4(ac) + ZnO(s)}\quad\Delta H=\SI{-50,52}{\kilo\calorie\per\mol}\\
								\ce{H2SO4(ac) + Zn(s) -> ZnSO4(ac) + H2(g)}\quad\Delta H=\SI{80,1}{\kilo\calorie\per\mol}\\
							\midrule
								\ce{\cancel{\ce{ZnSO4(ac)}} + \cancel{\ce{H2SO4(ac)}} + H2O(l) + Zn(s) -> \cancel{\ce{H2SO4(ac)}} + ZnO(s) + \cancel{\ce{ZnSO4(ac)}} + H2(g)}\\
							\bottomrule[1pt]
						\end{array}
					\]
			\onslide<3>
				\centering A la inversa de la Reacción~\ref{reac:H2SO4} le sumamos la Reacción~\ref{reac:ZnSO4}:
					\[\renewcommand\arraystretch{1.4}
						\arraycolsep=1.4pt
						\begin{array}{rl}\toprule[1pt]
								\ce{ZnSO4(ac) + H2O(l) -> H2SO4(ac) + ZnO(s)}\quad\Delta H=\SI{-50,52}{\kilo\calorie\per\mol}\\
								\ce{H2SO4(ac) + Zn(s) -> ZnSO4(ac) + H2(g)}\quad\Delta H=\SI{80,1}{\kilo\calorie\per\mol}\\
							\midrule
								\ce{H2O(l) + Zn(s) -> ZnO(s) + H2(g)}\quad\Delta H=\SI{29,58}{\kilo\calorie\per\mol}\\
							\bottomrule[1pt]
						\end{array}
					\]
			\onslide<4>
				\centering Comparando con la Reacción~\ref{reac:H2O}:
					\[\renewcommand\arraystretch{1.4}
						\arraycolsep=1.4pt
							\begin{array}{rl}\toprule[1pt]
								\ce{H2O(l) + Zn(s) -> ZnO(s) + H2(g)}\quad\Delta H=\SI{29,58}{\kilo\calorie\per\mol}\\
								\ce{2H2(g) + O2(g) -> 2H2O(l)}\quad\Delta H=\SI{136,6}{\kilo\calorie\per\mol}\\
							\bottomrule[1pt]
						\end{array}
					\]
				\centering Vemos que necesitamos dividir por $\rfrac{1}{2}$ la Reacción~\ref{reac:H2O}.
			\onslide<5>
				\centering Comparando con la Reacción~\ref{reac:H2O}:
					\[\renewcommand\arraystretch{1.4}
						\arraycolsep=1.4pt
						\begin{array}{rl}
							\toprule[1pt]
								\ce{H2O(l) + Zn(s) -> ZnO(s) + H2(g)}\quad\Delta H=\SI{29,58}{\kilo\calorie\per\mol}\\
								\frac{1}{2}\times\left(\ce{2H2(g) + O2(g) -> 2H2O(l)}\right)\quad\Delta H=\frac{1}{2}\cdot\SI{136,6}{\kilo\calorie\per\mol}\\
							\midrule
								\ce{\cancel{\ce{H2O(l)}} + Zn(s) + \cancel{\ce{H2(g)}} + 1/2O2(g) -> ZnO(s) + \cancel{\ce{H2(g)}} + \cancel{\ce{H2O(l)}}}\\
							\bottomrule[1pt]
						\end{array}
					\]
			\onslide<6->
				\centering Comparando con la Reacción~\ref{reac:H2O}:
					\[\renewcommand\arraystretch{1.4}
						\arraycolsep=1.4pt
						\begin{array}{rl}
							\toprule[1pt]
								\ce{H2O(l) + Zn(s) -> ZnO(s) + H2(g)}\quad\Delta H=\SI{29,58}{\kilo\calorie\per\mol}\\
								\ce{H2(g) + 1/2O2(g) -> H2O(l)}\quad\Delta H=\SI{68,3}{\kilo\calorie\per\mol}\\
							\midrule
								\ce{Zn(s) + 1/2O2(g) -> ZnO(s)}\quad\tcbhighmath[boxrule=0.4pt,arc=4pt,colframe=red,drop fuzzy shadow=black]{\Delta H=\SI{97,88}{\kilo\calorie\per\mol}}\\
							\bottomrule[1pt]
						\end{array}
					\]
		\end{overprint}
				}
\end{frame}
