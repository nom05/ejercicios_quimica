En un sistema aislado ocurren simultáneamente tres procesos en los que solo se permite la transferencia de calor pero no de materia entre ellos:
\begin{itemize}
	\item El calentamiento de \SI{150}{\gram} de una sustancia \ce{X} en fase líquida (\SI{20}{\celsius}) a vapor (\SI{150}{\celsius}). Datos de la sustancia: temperatura de ebullición: \SI{75}{\celsius}, variación de entalpía de vaporización \SI{23,2}{\kilo\joule\per\gram}, calor específico (gas): \SI{,21}{\joule\per\gram\per\celsius}, calor específico (líquido): \SI{,83}{\joule\per\gram\per\celsius}.
	\item La descomposición de \SI{500}{\gram} de hidrazina (\ce{N2H4}) en amoníaco (\ce{NH3}) y nitrógeno (\ce{N2}) mediante la reacción: \ce{N2H4(l) -> NH3(g) + N2(g)}. Datos de la reacción: entalpía estándar de formación de la hidrazina: \SI{50,4}{\kilo\joule\per\mol}, entalpía estándar de formación de \ce{NH3(g)}: \SI{-46,3}{\kilo\joule\per\mol}, $Mm(\ce{N2H4}) = \SI{32,04516}{\gram\per\mol}$.
	\item \SI{30}{\liter} de agua. Datos del agua: calor específico (líquido): \SI{1}{\calorie\per\gram\per\celsius}, calor específico (sólido y gaseoso): \SI{,5}{\calorie\per\gram\per\celsius}, densidad del agua: \SI{1}{\gram\per\milli\liter}.
\end{itemize}
\begin{enumerate}[label={\alph*)},font=\bfseries]
	\item Calcula el calor que absorbe la sustancia \ce{X}.
	\item Ajusta la reacción de la descomposición de la hidrazina, calcula su entalpía de descomposición y el calor desprendido.
	\item Averigua si el agua se va a enfriar o calentar y la temperatura final sabiendo que su temperatura inicial es \SI{50}{\celsius}.
\end{enumerate}
\resultadocmd{
				\SI{3489,21}{\kilo\joule};
				\SI{-1749.6}{\kilo\joule};
				\SI{36,14}{\celsius}
			}
