Un mol de un gas ideal inicialmente a la presión de \SI{2,0}{\atm} y ocupando un volumen de \SI{25}{\liter} se pasa a una presión final de \SI{1,0}{\atm} y un volumen de \SI{50}{\liter} mediante los procesos siguientes:
	\begin{enumerate}[label={\alph*)},font=\bfseries]
		\item expansión isobárica a \SI{2,0}{\atm} hasta un volumen de \SI{50}{\liter} + cambio de presión a volumen constante,
		\item cambio de presión desde \num{2} a \SI{1}{\atm} a volumen constante + expansión a presión constante hasta las condiciones finales,
		\item proceso isotérmico reversible entre el punto inicial y el final.
	\end{enumerate}
	Calcule el calor, trabajo y energía interna en cada uno de los procesos. Calcule los mismos parámetros si el proceso c) no fuese reversible, sino que se realizase a temperatura constante frente a una presión de \SI{1}{\atm}. Datos: $C_v = \num{1,5}$, $C_p = \num{2,5}$.
