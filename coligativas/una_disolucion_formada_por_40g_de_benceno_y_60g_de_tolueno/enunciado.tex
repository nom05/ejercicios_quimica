Una disolución formada por \SI{40}{\gram} de benceno, \ce{C6H6}, y \SI{60}{\gram} de tolueno, \ce{C7H8}, se encuentra a \SI{50}{\celsius}. Calcular:
	\begin{enumerate}[label={\alph*)},font=\bfseries]
		\item la presión parcial de cada componente en el vapor,
		\item la presión total de la mezcla gaseosa,
		\item la composición de vapor en \% en peso.
	\end{enumerate}
	 ($P_v(\text{benceno}) = \SI{271}{\torr}$ a \SI{50}{\celsius}, $P_v(\text{tolueno}) = \SI{92,6}{\torr}$ a \SI{50}{\celsius}).
\resultadocmd{
	\SI{119,30}{\torr};
	\SI{51,84}{\torr};
	\SI{66,11}{\percent}
		}
