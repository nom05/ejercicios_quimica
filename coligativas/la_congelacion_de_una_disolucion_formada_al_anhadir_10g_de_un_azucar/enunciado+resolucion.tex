\begin{frame}
	\frametitle{\ejerciciocmd}
	\framesubtitle{Enunciado}
	\textbf{
		Dadas las siguientes reacciones:
\begin{itemize}
    \item \ce{I2(g) + H2(g) -> 2 HI(g)}~~~$\Delta H_1 = \SI{-0,8}{\kilo\calorie}$
    \item \ce{I2(s) + H2(g) -> 2 HI(g)}~~~$\Delta H_2 = \SI{12}{\kilo\calorie}$
    \item \ce{I2(g) + H2(g) -> 2 HI(ac)}~~~$\Delta H_3 = \SI{-26,8}{\kilo\calorie}$
\end{itemize}
Calcular los parámetros que se indican a continuación:
\begin{description}%[label={\alph*)},font={\color{red!50!black}\bfseries}]
    \item[\texttt{a)}] Calor molar latente de sublimación del yodo.
    \item[\texttt{b)}] Calor molar de disolución del ácido yodhídrico.
    \item[\texttt{c)}] Número de calorías que hay que aportar para disociar en sus componentes el yoduro de hidrógeno gas contenido en un matraz de \SI{750}{\cubic\centi\meter} a \SI{25}{\celsius} y \SI{800}{\torr} de presión.
\end{description}
\resultadocmd{\SI{12,8}{\kilo\calorie}; \SI{-13,0}{\kilo\calorie}; \SI{12,9}{\calorie}}

		}
\end{frame}

\begin{frame}
	\frametitle{\ejerciciocmd}
	\framesubtitle{Datos del problema}
	\centering{\huge ¿$Mm(\text{azúcar})=Mm(\text{az})$?}
	\begin{center}
		\tcbhighmath[boxrule=0.4pt,arc=4pt,colframe=blue,drop fuzzy shadow=green]{m(\text{az})=\SI{10}{\gram}}\quad
		\tcbhighmath[boxrule=0.4pt,arc=4pt,colframe=blue,drop fuzzy shadow=green]{K_c(\text{az})=\SI{1,86}{\celsius\kilogram\per\mol}}\\[.3cm]
		\tcbhighmath[boxrule=0.4pt,arc=4pt,colframe=green,drop fuzzy shadow=blue]{V(\ce{H2O})=\SI{100}{\milli\liter}}\quad
		\tcbhighmath[boxrule=0.4pt,arc=4pt,colframe=green,drop fuzzy shadow=blue]{T_c(\ce{H2O})=\SI{0}{\celsius}}\quad
		\tcbhighmath[boxrule=0.4pt,arc=4pt,colframe=green,drop fuzzy shadow=blue]{d(\ce{H2O})=\SI{1}{\gram\per\milli\liter}}\\[.3cm]
		\tcbhighmath[boxrule=0.4pt,arc=4pt,colframe=orange,drop fuzzy shadow=red]{T_c(\ce{H2O}\text{+az})=\SI{-1,24}{\celsius}}
	\end{center}
\end{frame}

\begin{frame}
	\frametitle{\ejerciciocmd}
	\framesubtitle{Resolución (\rom{1}): determinación de la masa molecular del azúcar}
	\structure{Variación de temperatura:} $\Delta T=T_c(\ce{H2O})-T_c(\ce{H2O}\text{+az})\Rightarrow\Delta T=\SI{0}{\celsius}-(\SI{-1,24}{\celsius})=\SI{1,24}{\celsius}$
	\visible<2->{
		\structure{Según la expresión del descenso crioscópico:} $\Delta T=K_c\cdot\overbrace{m}^{\text{molalidad}}$\\
		Despejamos y calculamos la molalidad:
		$$
			m = \frac{\Delta T}{K_c}\Rightarrow m(\text{az})=\frac{\SI{1,24}{\cancel\celsius}}{\SI{1,86}{\cancel\celsius\kilogram\per\mol}}=\frac{2}{3}~\si{\mol\per\kilogram}
		$$
				}
	\visible<3->{
		\structure{\si{\kilogram} de disolvente (agua) usando la densidad:} $d(\ce{H2O})=\SI{1}{\gram\per\milli\liter}=\SI{1e-3}{\kilogram\per\milli\liter}$
		$$
			d=\frac{m}{V}\Rightarrow m=d\cdot V\Rightarrow m(\ce{H2O})=\SI{1e-3}{\kilogram\per\cancel\milli\liter}\cdot\SI{100}{\cancel\milli\liter}=\SI{,1}{\kilogram}
		$$
				}
	\visible<4->{
		\structure{Usando la definición de molalidad:} $m=\rfrac{n(\text{soluto})}{\text{\si{\kilogram} disolvente}}$
		\begin{overprint}
			\onslide<4>
				$$
					\text{molalidad}(\text{az}) = \frac{\overbrace{n(\text{az})}^{n=\rfrac{m}{Mm}}}{\text{\si{\kilogram} \ce{H2O}}}
				$$
			\onslide<5>
				$$
					\text{molalidad}(\text{az}) = \frac{m(\text{az})}{Mm(\text{az})\cdot\text{\si{\kilogram} \ce{H2O}}}
				$$
			\onslide<6>
				$$
					Mm(\text{az}) = \frac{m(\text{az})}{\text{molalidad}(\text{az})\cdot\text{\si{\kilogram} \ce{H2O}}}
				$$
			\onslide<7->
				$$
					Mm(\text{az}) = \frac{\SI{10}{\gram}}{\rfrac{2}{3}~\si{\mol\per\cancel\kilogram}\cdot\SI{,1}{\cancel\kilogram}}
				$$
		\end{overprint}
				}
	\visible<7->{
		\centering\tcbhighmath[boxrule=0.4pt,arc=4pt,colframe=blue,drop fuzzy shadow=green]{Mm(\text{az})=\SI{150}{\gram\per\mol}}
				}
\end{frame}
