\begin{frame}
	\frametitle{\ejerciciocmd}
	\framesubtitle{Enunciado}
	\textbf{
		Una reacción tiene una constante de velocidad de \SI{,017}{\per\second} a \SI{298}{\kelvin} y una energía libre de activación del \SI{27,235}{\kilo\joule\per\mol}. La adición de un catalizador disminuye dicha energía de activación hasta un \SI{33}{\percent} de su valor inicial. Calcule la nueva constante de velocidad.
\resultadocmd{ \SI{26,86}{\per\second} }

		}
\end{frame}

\begin{frame}
	\frametitle{\ejerciciocmd}
	\framesubtitle{Datos del problema}
	\centering{\huge ¿$Mm(\text{proteína})=Mm$?}
	\begin{center}
		\tcbhighmath[boxrule=0.4pt,arc=4pt,colframe=blue,drop fuzzy shadow=green]{\Pi=\SI{9,12}{\torr}=\SI{,012}{\atm}}\quad
		\tcbhighmath[boxrule=0.4pt,arc=4pt,colframe=blue,drop fuzzy shadow=green]{T=\SI{298,15}{\kelvin}}\\[.3cm]
		\tcbhighmath[boxrule=0.4pt,arc=4pt,colframe=blue,drop fuzzy shadow=green]{V(\text{disolución})=\SI{100}{\milli\liter}=\SI{,1}{\liter}}
	\end{center}
\end{frame}

\begin{frame}
	\frametitle{\ejerciciocmd}
	\framesubtitle{Resolución (\rom{1}): determinación de la masa molecular de la proteína}
	\structure{Ecuación de van't Hoff -- presión osmótica ($\Pi$), concentración ($M$):}
	$$
		\Pi = \overbrace{M}^{M = \rfrac{n}{V}}\vdot R\vdot T \Rightarrow \Pi = \frac{\overbrace{n}^{n=\rfrac{m}{Mm}}}{V}\vdot R\vdot T \Rightarrow
		\Pi = \frac{m}{Mm\vdot V}\vdot R\vdot T
	$$
	Despejamos $\Pi$ y sustituimos por los valores correspondientes:
	$$
		Mm = \frac{R\vdot T}{\Pi\vdot V}\vdot m\Rightarrow Mm = \frac{\SI{,082}{\cancel\atm\cancel\liter\per\mol\per\cancel\kelvin}\vdot\SI{298,15}{\cancel\kelvin}}{\SI{,012}{\cancel\atm}\vdot\SI{,1}{\cancel\liter}}\vdot\SI{2,70}{\gram}
	$$
	\centering\tcbhighmath[boxrule=0.4pt,arc=4pt,colframe=blue,drop fuzzy shadow=green]{Mm=\SI{55008,675}{\gram\per\mol}}
\end{frame}
