\begin{frame}
	\frametitle{\ejerciciocmd}
	\framesubtitle{Enunciado}
	\textbf{
		Dadas las siguientes reacciones:
\begin{itemize}
    \item \ce{I2(g) + H2(g) -> 2 HI(g)}~~~$\Delta H_1 = \SI{-0,8}{\kilo\calorie}$
    \item \ce{I2(s) + H2(g) -> 2 HI(g)}~~~$\Delta H_2 = \SI{12}{\kilo\calorie}$
    \item \ce{I2(g) + H2(g) -> 2 HI(ac)}~~~$\Delta H_3 = \SI{-26,8}{\kilo\calorie}$
\end{itemize}
Calcular los parámetros que se indican a continuación:
\begin{description}%[label={\alph*)},font={\color{red!50!black}\bfseries}]
    \item[\texttt{a)}] Calor molar latente de sublimación del yodo.
    \item[\texttt{b)}] Calor molar de disolución del ácido yodhídrico.
    \item[\texttt{c)}] Número de calorías que hay que aportar para disociar en sus componentes el yoduro de hidrógeno gas contenido en un matraz de \SI{750}{\cubic\centi\meter} a \SI{25}{\celsius} y \SI{800}{\torr} de presión.
\end{description}
\resultadocmd{\SI{12,8}{\kilo\calorie}; \SI{-13,0}{\kilo\calorie}; \SI{12,9}{\calorie}}

		}
\end{frame}

\begin{frame}
	\frametitle{\ejerciciocmd}
	\framesubtitle{Datos del problema}
	\centering{\huge ¿$Mm(\text{proteína})=Mm$?}
	\begin{center}
		\tcbhighmath[boxrule=0.4pt,arc=4pt,colframe=blue,drop fuzzy shadow=green]{\Pi=\SI{9,12}{\torr}=\SI{,012}{\atm}}\quad
		\tcbhighmath[boxrule=0.4pt,arc=4pt,colframe=blue,drop fuzzy shadow=green]{T=\SI{298,15}{\kelvin}}\\[.3cm]
		\tcbhighmath[boxrule=0.4pt,arc=4pt,colframe=blue,drop fuzzy shadow=green]{V(\text{disolución})=\SI{100}{\milli\liter}=\SI{,1}{\liter}}
	\end{center}
\end{frame}

\begin{frame}
	\frametitle{\ejerciciocmd}
	\framesubtitle{Resolución (\rom{1}): determinación de la masa molecular de la proteína}
	\structure{Ecuación de van't Hoff -- presión osmótica ($\Pi$), concentración ($M$):}
	$$
		\Pi = \overbrace{M}^{M = \rfrac{n}{V}}\vdot R\vdot T \Rightarrow \Pi = \frac{\overbrace{n}^{n=\rfrac{m}{Mm}}}{V}\vdot R\vdot T \Rightarrow
		\Pi = \frac{m}{Mm\vdot V}\vdot R\vdot T
	$$
	Despejamos $\Pi$ y sustituimos por los valores correspondientes:
	$$
		Mm = \frac{R\vdot T}{\Pi\vdot V}\vdot m\Rightarrow Mm = \frac{\SI{,082}{\cancel\atm\cancel\liter\per\mol\per\cancel\kelvin}\vdot\SI{298,15}{\cancel\kelvin}}{\SI{,012}{\cancel\atm}\vdot\SI{,1}{\cancel\liter}}\vdot\SI{2,70}{\gram}
	$$
	\centering\tcbhighmath[boxrule=0.4pt,arc=4pt,colframe=blue,drop fuzzy shadow=green]{Mm=\SI{55008,675}{\gram\per\mol}}
\end{frame}
