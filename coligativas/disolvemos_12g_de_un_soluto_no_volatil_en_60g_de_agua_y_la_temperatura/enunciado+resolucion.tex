\begin{frame}
	\frametitle{\ejerciciocmd}
	\framesubtitle{Enunciado}
	\textbf{
		Dadas las siguientes reacciones:
\begin{itemize}
    \item \ce{I2(g) + H2(g) -> 2 HI(g)}~~~$\Delta H_1 = \SI{-0,8}{\kilo\calorie}$
    \item \ce{I2(s) + H2(g) -> 2 HI(g)}~~~$\Delta H_2 = \SI{12}{\kilo\calorie}$
    \item \ce{I2(g) + H2(g) -> 2 HI(ac)}~~~$\Delta H_3 = \SI{-26,8}{\kilo\calorie}$
\end{itemize}
Calcular los parámetros que se indican a continuación:
\begin{description}%[label={\alph*)},font={\color{red!50!black}\bfseries}]
    \item[\texttt{a)}] Calor molar latente de sublimación del yodo.
    \item[\texttt{b)}] Calor molar de disolución del ácido yodhídrico.
    \item[\texttt{c)}] Número de calorías que hay que aportar para disociar en sus componentes el yoduro de hidrógeno gas contenido en un matraz de \SI{750}{\cubic\centi\meter} a \SI{25}{\celsius} y \SI{800}{\torr} de presión.
\end{description}
\resultadocmd{\SI{12,8}{\kilo\calorie}; \SI{-13,0}{\kilo\calorie}; \SI{12,9}{\calorie}}

		}
\end{frame}


\begin{frame}
	\frametitle{\ejerciciocmd}
	\framesubtitle{Datos del problema}
	\centering{\huge ¿$Mm(\text{soluto})=Mm$?}
	\begin{center}
		\tcbhighmath[boxrule=0.4pt,arc=4pt,colframe=blue,drop fuzzy shadow=green]{m(\text{soluto})=\SI{12}{\gram}}\quad
		\tcbhighmath[boxrule=0.4pt,arc=4pt,colframe=blue,drop fuzzy shadow=green]{K_e(\text{soluto})=\SI{,52}{\celsius\kilogram\per\mol}}\\[.3cm]
		\tcbhighmath[boxrule=0.4pt,arc=4pt,colframe=green,drop fuzzy shadow=blue]{m(\ce{H2O})=\SI{60}{\gram}=\SI{,060}{\kilogram}}\quad
		\tcbhighmath[boxrule=0.4pt,arc=4pt,colframe=orange,drop fuzzy shadow=red]{T_e(\ce{H2O}\text{+soluto})=\SI{374,45}{\kelvin}}\\[.3cm]
		\visible<2->{
			{\Large Pero también \ldots}
			\tcbhighmath[boxrule=0.4pt,arc=4pt,colframe=green,drop fuzzy shadow=blue]{T(\ce{H2O})=\SI{100}{\celsius}=\SI{373,15}{\kelvin}}
					}
	\end{center}
\end{frame}

\begin{frame}
	\frametitle{\ejerciciocmd}
	\framesubtitle{Resolución (\rom{1}): determinación de la masa molecular del soluto}
	\structure{Variación de temperatura:}\quad$\Delta T = T - T^0\Rightarrow\Delta T = \SI{374,45}{\kelvin} - \SI{373,15}{\kelvin} = \SI{1,3}{\kelvin} = \SI{1,3}{\celsius}$\\[.4cm]
	\visible<2->{
		\alert{\textbf{Recordad:}}\quad las escalas Kelvin y Celsius son centígradas. Por tanto $\Delta T(\si{\celsius}) = \Delta T(\si{\kelvin}) $\\[.4cm]
		\structure{Aumento ebulloscópico:}
		$$
			\Delta T = K_e\vdot m\Rightarrow m = \frac{\Delta T}{K_e}\Rightarrow m(\text{soluto}) = \frac{\SI{1,3}{\cancel\celsius}}{\SI{,52}{\cancel\celsius\kilogram\per\mol}} = \SI{2,5}{\mol\per\kilogram}
		$$
		\structure{Número de moles del soluto:}
		$$
			n(\text{soluto})=\SI{2,5}{\mol\per\cancel\kilogram}\vdot\SI{,060}{\kilogram}=\SI{,15}{\mol}
		$$
		Y finalmente:
		$$
			n = \frac{m}{Mm}\Rightarrow Mm = \frac{m}{n}\Rightarrow Mm = \frac{\SI{12}{\gram}}{\SI{,15}{\mol}}
		$$
		\centering\tcbhighmath[boxrule=0.4pt,arc=4pt,colframe=blue,drop fuzzy shadow=green]{Mm(\text{soluto})=\SI{80}{\gram\per\mol}}
				}
\end{frame}
