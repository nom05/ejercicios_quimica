\begin{frame}
	\frametitle{\ejerciciocmd}
	\framesubtitle{Enunciado}
	\textbf{
		Una reacción tiene una constante de velocidad de \SI{,017}{\per\second} a \SI{298}{\kelvin} y una energía libre de activación del \SI{27,235}{\kilo\joule\per\mol}. La adición de un catalizador disminuye dicha energía de activación hasta un \SI{33}{\percent} de su valor inicial. Calcule la nueva constante de velocidad.
\resultadocmd{ \SI{26,86}{\per\second} }

	}
\end{frame}

\begin{frame}
	\frametitle{\ejerciciocmd}
	\framesubtitle{Datos del problema}
	\begin{center}
		{\huge¿$m(\ce{CO2})$ en \si{\gram}?}\\[.4cm]
		\tcbhighmath[boxrule=0.4pt,arc=4pt,colframe=green,drop fuzzy shadow=orange]{m(\ce{LiOH}) = \SI{1,00}{\gram}}\\[.4cm]
		\tcbhighmath[boxrule=0.4pt,arc=4pt,colframe=red,drop fuzzy shadow=blue]{\ce{Li2CO3}}\quad
		\tcbhighmath[boxrule=0.4pt,arc=4pt,colframe=blue,drop fuzzy shadow=orange]{\ce{H2O}}\\[.4cm]
		Pero también ...\\[.4cm]
		\tcbhighmath[boxrule=0.4pt,arc=4pt,colframe=green,drop fuzzy shadow=orange]{Mm(\ce{LiOH}) = \SI{23,95}{\gram\per\mol}}\\[.4cm]
		\tcbhighmath[boxrule=0.4pt,arc=4pt,colframe=yellow,drop fuzzy shadow=black]{Mm(\ce{CO2}) = \SI{44,01}{\gram\per\mol}}
	\end{center}
\end{frame}

\begin{frame}
	\frametitle{\ejerciciocmd}
	\framesubtitle{Resolución}
	\structure{Ajustamos la reacción: \visible<4->{(comprobar que todos los átomos están ajustados)}}
	\begin{overprint}
		\onslide<1>
			$$
				\ce{LiOH(s) + CO2(g) -> Li2CO3(s) + H2O(l)}
			$$
		\onslide<2>
			$$
				\ce{
						$\tcbhighmath[boxrule=0.4pt,arc=4pt,colframe=green,drop fuzzy shadow=orange]{\ce{Li}}$
							OH(s) + CO2(g) -> 
						$\tcbhighmath[boxrule=0.4pt,arc=4pt,colframe=green,drop fuzzy shadow=orange]{\ce{Li2}}$
							CO3(s) + H2O(l)
					}
			$$
		\onslide<3>
			$$
				\ce{
						$\tcbhighmath[boxrule=0.4pt,arc=4pt,colframe=green,drop fuzzy shadow=orange]{\textbf{2}\ce{Li}}$
							OH(s) + CO2(g) -> 
						$\tcbhighmath[boxrule=0.4pt,arc=4pt,colframe=green,drop fuzzy shadow=orange]{\ce{Li2}}$
							CO3(s) + H2O(l)
				}
			$$
		\onslide<4->
			$$
				\ce{2LiOH(s) + CO2(g) -> Li2CO3(s) + H2O(l)}
			$$
	\end{overprint}
	\visible<4->{
		 \structure{Relacionamos estequiométricamente \ce{LiOH} y \ce{CO2}:} por cada mol de \ce{CO2} que reacciona, consumimos \SI{2}{\mol} de \ce{LiOH}
		 $$
		 	n(\ce{LiOH}) = 2 n(\ce{CO2})
		 $$
		 \structure{Usamos $n = \rfrac{m}{Mm}$:}
		 $$
		 	\frac{m(\ce{LiOH})}{Mm(\ce{LiOH})} = 2\vdot\frac{m(\ce{CO2})}{Mm(\ce{CO2})}\Rightarrow
		 	m(\ce{CO2}) = \frac{m(\ce{LiOH})\vdot Mm(\ce{CO2})}{2\vdot Mm(\ce{LiOH})}\Rightarrow
		 $$
		 $$
		 	m(\ce{CO2}) = \frac{\SI{1,00}{\gram}\vdot\SI{44,01}{\cancel\gram\per\cancel\mol}}{2\vdot\SI{23,95}{\cancel\gram\per\cancel\mol}}
		 $$
		 \begin{center}
 	 		\tcbhighmath[boxrule=0.4pt,arc=4pt,colframe=yellow,drop fuzzy shadow=black]{m(\ce{CO2}) = \SI{,92}{\gram}}
		 \end{center}
				}
\end{frame}