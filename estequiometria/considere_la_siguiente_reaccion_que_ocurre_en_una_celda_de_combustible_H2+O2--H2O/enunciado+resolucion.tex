\begin{frame}
    \frametitle{\ejerciciocmd}
    \framesubtitle{Enunciado}
    \textbf{
		Una reacción tiene una constante de velocidad de \SI{,017}{\per\second} a \SI{298}{\kelvin} y una energía libre de activación del \SI{27,235}{\kilo\joule\per\mol}. La adición de un catalizador disminuye dicha energía de activación hasta un \SI{33}{\percent} de su valor inicial. Calcule la nueva constante de velocidad.
\resultadocmd{ \SI{26,86}{\per\second} }

	}
\end{frame}

\begin{frame}
    \frametitle{\ejerciciocmd}
    \framesubtitle{Datos del problema}
    {\huge
        $$
            m(\ce{H2O})?
        $$
    }
    Reacción:\\[.3cm]
    \tcbhighmath[boxrule=0.4pt,arc=4pt,colframe=blue,drop fuzzy shadow=green]{\ce{H2(g)}}
    \ce{+}
    \tcbhighmath[boxrule=0.4pt,arc=4pt,colframe=red,drop fuzzy shadow=yellow]{\ce{O2(g)}}
    \ce{->}
    \tcbhighmath[boxrule=0.4pt,arc=4pt,colframe=green,drop fuzzy shadow=orange]{\ce{H2O(g)}}
    \\[.5 cm]
    \tcbhighmath[boxrule=0.4pt,arc=4pt,colframe=blue,drop fuzzy shadow=green]{m(\ce{H2(g)})=\SI{150}{\gram}}
    \tcbhighmath[boxrule=0.4pt,arc=4pt,colframe=red,drop fuzzy shadow=yellow]{m(\ce{O2(g)})=\SI{1500}{\gram}}\\[.5cm]
    \visible<2-|handout:0>{Pero también ...\\
        $$
            \tcbhighmath[boxrule=0.4pt,arc=4pt,colframe=blue,drop fuzzy shadow=green]{Mm(\ce{H2}) = \SI{2,02}{\gram\per\mol}}\quad
            \tcbhighmath[boxrule=0.4pt,arc=4pt,colframe=red,drop fuzzy shadow=yellow]{Mm(\ce{O2}) = \SI{32,00}{\gram\per\mol}}
        $$
        $$
            \tcbhighmath[boxrule=0.4pt,arc=4pt,colframe=green,drop fuzzy shadow=orange]{Mm(\ce{H2O(g)})=\SI{18,02}{\gram\per\mol}}
        $$
    }
\end{frame}

\begin{frame}
    \frametitle{\ejerciciocmd}
    \framesubtitle{Resolución (\rom{1}): Ajuste de la reacción y reactivo limitante}
    \begin{overprint}
        \onslide<1>
            \ce{H2(g) + O2(g) -> H2O(g)}
        \onslide<2>
            \ce{H2(g) + }
            \tcbhighmath[boxrule=0.4pt,arc=4pt,colframe=blue,drop fuzzy shadow=green]{\ce{O2(g)}}
            \ce{-> H2}
            \tcbhighmath[boxrule=0.4pt,arc=4pt,colframe=blue,drop fuzzy shadow=green]{\ce{O(g)}}
        \onslide<3>
            \ce{H2(g) + }
            \tcbhighmath[boxrule=0.4pt,arc=4pt,colframe=blue,drop fuzzy shadow=green]{\frac{1}{2}\ce{O2(g)}}
            \ce{-> H2}
            \tcbhighmath[boxrule=0.4pt,arc=4pt,colframe=blue,drop fuzzy shadow=green]{\ce{O(g)}}
        \onslide<4->
            \ce{H2(g) + \frac{1}{2}O2(g) -> H2O(g)}
    \end{overprint}
    \visible<2->{
        \begin{itemize}
            \item<4-> Según la estequiometría: $\frac{\SI{,5}{\mol}\text{ de }\ce{O2}}{\SI{1}{\mol}\text{ de }\ce{H2}}=\frac{\SI{1}{\mol}\text{ de }\ce{O2}}{\SI{2}{\mol}\text{ de }\ce{H2}}$.
            \item<5-> $n(\ce{H2})=\frac{\SI{150}{\cancel\gram}}{\SI{2,02}{\cancel\gram\per\mol}}=\SI{74,26}{\mol}$
            \item<6-> $n(\ce{O2})=\frac{\SI{1500}{\cancel\gram}}{\SI{32,00}{\cancel\gram\per\mol}}=\SI{46,88}{\mol}$
            \item<7-> Según las cantidades de los reactivos: $\frac{\SI{46,88}{\mol}\text{ de }\ce{O2}}{\SI{74,26}{\mol}\text{ de }\ce{H2}}=\frac{\SI{1}{\mol}\text{ de }\ce{O2}}{\SI{1,58}{\mol}\text{ de }\ce{H2}}>\frac{\SI{,5}{\mol}\text{ de }\ce{O2}}{\SI{1}{\mol}\text{ de }\ce{H2}}=\frac{\SI{1}{\mol}\text{ de }\ce{O2}}{\SI{2}{\mol}\text{ de }\ce{H2}}$.
            \item<8-> \ldots o $\SI{1,58}{\mol}\text{ de }\ce{H2}<\SI{2}{\mol}\text{ de }\ce{H2}$
        \end{itemize}
                }
    \visible<8->{
        \centering\myovalbox{\textcolor{yellow}{\textbf{El hidrógeno es el reactivo limitante}}}
                }
\end{frame}

\begin{frame}
    \frametitle{\ejerciciocmd}
    \framesubtitle{Resolución (\rom{2}): cálculo de la masa de agua}
    \begin{center}
        \textbf{\ce{H2(g) + \frac{1}{2}O2(g) -> H2O(g)}}
    \end{center}
    \structure{Según la estequiometría:}
        $$
            \SI{1}{\mol}\text{ de }\ce{H2 ->}\SI{1}{\mol}\text{ de } \ce{H2O}
        $$
    \begin{overprint}
        \onslide<2>
            $$
                n(\ce{H2}) = n(\ce{H2O})
            $$
        \onslide<3>
            $$
                \overbrace{n(\ce{H2})}^{n=\frac{m}{Mm}} = \underbrace{n(\ce{H2O})}_{n=\frac{m}{Mm}}
            $$
        \onslide<4->
            $$
                \frac{m(\ce{H2})}{Mm(\ce{H2})} = \frac{m(\ce{H2O})}{Mm(\ce{H2O})}
            $$
    \end{overprint}
    \visible<5->{
        \structure{Despejamos:}
        $$
            m(\ce{H2O})
            =
            m(\ce{H2})\cdot
            \frac{Mm(\ce{H2O})}{Mm(\ce{H2})}            
        $$
               }
       \visible<6->{
           \structure{Sustituimos:}
           $$
               \tcbhighmath[boxrule=0.4pt,arc=4pt,colframe=blue,drop fuzzy shadow=green]{m(\ce{H2O})}
               =
               \SI{150}{\gram}
               \cdot
               \frac{\SI{18,02}{\cancel\gram\per\cancel\mol}}{\SI{2,02}{\cancel\gram\per\cancel\mol}}
               =
               \tcbhighmath[boxrule=0.4pt,arc=4pt,colframe=blue,drop fuzzy shadow=green]{\SI{1338}{\gram}}
           $$
           O aproximadamente\ldots
           $$
               \tcbhighmath[boxrule=0.4pt,arc=4pt,colframe=blue,drop fuzzy shadow=green]{m(\ce{H2O})}
               =
               \SI{150}{\gram}
               \cdot
               \cancelto{9}{\frac{\SI{18}{\cancel\gram\per\cancel\mol}}{\SI{2}{\cancel\gram\per\cancel\mol}}}
               =
               \tcbhighmath[boxrule=0.4pt,arc=4pt,colframe=blue,drop fuzzy shadow=green]{\SI{1350}{\gram}}
           $$
                  }
\end{frame}
