Considere la siguiente reacción que ocurre en una celda de combustible:
\begin{center}
	\ce{H2(g) + O2(g) -> H2O(g)}
\end{center}
Esta reacción, realizada adecuadamente, produce agua y energía en forma de electricidad. Suponga que una celda de combustible se instala con \SI{150}{\gram} de hidrógeno gaseoso y \SI{1500}{\gram} de oxígeno gaseoso.\\ 
¿Cuántos gramos de agua pueden formarse?
\resultadocmd{ \SI{1338}{\gram} }
