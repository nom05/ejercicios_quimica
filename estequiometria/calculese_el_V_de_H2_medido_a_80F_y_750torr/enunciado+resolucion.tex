\begin{frame}
    \frametitle{\ejerciciocmd}
    \framesubtitle{Enunciado}
    \textbf{
	Dadas las siguientes reacciones:
\begin{itemize}
    \item \ce{I2(g) + H2(g) -> 2 HI(g)}~~~$\Delta H_1 = \SI{-0,8}{\kilo\calorie}$
    \item \ce{I2(s) + H2(g) -> 2 HI(g)}~~~$\Delta H_2 = \SI{12}{\kilo\calorie}$
    \item \ce{I2(g) + H2(g) -> 2 HI(ac)}~~~$\Delta H_3 = \SI{-26,8}{\kilo\calorie}$
\end{itemize}
Calcular los parámetros que se indican a continuación:
\begin{description}%[label={\alph*)},font={\color{red!50!black}\bfseries}]
    \item[\texttt{a)}] Calor molar latente de sublimación del yodo.
    \item[\texttt{b)}] Calor molar de disolución del ácido yodhídrico.
    \item[\texttt{c)}] Número de calorías que hay que aportar para disociar en sus componentes el yoduro de hidrógeno gas contenido en un matraz de \SI{750}{\cubic\centi\meter} a \SI{25}{\celsius} y \SI{800}{\torr} de presión.
\end{description}
\resultadocmd{\SI{12,8}{\kilo\calorie}; \SI{-13,0}{\kilo\calorie}; \SI{12,9}{\calorie}}

	}
\end{frame}

\begin{frame}
    \frametitle{\ejerciciocmd}
    \framesubtitle{Datos del problema}
    \begin{center}
        {\large ¿$V(\ce{H2})$?}
    \end{center}
    \structure{Reacción:}
    \tcbhighmath[boxrule=0.4pt,arc=4pt,colframe=red,drop fuzzy shadow=blue]{\ce{CO}}
    \ce{ + }
    \tcbhighmath[boxrule=0.4pt,arc=4pt,colframe=green,drop fuzzy shadow=red]{\ce{H2}}
    \ce{->}
    \ce{CH3CH2CH3}
    \ce{+}
    \ce{H2O}
    $$
        \tcbhighmath[boxrule=0.4pt,arc=4pt,colframe=red,drop fuzzy shadow=blue]{V(\ce{CO}) = \SI{28,5}{\liter}}\quad
        \tcbhighmath[boxrule=0.4pt,arc=4pt,colframe=red,drop fuzzy shadow=blue]{T(\ce{CO}) = 0+\SI{273,15}{\kelvin}}\quad
        \tcbhighmath[boxrule=0.4pt,arc=4pt,colframe=red,drop fuzzy shadow=blue]{P(\ce{CO}) = \SI{1}{\atm}}
    $$
    $$
        \tcbhighmath[boxrule=0.4pt,arc=4pt,colframe=green,drop fuzzy shadow=red]{T(\ce{H2}) = \SI{80}{\degree{F}}}\quad
        \tcbhighmath[boxrule=0.4pt,arc=4pt,colframe=green,drop fuzzy shadow=red]{P(\ce{H2}) = \frac{750}{760}~\si{\atm}}
    $$
\end{frame}

\begin{frame}
    \frametitle{\ejerciciocmd}
    \framesubtitle{Resolución (\rom{1}): ajuste de la reacción y volumen de \ce{H2}}
    \begin{center}
        \begin{overprint}
            \onslide<1>
                \ce{CO + H2 -> CH3CH2CH3 + H2O}
            \onslide<2>
                \tcbhighmath[boxrule=0.4pt,arc=4pt,colframe=red,drop fuzzy shadow=blue]{\ce{C}}
                \ce{O + H2 -> }
                \tcbhighmath[boxrule=0.4pt,arc=4pt,colframe=red,drop fuzzy shadow=blue]{\ce{C}}
                \ce{H3}
                \tcbhighmath[boxrule=0.4pt,arc=4pt,colframe=red,drop fuzzy shadow=blue]{\ce{C}}
                \ce{H2}
                \tcbhighmath[boxrule=0.4pt,arc=4pt,colframe=red,drop fuzzy shadow=blue]{\ce{C}}
                \ce{H3 + H2O}
            \onslide<3>
                \tcbhighmath[boxrule=0.4pt,arc=4pt,colframe=red,drop fuzzy shadow=blue]{\ce{3C}}
                \ce{O + H2 -> }
                \tcbhighmath[boxrule=0.4pt,arc=4pt,colframe=red,drop fuzzy shadow=blue]{\ce{C}}
                \ce{H3}
                \tcbhighmath[boxrule=0.4pt,arc=4pt,colframe=red,drop fuzzy shadow=blue]{\ce{C}}
                \ce{H2}
                \tcbhighmath[boxrule=0.4pt,arc=4pt,colframe=red,drop fuzzy shadow=blue]{\ce{C}}
                \ce{H3 + H2O}
            \onslide<4>
                \tcbhighmath[boxrule=0.4pt,arc=4pt,colframe=green,drop fuzzy shadow=blue]{\ce{3}}
                \ce{C}
                \tcbhighmath[boxrule=0.4pt,arc=4pt,colframe=green,drop fuzzy shadow=blue]{\ce{O}}
                \ce{ + H2 -> CH3CH2CH3 + H2}
                \tcbhighmath[boxrule=0.4pt,arc=4pt,colframe=green,drop fuzzy shadow=blue]{\ce{O}}
            \onslide<5>
                \tcbhighmath[boxrule=0.4pt,arc=4pt,colframe=green,drop fuzzy shadow=blue]{\ce{3}}
                \ce{C}
                \tcbhighmath[boxrule=0.4pt,arc=4pt,colframe=green,drop fuzzy shadow=blue]{\ce{O}}
                \ce{ + H2 -> CH3CH2CH3 +} 
                \tcbhighmath[boxrule=0.4pt,arc=4pt,colframe=green,drop fuzzy shadow=blue]{\ce{3}}
                \ce{H2}
                \tcbhighmath[boxrule=0.4pt,arc=4pt,colframe=green,drop fuzzy shadow=blue]{\ce{O}}
            \onslide<6>
                \ce{3CO +}
                \tcbhighmath[boxrule=0.4pt,arc=4pt,colframe=orange,drop fuzzy shadow=green]{\ce{H2}}
                \ce{ -> C}
                \tcbhighmath[boxrule=0.4pt,arc=4pt,colframe=orange,drop fuzzy shadow=green]{\ce{H3}}
                \ce{C}
                \tcbhighmath[boxrule=0.4pt,arc=4pt,colframe=orange,drop fuzzy shadow=green]{\ce{H2}}
                \ce{C}
                \tcbhighmath[boxrule=0.4pt,arc=4pt,colframe=orange,drop fuzzy shadow=green]{\ce{H3}}
                \ce{ +} 
                \tcbhighmath[boxrule=0.4pt,arc=4pt,colframe=orange,drop fuzzy shadow=green]{\ce{3H2}}
                \ce{O}\\[.6cm]
                (2 hidrógenos vs. 14 hidrógenos)
            \onslide<7>
                \ce{3CO +}
                \tcbhighmath[boxrule=0.4pt,arc=4pt,colframe=orange,drop fuzzy shadow=green]{\ce{7H2}}
                \ce{ -> C}
                \tcbhighmath[boxrule=0.4pt,arc=4pt,colframe=orange,drop fuzzy shadow=green]{\ce{H3}}
                \ce{C}
                \tcbhighmath[boxrule=0.4pt,arc=4pt,colframe=orange,drop fuzzy shadow=green]{\ce{H2}}
                \ce{C}
                \tcbhighmath[boxrule=0.4pt,arc=4pt,colframe=orange,drop fuzzy shadow=green]{\ce{H3}}
                \ce{ +} 
                \tcbhighmath[boxrule=0.4pt,arc=4pt,colframe=orange,drop fuzzy shadow=green]{\ce{3H2}}
                \ce{O}
            \onslide<8->
                \structure{Reacción ajustada:} \ce{3CO +7H2 -> CH3CH2CH3 + 3H2O}
        \end{overprint}
    \end{center}
    \visible<8->{
        \structure{Según la estequiometría:} \SI{3}{\mol} de \ce{CO} \ce{->} \SI{7}{\mol} de \ce{H2}
        \begin{overprint}
            \onslide<9>
                $$
                    7\cdot n(\ce{CO}) = 3\cdot n(\ce{H2})
                $$
            \onslide<10>
                $$
                    n(\ce{H2}) = \frac{7}{3}\cdot n(\ce{CO})
                $$
            \onslide<11>
                $$
                    \overbrace{n(\ce{H2})}^{PV=nRT\Rightarrow n=\frac{PV}{RT}} = \frac{7}{3}\cdot \underbrace{n(\ce{CO})}_{n=\frac{PV}{RT}}
                $$
            \onslide<12>
                $$
                    \frac{P(\ce{H2})\cdot V(\ce{H2})}{\cancel{R}\cdot T(\ce{H2})} = \frac{7}{3}\cdot\frac{P(\ce{CO})\cdot V(\ce{CO})}{\cancel{R}\cdot T(\ce{CO})}
                $$
            \onslide<13->
                $$
                    V(\ce{H2}) = \frac{7}{3}\cdot\frac{P(\ce{CO})}{P(\ce{H2})}\cdot\frac{T(\ce{H2})}{T(\ce{CO})}\cdot V(\ce{CO})
                $$
        \end{overprint}
                }
    \visible<14->{
        \structure{Cambio de unidades de \si{\degree{F}} a \si{\kelvin}:}
        \begin{overprint}
            \onslide<14>
                $$
                    T(\si{\celsius}) = \frac{5}{9}\cdot\left[T(\si{\degree{F}}) - 32 \right]
                $$
            \onslide<15->
                $$
                    T(\si{\kelvin}) = T(\si{\celsius}) + 273,15 = \frac{5}{9}\cdot\left[T(\si{\degree{F}}) - 32\right] + 273,15
                $$
        \end{overprint}
                }
    \visible<16->{
        \structure{Sustituimos por valores:}
            \begin{overprint}
                \onslide<16>
                    $$
                        V(\ce{H2}) = \frac{7}{3}\cdot\frac{\frac{760}{\cancel{760}}\si{\cancel\atm}}{\frac{750}{\cancel{760}}\si{\cancel\atm}}\cdot\frac{\frac{5}{9}\cdot\left[80 - 32\right]+\SI{273,15}{\cancel\kelvin}}{\SI{273,15}{\cancel\kelvin}}\cdot \SI{28,5}{\liter}
                    $$
                \onslide<17->
                    $$
                        \tcbhighmath[boxrule=0.4pt,arc=4pt,colframe=green,drop fuzzy shadow=red]{V(\ce{H2}) = \SI{74,0}{\liter}}
                    $$
            \end{overprint}
            }
\end{frame}
