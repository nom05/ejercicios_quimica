\SI{32,5}{\gram} de sulfato de cobre (\rom{2}) pentahidratado, \ce{CuSO4*5H2O}, se hacen reaccionar con la cantidad de agua resultante de condensar el vapor de agua contenido en un recipiente de \SI{1200}{\liter} que se mantiene a \SI{85}{\celsius} y cuya presión en el interior es de \SI{732}{\torr}. Calcule la concentración molal (\si{\Molal}) de la disolución de sulfato formada.\resultadocmd{\SI{,18}{\molal}}