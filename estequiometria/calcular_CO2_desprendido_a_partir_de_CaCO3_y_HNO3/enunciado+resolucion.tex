\begin{frame}
    \frametitle{\ejerciciocmd}
    \framesubtitle{Enunciado}
    \textbf{
		Dadas las siguientes reacciones:
\begin{itemize}
    \item \ce{I2(g) + H2(g) -> 2 HI(g)}~~~$\Delta H_1 = \SI{-0,8}{\kilo\calorie}$
    \item \ce{I2(s) + H2(g) -> 2 HI(g)}~~~$\Delta H_2 = \SI{12}{\kilo\calorie}$
    \item \ce{I2(g) + H2(g) -> 2 HI(ac)}~~~$\Delta H_3 = \SI{-26,8}{\kilo\calorie}$
\end{itemize}
Calcular los parámetros que se indican a continuación:
\begin{description}%[label={\alph*)},font={\color{red!50!black}\bfseries}]
    \item[\texttt{a)}] Calor molar latente de sublimación del yodo.
    \item[\texttt{b)}] Calor molar de disolución del ácido yodhídrico.
    \item[\texttt{c)}] Número de calorías que hay que aportar para disociar en sus componentes el yoduro de hidrógeno gas contenido en un matraz de \SI{750}{\cubic\centi\meter} a \SI{25}{\celsius} y \SI{800}{\torr} de presión.
\end{description}
\resultadocmd{\SI{12,8}{\kilo\calorie}; \SI{-13,0}{\kilo\calorie}; \SI{12,9}{\calorie}}

	}
\end{frame}

\begin{frame}
    \frametitle{\ejerciciocmd}
    \framesubtitle{Datos del problema}
    {\huge $$
        m(\ce{CO2})?\quad
        V(\ce{HNO3}\text{ comercial})?
    $$}
    Reacción: \tcbhighmath[boxrule=0.4pt,arc=4pt,colframe=blue,drop fuzzy shadow=red]{\ce{CaCO3(s)}} + 
    \tcbhighmath[boxrule=0.4pt,arc=4pt,colframe=red,drop fuzzy shadow=yellow]{\ce{HNO3(ac)}}
     \ce{->[\SI{95}{\percent}]}
     \tcbhighmath[boxrule=0.4pt,arc=4pt,colframe=green,drop fuzzy shadow=blue]{\ce{CO2(g)}} + \ce{Ca(NO3)2(ac)} + \ce{H2O(l)}
    \visible<1-|handout:0>{$$
            \tcbhighmath[boxrule=0.4pt,arc=4pt,colframe=blue,drop fuzzy shadow=red]{m(\ce{CaCO3}) = \SI{30}{\gram}}
        $$
        $$
            \tcbhighmath[boxrule=0.4pt,arc=4pt,colframe=red,drop fuzzy shadow=yellow]{V(\ce{HNO3}) = \SI{60}{\milli\liter} = \SI{0,060}{\liter}}\quad
            \tcbhighmath[boxrule=0.4pt,arc=4pt,colframe=red,drop fuzzy shadow=yellow]{[\ce{HNO3}] = \SI{2,5}{\Molar}}
        $$
        $$
            Rto = \SI{95}{\percent}
        $$
        $$
            \tcbhighmath[boxrule=0.4pt,arc=4pt,colframe=red,drop fuzzy shadow=yellow]{\text{pureza de \ce{HNO3} comercial} = \SI{64,2}{\percent}~m/m}\quad
            \tcbhighmath[boxrule=0.4pt,arc=4pt,colframe=red,drop fuzzy shadow=yellow]{d(\text{\ce{HNO3} comercial}) = \SI{1,39}{g\per\milli\liter}}
        $$
    }
    \visible<2-|handout:0>{Pero también ...\\
        $$
            \tcbhighmath[boxrule=0.4pt,arc=4pt,colframe=blue,drop fuzzy shadow=red]{Mm(\ce{CaCO3}) = \SI{100,09}{\gram\per\mol}}\quad
            \tcbhighmath[boxrule=0.4pt,arc=4pt,colframe=red,drop fuzzy shadow=yellow]{Mm(\ce{HNO3}) = \SI{63,01}{\gram\per\mol}}
        $$
        $$
            \tcbhighmath[boxrule=0.4pt,arc=4pt,colframe=green,drop fuzzy shadow=blue]{Mm(\ce{CO2}) = \SI{44,01}{\gram\per\mol}}
        $$
    }
\end{frame}

\begin{frame}
    \frametitle{\ejerciciocmd}
    \framesubtitle{Ejemplo de la vida cotidiana en un laboratorio de Química}
            \centering
            \begin{tikzpicture}
                \begin{scope}[spy using outlines={rectangle,magnification=3.5,connect spies,size=3.4cm,height=1.8cm}]
                    \node[inner sep=0,outer sep=0,anchor=south west] (image) at (0,0) 
                    {\polaroid{ 6}{.094}{/home/nicux/Documentos/docencia/repositorio/ejercicios/quimica_general-github/estequiometria/calcular_CO2_desprendido_a_partir_de_CaCO3_y_HNO3/figs/nitrico_comercial}{Un paseo por el laboratorio\ldots}};
                    \spy[red!70!black] on (2.28,5.20) in node (zoom) at (-2.00,7.40);
                    \spy[blue!70!black] on (4.15,4.62) in node (zoom) at (-2.00,4.40);
                \end{scope}
        \end{tikzpicture}
\end{frame}

\begin{frame}
    \frametitle{\ejerciciocmd}
    \framesubtitle{Resolución (\rom{1}): Ajuste de la reacción}
    \structure{Reacción:} \textbf{\ce{CaCO3(s) + HNO3(ac) -> CO2(g) + Ca(NO3)2(ac) + H2O(l)}}
    \structure{Por tanteo} en general es el más rápido. Contabilizar átomos en reactivos y átomos en los productos y ajustar las cantidades en ambos lados.\\
    \visible<2-|handout:0>{
        \centering\ce{CaCO3(s) + H}\tcbhighmath[boxrule=0.4pt,arc=4pt,colframe=blue,drop fuzzy shadow=red]{\ce{NO3(ac)}} \ce{-> CO2(g) + Ca}\tcbhighmath[boxrule=0.4pt,arc=4pt,colframe=blue,drop fuzzy shadow=red]{\ce{(NO3)2(ac)}} \ce{+ H2O(l)}\\[.3cm]
    }
    \visible<3-|handout:0>{
        \centering\ce{CaCO3(s) + }\tcbhighmath[boxrule=0.4pt,arc=4pt,colframe=blue,drop fuzzy shadow=red]{2}\ce{H}{\ce{NO3(ac)}} \ce{-> CO2(g) + Ca}\ce{(NO3)2(ac)} \ce{+ H2O(l)}\\
    }
    \visible<4-|handout:0>{
        \begin{flushleft}
            \structure{Método aritmético} mediante coeficientes y agrupando por átomos:\\
        \end{flushleft}

    }
    \visible<5-|handout:0>{
        \centering\colorbox{red}{\color{white}\textbf{a}}\ce{CaCO3(s)} + \colorbox{green}{\textbf{b}}\ce{HNO3(ac)} \ce{->} \colorbox{blue}{\color{white}\textbf{c}}\ce{CO2(g)} + \colorbox{orange}{\color{white}\textbf{d}}\ce{Ca(NO3)2(ac)} + \colorbox{yellow}{\textbf{e}}\ce{H2O(l)}\\
    }
    \begin{columns}
            \visible<6-|handout:0>{
                \column{.45\textwidth}
                    \begin{flushleft}
                        \ce{Ca}:\quad\colorbox{red}{\color{white}\textbf{a}}=\colorbox{orange}{\color{white}\textbf{d}}\\
                        \ce{C}:\quad\colorbox{red}{\color{white}\textbf{a}}=\colorbox{blue}{\color{white}\textbf{c}}\\
                        \ce{O}:\quad\textbf{3}\colorbox{red}{\color{white}\textbf{a}} + \textbf{3}\colorbox{green}{\textbf{b}} = \textbf{2}\colorbox{blue}{\color{white}\textbf{c}} + \textbf{6}\colorbox{orange}{\color{white}\textbf{d}} + \colorbox{yellow}{\textbf{e}}\\
                        \ce{H}:\quad\colorbox{green}{\textbf{b}} = \textbf{2}\colorbox{yellow}{\textbf{e}}\\
                        \ce{N}:\quad\colorbox{green}{\textbf{b}} = \textbf{2}\colorbox{orange}{\color{white}\textbf{d}}
                    \end{flushleft}
            }
            \visible<7-|handout:0>{
                \column{.45\textwidth}
                        \underline{Escogemos} \colorbox{red}{\color{white}\textbf{a}} = \textbf{1}\\
                        Entonces:
                        \colorbox{red}{\color{white}\textbf{a}}=\colorbox{orange}{\color{white}\textbf{d}}=\colorbox{blue}{\color{white}\textbf{c}}=\textbf{1}\\
                        \colorbox{green}{\textbf{b}} = \textbf{2}\colorbox{orange}{\color{white}\textbf{d}} = \textbf{2}\\
                        \colorbox{green}{\textbf{b}} = \textbf{2}\colorbox{yellow}{\textbf{e}}$\Rightarrow$\colorbox{yellow}{\textbf{e}} = \colorbox{green}{\textbf{b}} / \textbf{2} = \textbf{1}
            }
    \end{columns}
    \visible<8-|handout:0>{
        \centering
        \colorbox{red}{\color{white}\textbf{1}}
        \colorbox{green}{\textbf{2}}
        \colorbox{blue}{\color{white}\textbf{1}}
        \colorbox{orange}{\color{white}\textbf{1}}
        \colorbox{yellow}{\textbf{1}}\\
    }
    \visible<9-|handout:0>{
        \colorbox{red}{\color{white}\textbf{1}}\ce{CaCO3(s)} + \colorbox{green}{\textbf{2}}\ce{HNO3(ac)} \ce{->} \colorbox{blue}{\color{white}\textbf{1}}\ce{CO2(g)} + \colorbox{orange}{\color{white}\textbf{1}}\ce{Ca(NO3)2(ac)} + \colorbox{yellow}{\textbf{1}}\ce{H2O(l)}
    }
\end{frame}

\begin{frame}
    \frametitle{\ejerciciocmd}
    \framesubtitle{Resolución (\rom{2}): volumen de ácido nítrico comercial}
    \structure{Resumen:} a partir de la definición de la concentración molar obtenemos el número de moles de disolución de \ce{HNO3} ($n(\ce{HNO3})$)\footnote{El número de moles se utiliza en el siguiente apartado}. Con estos moles, sabremos su masa a través de la relación entre la masa y masa molecular. Esa masa es del ácido puro que, por un factor de proporcionalidad empleando la pureza, nos va a permitir saber la masa del \ce{HNO3} comercial. Con la expresión de la densidad, obtenemos el volume.
    \visible<2->{
        \structure{Paso 1:} partimos de la definición de molaridad e despejamos $n(\ce{HNO3})$:
        \begin{equation}\label{eq:molaridad}
            M = \frac{n}{V}\Rightarrow n = M\cdot V
        \end{equation}
        $$
            n(\ce{HNO3})= [\ce{HNO3}]\cdot V(\ce{HNO3})
        $$
        $$
            n(\ce{HNO3})= \SI{2,5}{\mol\per\cancel\liter}\cdot\SI{0,060}{\cancel\liter} = \tcbhighmath[boxrule=0.4pt,arc=4pt,colframe=blue,drop fuzzy shadow=red]{\SI{0,15}{\mol}\text{ de \ce{HNO3}}}
        $$
    }
    \visible<3->{
        \structure{Paso 2:} introducimos la relación de $n$ con la masa ($m$) y la masa molecular ($Mm$) en la ecuación~\eqref{eq:molaridad}:
        \begin{equation}\label{eq:mol-Mm}
            n = \frac{m}{Mm}
        \end{equation}
        \begin{equation}\label{eq:m_puro_dis}
            \frac{m_{\text{puro}}(\ce{HNO3})}{Mm(\ce{HNO3})} = [\ce{HNO3}]\cdot V\Rightarrow \colorboxed{orange}{m_{\text{puro}}(\ce{HNO3})} = [\ce{HNO3}]\cdot V_{\text{puro}}(\ce{HNO3})\cdot Mm(\ce{HNO3})
        \end{equation}
    }
\end{frame}

\begin{frame}
    \frametitle{\ejerciciocmd}
    \framesubtitle{Resolución (\rom{2}): volumen de ácido nítrico comercial}
    \structure{Paso 3:} Esta $m$ corresponde a la sustancia pura ¿Cuál es la relación entre el \ce{HNO3} puro y el comercial?
    \begin{overprint}
        \onslide<1>
            $$
                m_{\text{puro}}(\ce{HNO3})
                \quad ??\quad 
                m_{\text{comercial}}(\ce{HNO3})
            $$
        \onslide<2>    
            $$
                m_{\text{puro}}(\ce{HNO3})
                \quad <\quad
                m_{\text{comercial}}(\ce{HNO3})
            $$
        \onslide<3->
            \begin{equation}\label<5->{eq:m_teo-com}
                \colorboxed{orange}{m_{\text{puro}}(\ce{HNO3})}
                \quad =\quad
                \colorboxed{red}{\frac{64,2}{100}}\Cline[blue]{m_{\text{comercial}}(\ce{HNO3})}
            \end{equation}
    \end{overprint}
    \visible<4->{
        \structure{Paso 4:} Y con la expresión de la densidad ($d$)
        \begin{equation}\label{eq:dens}
            d_{\text{comercial}}(\ce{HNO3}) = \frac{m_{\text{comercial}}(\ce{HNO3})}{V_{\text{comercial}}(\ce{HNO3})}\Rightarrow \Cline[blue]{m_{\text{comercial}}(\ce{HNO3})} = \Cline[green]{V_{\text{comercial}}(\ce{HNO3})\cdot d_{\text{comercial}}(\ce{HNO3})}
        \end{equation}
        }
    \visible<5->{
        Unimos las ecuaciones~\eqref{eq:m_teo-com} y \eqref{eq:dens}:
        \begin{equation}
            \colorboxed{orange}{m_{\text{puro}}(\ce{HNO3})}
            \quad =\quad
            \colorboxed{red}{\frac{64,2}{100}}\cdot\Cline[green]{V_{\text{comercial}}(\ce{HNO3})\cdot d_{\text{comercial}}(\ce{HNO3})}
        \end{equation}
        }
    \visible<6->{
        \structure{Paso 5:} Igualamos con la ecuación~\eqref{eq:m_puro_dis} que también hacía referencia a la $\colorboxed{orange}{m_{\text{puro}}}$:
        \begin{equation}
           \colorboxed{orange}{m_{\text{puro}}(\ce{HNO3})} = [\ce{HNO3}]\cdot V_{\text{puro}}(\ce{HNO3})\cdot Mm(\ce{HNO3}) 
            =
            \colorboxed{red}{\frac{64,2}{100}}\cdot\Cline[green]{V_{\text{comercial}}(\ce{HNO3})\cdot d_{\text{comercial}}(\ce{HNO3})}
        \end{equation}
    }
\end{frame}

\begin{frame}
    \frametitle{\ejerciciocmd}
    \framesubtitle{Resolución (\rom{2}): volumen de ácido nítrico comercial}
    \structure{Paso 6:} Despejamos $V_{\text{comercial}}$:
    \begin{equation}
        \Cline[green]{V_{\text{comercial}}(\ce{HNO3})} = \colorboxed{red}{\frac{100}{64,2}}\cdot\frac{[\ce{HNO3}]\cdot V_{\text{puro}}(\ce{HNO3})\cdot Mm(\ce{HNO3})}{\Cline[green]{d_{\text{comercial}}(\ce{HNO3})}} 
    \end{equation}
    \visible<2->{\structure{Paso 6:} Sustituimos valores:
    $$
        \tcbhighmath[boxrule=0.4pt,arc=4pt,colframe=blue,drop fuzzy shadow=red]{\Cline[green]{V_{\text{comercial}}}} = \colorboxed{red}{\frac{100}{64,2}}\cdot\frac{\SI{2,5}{\cancel\mol\per\cancel\liter}\cdot \SI{0,060}{\cancel\liter}\cdot\SI{63,01}{\cancel\gram\per\cancel\mol}}{\Cline[green]{\SI{1,394}{\cancel\gram\per\milli\liter}}} = \tcbhighmath[boxrule=0.4pt,arc=4pt,colframe=blue,drop fuzzy shadow=red]{\SI{10,6}{\milli\liter}}
    $$}
\end{frame}

\begin{frame}
    \frametitle{\ejerciciocmd}
    \framesubtitle{Resolución (\rom{3}): gramos de dióxido de carbono}
    \structure{Reacción:} \textbf{\ce{CaCO3(s) + 2HNO3(ac) ->[\SI{95}{\percent}] CO2(g) + Ca(NO3)2(ac) + H2O(l)}}
    \structure{Paso 1:} Número de moles de \ce{CaCO3} (Ecuación~\eqref{eq:mol-Mm}):
    $$
        n=\frac{m}{Mm}\Rightarrow n(\ce{CaCO3})=\frac{\SI{30}{\cancel\gram}}{\SI{100,09}{\cancel\gram\per\mol}}=\SI{0,30}{\mol}\text{ de \ce{CaCO3}}
    $$
    \visible<2->{\structure{Paso 2:} Encontrar el reactivo limitante.\\[.5cm]
    Según la estequiometría: $\quad\SI{1}{\mol}~\ce{CaCO3}\ce{->}\SI{2}{\mol}~\ce{HNO3}$\\
    Tenemos:\quad\quad\quad\quad\quad\quad\quad\quad $\quad\SI{0,30}{\mol}~\ce{CaCO3}\ce{->}\SI{0,15}{\mol}~\ce{HNO3}$}\\[.5cm]
    \visible<3->{Para hacer \underline{reaccionar \SI{0,30}{\mol} de \ce{CaCO3}} necesitaríamos \underline{\SI{0,60}{\mol} de \ce{HNO3}} (el doble), pero \underline{hay \SI{0,15}{\mol} de \ce{HNO3}}.}\\[.5cm]
    \visible<4->{\centering\myovalbox{\textcolor{white}{\ce{HNO3} es el reactivo limitante}}}
\end{frame}

\begin{frame}
    \frametitle{\ejerciciocmd}
    \framesubtitle{Resolución (\rom{3}): gramos de dióxido de carbono}
    \structure{Reacción:} \textbf{\ce{CaCO3(s) + 2HNO3(ac) ->[\SI{95}{\percent}] CO2(g) + Ca(NO3)2(ac) + H2O(l)}}
    \structure{Paso 3:} Según la estequiometría: $\quad\SI{2}{\mol}~\ce{HNO3}\ce{->}\SI{1}{\mol}~\ce{CO2}$\\[.3cm]
    \begin{overprint}
        \onslide<2>
            $$
                n(\ce{HNO3})
                \quad ??\quad 
                n(\ce{CO2})
            $$
        \onslide<3>    
            $$
                n(\ce{HNO3})
                \quad >\quad
                n(\ce{CO2})
            $$
        \onslide<4>
            $$
                n(\ce{HNO3})
                \quad =\quad
                2n(\ce{CO2})
            $$
        \onslide<5>
            Empleando una regla de tres:
            $$
                \left.
                \begin{aligned}
                    \quad\SI{2}{\mol}~\ce{HNO3}\ce{->}\SI{1}{\mol}~\ce{CO2}\\
                    \quad n(\ce{HNO3})\quad\ce{->}x\qquad\qquad
                \end{aligned}
                \right\}
                \qquad x = \frac{n(\ce{HNO3})\cdot\SI{1}{\mol}~\ce{CO2}}{\SI{2}{\mol}~\ce{HNO3}}
            $$
        \onslide<6->
            Empleando una regla de tres:
            $$
                \left.
                \begin{aligned}
                    \quad\SI{2}{\mol}~\ce{HNO3}\ce{->}\SI{1}{\mol}~\ce{CO2}\\
                    \quad n(\ce{HNO3})\quad\ce{->}x\qquad\qquad
                \end{aligned}
                \right\}
                \qquad x = \frac{n(\ce{HNO3})\cdot\SI{1}{\mol}~\ce{CO2}}{\SI{2}{\mol}~\ce{HNO3}}
            $$
            Pero $x$ é $n(\ce{CO2})$ por lo que llegamos a la misma expresión:
            \begin{equation}\label<6->{eq:n_HNO3-n_CO2}
                \qquad x = n(\ce{CO2})\Rightarrow n(\ce{HNO3}) = 2n(\ce{CO2})            
            \end{equation}
    \end{overprint}
    \visible<7->{
        \structure{Paso 4:} Rendimiento de la reacción - Experimentalmente se encontró que no todos los moles de \ce{HNO3} reaccionan para dar \ce{CO2} (\SI{95}{\percent} de rendimiento).
        \begin{overprint}
            \onslide<7>
                $$
                    n_{\text{teo}}(\ce{CO2})
                    \quad ??\quad 
                    n_{\text{real}}(\ce{CO2})
                $$
            \onslide<8>    
                $$
                    n_{\text{teo}}(\ce{CO2})
                    \quad >\quad 
                    n_{\text{real}}(\ce{CO2})
                $$
            \onslide<9->
                $$
                    \frac{95}{100}
                    n_{\text{teo}}(\ce{CO2})
                    \quad =\quad
                    n_{\text{real}}(\ce{CO2})
                    \Rightarrow
                    n_{\text{teo}}(\ce{CO2})
                    =
                    \frac{100}{95}
                    n_{\text{real}}(\ce{CO2})
                $$
        \end{overprint}
            }
\end{frame}

\begin{frame}
    \frametitle{\ejerciciocmd}
    \framesubtitle{Resolución (\rom{3}): gramos de dióxido de carbono}
    \structure{Paso 5:} Introducimos el número de moles reales en la Ecuación~\eqref{eq:n_HNO3-n_CO2}
    \begin{overprint}
        \onslide<1>
            $$
                n(\ce{HNO3}) = 2\cdot\overbrace{n_{\text{teo}}(\ce{CO2})}^{n_{\text{teo}}(\ce{CO2})=\frac{100}{95}n_{\text{real}}(\ce{CO2})}
            $$
        \onslide<2>
            $$
                n(\ce{HNO3}) = 2\cdot\frac{100}{95}n_{\text{real}}(\ce{CO2})
            $$
        \onslide<3->
            $$
                n(\ce{HNO3}) = \frac{\cancelto{200}{2\cdot 100}}{95}n_{\text{real}}(\ce{CO2})
            $$
    \end{overprint}
    \visible<4->{
        \structure{Paso 6:} Obtenemos $m(\ce{CO2})$ y despejamos:
        \begin{overprint}
            \onslide<4>
                $$
                    \overbrace{n(\ce{HNO3})}^{[\ce{HNO3}]\cdot V(\ce{HNO3})}
                    \quad =\quad
                    \frac{200}{95}\underbrace{n_{\text{real}}(\ce{CO2})}_{\frac{m_{\text{real}}(\ce{CO2})}{Mm(\ce{CO2})}}
                $$
            \onslide<5>
                $$
                    [\ce{HNO3}]\cdot V(\ce{HNO3})
                    \quad =\quad
                    \frac{200}{95}\cdot\frac{m_{\text{real}}(\ce{CO2})}{Mm(\ce{CO2})}
                $$
            \onslide<6>
                $$
                    [\ce{HNO3}]\cdot V(\ce{HNO3})\cdot Mm(\ce{CO2})
                    \quad =\quad
                    \frac{200}{95}\cdot m_{\text{real}}(\ce{CO2})
                $$
            \onslide<7->
                $$
                    \frac{95}{200}\cdot [\ce{HNO3}]\cdot V(\ce{HNO3})\cdot Mm(\ce{CO2})
                    \quad =\quad
                    m_{\text{real}}(\ce{CO2})
                $$
        \end{overprint}
                }
        \visible<8->{
            \structure{Paso 7:} Sustituimos y resolvemos:
                $$
                    m(\ce{CO2})
                    \quad =\quad
                    \frac{95}{200}\cdot\SI{2,5}{\cancel\mol\per\cancel\liter}\cdot\SI{0,060}{\cancel\liter}\cdot\SI{44,01}{\gram\per\cancel\mol}
                $$
                $$
                    \tcbhighmath[boxrule=0.4pt,arc=4pt,colframe=blue,drop fuzzy shadow=red]{m(\ce{CO2})=\SI{3}{\gram}}
                $$
                    }
\end{frame}
