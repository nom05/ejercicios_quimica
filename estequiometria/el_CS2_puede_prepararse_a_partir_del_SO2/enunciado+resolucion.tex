\begin{frame}
	\frametitle{\ejerciciocmd}
	\framesubtitle{Enunciado}
	\textbf{
		Una reacción tiene una constante de velocidad de \SI{,017}{\per\second} a \SI{298}{\kelvin} y una energía libre de activación del \SI{27,235}{\kilo\joule\per\mol}. La adición de un catalizador disminuye dicha energía de activación hasta un \SI{33}{\percent} de su valor inicial. Calcule la nueva constante de velocidad.
\resultadocmd{ \SI{26,86}{\per\second} }

		}
\end{frame}

\begin{frame}
	\frametitle{\ejerciciocmd}
	\framesubtitle{Datos del problema}
	\centering{\huge m(\ce{CS2})?}
	\begin{center}
		\tcbhighmath[boxrule=0.4pt,arc=4pt,colframe=green,drop fuzzy shadow=blue]{m(\ce{SO2})=\SI{450}{\kilogram}}
		\tcbhighmath[boxrule=0.4pt,arc=4pt,colframe=orange,drop fuzzy shadow=green]{\si{\percent}\text{ rendimiento de reacción}=\SI{82}{\percent}}
	\end{center}
	\visible<2-|handout:0>{Pero también\ldots\\
		$$
			\tcbhighmath[boxrule=0.4pt,arc=4pt,colframe=green,drop fuzzy shadow=blue]{Mm(\ce{SO2}) = \SI{64,06}{\gram\per\mol}}
			\tcbhighmath[boxrule=0.4pt,arc=4pt,colframe=red,drop fuzzy shadow=green]{Mm(\ce{CS2}) = \SI{76,14}{\gram\per\mol}}
		$$
		Reacción sin ajustar: 
		\ce{C(s) +}
		\tcbhighmath[boxrule=0.4pt,arc=4pt,colframe=green,drop fuzzy shadow=blue]{\ce{SO2}}
		\ce{->[\SI{82}{\percent}]}
		\tcbhighmath[boxrule=0.4pt,arc=4pt,colframe=red,drop fuzzy shadow=green]{\ce{CS2(s)}}
		\ce{ + CO(g)}
	}
\end{frame}

\begin{frame}
	\frametitle{\ejerciciocmd}
	\framesubtitle{Resolución (\rom{1}): ajuste de la reacción}
	\begin{block}{Método del tanteo}
		\begin{overprint}
			\onslide<1>
				\centering\ce{C +}
				\tcbhighmath[boxrule=0.4pt,arc=4pt,colframe=green,drop fuzzy shadow=blue]{\ce{\textbf{S}O2}}
				\ce{->}
				\tcbhighmath[boxrule=0.4pt,arc=4pt,colframe=green,drop fuzzy shadow=blue]{\ce{C\textbf{S2}}}
				\ce{+ CO} (\textbf{\underline{sin ajustar}})
			\onslide<2>
				\centering\ce{C +}
				\tcbhighmath[boxrule=0.4pt,arc=4pt,colframe=green,drop fuzzy shadow=blue]{\ce{\textbf{2S}O2}}
				\ce{->}
				\tcbhighmath[boxrule=0.4pt,arc=4pt,colframe=green,drop fuzzy shadow=blue]{\ce{C\textbf{S2}}}
				\ce{+ CO} (\textbf{\underline{ajustado}})
			\onslide<3>
				\centering\ce{C + }
				\tcbhighmath[boxrule=0.4pt,arc=4pt,colframe=red,drop fuzzy shadow=green]{\ce{\textbf{2}S\textbf{O2}}}
				\ce{->}
				\ce{CS2 +}
				\tcbhighmath[boxrule=0.4pt,arc=4pt,colframe=red,drop fuzzy shadow=green]{\ce{C\textbf{O}}} (\textbf{\underline{sin ajustar}})
			\onslide<4>
				\centering\ce{C + }
				\tcbhighmath[boxrule=0.4pt,arc=4pt,colframe=red,drop fuzzy shadow=green]{\ce{\textbf{2}S\textbf{O2}}}
				\ce{->}
				\ce{CS2 +}
				\tcbhighmath[boxrule=0.4pt,arc=4pt,colframe=red,drop fuzzy shadow=green]{\ce{\textbf{4}C\textbf{O}}} (\textbf{\underline{ajustado}})
			\onslide<5>
				\centering\tcbhighmath[boxrule=0.4pt,arc=4pt,colframe=orange,drop fuzzy shadow=blue]{\textbf{\ce{C}}}
				\ce{ + 2SO2 ->}
				\tcbhighmath[boxrule=0.4pt,arc=4pt,colframe=orange,drop fuzzy shadow=blue]{\ce{\textbf{C}S2}}
				\ce{+}
				\tcbhighmath[boxrule=0.4pt,arc=4pt,colframe=orange,drop fuzzy shadow=blue]{\ce{\textbf{4C}O}} (\textbf{\underline{sin ajustar}})
			\onslide<6>
				\centering\tcbhighmath[boxrule=0.4pt,arc=4pt,colframe=orange,drop fuzzy shadow=blue]{\textbf{\ce{5C}}}
				\ce{ + 2SO2 ->}
				\tcbhighmath[boxrule=0.4pt,arc=4pt,colframe=orange,drop fuzzy shadow=blue]{\ce{\textbf{C}S2}}
				\ce{+}
				\tcbhighmath[boxrule=0.4pt,arc=4pt,colframe=orange,drop fuzzy shadow=blue]{\ce{\textbf{4C}O}} (\textbf{\underline{ajustado}})
			\onslide<7->
				\centering\textbf{\ce{5C + 2SO2 -> CS2 + 4CO}}
		\end{overprint}
	\end{block}
	\visible<7->{
	\begin{alertblock}{Método aritmético:}
		\centering\colorbox{red}{\color{white}\textbf{a}}\ce{C} + \colorbox{green}{\textbf{b}}\ce{SO2} \ce{->} \colorbox{blue}{\color{white}\textbf{c}}\ce{CS2} +  \colorbox{orange}{\color{white}\textbf{d}}\ce{CO}\\
		\begin{columns}
			\visible<8-|handout:0>{
				\column{.45\textwidth}
				\begin{flushleft}
					\ce{C}:\quad\colorbox{red}{\color{white}\textbf{a}} = \colorbox{blue}{\color{white}\textbf{c}} + \colorbox{orange}{\color{white}\textbf{d}}\\
					\ce{S}:\quad\colorbox{green}{\textbf{b}} = \textbf{2}\colorbox{blue}{\color{white}\textbf{c}}\\
					\ce{O}:\quad \textbf{2}\colorbox{green}{\textbf{b}} = \colorbox{orange}{\color{white}\textbf{d}}\\
				\end{flushleft}
								}
			\visible<9-|handout:0>{
				\column{.45\textwidth}
				$a$ depende de $c$ y $d$.\\
				$c$ y $d$ dependen de $b$.\\
				Escogemos $b$ para que $c$ sea entero:\\
				\colorbox{green}{\textbf{b}} = 2\\
				Entonces:
				\colorbox{blue}{\color{white}\textbf{c}}$=\rfrac{2}{2}=1$\\
				\colorbox{orange}{\color{white}\textbf{d}}$=2\times 2=4$\\
				\colorbox{red}{\color{white}\textbf{a}}$= 1+4=5$\\
			}
		\end{columns}
		\visible<10-|handout:0>{
			\centering
			\colorbox{red}{\color{white}\textbf{5}}
			\colorbox{green}{\color{white}\textbf{2}}
			\colorbox{blue}{\color{white}\textbf{1}}
			\colorbox{orange}{\color{white}\textbf{4}}
			\\[.5cm]
								}
		\visible<10-|handout:0>{
			\centering\colorbox{red}{\color{white}\textbf{5}}\ce{C} + \colorbox{green}{\textbf{2}}\ce{SO2} \ce{->} \colorbox{blue}{\color{white}\textbf{1}}\ce{CS2} +  \colorbox{orange}{\color{white}\textbf{4}}\ce{CO}\\
								}
	\end{alertblock}
	}
\end{frame}

\begin{frame}
	\frametitle{\ejerciciocmd}
	\framesubtitle{Resolución (\rom{2}): masa real de \ce{CS2}}
	\structure{Reacción:} \ce{5C + 2SO2 ->[\SI{82}{\percent}] CS2 + 4CO}
	\structure{Según la estequiometría:} \SI{2}{\mol} de \ce{SO2} reaccionan con \SI{1}{\mol} de \ce{CS2}, entonces:
	\begin{overprint}
		\onslide<1>
			$$
				n(\ce{SO2}) = 2\cdot n_{\text{teórico}}(\ce{CS2})
			$$
			Como la reacción tiene un rendimiento menor de \SI{100}{\percent} (concretamente \SI{82}{\percent})
			$$
				n_{\text{teórico}}(\ce{CS2}) > n_{\text{real}}(\ce{CS2})
			$$
		\onslide<2>
			$$
				n(\ce{SO2}) = 2\cdot n_{\text{teórico}}(\ce{CS2})
			$$
			Como la reacción tiene un rendimiento menor de \SI{100}{\percent} (concretamente \SI{82}{\percent})
			$$
				\frac{82}{100}\cdot n_{\text{teórico}}(\ce{CS2}) = n_{\text{real}}(\ce{CS2})\Rightarrow	n_{\text{teórico}}(\ce{CS2}) = \frac{100}{82}\cdot n_{\text{real}}(\ce{CS2})
			$$
		\onslide<3>
			$$
				n(\ce{SO2}) = 2\cdot\overbrace{n_{\text{teórico}}(\ce{CS2})}^{n_{\text{teórico}}(\ce{CS2}) = \frac{100}{82}\cdot n_{\text{real}}(\ce{CS2})}
			$$
		\onslide<4>
			$$
				\overbrace{n(\ce{SO2})}^{n=\rfrac{m}{Mm}} = 2\cdot\frac{100}{82}\cdot\underbrace{n_{\text{real}}(\ce{CS2})}_{n=\rfrac{m}{Mm}}
			$$
		\onslide<5>
			$$
				\frac{m(\ce{SO2})}{Mm(\ce{SO2})} = \overbrace{2\cdot\frac{100}{82}}^{\frac{100}{41}}\cdot\frac{m_{\text{real}}(\ce{CS2})}{Mm(\ce{CS2})}
			$$
		\onslide<6->
			$$
				m_{\text{real}}(\ce{CS2}) = m(\ce{SO2})\cdot\frac{41}{100}\cdot\frac{Mm(\ce{CS2})}{Mm(\ce{SO2})}
			$$
	\end{overprint}
	\visible<6>{
		Sustituimos por sus correspondientes valores:
		$$
				m_{\text{real}}(\ce{CS2}) = \SI{450}{\kilogram}\cdot\frac{41}{100}\cdot\frac{\SI{76,14}{\cancel\gram\per\cancel\mol}}{\SI{64,06}{\cancel\gram\per\cancel\mol}}
		$$
		\centering\tcbhighmath[boxrule=0.4pt,arc=4pt,colframe=red,drop fuzzy shadow=green]{m_{\text{real}}(\ce{CS2}) = \SI{219}{\kilogram}}
				}
\end{frame}
