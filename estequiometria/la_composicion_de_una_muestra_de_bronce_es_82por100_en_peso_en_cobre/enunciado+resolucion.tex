\begin{frame}
    \frametitle{\ejerciciocmd}
    \framesubtitle{Enunciado}
    \textbf{
		Dadas las siguientes reacciones:
\begin{itemize}
    \item \ce{I2(g) + H2(g) -> 2 HI(g)}~~~$\Delta H_1 = \SI{-0,8}{\kilo\calorie}$
    \item \ce{I2(s) + H2(g) -> 2 HI(g)}~~~$\Delta H_2 = \SI{12}{\kilo\calorie}$
    \item \ce{I2(g) + H2(g) -> 2 HI(ac)}~~~$\Delta H_3 = \SI{-26,8}{\kilo\calorie}$
\end{itemize}
Calcular los parámetros que se indican a continuación:
\begin{description}%[label={\alph*)},font={\color{red!50!black}\bfseries}]
    \item[\texttt{a)}] Calor molar latente de sublimación del yodo.
    \item[\texttt{b)}] Calor molar de disolución del ácido yodhídrico.
    \item[\texttt{c)}] Número de calorías que hay que aportar para disociar en sus componentes el yoduro de hidrógeno gas contenido en un matraz de \SI{750}{\cubic\centi\meter} a \SI{25}{\celsius} y \SI{800}{\torr} de presión.
\end{description}
\resultadocmd{\SI{12,8}{\kilo\calorie}; \SI{-13,0}{\kilo\calorie}; \SI{12,9}{\calorie}}

	}
\end{frame}

\begin{frame}
    \frametitle{\ejerciciocmd}
    \framesubtitle{Datos del problema}
    {\huge
        $$
            m(\ce{CuSO4})?
        $$
    }
    Relación entre la aleación y el sulfato de cobre:\\[.3cm]
    $$
        \tcbhighmath[boxrule=0.4pt,arc=4pt,colframe=blue,drop fuzzy shadow=green]{\ce{Cu}}
        \ce{ -> }
        \tcbhighmath[boxrule=0.4pt,arc=4pt,colframe=green,drop fuzzy shadow=orange]{\ce{CuSO4}}
    $$\\[.5 cm]
    $$
        \SI{1}{\mol}\text{ de }\ce{Cu ->}\SI{1}{\mol}\text{ de }\ce{CuSO4}
    $$
    $$
        \tcbhighmath[boxrule=0.4pt,arc=4pt,colframe=blue,drop fuzzy shadow=green]{n(\ce{Cu})}
        =
        \tcbhighmath[boxrule=0.4pt,arc=4pt,colframe=green,drop fuzzy shadow=orange]{n(\ce{CuSO4})}
    $$
    \tcbhighmath[boxrule=0.4pt,arc=4pt,colframe=blue,drop fuzzy shadow=green]{\%\text{pureza}(\ce{Cu})=\SI{82}{\percent}}
    \tcbhighmath[boxrule=0.4pt,arc=4pt,colframe=blue,drop fuzzy shadow=green]{d_{\text{bronce}}(\ce{Cu})=\SI{8,7}{\gram\per\milli\liter}}
    \tcbhighmath[boxrule=0.4pt,arc=4pt,colframe=blue,drop fuzzy shadow=green]{V_{\text{bronce}}(\ce{Cu})=\SI{62}{\milli\liter}}\\[.5cm]
    \visible<2-|handout:0>{
        y las masas moleculares o molares:\\
        $$
            \tcbhighmath[boxrule=0.4pt,arc=4pt,colframe=blue,drop fuzzy shadow=green]{Mat(\ce{Cu}) = \SI{63,55}{\gram\per\mol}}\quad
            \tcbhighmath[boxrule=0.4pt,arc=4pt,colframe=green,drop fuzzy shadow=orange]{Mm(\ce{PCl3})=\SI{159,61}{\gram\per\mol}}
        $$
    }
\end{frame}

\begin{frame}
    \frametitle{\ejerciciocmd}
    \framesubtitle{Resolución (\rom{1}): masa del sulfato de cobre}
    \structure{Paso 1:} Relación entre la masa de aleación y el cobre que contiene.
    \begin{overprint}
        \onslide<1>
            $$
                m_{\text{bronce}}(\ce{Cu})
                \quad ??\quad 
                m(\ce{Cu})
            $$
        \onslide<2>    
            $$
                m_{\text{bronce}}(\ce{Cu})
                \quad >\quad 
                m(\ce{Cu})
            $$
        \onslide<3->
            $$
                \frac{82}{100}m_{\text{bronce}}(\ce{Cu})
                \quad =\quad 
                m(\ce{Cu})
            $$
    \end{overprint}
    \visible<4->{
        \structure{Paso 2:} Utilizar la relación molar entre \ce{Cu} y \ce{CuSO4} para obtener la expresión del sulfato en función del bronce
               }
     \begin{overprint}
         \onslide<4>
              $$
                 n(\ce{Cu})
                 =
                 n(\ce{CuSO4})
             $$
         \onslide<5>
              $$
                \overbrace{n(\ce{Cu})}^{n=\frac{m}{Mm}}
                =
                \underbrace{n(\ce{CuSO4})}_{n=\frac{m}{Mm}}
              $$
         \onslide<6->
              $$
                  \frac{m(\ce{Cu})}{Mm(\ce{Cu})}
                  =
                  \frac{m(\ce{CuSO4})}{Mm(\ce{CuSO4})}
              $$
     \end{overprint}
     \visible<7->{
         \structure{Paso 3:} Unimos las expresiones de los pasos anteriores, añadimos la densidad y el volumen del bronce y despejamos
                 }
      \begin{overprint}
         \onslide<7>
             $$
                 \frac{\overbrace{m(\ce{Cu})}^{m(\ce{Cu})=\frac{82}{100}m_{\text{bronce}}(\ce{Cu})}}{Mm(\ce{Cu})}
                 =
                 \frac{m(\ce{CuSO4})}{Mm(\ce{CuSO4})}
             $$
         \onslide<8>
             $$
                 \frac{82}{100}
                 \frac{m_{\text{bronce}}(\ce{Cu})}{Mm(\ce{Cu})}
                 =
                 \frac{m(\ce{CuSO4})}{Mm(\ce{CuSO4})}
             $$
         \onslide<9>
             $$
                 \frac{82}{100}
                 \frac{\overbrace{m_{\text{bronce}}(\ce{Cu})}^{d=\frac{m}{V}\Rightarrow m=d\cdot V}}{Mm(\ce{Cu})}
                 =
                 \frac{m(\ce{CuSO4})}{Mm(\ce{CuSO4})}
             $$
         \onslide<10>
             $$
                 \frac{82}{100}
                 \frac{d_{\text{bronce}}(\ce{Cu})\cdot V_{\text{bronce}}(\ce{Cu})}{Mm(\ce{Cu})}
                 =
                 \frac{m(\ce{CuSO4})}{Mm(\ce{CuSO4})}
             $$
          \onslide<11->
             $$
                 m(\ce{CuSO4})
                 =
                 \frac{82}{100}\cdot
                 d_{\text{bronce}}(\ce{Cu})\cdot V_{\text{bronce}}(\ce{Cu})\cdot
                 \frac{Mm(\ce{CuSO4})}{Mm(\ce{Cu})}
             $$
     \end{overprint}
     \visible<12->{
         \structure{Paso 4:} Sustituimos
            $$
                \tcbhighmath[boxrule=0.4pt,arc=4pt,colframe=blue,drop fuzzy shadow=green]{m(\ce{CuSO4})}
                =
                \frac{82}{100}\cdot
                \SI{8,7}{\gram\per\cancel\milli\liter}\cdot \SI{62}{\cancel\milli\liter}\cdot
                \frac{\SI{159,61}{\cancel\gram\per\cancel\mol}}{\SI{63,55}{\cancel\gram\per\cancel\mol}}
                = \tcbhighmath[boxrule=0.4pt,arc=4pt,colframe=blue,drop fuzzy shadow=green]{\SI{1111}{\gram}}
            $$
                  }
\end{frame}
