\begin{frame}
    \frametitle{\ejerciciocmd}
    \framesubtitle{Enunciado}
    \textbf{
		Dadas las siguientes reacciones:
\begin{itemize}
    \item \ce{I2(g) + H2(g) -> 2 HI(g)}~~~$\Delta H_1 = \SI{-0,8}{\kilo\calorie}$
    \item \ce{I2(s) + H2(g) -> 2 HI(g)}~~~$\Delta H_2 = \SI{12}{\kilo\calorie}$
    \item \ce{I2(g) + H2(g) -> 2 HI(ac)}~~~$\Delta H_3 = \SI{-26,8}{\kilo\calorie}$
\end{itemize}
Calcular los parámetros que se indican a continuación:
\begin{description}%[label={\alph*)},font={\color{red!50!black}\bfseries}]
    \item[\texttt{a)}] Calor molar latente de sublimación del yodo.
    \item[\texttt{b)}] Calor molar de disolución del ácido yodhídrico.
    \item[\texttt{c)}] Número de calorías que hay que aportar para disociar en sus componentes el yoduro de hidrógeno gas contenido en un matraz de \SI{750}{\cubic\centi\meter} a \SI{25}{\celsius} y \SI{800}{\torr} de presión.
\end{description}
\resultadocmd{\SI{12,8}{\kilo\calorie}; \SI{-13,0}{\kilo\calorie}; \SI{12,9}{\calorie}}

	}
\end{frame}

\begin{frame}
    \frametitle{\ejerciciocmd}
    \framesubtitle{Datos del problema}
    {\huge $$
        ¿m_{\text{impuro}}(\ce{KClO3})?
    $$}
    Reacción: 
    \tcbhighmath[boxrule=0.4pt,arc=4pt,colframe=blue,drop fuzzy shadow=red]{\ce{KClO3}(s)}
    \ce{->}
    \ce{KCl(s)}
    \ce{+}
    \tcbhighmath[boxrule=0.4pt,arc=4pt,colframe=blue,drop fuzzy shadow=green]{\ce{O2(g)}}
        $$
            \tcbhighmath[boxrule=0.4pt,arc=4pt,colframe=blue,drop fuzzy shadow=red]{\text{pureza de \ce{KClO3}} = \SI{90}{\percent}}
        $$
        $$
            \tcbhighmath[boxrule=0.4pt,arc=4pt,colframe=blue,drop fuzzy shadow=green]{T(\ce{O2}) = \SI{298,15}{\kelvin}}\quad
            \tcbhighmath[boxrule=0.4pt,arc=4pt,colframe=blue,drop fuzzy shadow=green]{P(\ce{O2}) = 745-24~\si{\torr}}\quad
            \tcbhighmath[boxrule=0.4pt,arc=4pt,colframe=blue,drop fuzzy shadow=green]{V(\ce{O2}) = \SI{10}{\liter}}
        $$\\[.6cm]
    Y la masa molecular\ldots
        $$
            \tcbhighmath[boxrule=0.4pt,arc=4pt,colframe=blue,drop fuzzy shadow=red]{Mm(\ce{KClO3})=\SI{122,55}{\gram\per\mol}}
        $$
\end{frame}

\begin{frame}
    \frametitle{\ejerciciocmd}
    \framesubtitle{Resolución (\rom{1}): Ajuste de la reacción}
    \structure{Reacción:}
    \visible<1-|handout:0>{
        \centering\ce{KCl}\tcbhighmath[boxrule=0.4pt,arc=4pt,colframe=blue,drop fuzzy shadow=red]{\ce{O3}} \ce{-> KCl +}\tcbhighmath[boxrule=0.4pt,arc=4pt,colframe=blue,drop fuzzy shadow=red]{\ce{O2}}\\[.3cm]
                          }
    \visible<2-|handout:0>{
        \centering\ce{KCl}\tcbhighmath[boxrule=0.4pt,arc=4pt,colframe=blue,drop fuzzy shadow=red]{\ce{O3}} \ce{-> KCl +}\tcbhighmath[boxrule=0.4pt,arc=4pt,colframe=blue,drop fuzzy shadow=red]{\ce{\frac{3}{2}O2}}\\[.3cm]
                          }
    \visible<2-|handout:0>{
        \centering\ce{2KCl}\tcbhighmath[boxrule=0.4pt,arc=4pt,colframe=blue,drop fuzzy shadow=red]{\ce{O3}} \ce{-> 2KCl +}\tcbhighmath[boxrule=0.4pt,arc=4pt,colframe=blue,drop fuzzy shadow=red]{\ce{3O2}}\\[.3cm]
                          }
\end{frame}

\begin{frame}
    \frametitle{\ejerciciocmd}
    \framesubtitle{Resolución (\rom{2}): masa de \ce{KClO3} impuro}
    \structure{Reacción:} \ce{2KClO3 -> 2KCl +3O2}
    \structure{Según la estequiometría:} \SI{2}{\mol} de \ce{KClO3 ->} \SI{3}{\mol} de \ce{O2}. \visible<2->{Por tanto:
    \begin{overprint}
        \onslide<2>
            $$
                3n(\ce{KClO3}) = 2n(\ce{O2})
            $$
        \onslide<3>
            $$
                3\overbrace{n(\ce{KClO3})}^{n=\frac{m}{Mm}} = 2\underbrace{n(\ce{O2})}_{PV=nRT\Rightarrow n=\frac{PV}{RT}}
            $$
        \onslide<4>
            $$
                3\cdot\frac{m_{\text{puro}}(\ce{KClO3})}{Mm(\ce{KClO3})} = 2\cdot\frac{P(\ce{O2})\cdot V(\ce{O2})}{R\cdot T(\ce{O2})}
            $$
        \onslide<5->
            $$
                m_{\text{puro}}(\ce{KClO3}) = \frac{2}{3}\cdot Mm(\ce{KClO3})\cdot\frac{P(\ce{O2})\cdot V(\ce{O2})}{R\cdot T(\ce{O2})}
            $$
    \end{overprint}
    }
    \visible<6->{
        \structure{Masa de \ce{KClO3} impura:}
        \begin{overprint}
            \onslide<6>
                $$
                    m_{\text{puro}}(\ce{KClO3}) < m_{\text{impuro}}(\ce{KClO3})
                $$
            \onslide<7->
                $$
                    m_{\text{puro}}(\ce{KClO3}) = \frac{9\cancel{0}}{10\cancel{0}} m_{\text{impuro}}(\ce{KClO3})
                $$
        \end{overprint}
                }
    \visible<8->{
        \structure{Igualando las dos expresiones de $m_{\text{puro}}(\ce{KClO3})$:}
        \begin{overprint}
            \onslide<8>
                $$
                    m_{\text{puro}}(\ce{KClO3}) = m_{\text{puro}}(\ce{KClO3})
                $$
            \onslide<9>
                $$
                    \overbrace{m_{\text{puro}}(\ce{KClO3})}^{m_{\text{puro}}(\ce{KClO3}) = \frac{9}{10} m_{\text{impuro}}(\ce{KClO3})}
                    =
                    \underbrace{m_{\text{puro}}(\ce{KClO3})}_{m_{\text{puro}}(\ce{KClO3}) = \frac{2}{3}\cdot Mm(\ce{KClO3})\cdot\frac{P(\ce{O2})\cdot V(\ce{O2})}{R\cdot T(\ce{O2})}}
                $$
            \onslide<10>
                $$
                    \frac{9}{10} m_{\text{impuro}}(\ce{KClO3})
                    =
                    \frac{2}{3}\cdot Mm(\ce{KClO3})\cdot\frac{P(\ce{O2})\cdot V(\ce{O2})}{R\cdot T(\ce{O2})}
                $$
            \onslide<11->
                $$
                    m_{\text{impuro}}(\ce{KClO3})
                    =
                    \overbrace{\frac{10}{9}\cdot\frac{2}{3}}^{\frac{20}{27}}
                    \cdot Mm(\ce{KClO3})\cdot P(\ce{O2})\cdot\frac{V(\ce{O2})}{R\cdot T(\ce{O2})}
                $$
        \end{overprint}
                }
        \visible<12->{
            \structure{Sustituyendo por los valores:}
            \begin{overprint}
                \onslide<12>
                    $$
                        m_{\text{impuro}}(\ce{KClO3})
                        =
                        \frac{20}{27}
                        \cdot \SI{122,549}{\gram\per\cancel\mol}\cdot \frac{745-24}{760}~\si{\cancel\atm}\cdot\frac{\SI{10}{\cancel\liter}}{\SI{0,082}{\cancel\atm\cancel\liter\per\cancel\mol\per\cancel\kelvin}\cdot\SI{298,15}{\cancel\kelvin}}
                    $$
                \onslide<13->
                $$
                    \tcbhighmath[boxrule=0.4pt,arc=4pt,colframe=blue,drop fuzzy shadow=red]{
                        m_{\text{impuro}}(\ce{KClO3})
                        =
                        \SI{35}{\gram}
                    }
                $$
            \end{overprint}
                    }
\end{frame}
