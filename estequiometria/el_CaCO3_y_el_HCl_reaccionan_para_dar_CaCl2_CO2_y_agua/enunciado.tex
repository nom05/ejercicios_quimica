El carbonato de calcio (\ce{CaCO3}) y el ácido clorhídrico (\ce{HCl}) reaccionan para dar cloruro de calcio (\ce{CaCl2}), dióxido de carbono (\ce{CO2}) y agua (\ce{H2O}). Calcular los \si{\milli\liter} de ácido \SI{5}{\Molar} necesario para producir \SI{100}{\liter} de dióxido de carbono a \SI{20}{\celsius} y \SI{750}{\torr} si el Rto de la reacción es del \SI{90}{\percent}.
\resultadocmd{ \SI{1824}{\milli\liter} }
