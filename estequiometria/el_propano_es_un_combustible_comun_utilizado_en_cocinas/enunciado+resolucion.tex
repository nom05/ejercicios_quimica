\begin{frame}
	\frametitle{\ejerciciocmd}
	\framesubtitle{Enunciado}
	\textbf{
		Dadas las siguientes reacciones:
\begin{itemize}
    \item \ce{I2(g) + H2(g) -> 2 HI(g)}~~~$\Delta H_1 = \SI{-0,8}{\kilo\calorie}$
    \item \ce{I2(s) + H2(g) -> 2 HI(g)}~~~$\Delta H_2 = \SI{12}{\kilo\calorie}$
    \item \ce{I2(g) + H2(g) -> 2 HI(ac)}~~~$\Delta H_3 = \SI{-26,8}{\kilo\calorie}$
\end{itemize}
Calcular los parámetros que se indican a continuación:
\begin{description}%[label={\alph*)},font={\color{red!50!black}\bfseries}]
    \item[\texttt{a)}] Calor molar latente de sublimación del yodo.
    \item[\texttt{b)}] Calor molar de disolución del ácido yodhídrico.
    \item[\texttt{c)}] Número de calorías que hay que aportar para disociar en sus componentes el yoduro de hidrógeno gas contenido en un matraz de \SI{750}{\cubic\centi\meter} a \SI{25}{\celsius} y \SI{800}{\torr} de presión.
\end{description}
\resultadocmd{\SI{12,8}{\kilo\calorie}; \SI{-13,0}{\kilo\calorie}; \SI{12,9}{\calorie}}

	}
\end{frame}

\begin{frame}
	\frametitle{\ejerciciocmd}
	\framesubtitle{Datos del problema}
	\begin{center}
		{\huge¿$V(\ce{O2})$?} \\[.4cm]
		\tcbhighmath[boxrule=0.4pt,arc=4pt,colframe=black,drop fuzzy shadow=orange]{m(\ce{C3H8}) = \SI{1,00}{\gram}}\\[.4cm]
		\tcbhighmath[boxrule=0.4pt,arc=4pt,colframe=blue,drop fuzzy shadow=green]{T(\ce{O2}) = \SI{0}{\celsius} = \SI{273,15}{\kelvin}}\quad
		\tcbhighmath[boxrule=0.4pt,arc=4pt,colframe=blue,drop fuzzy shadow=green]{P(\ce{O2}) = \SI{1}{\atm}}\\[.4cm]
		Pero también ...\\[.4cm]
		\tcbhighmath[boxrule=0.4pt,arc=4pt,colframe=black,drop fuzzy shadow=orange]{Mm(\ce{C3H8}) = \SI{44,06}{\gram\per\mol}}
	\end{center}
\end{frame}

\begin{frame}
	\frametitle{\ejerciciocmd}
	\framesubtitle{Resolución}
	\structure{Ajustamos la ecuación de combustión del \ce{C3H8}:} La combustión de los alcanos, alquenos y alquinos siempre produce \ce{CO2} y \ce{H2O}. El orden para ajustar fácilmente es: \underline{primero carbonos, segundo hidrógenos y finalmente oxígenos.}
	\begin{overprint}
		\onslide<1>
			$$		
				\ce{
						$\tcbhighmath[boxrule=0.4pt,arc=4pt,colframe=black,drop fuzzy shadow=orange]{\ce{C3}}$
							H8 + O2 ->
						$\tcbhighmath[boxrule=0.4pt,arc=4pt,colframe=black,drop fuzzy shadow=orange]{\ce{\textbf{3}C}}$
							O2 + H2O
					}
			$$
		\onslide<2>
			$$		
				\ce{
							C3
						$\tcbhighmath[boxrule=0.4pt,arc=4pt,colframe=red,drop fuzzy shadow=orange]{\ce{H8}}$
							+ O2 -> 3CO2 +
						$\tcbhighmath[boxrule=0.4pt,arc=4pt,colframe=red,drop fuzzy shadow=orange]{\ce{\textbf{4}H2}}$
					 		O
				}
			$$
		\onslide<3>
			$$
				\ce{
							C3H8 + 
						$\underset{\SI{10}{\text{oxígenos}}}{\tcbhighmath[boxrule=0.4pt,arc=4pt,colframe=blue,drop fuzzy shadow=orange]{\ce{\textbf{5}O2}}}$
							->
						$\underset{\SI{6}{\text{oxígenos}}}{\tcbhighmath[boxrule=0.4pt,arc=4pt,colframe=blue,drop fuzzy shadow=orange]{3}
							\ce{C}
						\tcbhighmath[boxrule=0.4pt,arc=4pt,colframe=blue,drop fuzzy shadow=orange]{\ce{O2}}}$
							+
						$\underset{\SI{4}{\text{oxígenos}}}{\tcbhighmath[boxrule=0.4pt,arc=4pt,colframe=blue,drop fuzzy shadow=orange]{4}
							\ce{H2}
						\tcbhighmath[boxrule=0.4pt,arc=4pt,colframe=blue,drop fuzzy shadow=orange]{\ce{O}}}$
				}
			$$
		\onslide<4->
			$$
				\ce{C3H8(g) + 5O2(g) -> 3CO2(g) + 4H2O(l)}
			$$
	\end{overprint}
	\visible<4->{
		\structure{Usamos la estequiometría para relacionar \ce{C3H8} y \ce{O2}:} $5n(\ce{C3H8}) = n(\ce{O2})$
		\structure{Empleamos la ec. de gases ideales ($PV = nRT$) y la relación entre $n$ y $m$ y despejamos:}
		$$
			5n(\ce{C3H8}) = \underbrace{n(\ce{O2})}_{n = \frac{PV}{RT}}\Rightarrow
			5\frac{m(\ce{C3H8})}{Mn(\ce{C3H8})} = \frac{P_{\ce{O2}}\vdot V(\ce{O2})}{R\vdot T(\ce{O2})}\Rightarrow
			V(\ce{O2}) = \frac{5\vdot m(\ce{C3H8})\vdot R\vdot T(\ce{O2})}{Mn(\ce{C3H8})\vdot P_{\ce{O2}}}
		$$
		$$
			V(\ce{O2}) = \frac{5\vdot\SI{1,00}{\cancel\gram}\vdot\SI{,082}{\cancel\atm\liter\per\cancel\mol\per\cancel\kelvin}\vdot\SI{273,15}{\cancel\kelvin}}{\SI{44,06}{\cancel\gram\per\cancel\mol}\vdot\SI{1}{\cancel\atm}}
		$$
		$$
			\tcbhighmath[boxrule=0.4pt,arc=4pt,colframe=blue,drop fuzzy shadow=green]{V(\ce{O2}) = \SI{2,54}{\liter}}
		$$
				}
\end{frame}

%5.- El propano, es un combustible común utilizado en cocinas y calefacciones. ¿Qué volumen de oxígeno se consume en la combustión de 1,00 g de propano a 0ºC y 1 atm de presión?
%R: 2,5 L 