\begin{frame}
    \frametitle{Anexo 1}
    \framesubtitle{Como curiosidad, únicamente para interesados (no entra en el examen)}
    \structure{Relación entre las dos formas de la ``Ecuación de Van der Waals''}
    Partiendo de una de las formas y multiplicando toda la igualdad por $V^2$:
    \begin{overprint}
        \onslide<1>
                $$
                    P\vdot\left(V-n\vdot b\right)+\frac{n^2\vdot a}{V^2}\vdot\left(V-n\vdot b\right) = n\vdot R\vdot T\Rightarrow
                    P\vdot V^2\vdot\left(V-n\vdot b\right)+\cancel{V^2}\vdot\frac{n^2\vdot a}{\cancel{V^2}}\vdot\left(V-n\vdot b\right) = n\vdot R\vdot T\vdot V^2
                $$
            \onslide<2>
                $$
                    P\vdot V^2\vdot\left(V-n\vdot b\right)+n^2\vdot a\vdot\left(V-n\vdot b\right) = n\vdot R\vdot T\vdot V^2
                $$
            \onslide<3->
                $$
                    P\vdot V^2\vdot\left(V-n\vdot b\right) = n\vdot R\vdot T\vdot V^2 - n^2\vdot a\vdot\left(V-n\vdot b\right) 
                $$
    \end{overprint}
    \visible<4->{
        Multiplicamos a ambos lados de la igualdad por $\frac{1}{n\vdot R\vdot T\vdot \left(V-n\vdot b\right)\vdot V}$
        \begin{overprint}
            \onslide<4>
                $$
                    \scriptstyle
                    \frac{1}{n\vdot R\vdot T\vdot \left(V-n\vdot b\right)\vdot V}\vdot
                    P\vdot V^2\vdot\left(V-n\vdot b\right) = 
                    \frac{1}{n\vdot R\vdot T\vdot \left(V-n\vdot b\right)\vdot V}\vdot
                    \left[
                    n\vdot R\vdot T\vdot V^2 - n^2\vdot a\vdot\left(V-n\vdot b\right) 
                    \right]
                $$
            \onslide<5>
                $$
                    \frac{P\vdot V^2\vdot\left(V-n\vdot b\right)}{n\vdot R\vdot T\vdot \left(V-n\vdot b\right)\vdot V}
                    = 
                    \frac{n\vdot R\vdot T\vdot V^2 - n^2\vdot a\vdot\left(V-n\vdot b\right)}{n\vdot R\vdot T\vdot \left(V-n\vdot b\right)\vdot V}
                $$
            \onslide<6>
                $$
                    \frac{P\vdot V^2\vdot\left(V-n\vdot b\right)}{n\vdot R\vdot T\vdot \left(V-n\vdot b\right)\vdot V}
                    = 
                    \frac{n\vdot R\vdot T\vdot V^2}{n\vdot R\vdot T\vdot \left(V-n\vdot b\right)\vdot V}
                    -
                    \frac{n^2\vdot a\vdot\left(V-n\vdot b\right)}{n\vdot R\vdot T\vdot \left(V-n\vdot b\right)\vdot V}
                $$
            \onslide<7>
                $$
                    \frac{P\vdot V\cancel{^2}\vdot\cancel{\left(V-n\vdot b\right)}}{n\vdot R\vdot T\vdot \cancel{\left(V-n\vdot b\right)}\vdot\cancel{V}}
                    = 
                    \frac{\cancel{n\vdot R\vdot T}\vdot V\cancel{^2}}{\cancel{n\vdot R\vdot T}\vdot\left(V-n\vdot b\right)\vdot\cancel{V}}
                    -
                    \frac{n\cancel{^2}\vdot a\vdot\cancel{\left(V-n\vdot b\right)}}{\cancel{n}\vdot R\vdot T\vdot\cancel{\left(V-n\vdot b\right)}\vdot V}
                $$
            \onslide<8>
                $$
                    \frac{P\vdot V}{n\vdot R\vdot T}
                    = 
                    \frac{V}{\left(V-n\vdot b\right)}
                    -
                    \frac{n\vdot a}{R\vdot T\vdot V}
                $$
        \end{overprint}
    }
\end{frame}
