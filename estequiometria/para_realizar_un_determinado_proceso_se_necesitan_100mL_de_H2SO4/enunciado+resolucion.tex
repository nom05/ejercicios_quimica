\begin{frame}
	\frametitle{\ejerciciocmd}
	\framesubtitle{Enunciado}
	\textbf{
		Una reacción tiene una constante de velocidad de \SI{,017}{\per\second} a \SI{298}{\kelvin} y una energía libre de activación del \SI{27,235}{\kilo\joule\per\mol}. La adición de un catalizador disminuye dicha energía de activación hasta un \SI{33}{\percent} de su valor inicial. Calcule la nueva constante de velocidad.
\resultadocmd{ \SI{26,86}{\per\second} }

		}
\end{frame}

\begin{frame}
	\frametitle{\ejerciciocmd}
	\framesubtitle{Datos del problema}
	{\huge
		$$
			V(\ce{H2SO4})\text{ concentrado y no puro?}
		$$}
	\begin{center}
		\tcbhighmath[boxrule=0.4pt,arc=4pt,colframe=green,drop fuzzy shadow=blue]{V_{
				\begin{matrix}
					\text{diluido}\footnote{NOTA: \textbf{Disolver} (Separar las partículas o moléculas de un sólido, un líquido o un gas en un líquido de forma que queden incorporadas a él) no es lo mismo que \textbf{diluir} (aumentar volumen de disolvente) en Química}\\
					\text{no puro}
				\end{matrix}
				}(\ce{H2SO4})=\SI{100}{\milli\liter}=\SI{100}{\cubic\centi\meter}}\\[.2cm]
		\tcbhighmath[boxrule=0.4pt,arc=4pt,colframe=green,drop fuzzy shadow=blue]{\% \text{pureza}_{
			\begin{matrix}
				\text{diluido}\\
				\text{no puro}
			\end{matrix}
			}(\ce{H2SO4})=\SI{20}{\percent}}
		\tcbhighmath[boxrule=0.4pt,arc=4pt,colframe=green,drop fuzzy shadow=blue]{d_{
			\begin{matrix}
				\text{diluido}\\
				\text{no puro}
			\end{matrix}
			}(\ce{H2SO4})=\SI{1,14}{\gram\per\cubic\centi\meter}=\SI{1,14}{\gram\per\milli\liter}}\\[.4cm]
%%%%%%%%%%%%%%%%%%%%%%%%%%%%%%%%%%%%%%%%%%%%%%%%%%%%%%%%%
		\tcbhighmath[boxrule=0.4pt,arc=4pt,colframe=blue,drop fuzzy shadow=orange]{\text{\% pureza}_{
			\begin{matrix}
				\text{concentrado}\\
				\text{no puro}
			\end{matrix}
			}(\ce{H2SO4})=\SI{98}{\percent}}\\[.2cm]
		\tcbhighmath[boxrule=0.4pt,arc=4pt,colframe=blue,drop fuzzy shadow=orange]{d_{
			\begin{matrix}
				\text{concentrado}\\
				\text{no puro}
			\end{matrix}
			}(\ce{H2SO4})=\SI{1,84}{\gram\per\cubic\centi\meter}=\SI{1,84}{\gram\per\milli\liter}}
	\end{center}
\end{frame}

\begin{frame}
	\frametitle{\ejerciciocmd}
	\framesubtitle{Resolución (\rom{1}): determinar volumen de \ce{H2SO4} concentrado no puro}
	\structure{Nota importante:} Nos tenemos que dar cuenta que el número de moléculas de \ce{H2SO4} (y, por tanto, número de moles) es el mismo pero en diferente volumen. Usamos la parte donde tenemos más información (diluido) para obtener lo que necesitamos.\\[.4cm]
	\begin{overprint}
		\onslide<1>
			\begin{center}
				\myovalbox{\textcolor{white}{		
						$n_{
							\begin{matrix}
								\text{concentrado}\\
								\text{puro}
							\end{matrix}
						}(\ce{H2SO4}) = n_{
							\begin{matrix}
								\text{diluido}\\
								\text{puro}
							\end{matrix}
						}(\ce{H2SO4})$
						\quad\quad $V_{\text{diluido}} > V_{\text{concentrado}}$
				}}
			\end{center}
		\onslide<2>
			$$
						\overbrace{\frac{m_{
							\begin{matrix}
								\text{concentrado}\\
								\text{puro}
							\end{matrix}
						}}{\cancel{Mm(\ce{H2SO4})}}}^{n=\rfrac{m}{Mm}} = 
						\underbrace{\frac{m_{
							\begin{matrix}
								\text{diluido}\\
								\text{puro}
							\end{matrix}
						}}{\cancel{Mm(\ce{H2SO4})}}}_{n=\rfrac{m}{Mm}}
			$$
		\onslide<3>
			$$
				m_{
						\begin{matrix}
							\text{concentrado}\\
							\text{puro}
						\end{matrix}
					} = 
				m_{
						\begin{matrix}
							\text{diluido}\\
							\text{puro}
						\end{matrix}
					}
			$$
		\onslide<4>
			$$
				d=\frac{m}{V}\Rightarrow
					m_{
						\begin{matrix}
							\text{diluido}\\
							\text{no puro}
						\end{matrix}
					}
					=
						d_{
							\begin{matrix}
								\text{diluido}\\
								\text{no puro}
							\end{matrix}
						}
					\vdot
						V_{
							\begin{matrix}
								\text{diluido}\\
								\text{no puro}
							\end{matrix}
						}
					\Rightarrow
					m_{
						\begin{matrix}
							\text{diluido}\\
							\text{no puro}
						\end{matrix}
					}
					= \SI{1,14}{\gram\per\cancel\milli\liter}\vdot\SI{100}{\cancel\milli\liter} = \SI{114}{\gram}
			$$
		\onslide<5>
			$$
				m_{
					\begin{matrix}
						\text{diluido}\\
						\text{puro}
					\end{matrix}
				}
				=
					m_{
						\begin{matrix}
							\text{diluido}\\
							\text{no puro}
						\end{matrix}
					}
					\vdot
					\overbrace{\num{,20}}^{\text{\% pureza diluido = \SI{20}{\percent}}}
				\Rightarrow
				m_{
					\begin{matrix}
						\text{diluido}\\
						\text{puro}
					\end{matrix}
				}
				=
					\SI{114}{\gram}
					\vdot
					\num{,20}
				=
					\SI{22,8}{\gram}
				=
				m_{
					\begin{matrix}
						\text{concentrado}\\
						\text{puro}
					\end{matrix}
				}
			$$
		\onslide<6>
			$$
				m_{
					\begin{matrix}
						\text{concentrado}\\
						\text{puro}
					\end{matrix}
				}
				=
				\SI{22,8}{\gram}
				\Rightarrow
				m_{
					\begin{matrix}
						\text{concentrado}\\
						\text{no puro}
					\end{matrix}
				}
				=
					\frac{\SI{22,8}{\gram}}{\underbrace{\num{,98}}_{\text{\% pureza concentrado} = \SI{98}{\percent}}}
				=
					\SI{23,27}{\gram}
			$$
		\onslide<7->
			$$
				m_{
					\begin{matrix}
						\text{concentrado}\\
						\text{no puro}
					\end{matrix}
				}
				=
				\SI{23,27}{\gram}
				\Rightarrow
				d=\frac{m}{V}
				\Rightarrow
				V=\frac{m}{d}
				\Rightarrow
				V_{
					\begin{matrix}
						\text{concentrado}\\
						\text{no puro}
					\end{matrix}
				}
				=
					\overbrace{\frac{m_{
							\begin{matrix}
								\text{concentrado}\\
								\text{no puro}
							\end{matrix}
					}}{
					\underbrace{d_{
						\begin{matrix}
							\text{concentrado}\\
							\text{no puro}
						\end{matrix}
					}}_{\SI{1,84}{\cancel\gram\per\milli\liter}}
						}}^{\SI{23,27}{\cancel\gram}}
			$$
		\end{overprint}
	
	\visible<7->{
		\centering\tcbhighmath[boxrule=0.4pt,arc=4pt,colframe=green,drop fuzzy shadow=blue]{V_{
				\begin{matrix}
					\text{concentrado}\\
					\text{no puro}
				\end{matrix}
			}(\ce{H2SO4})=\SI{12,64}{\milli\liter}}
				}
\end{frame}
