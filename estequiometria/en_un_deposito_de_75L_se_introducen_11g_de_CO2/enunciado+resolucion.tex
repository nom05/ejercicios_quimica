\begin{frame}
    \frametitle{\ejerciciocmd}
    \framesubtitle{Enunciado}
    \textbf{
	 Dadas las siguientes reacciones:
\begin{itemize}
    \item \ce{I2(g) + H2(g) -> 2 HI(g)}~~~$\Delta H_1 = \SI{-0,8}{\kilo\calorie}$
    \item \ce{I2(s) + H2(g) -> 2 HI(g)}~~~$\Delta H_2 = \SI{12}{\kilo\calorie}$
    \item \ce{I2(g) + H2(g) -> 2 HI(ac)}~~~$\Delta H_3 = \SI{-26,8}{\kilo\calorie}$
\end{itemize}
Calcular los parámetros que se indican a continuación:
\begin{description}%[label={\alph*)},font={\color{red!50!black}\bfseries}]
    \item[\texttt{a)}] Calor molar latente de sublimación del yodo.
    \item[\texttt{b)}] Calor molar de disolución del ácido yodhídrico.
    \item[\texttt{c)}] Número de calorías que hay que aportar para disociar en sus componentes el yoduro de hidrógeno gas contenido en un matraz de \SI{750}{\cubic\centi\meter} a \SI{25}{\celsius} y \SI{800}{\torr} de presión.
\end{description}
\resultadocmd{\SI{12,8}{\kilo\calorie}; \SI{-13,0}{\kilo\calorie}; \SI{12,9}{\calorie}}

    	}
\end{frame}

\begin{frame}
    \frametitle{\ejerciciocmd}
    \framesubtitle{Datos del problema}
    \begin{center}
        \begin{enumerate}[label={\alph*)},font={\color{red!50!black}\bfseries}]
            \item {\large ¿$P(\ce{H2})$? ¿$P(\ce{CO2})$? ¿$P(\ce{CH4})$?}
            \item {\large ¿$m(\ce{CH4})$?}
            \item {\large ¿$d_T$?}
        \end{enumerate}
    \end{center}

    $$
        \tcbhighmath[boxrule=0.4pt,arc=4pt,colframe=blue,drop fuzzy shadow=red]{V_T = \SI{75}{\liter}}\quad
        \tcbhighmath[boxrule=0.4pt,arc=4pt,colframe=blue,drop fuzzy shadow=red]{P_T = \frac{1782}{760}~\si{\atm}}\quad
        \tcbhighmath[boxrule=0.4pt,arc=4pt,colframe=blue,drop fuzzy shadow=red]{T = 27+273,15~\si{\kelvin}}
    $$
    $$
        \tcbhighmath[boxrule=0.4pt,arc=4pt,colframe=red,drop fuzzy shadow=blue]{m(\ce{CO2}) = \SI{11}{\gram}}\quad
        \tcbhighmath[boxrule=0.4pt,arc=4pt,colframe=red,drop fuzzy shadow=blue]{Mm(\ce{CO2}) = \SI{44,01}{\gram\per\mol}}
    $$
    $$
        \tcbhighmath[boxrule=0.4pt,arc=4pt,colframe=green,drop fuzzy shadow=red]{m(\ce{H2}) = \SI{8}{\gram}}\quad
        \tcbhighmath[boxrule=0.4pt,arc=4pt,colframe=green,drop fuzzy shadow=red]{Mm(\ce{H2}) = \SI{2,02}{\gram\per\mol}}
    $$
    $$
        \tcbhighmath[boxrule=0.4pt,arc=4pt,colframe=yellow,drop fuzzy shadow=red]{Mm(\ce{CH4}) = \SI{16,04}{\gram\per\mol}}
    $$
\end{frame}

\begin{frame}
    \frametitle{\ejerciciocmd}
    \framesubtitle{Resolución (\rom{1}): Presiones parciales de los gases}
    \structure{Combinamos la ecuación de los gases ideales con la fracción molar para un gas $A$:}
    \begin{overprint}
        \onslide<1>
            $$
                x_A = \frac{n_A}{n_T}
            $$
        \onslide<2>
            $$
                x_A = \frac{\overbrace{n_A}^{n=\frac{m}{Mm}}}{\underbrace{n_T}_{PV=nRT\Rightarrow n=\frac{PV}{RT}}}
            $$
        \onslide<3>
            $$
                x_A = \frac{\frac{m_A}{Mm_A}}{\frac{P_T\cdot V_T}{R\cdot T}}
            $$
        \onslide<4->
            $$
                x_A = \frac{m_A\cdot R\cdot T}{Mm_A\cdot P_T\cdot V_T}
            $$
    \end{overprint}
    \visible<4->{
        \structure{Usamos la relación entre la $P$ parcial y la $P$ total de una mezcla de gases:}
        \begin{overprint}
            \onslide<4>
                $$
                    P_A = x_A\cdot P_T
                $$
            \onslide<5>
                $$
                    P_A = \overbrace{x_A}^{x_A = \frac{m_A\cdot R\cdot T}{Mm_A\cdot P_T\cdot V_T}}\cdot P_T
                $$
            \onslide<6>
                $$
                    P_A = \frac{m_A\cdot R\cdot T}{Mm_A\cdot P_T\cdot V_T}\cdot P_T
                $$
            \onslide<7>
                $$
                    P_A = \frac{m_A\cdot R\cdot T}{Mm_A\cdot\cancel{P_T}\cdot V_T}\cdot\cancel{P_T}
                $$
            \onslide<8->
                $$
                    P_A = \frac{m_A\cdot R\cdot T}{Mm_A\cdot V_T}
                $$
        \end{overprint}
                }
    \visible<9->{
        \structure{Sustituimos por los valores y calculamos la $P$ para \ce{H2} y \ce{CO2}:}
        $$
            P(\ce{H2}) = \frac{\SI{8}{\cancel\gram}\cdot\SI{0,082}{\atm\cancel\liter\per\cancel\mol\per\cancel\kelvin}\cdot\SI{300,15}{\cancel\kelvin}}{\SI{2,002}{\cancel\gram\per\cancel\mol}\cdot\SI{75}{\cancel\liter}}\Rightarrow
            \tcbhighmath[boxrule=0.4pt,arc=4pt,colframe=blue,drop fuzzy shadow=red]{P(\ce{H2}) = \SI{1,30}{\atm}}
        $$
        $$
            P(\ce{O2}) = \frac{\SI{11}{\cancel\gram}\cdot\SI{0,082}{\atm\cancel\liter\per\cancel\mol\per\cancel\kelvin}\cdot\SI{300,15}{\cancel\kelvin}}{\SI{44,01}{\cancel\gram\per\cancel\mol}\cdot\SI{75}{\cancel\liter}}\Rightarrow
            \tcbhighmath[boxrule=0.4pt,arc=4pt,colframe=green,drop fuzzy shadow=red]{P(\ce{O2}) = \SI{0,08}{\atm}}
        $$
                }
    \visible<10->{
        \structure{Con la Ley de Dalton (Ecuación~\eqref{eq:Dalton}) obtenemos la $P$ del tercer gas:}
        \begin{overprint}
            \onslide<10>
                $$
                    P_T = \sum_{i=1}^{n} P_i\Rightarrow\text{Si }n=3\text{ entonces: } P_T = P_1 + P_2 + P_3
                $$
            \onslide<11>
                $$
                    P_T = P(\ce{H2}) + P(\ce{CO2}) + P(\ce{CH4})\Rightarrow  P(\ce{CH4}) = P_T - P(\ce{H2}) - P(\ce{CO2})
                $$
            \onslide<12>
                $$
                    P(\ce{CH4}) = \frac{1782}{760}~\si{\atm} - \SI{1,30}{\atm} - \SI{0,08}{\atm}
                $$
            \onslide<13>
                $$
                    \tcbhighmath[boxrule=0.4pt,arc=4pt,colframe=blue,drop fuzzy shadow=green]{P(\ce{CH4}) = \SI{0,96}{\atm}}
                $$
        \end{overprint}
                 }
\end{frame}

\begin{frame}
    \frametitle{\ejerciciocmd}
    \framesubtitle{Resolución (\rom{2}): masa de metano}
    \structure{Previamente obtuvimos la expresión:}
    $$
        P_A = \frac{m_A\cdot R\cdot T}{Mm_A\cdot V_T}
    $$
    \visible<2->{
        \structure{Despejando $m_A$:}
        \begin{overprint}
            \onslide<2>
                $$
                    P_A = \frac{m_A\cdot R\cdot T}{Mm_A\cdot V_T}
                $$
            \onslide<3->
                $$
                    m_A = \frac{P_A\cdot Mm_A\cdot V_T}{R\cdot T}
                $$
        \end{overprint}
                }
    \visible<3->{
        \structure{Sustituyendo por los valores del \ce{CH4}:}
        $$
            m_A = \frac{\SI{0,96}{\cancel\atm}\cdot \SI{16,04}{\gram\per\cancel\mol}\cdot\SI{75}{\cancel\liter}}{\SI{0,082}{\cancel\atm\cancel\liter\per\cancel\mol\per\cancel\kelvin}\cdot\SI{300,15}{\cancel\kelvin}}
        $$
        $$
            \tcbhighmath[boxrule=0.4pt,arc=4pt,colframe=green,drop fuzzy shadow=red]{m(\ce{CH4}) = \SI{47,0}{\gram}}
        $$
                }
\end{frame}

\begin{frame}
    \frametitle{\ejerciciocmd}
    \framesubtitle{Resolución (\rom{3}): densidad de la mezcla de gases}
    \structure{La densidad viene dada por esta expresión general:}
    $$
        d = \frac{m}{V}
    $$
    \begin{block}{Método de la masa molecular media}
        \structure{Partiendo de la ecuación de los gases ideales en función de la densidad (Ecuación~\eqref{eq:densidad_gases}):}
        \begin{overprint}
            \onslide<1-6>
                $$
                    d = \frac{P_T\cdot\overline{Mm}}{R\cdot T}
                $$
            \onslide<7>
                $$
                    d = \frac{\SI{2,345}{\cancel\atm}\cdot\SI{9,14}{\gram\per\cancel\mol}}{\SI{0,082}{\cancel\atm\liter\per\cancel\mol\per\cancel\kelvin}\cdot\SI{300,15}{\cancel\kelvin}}
                $$
            \onslide<8->
                $$
                    \tcbhighmath[boxrule=0.4pt,arc=4pt,colframe=green,drop fuzzy shadow=red]{d = \SI{0,88}{\gram\per\liter}}
                $$
        \end{overprint}
        \begin{overprint}
            \onslide<2>
                $$
                    \overline{Mm} = \sum_{i=1}^{n}x_i\cdot Mm_i
                $$
            \onslide<3>
                $$
                    \overline{Mm} = \sum_{i=1}^{n}x_i\cdot Mm_i = x(\ce{CO2})\cdot Mm(\ce{CO2}) + x(\ce{H2})\cdot Mm(\ce{H2}) + x(\ce{CH4})\cdot Mm(\ce{CH4})
                $$
            \onslide<4>
                $$
                    \overline{Mm} = \frac{n(\ce{CO2})}{n_T}\cdot Mm(\ce{CO2}) + \frac{n(\ce{H2})}{n_T}\cdot Mm(\ce{H2}) + \frac{n(\ce{CH4})}{n_T}\cdot Mm(\ce{CH4})
                $$
            \onslide<5>
                $$
                    \overline{Mm} = \frac{\SI{0,25}{\mol}}{\SI{7,14}{\mol}}\cdot Mm(\ce{CO2}) + \frac{\SI{4,00}{\mol}}{\SI{7,14}{\mol}}\cdot Mm(\ce{H2}) + \frac{\SI{2,89}{\mol}}{\SI{7,14}{\mol}}\cdot Mm(\ce{CH4})
                $$
            \onslide<6-7>
                $$
                    \overline{Mm} = \frac{\SI{0,25}{\mol}}{\SI{7,14}{\mol}}\cdot\SI{44,01}{\gram\per\mol} + \frac{\SI{4,00}{\mol}}{\SI{7,14}{\mol}}\cdot\SI{2,02}{\gram\per\mol} + \frac{\SI{2,89}{\mol}}{\SI{7,14}{\mol}}\cdot\SI{16,04}{\gram\per\mol}
                    =
                    \SI{9,14}{\gram\per\mol}
                $$
        \end{overprint}
    \end{block}
    \visible<9->{
        \begin{exampleblock}{Método de masa total}
            \begin{overprint}
                \onslide<9>
                    $$
                        d = \frac{m_T}{V}
                    $$                    
                \onslide<10>
                    $$
                        d = \frac{\sum_{i=1}^{n}m_i}{V}
                    $$
                \onslide<11>
                    $$
                        \tcbhighmath[boxrule=0.4pt,arc=4pt,colframe=green,drop fuzzy shadow=red]{d = \frac{(11+8+47)~\si{\gram}}{\SI{75}{\liter}} = \SI{0,88}{\gram\per\liter}}
                    $$
                    \structure{VER ANEXO 2 para la demostración sobre la equivalencia entre los dos métodos}
            \end{overprint}
        \end{exampleblock}
               }
\end{frame}

%\subsection[Anexo]{Anexo: relación entre masa molecular y densidad en la ecuación de los gases ideales}
\begin{frame}
    \frametitle{Anexo}
    \framesubtitle{Apartado (\rom{3})}
    En el apartado c se pide la densidad ($d$). Partiendo de la ecuación de los gases ideales se obtiene la densidad:
    $$
        P_T\cdot V = \overbrace{n}^{n=\frac{m}{Mm}}\cdot R\cdot T
    $$
    $$
        P_T=\overbrace{\frac{m}{V}}^{d=\frac{m}{V}}\cdot\frac{R\cdot T}{\overline{Mm}} = d\cdot\frac{R\cdot T}{\overline{Mm}}
        \Rightarrow
        d=\frac{P_T\cdot\overline{Mm}}{R\cdot T}
    $$
    Esta $\overline{Mm}$ viene dada por:
    $$
        \overline{Mm}=\sum_{i=1}^{n}x_i\cdot Mm_i = \sum_{i=1}^{n}\frac{n_i}{n_T}\cdot Mm_i =\frac{1}{n_T}\sum_{i=1}^{n}\overbrace{n_i}^{n_i=\frac{m_i}{Mm_i}}\cdot Mm_i= \frac{1}{\underbrace{n_T}_{\frac{P_T\cdot V}{R\cdot T}}}\sum_{i=1}^{n}\frac{m_i}{\cancel{Mm_i}}\cdot\cancel{Mm_i}
    $$
    $$
        \overline{Mm}=\frac{R\cdot T}{P_T\cdot V}\sum_{i=1}^{n}m_i
    $$
    Sustituyendo en la ecuación de los gases ideales en función de la densidad:
    $$
        d=\frac{P_T\cdot\overbrace{\overline{Mm}}^{\overline{Mm}=\frac{R\cdot T}{P_T\cdot V}\sum_{i=1}^{n}m_i}}{R\cdot T}=
        \frac{\cancel{P_T}\cdot\frac{\cancel{R\cdot T}}{\cancel{P_T}\cdot V}\sum_{i=1}^{n}m_i}{\cancel{R\cdot T}}=
        \frac{\sum_{i=1}^{n}m_i}{V}
    $$
    Es decir, la densidad de una mezcla de gases ideales es la suma de las masas dividido el volumen que ocupan. No es necesario ni $P_T$, ni $T$, ni $\overline{Mm}$.
\end{frame}


