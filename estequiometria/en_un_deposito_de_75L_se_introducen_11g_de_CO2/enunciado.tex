En un depósito de \SI{75}{\liter} se introducen \SI{11}{\gram} de dióxido de carbono, \SI{8}{\gram} de hidrógeno y una masa desconocida de metano. La mezcla ejerce una presión de \SI{1782}{\torr} a \SI{27}{\celsius}. Calcular:
\begin{enumerate}[label={\alph*)},font={\color{red!50!black}\bfseries}]
    \item La presión parcial de cada gas.
    \item La masa de metano.
    \item La densidad de la mezcla gaseosa.
\end{enumerate}
\resultadocmd{
            \SI{1,30}{\atm}, \SI{0,08}{\atm}, \SI{0,96}{\atm};
            \SI{47,0}{\gram};
            \SI{0,88}{\gram\per\liter}
}
