%\subsection[Anexo]{Anexo: relación entre masa molecular y densidad en la ecuación de los gases ideales}
\begin{frame}
    \frametitle{Anexo}
    \framesubtitle{Apartado (\rom{3})}
    En el apartado c se pide la densidad ($d$). Partiendo de la ecuación de los gases ideales se obtiene la densidad:
    $$
        P_T\cdot V = \overbrace{n}^{n=\frac{m}{Mm}}\cdot R\cdot T
    $$
    $$
        P_T=\overbrace{\frac{m}{V}}^{d=\frac{m}{V}}\cdot\frac{R\cdot T}{\overline{Mm}} = d\cdot\frac{R\cdot T}{\overline{Mm}}
        \Rightarrow
        d=\frac{P_T\cdot\overline{Mm}}{R\cdot T}
    $$
    Esta $\overline{Mm}$ viene dada por:
    $$
        \overline{Mm}=\sum_{i=1}^{n}x_i\cdot Mm_i = \sum_{i=1}^{n}\frac{n_i}{n_T}\cdot Mm_i =\frac{1}{n_T}\sum_{i=1}^{n}\overbrace{n_i}^{n_i=\frac{m_i}{Mm_i}}\cdot Mm_i= \frac{1}{\underbrace{n_T}_{\frac{P_T\cdot V}{R\cdot T}}}\sum_{i=1}^{n}\frac{m_i}{\cancel{Mm_i}}\cdot\cancel{Mm_i}
    $$
    $$
        \overline{Mm}=\frac{R\cdot T}{P_T\cdot V}\sum_{i=1}^{n}m_i
    $$
    Sustituyendo en la ecuación de los gases ideales en función de la densidad:
    $$
        d=\frac{P_T\cdot\overbrace{\overline{Mm}}^{\overline{Mm}=\frac{R\cdot T}{P_T\cdot V}\sum_{i=1}^{n}m_i}}{R\cdot T}=
        \frac{\cancel{P_T}\cdot\frac{\cancel{R\cdot T}}{\cancel{P_T}\cdot V}\sum_{i=1}^{n}m_i}{\cancel{R\cdot T}}=
        \frac{\sum_{i=1}^{n}m_i}{V}
    $$
    Es decir, la densidad de una mezcla de gases ideales es la suma de las masas dividido el volumen que ocupan. No es necesario ni $P_T$, ni $T$, ni $\overline{Mm}$.
\end{frame}

