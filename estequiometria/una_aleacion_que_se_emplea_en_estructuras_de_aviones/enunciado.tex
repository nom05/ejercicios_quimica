Una aleación que se emplea en estructuras de aviones está formada por $93,7$ por ciento de aluminio y $6,3$ por ciento de cobre. La aleación tiene una densidad de \SI{2,85}{\gram\per\cubic\centi\meter}. Una pieza de \SI{0,691}{\cubic\centi\meter} de esta aleación reacciona con un exceso de ácido clorhídrico (ac). Si suponemos que todo el aluminio pero nada del cobre reacciona con el ácido clorhídrico (ac), ¿qué masa de hidrógeno se obtiene?
    Reacción: \ce{Al(s) + HCl(ac) -> AlCl3 + H2(g)}
\resultadocmd{ \SI{0,21}{\gram} }
