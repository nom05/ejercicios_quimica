\begin{frame}
    \frametitle{\ejerciciocmd}
    \framesubtitle{Enunciado}
    \textbf{
		Una reacción tiene una constante de velocidad de \SI{,017}{\per\second} a \SI{298}{\kelvin} y una energía libre de activación del \SI{27,235}{\kilo\joule\per\mol}. La adición de un catalizador disminuye dicha energía de activación hasta un \SI{33}{\percent} de su valor inicial. Calcule la nueva constante de velocidad.
\resultadocmd{ \SI{26,86}{\per\second} }

	}
\end{frame}

\begin{frame}
    \frametitle{\ejerciciocmd}
    \framesubtitle{Datos del problema}
    {\huge $$
        m(\ce{H2})?
    $$}
    Reacción: 
    \tcbhighmath[boxrule=0.4pt,arc=4pt,colframe=red,drop fuzzy shadow=yellow]{\ce{Al}}
    + 
    \ce{HCl(exceso)}
    \ce{->}
    \ce{AlCl3}
    +
    \tcbhighmath[boxrule=0.4pt,arc=4pt,colframe=green,drop fuzzy shadow=blue]{\ce{H2}}
    \visible<1-|handout:0>{
        $$
            \tcbhighmath[boxrule=0.4pt,arc=4pt,colframe=red,drop fuzzy shadow=yellow]{\text{pureza \ce{Al}} = \SI{93,7}{\percent}}
            \quad
            \tcbhighmath[boxrule=0.4pt,arc=4pt,colframe=red,drop fuzzy shadow=yellow]{V(\ce{Al}) = \SI{0,691}{\cubic\centi\meter}}
            \quad
            \tcbhighmath[boxrule=0.4pt,arc=4pt,colframe=red,drop fuzzy shadow=yellow]{d(\ce{Al}) = \SI{2,85}{\gram\per\cubic\centi\meter}}
        $$
                          }
    \visible<2-|handout:0>{Pero también ...\\
        $$
            \tcbhighmath[boxrule=0.4pt,arc=4pt,colframe=red,drop fuzzy shadow=yellow]{Mat(\ce{Al}) = \SI{26,98}{\gram\per\mol}}
            \quad
            \tcbhighmath[boxrule=0.4pt,arc=4pt,colframe=green,drop fuzzy shadow=blue]{Mm(\ce{H2}) = \SI{2,02}{\gram\per\mol}}
        $$
                          }
\end{frame}

\begin{frame}
    \frametitle{\ejerciciocmd}
    \framesubtitle{Resolución (\rom{1}): ajuste de la ecuación de la reacción química}
    \structure{Por tanteo:}\\[.3cm]
    \begin{overprint}
        \onslide<1>
            \centering\ce{Al + HCl -> AlCl3 + H2}
        \onslide<2>
            \centering\ce{Al + H}\tcbhighmath[boxrule=0.4pt,arc=4pt,colframe=blue,drop fuzzy shadow=red]{\ce{Cl}} \ce{-> Al}\tcbhighmath[boxrule=0.4pt,arc=4pt,colframe=blue,drop fuzzy shadow=red]{\ce{Cl3}} \ce{+ H2}\\[.3cm]
        \onslide<3>
            \centering\tcbhighmath[boxrule=0.4pt,arc=4pt,colframe=blue,drop fuzzy shadow=red]{Al+\ce{3HCl}} \ce{->}\tcbhighmath[boxrule=0.4pt,arc=4pt,colframe=blue,drop fuzzy shadow=red]{\ce{AlCl3}} \ce{+ H2}\\[.3cm]
        \onslide<4>
            \centering\ce{Al + }\tcbhighmath[boxrule=0.4pt,arc=4pt,colframe=green,drop fuzzy shadow=blue]{\ce{3H}}\ce{Cl} \ce{-> AlCl3 +} \tcbhighmath[boxrule=0.4pt,arc=4pt,colframe=green,drop fuzzy shadow=blue]{\ce{H2}}\\[.3cm]
        \onslide<5>
            \centering\ce{Al + }\tcbhighmath[boxrule=0.4pt,arc=4pt,colframe=green,drop fuzzy shadow=blue]{\ce{3H}}\ce{Cl} \ce{-> AlCl3 +} \tcbhighmath[boxrule=0.4pt,arc=4pt,colframe=green,drop fuzzy shadow=blue]{\frac{3}{2}\ce{H2}}\\[.3cm]
        \onslide<6>
            \centering$2\times\left(\ce{Al + 3HCl -> AlCl3 +\frac{3}{2}H2}\right)$\\[.3cm]
        \onslide<7>
            \centering\ce{2Al + 6HCl -> 2AlCl3 +3H2}\\[.3cm]
   \end{overprint}
\end{frame}

\begin{frame}
    \frametitle{\ejerciciocmd}
    \framesubtitle{Resolución (\rom{2}): cálculo de la masa de hidrógeno}
    \structure{Reacción:} \ce{2Al + 6HCl -> 2AlCl3 +3H2}\\[.3cm]
    \structure{Paso 1:} $m_{\text{puro}}(\ce{Al})$ vs. $m_{\text{aleación}}(\ce{Al})$
    \begin{overprint}
        \onslide<2>
            $$
                m_{\text{puro}}(\ce{Al})
                \quad ??\quad
                m_{\text{aleación}}(\ce{Al})
            $$
        \onslide<3>    
            $$
                m_{\text{puro}}(\ce{Al})
                \quad <\quad 
                m_{\text{aleación}}(\ce{Al})
            $$
        \onslide<4>
            $$
                m_{\text{puro}}(\ce{Al})
                \quad =\quad
                \frac{93,7}{100}
                m_{\text{aleación}}(\ce{Al})
            $$
        \onslide<5>
            $$
                m_{\text{puro}}(\ce{Al})
                \quad =\quad
                \frac{93,7}{100}
                \underbrace{m_{\text{aleación}}(\ce{Al})}_{d=\frac{m}{V}\Rightarrow m=d\cdot V}
            $$
        \onslide<6->
            $$
                m_{\text{puro}}(\ce{Al})
                \quad =\quad
                \frac{93,7}{100}
                d_{\text{aleación}}(\ce{Al})\cdot V_{\text{aleación}}(\ce{Al})
            $$
    \end{overprint}
    \visible<7->{
        \structure{Paso 2:} Número de moles de \ce{Al} puros
                }
    \begin{overprint}
        \onslide<7>
            $$
                m_{\text{puro}}(\ce{Al})
                \quad =\quad
                \frac{93,7}{100}
                d_{\text{aleación}}(\ce{Al})\cdot V_{\text{aleación}}(\ce{Al})
            $$
        \onslide<8>
            $$
                \overbrace{m_{\text{puro}}(\ce{Al})}^{n=\frac{m}{Mm}\Rightarrow m=n\cdot Mm}
                \quad =\quad
                \frac{93,7}{100}
                d_{\text{aleación}}(\ce{Al})\cdot V_{\text{aleación}}(\ce{Al})
            $$
        \onslide<9>
            $$
                n(\ce{Al})\cdot Mm(\ce{Al})
                \quad =\quad
                \frac{93,7}{100}
                d_{\text{aleación}}(\ce{Al})\cdot V_{\text{aleación}}(\ce{Al})
            $$
        \onslide<10->
            $$
                n(\ce{Al})
                \quad =\quad
                \frac{93,7}{100}
                \frac{d_{\text{aleación}}(\ce{Al})\cdot V_{\text{aleación}}(\ce{Al})}{Mm(\ce{Al})}
            $$
    \end{overprint}
    \visible<11->{
        \structure{Paso 3:} Relación estequiométrica entre el \ce{Al} y el \ce{H2}\\[.3cm]
        \centering\SI{2}{\mol} de \ce{Al}\ce{->}\SI{3}{\mol} de \ce{H2}\\[.3cm]
                }
    \begin{overprint}
        \onslide<12>
            $$
                n(\ce{Al})
                \quad ??\quad
                n(\ce{H2})
            $$
        \onslide<13>    
            $$
                n(\ce{Al})
                \quad <\quad
                n(\ce{H2})
            $$
        \onslide<14>
            $$
                3n(\ce{Al})
                \quad =\quad
                2n(\ce{H2})
            $$
        \onslide<15>
            $$
                n(\ce{Al})
                \quad =\quad
                \frac{2}{3}n(\ce{H2})
            $$
    \end{overprint}
\end{frame}

\begin{frame}
    \frametitle{\ejerciciocmd}
    \framesubtitle{Resolución (\rom{2}): cálculo de la masa del \ce{H2}}
    \structure{Paso 4:} Igualando expresiones previas.
    \begin{overprint}
        \onslide<1>
            $$
                \frac{2}{3}n(\ce{H2})
                \quad =\quad
                \frac{93,7}{100}
                \frac{d_{\text{aleación}}(\ce{Al})\cdot V_{\text{aleación}}(\ce{Al})}{Mm(\ce{Al})}
            $$
        \onslide<2->
            $$
                n(\ce{H2})
                \quad =\quad
                \frac{93,7}{100}\cdot
                \frac{3}{2}
                \frac{d_{\text{aleación}}(\ce{Al})\cdot V_{\text{aleación}}(\ce{Al})}{Mm(\ce{Al})}
            $$
    \end{overprint}
    \visible<3->{
        \structure{Paso 5:} Introduciendo y despejando $m(\ce{H2})$.
                }
        \begin{overprint}
            \onslide<3>
                $$
                    \overbrace{n(\ce{H2})}^{n=\frac{m}{Mm}}
                    \quad =\quad
                    \frac{93,7}{100}\cdot
                    \frac{3}{2}
                    \frac{d_{\text{aleación}}(\ce{Al})\cdot V_{\text{aleación}}(\ce{Al})}{Mm(\ce{Al})}
                $$
            \onslide<4>
                $$
                    \frac{m(\ce{H2})}{Mm(\ce{H2})}
                    \quad =\quad
                    \frac{93,7}{100}\cdot
                    \frac{3}{2}
                    \frac{d_{\text{aleación}}(\ce{Al})\cdot V_{\text{aleación}}(\ce{Al})}{Mm(\ce{Al})}
                $$
            \onslide<5->
                $$
                    m(\ce{H2})
                    \quad =\quad
                    \frac{93,7}{100}\cdot
                    \frac{3}{2}
                    \frac{Mm(\ce{H2})}{Mm(\ce{Al})}
                    d_{\text{aleación}}(\ce{Al})\cdot V_{\text{aleación}}(\ce{Al})
                $$
        \end{overprint}
            \visible<6->{
                \structure{Paso 6:} Operamos
                $$
                    m(\ce{H2})
                    \quad =\quad
                    \frac{93,7}{100}\cdot
                    \frac{3}{2}\cdot
                    \frac{\SI{2,02}{\cancel\gram\per\cancel\mol}}{\SI{26,48}{\cancel\gram\per\cancel\mol}}\cdot
                    \SI{2,85}{\gram\per\cancel\cubic\centi\meter}\cdot\SI{0,691}{\cancel\cubic\centi\meter}
                $$
                        }
            \visible<7->{
                $$
                    \tcbhighmath[boxrule=0.4pt,arc=4pt,colframe=green,drop fuzzy shadow=blue]{m(\ce{H2})
                    \quad =\quad
                    \SI{0,21}{\gram}}
                $$
                        }
\end{frame}
