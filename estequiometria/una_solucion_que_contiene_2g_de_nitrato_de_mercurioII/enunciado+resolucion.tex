\begin{frame}
	\frametitle{\ejerciciocmd}
	\framesubtitle{Enunciado}
	\textbf{
		Dadas las siguientes reacciones:
\begin{itemize}
    \item \ce{I2(g) + H2(g) -> 2 HI(g)}~~~$\Delta H_1 = \SI{-0,8}{\kilo\calorie}$
    \item \ce{I2(s) + H2(g) -> 2 HI(g)}~~~$\Delta H_2 = \SI{12}{\kilo\calorie}$
    \item \ce{I2(g) + H2(g) -> 2 HI(ac)}~~~$\Delta H_3 = \SI{-26,8}{\kilo\calorie}$
\end{itemize}
Calcular los parámetros que se indican a continuación:
\begin{description}%[label={\alph*)},font={\color{red!50!black}\bfseries}]
    \item[\texttt{a)}] Calor molar latente de sublimación del yodo.
    \item[\texttt{b)}] Calor molar de disolución del ácido yodhídrico.
    \item[\texttt{c)}] Número de calorías que hay que aportar para disociar en sus componentes el yoduro de hidrógeno gas contenido en un matraz de \SI{750}{\cubic\centi\meter} a \SI{25}{\celsius} y \SI{800}{\torr} de presión.
\end{description}
\resultadocmd{\SI{12,8}{\kilo\calorie}; \SI{-13,0}{\kilo\calorie}; \SI{12,9}{\calorie}}

		}
\end{frame}

\begin{frame}
	\frametitle{\ejerciciocmd}
	\framesubtitle{Datos del problema}
	\centering{\huge m(\ce{HgS})?}
	\begin{center}
		\tcbhighmath[boxrule=0.4pt,arc=4pt,colframe=green,drop fuzzy shadow=blue]{m(\ce{Hg(NO3)2})=\SI{2,00}{\gram}}
		\tcbhighmath[boxrule=0.4pt,arc=4pt,colframe=orange,drop fuzzy shadow=green]{m(\ce{Na2S})=\SI{2,00}{\gram}}
	\end{center}
	\visible<2-|handout:0>{Pero también\ldots\\
		$$
			\tcbhighmath[boxrule=0.4pt,arc=4pt,colframe=green,drop fuzzy shadow=blue]{Mm(\ce{Hg(NO3)2}) = \SI{324,6}{\gram\per\mol}}\quad
			\tcbhighmath[boxrule=0.4pt,arc=4pt,colframe=yellow,drop fuzzy shadow=green]{Mm(\ce{Na2S}) = \SI{78,04}{\gram\per\mol}}
		$$
		$$
			\tcbhighmath[boxrule=0.4pt,arc=4pt,colframe=red,drop fuzzy shadow=green]{Mm(\ce{HgS}) = \SI{232,66}{\gram\per\mol}}
		$$
		Reacción sin ajustar: 
		\tcbhighmath[boxrule=0.4pt,arc=4pt,colframe=green,drop fuzzy shadow=blue]{\ce{Hg(NO3)2}}
		\ce{+}
		\tcbhighmath[boxrule=0.4pt,arc=4pt,colframe=yellow,drop fuzzy shadow=green]{\ce{Na2S}}
		\ce{->}
		\tcbhighmath[boxrule=0.4pt,arc=4pt,colframe=red,drop fuzzy shadow=green]{\ce{HgS}}
		\ce{ + NaNO3}
	}
\end{frame}

\begin{frame}
	\frametitle{\ejerciciocmd}
	\framesubtitle{Resolución (\rom{1}): ajuste de la reacción}
	\structure{Reacción:} 
	\begin{overprint}
		\onslide<1>
			\centering
			\tcbhighmath[boxrule=0.4pt,arc=4pt,colframe=green,drop fuzzy shadow=blue]{\ce{Hg}}
			\ce{(NO3)2 + Na2S -> }
			\tcbhighmath[boxrule=0.4pt,arc=4pt,colframe=green,drop fuzzy shadow=blue]{\ce{Hg}}
			\ce{S + NaNO3} (\textbf{\underline{\ce{Hg} ajustado}})
		\onslide<2>
			\centering
			\ce{Hg}
			\tcbhighmath[boxrule=0.4pt,arc=4pt,colframe=yellow,drop fuzzy shadow=green]{\ce{(NO3)\textbf{2}}}
			\ce{ + Na2S -> HgS + }
			\tcbhighmath[boxrule=0.4pt,arc=4pt,colframe=yellow,drop fuzzy shadow=green]{\ce{Na(NO3)\textbf{1}}}\quad\quad(\textbf{\underline{\ce{NO3-} sin ajustar}})
		\onslide<3>
			\centering
			\ce{Hg}
			\tcbhighmath[boxrule=0.4pt,arc=4pt,colframe=yellow,drop fuzzy shadow=green]{\ce{(NO3)\textbf{2}}}
			\ce{ + Na2S -> HgS + }
			\tcbhighmath[boxrule=0.4pt,arc=4pt,colframe=yellow,drop fuzzy shadow=green]{\ce{\textbf{2}Na(NO3)\textbf{1}}}\quad\quad(\textbf{\ce{NO3-} \underline{ajustado}})
		\onslide<4>
			\centering
			\ce{Hg(NO3)2 +}
			\tcbhighmath[boxrule=0.4pt,arc=4pt,colframe=red,drop fuzzy shadow=blue]{\ce{Na2}}
			\ce{S -> HgS + }
			\tcbhighmath[boxrule=0.4pt,arc=4pt,colframe=red,drop fuzzy shadow=blue]{\ce{2Na}}
			\ce{NO3} (\textbf{\ce{Na} \underline{ajustado}})
		\onslide<5>
			\centering
			\ce{Hg(NO3)2 + Na2}
			\tcbhighmath[boxrule=0.4pt,arc=4pt,colframe=orange,drop fuzzy shadow=green]{\ce{S}}
			\ce{ -> Hg}
			\tcbhighmath[boxrule=0.4pt,arc=4pt,colframe=orange,drop fuzzy shadow=green]{\ce{S}}
			\ce{ + \textbf{2}NaNO3} (\textbf{\ce{S} \underline{ajustado}})
		\onslide<6->
			\centering\ce{Hg(NO3)2 + Na2S -> HgS + \textbf{2}NaNO3}
	\end{overprint}
\end{frame}

\begin{frame}
	\frametitle{\ejerciciocmd}
	\framesubtitle{Resolución (\rom{2}): averiguar quién es el reactivo limitante}
	\structure{Reacción:} \ce{Hg(NO3)2 + Na2S -> HgS + 2NaNO3}
	\structure{Relación estequiométrica entre los reactivos:} $n(\ce{Hg(NO3)2}) = n(\ce{Na2S})$ (relación $1:1$)
	\visible<2>{
		\structure{Cálculo de número de moles de reactivos:} Empleamos $n=\rfrac{m}{Mm}$
		$$
			n(\ce{Hg(NO3)2}) = \frac{\SI{2}{\cancel\gram}}{\SI{324,6}{\cancel\gram\per\mol}} = \SI{6,161e-3}{\mol}
		$$
		$$
			n(\ce{Na2S}) = \frac{\SI{2}{\cancel\gram}}{\SI{78,04}{\cancel\gram\per\mol}} = \SI{2,562e-2}{\mol}
		$$
				}
	\visible<3->{
		\structure{Comparamos cantidades:}
		$$
			n(\ce{Hg(NO3)2}) = \SI{6,161e-3}{\mol} < \SI{2,562e-2}{\mol}=n(\ce{Na2S})
		$$
		Cuando deberíamos tener la relación $1:1$. Vemos que hay un \underline{déficit} de \underline{\ce{Hg(NO3)2}}, por lo que \underline{se agotará antes que el \ce{Na2S}}.
		\centering\myovalbox{\textcolor{yellow}{\textbf{\ce{Hg(NO3)2} es el \underline{reactivo limitante}}}}
				}
\end{frame}

\begin{frame}
	\frametitle{\ejerciciocmd}
	\framesubtitle{Resolución (\rom{3}): masa de \ce{HgS}}
	\structure{Reacción:} \ce{Hg(NO3)2 + Na2S -> HgS + 2NaNO3}
	\structure{Relación estequiométrica entre reactivo limitante y producto:} $n(\ce{Hg(NO3)2}) = n(\ce{HgS})$
	\visible<2->{
		\structure{Cálculo de la masa deseada:} Empleamos $n=\rfrac{m}{Mm}$
		\begin{overprint}
			\onslide<2>
				$$
					n(\ce{Hg(NO3)2}) = \overbrace{n(\ce{HgS})}^{n=\frac{m}{Mm}}
				$$
			\onslide<3>
				$$
					n(\ce{Hg(NO3)2}) = \frac{m(\ce{HgS})}{Mm(\ce{HgS})}
				$$
			\onslide<4>
				$$
					m(\ce{HgS}) = n(\ce{Hg(NO3)2})\cdot Mm(\ce{HgS})
				$$
			\onslide<5->
				$$
					m(\ce{HgS}) = n(\ce{Hg(NO3)2})\cdot Mm(\ce{HgS})\Rightarrow m(\ce{HgS}) = \SI{6,161e-3}{\cancel\mol}\cdot\SI{232,66}{\gram\per\cancel\mol}
				$$
		\end{overprint}
				}
	\visible<5->{
		\centering\tcbhighmath[boxrule=0.4pt,arc=4pt,colframe=red,drop fuzzy shadow=green]{m(\ce{HgS}) = \SI{1,43}{\gram}}
				}
\end{frame}
