En un recipiente completamente cerrado una mezcla de \SI{640}{\gram} de dióxido de azufre, \ce{SO2(g)}, y \SI{480}{\gram} de oxígeno, \ce{O2(g)}, se pasan sobre un catalizador. Al acabarse la reacción se formaron \SI{640}{\gram} de trióxido de azufre, \ce{SO3(g)}.
\begin{enumerate}
	\item Señale el reactivo limitante.
	\item Calcule el rendimiento de la reacción.
	\item Si no se pierden los reactivos, ¿cuántos moles de reactivos no reaccionaron?
\end{enumerate}
DATOS: $Mm(\si{\gram\per\mol})$: \ce{SO2}  \num{64}; \ce{SO3}  \num{80}; \ce{O2}  \num{32}.