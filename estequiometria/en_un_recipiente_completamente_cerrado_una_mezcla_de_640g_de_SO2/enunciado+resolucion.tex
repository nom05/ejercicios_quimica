\begin{frame}
	\frametitle{\ejerciciocmd}
	\framesubtitle{Enunciado}
	\textbf{
		Dadas las siguientes reacciones:
\begin{itemize}
    \item \ce{I2(g) + H2(g) -> 2 HI(g)}~~~$\Delta H_1 = \SI{-0,8}{\kilo\calorie}$
    \item \ce{I2(s) + H2(g) -> 2 HI(g)}~~~$\Delta H_2 = \SI{12}{\kilo\calorie}$
    \item \ce{I2(g) + H2(g) -> 2 HI(ac)}~~~$\Delta H_3 = \SI{-26,8}{\kilo\calorie}$
\end{itemize}
Calcular los parámetros que se indican a continuación:
\begin{description}%[label={\alph*)},font={\color{red!50!black}\bfseries}]
    \item[\texttt{a)}] Calor molar latente de sublimación del yodo.
    \item[\texttt{b)}] Calor molar de disolución del ácido yodhídrico.
    \item[\texttt{c)}] Número de calorías que hay que aportar para disociar en sus componentes el yoduro de hidrógeno gas contenido en un matraz de \SI{750}{\cubic\centi\meter} a \SI{25}{\celsius} y \SI{800}{\torr} de presión.
\end{description}
\resultadocmd{\SI{12,8}{\kilo\calorie}; \SI{-13,0}{\kilo\calorie}; \SI{12,9}{\calorie}}

		}
\end{frame}

\begin{frame}
	\frametitle{\ejerciciocmd}
	\framesubtitle{Datos del problema}
	\begin{center}
		{\huge
			¿Reactivo limitante?, ¿$\eta$?, ¿$n$ de reactivos sin reaccionar?\\[.4cm]
		}
		\tcbhighmath[boxrule=0.4pt,arc=4pt,colframe=green,drop fuzzy shadow=blue]{m(\ce{SO2})=\SI{640}{\gram}}\quad
		\tcbhighmath[boxrule=0.4pt,arc=4pt,colframe=blue,drop fuzzy shadow=green]{m(\ce{O2})=\SI{480}{\gram}}\quad
		\tcbhighmath[boxrule=0.4pt,arc=4pt,colframe=red,drop fuzzy shadow=black]{m(\ce{SO3})=\SI{640}{\gram}}\\[.4cm]
		\tcbhighmath[boxrule=0.4pt,arc=4pt,colframe=green,drop fuzzy shadow=blue]{Mm(\ce{SO2})=\SI{64}{\gram}}\quad
		\tcbhighmath[boxrule=0.4pt,arc=4pt,colframe=blue,drop fuzzy shadow=green]{Mm(\ce{O2})=\SI{32}{\gram}}\quad
		\tcbhighmath[boxrule=0.4pt,arc=4pt,colframe=red,drop fuzzy shadow=black]{Mm(\ce{SO3})=\SI{80}{\gram}}\\[.4cm]
	\end{center}
\end{frame}

\begin{frame}
	\frametitle{\ejerciciocmd}
	\framesubtitle{Resolución (\rom{1}): reactivo limitante}
	\begin{overprint}
		\onslide<1>
			\structure{Reacción química (sin ajustar):} \ce{SO2(g) + O2(g) -> SO3(g)}\\[.6cm]
		\onslide<2>
			\structure{Reacción química (ajustada):} \ce{SO2(g) + 1/2O2(g) -> SO3(g)}\\[.6cm]
		\onslide<3->
			\structure{Reacción química (ajustada):}
				\ce{
						\colorbox{yellow}{\textbf{1}}
							SO2(g) + 
						\colorbox{orange}{\textbf{\ce{1/2}}}
							O2(g) -> 
						\colorbox{blue}{\color{white}\textbf{1}}
							SO3(g)
					}\\[.6cm]
	\end{overprint}
	\begin{overprint}
		\onslide<1>
			\structure{Datos del problema:}
			\begin{center}
				\begin{tabular}{cSSS}
					\toprule
						Propiedad				&	{\ce{SO2(g)}}	&	{\ce{O2(g)}}	&	{\ce{SO3(g)}}	\\
					\midrule
						Mm (\si{\gram\per\mol})	&	64				&	32				&	80				\\
						m (\si{\gram})			&	640				&	480				&	640				\\
					\bottomrule
				\end{tabular}
			\end{center}
		\onslide<2>
			\structure{Calculamos el n"o de moles del problema:} $n = \rfrac{m}{Mm}$
			\begin{center}
				\begin{tabular}{cSSS}
					\toprule
						Propiedad				&	{\ce{SO2(g)}}	&	{\ce{O2(g)}}	&	{\ce{SO3(g)}}	\\
					\midrule
						Mm (\si{\gram\per\mol})	&	64				&	32				&	80				\\
						m (\si{\gram})			&	640				&	480				&	640				\\
						n (\si{\mol})			&	10				&	15				&	8				\\
					\bottomrule
				\end{tabular}
			\end{center}
		\onslide<3>
			\structure{Esos n"os de moles hay que distinguirlos entre iniciales y finales, se añaden los que reaccionan:} Usamos la estequiometría de la reacción
			\begin{center}
				\begin{tabular}{cSSS}
					\toprule
						Propiedad						&	{\ce{SO2(g)}}	&	{\ce{O2(g)}}		&	{\ce{SO3(g)}}	\\
					\midrule
						Mm (\si{\gram\per\mol})			&	64				&	32					&	80				\\
						m (\si{\gram})					&	640				&	480					&	640				\\
						n (\si{\mol})					&	10				&	15					&	8				\\
						n iniciales (\si{\mol})			&	10				&	15					&	0				\\
						n reaccionan (\si{\mol})		&	{$-\colorbox{yellow}{\textbf{1}}x$}							&
															{$-\colorbox{orange}{\textbf{\ce{1/2}}}x$}					&	
															{$+\colorbox{blue}{\color{white}\textbf{1}}x$}				\\
						n finales teóricos (\si{\mol})	&	{$10-x$}		&	{$15-\num{,5}x$}	&	{$x$}			\\
					\bottomrule
				\end{tabular}
			\end{center}
				\onslide<4>
					\structure{Tenemos dos alternativas para el reactivo limitante:} suponemos las dos situaciones y usamos la que es químicamente posible.
					\begin{center}
						\begin{tabular}{cSSSl}
							\toprule
								Propiedad												&	{\ce{SO2(g)}}	&	{\ce{O2(g)}}		&	{\ce{SO3(g)}}	&										\\
							\midrule
								Mm (\si{\gram\per\mol})									&	64				&	32					&	80				&										\\
								m (\si{\gram})											&	640				&	480					&	640				&										\\
								n (\si{\mol})											&	10				&	15					&	8				&										\\
								n iniciales (\si{\mol})									&	10				&	15					&	0				&										\\
								n reaccionan (\si{\mol})								&	{$-\colorbox{yellow}{\textbf{1}}x$}							&
																							{$-\colorbox{orange}{\textbf{\ce{1/2}}}x$}					&	
																							{$+\colorbox{blue}{\color{white}\textbf{1}}x$}				&										\\
								n finales teóricos (\si{\mol}) \textbf{posibilidad 1}	&	{$10-x=0$}		&	{$15-\num{,5}x$}	&	{$x$}			&										\\
								Si $x=10$												&	0				&	{\colorbox{green}{\num{10}}}	&
																																		10				&	\colorbox{green}{\textbf{POSIBLE}}	\\
								n finales teóricos (\si{\mol}) \textbf{posibilidad 2}	&	{$10-x$}		&	{$15-\num{,5}x=0$}	&	{$x$}			&										\\
								Si $x=30$												&	{\colorbox{red}{\cancel{\num{-20}}}}	&
																												0					&	30				&	\colorbox{red}{\textbf{IMPOSIBLE}}	\\
							\bottomrule
						\end{tabular}
						\tcbhighmath[boxrule=0.4pt,arc=4pt,colframe=green,drop fuzzy shadow=blue]{\text{El reactivo limitante es el \ce{SO2}}} porque es el que se agota antes.
					\end{center}
	\end{overprint}
\end{frame}

\begin{frame}
	\frametitle{\ejerciciocmd}
	\framesubtitle{Resolución (\rom{2}): rendimiento de la reacción}
	\structure{Reacción química (ajustada):} \ce{SO2(g) + 1/2O2(g) -> SO3(g)}\\[.6cm]
	\begin{center}
		\begin{tabular}{cSSSl}
			\toprule
				Propiedad				&	{\ce{SO2(g)}}	&	{\ce{O2(g)}}		&	{\ce{SO3(g)}}	&										\\
			\midrule
				n (\si{\mol})			&	10				&	15					&	8				&	DATOS DEL EJERCICIO					\\
				n finales (\si{\mol})	&	0				&	10					&	10				&	TEÓRICOS (calculados previamente)	\\
				n finales (\si{\mol})	&	{?}				&	{?}					&	8				&	REALES								\\
			\bottomrule
		\end{tabular}
	\end{center}
	\structure{Definición de rendimiento:} usando el \underline{n"o de moles} de alguno de os \underline{productos}
	$$
		\eta = \frac{n_{\text{reales}}}{n_{\text{teóricos}}}\times 100
	$$
	\structure{Usando los datos de la tabla:}
	$$
		\tcbhighmath[boxrule=0.4pt,arc=4pt,colframe=red,drop fuzzy shadow=black]{
			\eta = \frac{8}{10}\times 100 = \SI{80}{\percent}
																				}
	$$
\end{frame}

\begin{frame}
	\frametitle{\ejerciciocmd}
	\framesubtitle{Resolución (\rom{3}): cantidad de reactivos sin reaccionar}
	\structure{Reacción química (ajustada):} \ce{SO2(g) + 1/2O2(g) -> SO3(g)}\\[.6cm]
	\begin{overprint}
		\onslide<1>
			\begin{center}
				\begin{tabular}{cSSSl}
					\toprule
						Propiedad					&	{\ce{SO2(g)}}	&	{\ce{O2(g)}}		&	{\ce{SO3(g)}}	&										\\
					\midrule
						n (\si{\mol})				&	10				&	15					&	8				&	DATOS DEL EJERCICIO					\\
						n finales (\si{\mol})		&	0				&	10					&	10				&	TEÓRICOS (calculados previamente)	\\
						n finales (\si{\mol})		&	{?}				&	{?}					&	8				&	REALES								\\
						n reaccionan (\si{\mol})	&	{$-x$}			&	{$-\ce{1/2}x$}		&	{$+x$}			&										\\
					\bottomrule
				\end{tabular}
			\end{center}
			\structure{Si tenemos \SI{8}{\mol} de producto únicamente ...} tenemos que usar $x=8$.
		\onslide<2->
		\begin{center}
			\begin{tabular}{cSSSl}
				\toprule
					Propiedad					&	{\ce{SO2(g)}}	&	{\ce{O2(g)}}		&	{\ce{SO3(g)}}	&										\\
				\midrule
					n iniciales (\si{\mol})		&	10				&	15					&	0				&	DATOS DEL EJERCICIO					\\
					n finales (\si{\mol})		&	{?}				&	{?}					&	8				&	REALES								\\
					n reaccionan (\si{\mol})	&	{$-x$}			&	{$-\ce{1/2}x$}		&	{$+x$}			&	(Según la estequiometría)			\\
					n reaccionan (\si{\mol})	&	-8				&	-4					&	+8				&	$x=8$								\\
					n finales (\si{\mol})		&	{$10-8=2$}		&	{15-4=11}			&	8				&	Por el rendimiento de reacción	\\
				\toprule
			\end{tabular}
		\end{center}
	\end{overprint}
	\visible<2->{
		Como estamos en un recipiente cerrado y no puede entrar ni salir nada solo ocurre la reacción y si no se completa la única solución es que queden sin reaccionar los reactivos. En otras reacciones, un rendimiento inferior al \SI{100}{\percent} puede indicar también que se ha perdido parte de los reactivos o del producto.
		\begin{center}
			\tcbhighmath[boxrule=0.4pt,arc=4pt,colframe=green,drop fuzzy shadow=blue]{n_{\text{sin reaccionar}}(\ce{SO2})=\SI{2}{\mol}}\qquad
			\tcbhighmath[boxrule=0.4pt,arc=4pt,colframe=blue,drop fuzzy shadow=green]{n_{\text{sin reaccionar}}(\ce{O2})=\SI{11}{\mol}}
		\end{center}

				}
\end{frame}