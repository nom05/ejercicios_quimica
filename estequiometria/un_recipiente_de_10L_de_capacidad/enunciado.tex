Un recipiente de \SI{10,0}{\liter} de capacidad contiene \SI{30,0}{\gram} de oxígeno (\ce{O2}) a \SI{77}{\celsius}. Se abre el recipiente y se deja salir el gas libremente (no entra nada de aire). Si la presión atmosférica es de \SI{1}{\atm} calcule:
\begin{enumerate}[label={\alph*)},font={\color{red!50!black}\bfseries}]
	\item La masa de oxígeno que sale del recipiente.
	\item La temperatura a la que debería estar el oxígeno que queda en el recipiente para que se encontrase a la presión inicial.
\end{enumerate}
\resultadocmd{
			\SI{18,85}{\gram};
			\SI{669,38}{\celsius}
}
