Se hacen reaccionar \SI{3,6}{\gram} de sulfuro de calcio con \SI{10}{\milli\liter} de una disolución de ácido clorhídrico del \SI{36}{\percent} en peso y $d$: \SI{1,19}{\gram\per\milli\liter}, para producir cloruro de calcio y sulfuro de hidrógeno. Si el Rto de la reacción fue del \SI{88}{\percent}, calcular:
\begin{enumerate}[label={\alph*)},font={\color{red!50!black}\bfseries}]
	 \item Cantidad de cloruro de calcio que se forma.
 	\item $V$ de sulfuro de hidrógeno recogido sobre agua a \SI{25}{\celsius} y a una $P$ de \SI{746}{\torr}. La $P_v$ del agua a \SI{25}{\celsius} es \SI{24}{\torr}.
 \end{enumerate}
\resultadocmd{
            \SI{4,9}{\gram};
            \SI{1,1}{\liter}
}
