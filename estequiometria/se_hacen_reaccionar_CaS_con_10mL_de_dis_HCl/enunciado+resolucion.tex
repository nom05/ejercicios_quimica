\begin{frame}
    \frametitle{\ejerciciocmd}
    \framesubtitle{Enunciado}
    \textbf{
		Dadas las siguientes reacciones:
\begin{itemize}
    \item \ce{I2(g) + H2(g) -> 2 HI(g)}~~~$\Delta H_1 = \SI{-0,8}{\kilo\calorie}$
    \item \ce{I2(s) + H2(g) -> 2 HI(g)}~~~$\Delta H_2 = \SI{12}{\kilo\calorie}$
    \item \ce{I2(g) + H2(g) -> 2 HI(ac)}~~~$\Delta H_3 = \SI{-26,8}{\kilo\calorie}$
\end{itemize}
Calcular los parámetros que se indican a continuación:
\begin{description}%[label={\alph*)},font={\color{red!50!black}\bfseries}]
    \item[\texttt{a)}] Calor molar latente de sublimación del yodo.
    \item[\texttt{b)}] Calor molar de disolución del ácido yodhídrico.
    \item[\texttt{c)}] Número de calorías que hay que aportar para disociar en sus componentes el yoduro de hidrógeno gas contenido en un matraz de \SI{750}{\cubic\centi\meter} a \SI{25}{\celsius} y \SI{800}{\torr} de presión.
\end{description}
\resultadocmd{\SI{12,8}{\kilo\calorie}; \SI{-13,0}{\kilo\calorie}; \SI{12,9}{\calorie}}

	}
\end{frame}

\begin{frame}
    \frametitle{\ejerciciocmd}
    \framesubtitle{Datos del problema}
    {\huge $$
        m(\ce{CaCl2})?\quad
        V(\ce{H2S})?
    $$}
    Reacción: 
    \tcbhighmath[boxrule=0.4pt,arc=4pt,colframe=blue,drop fuzzy shadow=red]{\ce{CaS}(s)}\ce{+} 
    \tcbhighmath[boxrule=0.4pt,arc=4pt,colframe=red,drop fuzzy shadow=yellow]{\ce{HCl(ac)}}
    \ce{->[\SI{88}{\percent}]}
    \tcbhighmath[boxrule=0.4pt,arc=4pt,colframe=green,drop fuzzy shadow=blue]{\ce{CaCl2(ac)}}\ce{+}
    \tcbhighmath[boxrule=0.4pt,arc=4pt,colframe=blue,drop fuzzy shadow=green]{\ce{H2S(g)}}
        $$
            \tcbhighmath[boxrule=0.4pt,arc=4pt,colframe=blue,drop fuzzy shadow=red]{m(\ce{CaS}) = \SI{3,6}{\gram}}
        $$
        $$
            \tcbhighmath[boxrule=0.4pt,arc=4pt,colframe=red,drop fuzzy shadow=yellow]{V(\ce{HCl}) = \SI{10}{\milli\liter} = \SI{0,010}{\liter}}\quad
            \tcbhighmath[boxrule=0.4pt,arc=4pt,colframe=red,drop fuzzy shadow=yellow]{\text{pureza de \ce{HCl}} = \SI{36}{\percent}}\quad
            \tcbhighmath[boxrule=0.4pt,arc=4pt,colframe=red,drop fuzzy shadow=yellow]{d(\ce{HCl}) = \SI{1,19}{\gram\per\milli\liter}}
        $$
        $$
            Rto = \SI{88}{\percent}
        $$
        $$
            \tcbhighmath[boxrule=0.4pt,arc=4pt,colframe=blue,drop fuzzy shadow=green]{T(\ce{H2S}) = \SI{298,15}{\kelvin}}\quad
            \tcbhighmath[boxrule=0.4pt,arc=4pt,colframe=blue,drop fuzzy shadow=green]{P(\ce{H2S}) = 746-24~\si{\torr}}
        $$
    \visible<2-|handout:0>{Y las masas moleculares\ldots
        $$
            \tcbhighmath[boxrule=0.4pt,arc=4pt,colframe=blue,drop fuzzy shadow=red]{Mm(\ce{CaS})=\SI{72,143}{\gram\per\mol}} 
            \tcbhighmath[boxrule=0.4pt,arc=4pt,colframe=red,drop fuzzy shadow=yellow]{Mm(\ce{HCl})=\SI{36,46}{\gram\per\mol}}
            \tcbhighmath[boxrule=0.4pt,arc=4pt,colframe=green,drop fuzzy shadow=blue]{Mm(\ce{CaCl2})=\SI{110}{\gram\per\mol}}
        $$
    }
\end{frame}

\begin{frame}
    \frametitle{\ejerciciocmd}
    \framesubtitle{Resolución (\rom{1}): Ajuste de la reacción}
    \structure{Reacción:}
    \visible<1-|handout:0>{
        \centering\ce{CaS + H}\tcbhighmath[boxrule=0.4pt,arc=4pt,colframe=blue,drop fuzzy shadow=red]{\ce{Cl}} \ce{-> Ca}\tcbhighmath[boxrule=0.4pt,arc=4pt,colframe=blue,drop fuzzy shadow=red]{\ce{Cl2(ac)}} \ce{+ H2S}\\[.3cm]
    }
    \visible<2-|handout:0>{
        \centering\ce{CaS + }\tcbhighmath[boxrule=0.4pt,arc=4pt,colframe=blue,drop fuzzy shadow=red]{2\ce{HCl}}\ce{-> Ca}\tcbhighmath[boxrule=0.4pt,arc=4pt,colframe=blue,drop fuzzy shadow=red]{\ce{Cl2}}\ce{+ H2S}\\[.6cm]
    }
    \visible<3-|handout:0>{
        \centering\tcbhighmath[boxrule=0.4pt,arc=4pt,colframe=blue,drop fuzzy shadow=red]{\ce{CaS + 2HCl -> CaCl2 + H2S}}
    }
\end{frame}

\begin{frame}
    \frametitle{\ejerciciocmd}
    \framesubtitle{Resolución (\rom{2}): averiguar qué reactivo es el limitante}
    \structure{Reacción:} \ce{CaS + 2HCl -> CaCl2 + H2S}
    \structure{Número de moles de \ce{CaS}:}
    $$
        n(\ce{CaS}) = \frac{m(\ce{CaS})}{Mm(\ce{CaS})}\Rightarrow n(\ce{CaS}) = \frac{\SI{3,6}{\gram}}{\SI{72,143}{\gram\per\mol}} = \SI{0,050}{\mol}
    $$
    \visible<2->{
        \structure{Número de moles de \ce{HCl}:}
        \begin{overprint}
            \onslide<2>
                $$
                    m_{\text{puro}}(\ce{HCl}) < m_{\text{comercial}}(\ce{HCl}) 
                $$
            \onslide<3>
                $$
                    m_{\text{puro}}(\ce{HCl}) = \frac{36}{100} m_{\text{comercial}}(\ce{HCl}) 
                $$
            \onslide<4>
                $$
                    \overbrace{m_{\text{puro}}(\ce{HCl})}^{n=\frac{m}{Mm}\Rightarrow m=n\cdot Mm} = \frac{36}{100} \underbrace{m_{\text{comercial}}(\ce{HCl})}_{d=\frac{m}{V}\Rightarrow m=d\cdot V}
                $$
            \onslide<5>
                $$
                    n(\ce{HCl})\cdot Mm(\ce{HCl}) = \frac{36}{100} d_{\text{comercial}}(\ce{HCl})\cdot V_{\text{comercial}}(\ce{HCl})
                $$
            \onslide<6>
                $$
                    n(\ce{HCl}) = \frac{36}{100} \frac{d_{\text{comercial}}(\ce{HCl})\cdot V_{\text{comercial}}(\ce{HCl})}{Mm(\ce{HCl})}
                $$
            \onslide<7->
                $$
                    n(\ce{HCl}) = \frac{36}{100} \frac{\SI{1,19}{\cancel\gram\per\cancel\milli\liter}\cdot \SI{10}{\cancel\milli\liter}}{\SI{36,46}{\cancel\gram\per\mol}} = \SI{0,117}{\mol}
                $$
        \end{overprint}
    }
    \visible<8->{
        \structure{Según estequiometría:}
        $$
            \frac{n(HCl)}{n(CaS)}=\frac{2}{1}
        $$
        \structure{Según las cantidades del problema:}
        $$
            \frac{n(HCl)}{n(CaS)}=\frac{0,117}{0,050}=\frac{2,34}{1}
        $$
                }
    \visible<9->{
        \structure{Comparando:} (lo que reacciona de \ce{HCl}) $2<2,34$ (lo que tenemos de \ce{HCl}). 
        {\centering\myovalbox{\textcolor{white}{\ce{CaS} es el reactivo limitante por el exceso de \ce{HCl}}}}
                }
\end{frame}

\begin{frame}
    \frametitle{\ejerciciocmd}
    \framesubtitle{Resolución (\rom{3}): masa de \ce{CaCl2}}
    \structure{Según estequiometría:} $n(\ce{CaS}) = n_{\text{teo}}(\ce{CaCl2})$
    \structure{Rendimiento de reacción:}
    \begin{overprint}
        \onslide<2->
            $$
                \frac{88}{100}n_{\text{teo}}(\ce{CaCl2}) = n_{\text{real}}(\ce{CaCl2})\Rightarrow n_{\text{teo}}(\ce{CaCl2}) = \frac{100}{88}n_{\text{real}}(\ce{CaCl2})
            $$
    \end{overprint}
    \visible<3->{
        \structure{Uniendo ambas expresiones:}
        \begin{overprint}
            \onslide<3>
                $$
                    n(\ce{CaS}) = \frac{100}{88}n_{\text{real}}(\ce{CaCl2})
                $$
            \onslide<4>
                $$
                    \overbrace{n(\ce{CaS})}^{n=\frac{m}{Mm}} = \frac{100}{88}\underbrace{n_{\text{real}}(\ce{CaCl2})}_{n=\frac{m}{Mm}}
                $$
            \onslide<5>
                $$
                    \frac{m(\ce{CaS})}{Mm(\ce{CaS})} = \frac{100}{88}\frac{m_{\text{real}}(\ce{CaCl2})}{Mm(\ce{CaCl2})}
                $$
            \onslide<6->
                $$
                    m_{\text{real}}(\ce{CaCl2}) = \frac{88}{100}m(\ce{CaS})\cdot\frac{Mm(\ce{CaCl2})}{Mm(\ce{CaS})}
                $$
        \end{overprint}
                }
    \visible<7->{
        \structure{Sustituyendo por los valores:}
        $$
            m_{\text{real}}(\ce{CaCl2}) = \frac{88}{100}\SI{3,6}{\gram}\cdot\frac{\SI{110,984}{\cancel\gram\per\cancel\mol}}{\SI{72,143}{\cancel\gram\per\cancel\mol}}
        $$
        $$
            \tcbhighmath[boxrule=0.4pt,arc=4pt,colframe=blue,drop fuzzy shadow=red]{m_{\text{real}}(\ce{CaCl2}) = \SI{4,9}{\gram}}
        $$
                }
\end{frame}

\begin{frame}
    \frametitle{\ejerciciocmd}
    \framesubtitle{Resolución (\rom{4}): volumen de \ce{H2S}}
    \structure{Según la estequiometría:}  $n(\ce{CaS}) = n_{\text{teo}}(\ce{H2S})$
    \structure{Teniendo en cuenta el rendimiento de reacción (análogo a \ce{CaCl2}):}
    \begin{overprint}
        \onslide<1>
            $$
                \overbrace{n(\ce{CaS})}^{n=\frac{m}{Mm}} = \frac{100}{88}\underbrace{n_{\text{real}}(\ce{H2S})}_{PV=nRT\Rightarrow n=\frac{PV}{RT}}
            $$
        \onslide<2>
            $$
                \frac{m(\ce{CaS})}{Mm(\ce{CaS})} = \frac{100}{88}\frac{P(\ce{H2S})\cdot V(\ce{H2S})}{R\cdot T(\ce{H2S})}
            $$
        \onslide<3->
            $$
                V(\ce{H2S})
                =
                \frac{88}{100}\frac{m(\ce{CaS})}{Mm(\ce{CaS})}\cdot\frac{R\cdot T(\ce{H2S})}{P(\ce{H2S})}
            $$
    \end{overprint}
    \visible<4->{
        \begin{center}
            \alert{¡¡Recordad que tenemos una mezcla de gases: \ce{H2S}+agua!!}
        \end{center}
                }
    \visible<5->{
        $$
            V(\ce{H2S})
            =
            \frac{88}{100}\frac{\SI{3,6}{\cancel\gram}}{\SI{72,143}{\cancel\gram\per\cancel\mol}}\cdot\frac{\SI{0,082}{\cancel\atm\liter\per\cancel\mol\per\cancel\kelvin}\cdot \SI{298,15}{\cancel\kelvin}}{\colorbox{red}{\color{white}$\frac{746-24}{760}$}~\si{\cancel\atm}}
        $$
        $$
            \tcbhighmath[boxrule=0.4pt,arc=4pt,colframe=blue,drop fuzzy shadow=red]{V(\ce{H2S})
            =
            \SI{1,1}{\liter}}
        $$
                }
\end{frame}
