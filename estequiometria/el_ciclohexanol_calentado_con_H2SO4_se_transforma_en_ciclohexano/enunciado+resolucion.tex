\begin{frame}
	\frametitle{\ejerciciocmd}
	\framesubtitle{Enunciado}
	\textbf{
		Dadas las siguientes reacciones:
\begin{itemize}
    \item \ce{I2(g) + H2(g) -> 2 HI(g)}~~~$\Delta H_1 = \SI{-0,8}{\kilo\calorie}$
    \item \ce{I2(s) + H2(g) -> 2 HI(g)}~~~$\Delta H_2 = \SI{12}{\kilo\calorie}$
    \item \ce{I2(g) + H2(g) -> 2 HI(ac)}~~~$\Delta H_3 = \SI{-26,8}{\kilo\calorie}$
\end{itemize}
Calcular los parámetros que se indican a continuación:
\begin{description}%[label={\alph*)},font={\color{red!50!black}\bfseries}]
    \item[\texttt{a)}] Calor molar latente de sublimación del yodo.
    \item[\texttt{b)}] Calor molar de disolución del ácido yodhídrico.
    \item[\texttt{c)}] Número de calorías que hay que aportar para disociar en sus componentes el yoduro de hidrógeno gas contenido en un matraz de \SI{750}{\cubic\centi\meter} a \SI{25}{\celsius} y \SI{800}{\torr} de presión.
\end{description}
\resultadocmd{\SI{12,8}{\kilo\calorie}; \SI{-13,0}{\kilo\calorie}; \SI{12,9}{\calorie}}

	}
\end{frame}

\begin{frame}
	\frametitle{\ejerciciocmd}
	\framesubtitle{Datos del problema}
	\begin{center}
		{\huge¿$m(\ce{C6H11OH})$?} \\[.4cm]
		\ce{C6H11OH(l) ->[$\SI{83}{\percent}$] C6H10(l) + H2O(l)}\\[.4cm]
		\tcbhighmath[boxrule=0.4pt,arc=4pt,colframe=black,drop fuzzy shadow=yellow]{\eta = \SI{83}{\percent}}\\[.4cm]
		\tcbhighmath[boxrule=0.4pt,arc=4pt,colframe=green,drop fuzzy shadow=orange]{m(\ce{C6H10}) = \SI{25}{\gram}}\\[.4cm]
		Pero también ...\\[.4cm]
		\tcbhighmath[boxrule=0.4pt,arc=4pt,colframe=blue,drop fuzzy shadow=green]{Mm(\ce{C6H11OH}) = \SI{100,16}{\gram\per\mol}}\quad
		\tcbhighmath[boxrule=0.4pt,arc=4pt,colframe=green,drop fuzzy shadow=orange]{Mm(\ce{C6H10}) = \SI{82,14}{\gram\per\mol}}\\[.4cm]
	\end{center}
\end{frame}

\begin{frame}
	\frametitle{\ejerciciocmd}
	\framesubtitle{Resolución}
	\structure{La reacción ya está ajustada (comprobar siempre):}
	\begin{center}
		\ce{C6H11OH(l) ->[$\SI{83}{\percent}$] C6H10(l) + H2O(l)}
	\end{center}
	\structure{Relación estequiométrica:} $n_{\text{real}}(\ce{C6H11OH}) = n_{\text{teórico}}(\ce{C6H10})$
	\structure{Solo el \SI{83}{\percent} del n"o de moles de \ce{C6H11OH} se convierte en \ce{C6H10},} por tanto:
	$$
		\overbrace{\num{,83}}^{\SI{83}{\percent}}\times n_{\text{real}}(\ce{C6H11OH}) = n_{\text{real}}(\ce{C6H10})\overset{n = \rfrac{m}{Mm}}{\Rightarrow}
		\num{,83}\vdot\frac{m(\ce{C6H11OH})}{Mm(\ce{C6H11OH})} = \frac{m(\ce{C6H10})}{Mm(\ce{C6H10})}
	$$
	$$
		m(\ce{C6H11OH}) = \frac{m(\ce{C6H10})\vdot Mm(\ce{C6H11OH})}{\num{,83}\vdot Mm(\ce{C6H10})}
	$$
	$$
		\tcbhighmath[boxrule=0.4pt,arc=4pt,colframe=blue,drop fuzzy shadow=green]{m(\ce{C6H11OH}) = \SI{36,73}{\gram}}
	$$
\end{frame}