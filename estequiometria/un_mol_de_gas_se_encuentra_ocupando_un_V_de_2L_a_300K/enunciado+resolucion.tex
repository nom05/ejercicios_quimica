\begin{frame}
    \frametitle{\ejerciciocmd}
    \framesubtitle{Enunciado}
    \textbf{
		Una reacción tiene una constante de velocidad de \SI{,017}{\per\second} a \SI{298}{\kelvin} y una energía libre de activación del \SI{27,235}{\kilo\joule\per\mol}. La adición de un catalizador disminuye dicha energía de activación hasta un \SI{33}{\percent} de su valor inicial. Calcule la nueva constante de velocidad.
\resultadocmd{ \SI{26,86}{\per\second} }

	}
\end{frame}

\begin{frame}
    \frametitle{\ejerciciocmd}
    \framesubtitle{Datos del problema}
    \begin{center}
        {\large ¿$P$ para gas ideal y para reales?}
        {\large ¿Cuál se desvía más del comportamiento ideal?}
    \end{center}
    $$
        \tcbhighmath[boxrule=0.4pt,arc=4pt,colframe=blue,drop fuzzy shadow=red]{n = \SI{1}{\mol}}\quad
        \tcbhighmath[boxrule=0.4pt,arc=4pt,colframe=blue,drop fuzzy shadow=red]{V = \SI{2}{\liter}}\quad
        \tcbhighmath[boxrule=0.4pt,arc=4pt,colframe=blue,drop fuzzy shadow=red]{T = \SI{300}{\kelvin}}
    $$
    \begin{center}
        \begin{tabular}{lSS}
            \toprule
                     &\textbf{a}&\textbf{b}\\
                     &\si{\square\liter\atm\per\square\mol}&\si{\liter\per\mol}\\
            \midrule
            \ce{Cl2} &   6,49   & 0,0562   \\
            \ce{CO2} &   3,59   & 0,0427   \\
            \ce{CO}  &   1,49   & 0,0399   \\
            \bottomrule
        \end{tabular}
    \end{center}
\end{frame}

\begin{frame}
    \frametitle{\ejerciciocmd}
    \framesubtitle{Resolución (\rom{1}): $P$ de gas ideal}
    \structure{Ecuación de gases ideales:} Recordamos ecuación general de los gases ideales, despejamos y sustituimos.
    \begin{overprint}
        \onslide<1>
            $$
                P\vdot V=n\vdot R\vdot T
            $$
        \onslide<2>
            $$
                P = \frac{n\vdot R\vdot T}{V}
            $$
        \onslide<3->
            $$
                P = \frac{\SI{1}{\cancel\mol}\vdot\SI{0,082}{\atm\cancel\liter\per\cancel\mol\per\cancel\kelvin}\vdot\SI{300}{\cancel\kelvin}}{\SI{2}{\cancel\liter}}
            $$
    \end{overprint}
    \visible<3->{
        $$
            \tcbhighmath[boxrule=0.4pt,arc=4pt,colframe=blue,drop fuzzy shadow=red]{P = \SI{12,3}{\atm}}
        $$
        \structure{Tened en mente:}
        $$
            \frac{P\vdot V}{n\vdot R\vdot T} = 1
        $$
                }
    \visible<4->{
        \structure{Ecuación de Van der Waals:}
        $$
            \frac{P\vdot V}{n\vdot R\vdot T} = \underbrace{\frac{V}{V-n\vdot b}-\frac{n\vdot a}{R\vdot T\vdot V}}_{\tcbhighmath[boxrule=0.4pt,arc=4pt,colframe=blue,drop fuzzy shadow=red]{1\text{ si el gas es ideal}}}
        $$
                }
\end{frame}

\begin{frame}
    \frametitle{\ejerciciocmd}
    \framesubtitle{Resolución (\rom{2}): Gases reales}
    \structure{Operamos:} añadimos columnas calculando los factores que distancian al gas de la idealidad.\\[.4cm]
    \begin{overprint}
        \onslide<2>
            \begin{center}
                \begin{tabular}{lSS}
                    \toprule
                    &\textbf{a}&\textbf{b}\\
                    &\si{\square\liter\atm\per\square\mol}&\si{\liter\per\mol}\\
                    \midrule
                    \ce{Cl2} &   6,49   & 0,0562   \\
                    \ce{CO2} &   3,59   & 0,0427   \\
                    \ce{CO}  &   1,49   & 0,0399   \\
                    \bottomrule
                \end{tabular}
            \end{center}
        \onslide<3>
            \begin{center}
                \begin{tabular}{lSSS}
                    \toprule
                    &\textbf{a}
                    &\textbf{b}
                    &$\frac{V}{V-n\vdot b}$\\
                    &\si{\square\liter\atm\per\square\mol}
                    &\si{\liter\per\mol}
                    & \\
                    \midrule
                    \ce{Cl2} &   6,49   & 0,0562 & 1,0289 \\
                    \ce{CO2} &   3,59   & 0,0427 & 1,0218 \\
                    \ce{CO}  &   1,49   & 0,0399 & 1,0204 \\
                    \bottomrule
                \end{tabular}
            \end{center}
        \onslide<4>
            \begin{center}
                \begin{tabular}{lSSSS}
                    \toprule
                    &\textbf{a}
                    &\textbf{b}
                    &$\frac{V}{V-n\vdot b}$
                    &$\frac{n\vdot a}{R\vdot T\vdot V}$\\
                    &\si{\square\liter\atm\per\square\mol}
                    &\si{\liter\per\mol}
                    &
                    &\\
                    \midrule
                    \ce{Cl2} &   6,49   & 0,0562 & 1,0289 & 0,1319 \\
                    \ce{CO2} &   3,59   & 0,0427 & 1,0218 & 0,0730 \\
                    \ce{CO}  &   1,49   & 0,0399 & 1,0204 & 0,0303 \\
                    \bottomrule
                \end{tabular}
            \end{center}
        \onslide<5>
            \begin{center}
                \begin{tabular}{lSSSSS}
                    \toprule
                    &\textbf{a}
                    &\textbf{b}
                    &$\frac{V}{V-n\vdot b}$
                    &$\frac{n\vdot a}{R\vdot T\vdot V}$
                    &$\frac{V}{V-n\vdot b}\text{--}\frac{n\vdot a}{R\vdot T\vdot V}$\\
                    &\si{\square\liter\atm\per\square\mol}
                    &\si{\liter\per\mol}
                    &
                    &
                    &\\
                    \midrule
                    \ce{Cl2} &   6,49   & 0,0562 & 1,0289 & 0,1319 & 0,8970 \\
                    \ce{CO2} &   3,59   & 0,0427 & 1,0218 & 0,0730 & 0,9488 \\
                    \ce{CO}  &   1,49   & 0,0399 & 1,0204 & 0,0303 & 0,9901 \\
                    \bottomrule
                \end{tabular}
            \end{center}
        \onslide<6->
            \begin{center}
                \begin{tabular}{lSSSSSS}
                    \toprule
                    gas
                    &\textbf{a}
                    &\textbf{b}
                    &$\frac{V}{V-n\vdot b}$
                    &$\frac{n\vdot a}{R\vdot T\vdot V}$
                    &$\frac{V}{V-n\vdot b}\text{--}\frac{n\vdot a}{R\vdot T\vdot V}$
                    &$P\text{ real}$ \\
                    &\si{\square\liter\atm\per\square\mol}
                    &\si{\liter\per\mol}
                    &
                    &
                    &
                    &\si{\atm}\\
                    \midrule
                    \ce{Cl2} &   6,49   & 0,0562 & 1,0289 & 0,1319 & 0,8970 & 11,03 \\
                    \ce{CO2} &   3,59   & 0,0427 & 1,0218 & 0,0730 & 0,9488 & 11,67 \\
                    \ce{CO}  &   1,49   & 0,0399 & 1,0204 & 0,0303 & 0,9901 & 12,18 \\
                    \bottomrule
                \end{tabular}
            \end{center}
    \end{overprint}
    \visible<7->{
        \structure{¿Cuál es el valor que se distancia más de 1 en la columna 6?}\\[.6cm]
               }
    \visible<8->{\huge \centering El \ce{Cl2} ($0,8970$) y tiene una $P(\ce{Cl2})=\SI{11,03}{\atm}$}
\end{frame}

%\subsection[Anexo]{Anexo: relación entre masa molecular y densidad en la ecuación de los gases ideales}
\begin{frame}
    \frametitle{Anexo}
    \framesubtitle{Como curiosidad, únicamente para interesados (no entra en el examen)}
    \structure{Relación entre las dos formas de la ``Ecuación de Van der Waals''}
    Partiendo de una de las formas y multiplicando toda la igualdad por $V^2$:
    \begin{overprint}
        \onslide<1>
                $$
                    P\vdot\left(V-n\vdot b\right)+\frac{n^2\vdot a}{V^2}\vdot\left(V-n\vdot b\right) = n\vdot R\vdot T\Rightarrow
                    P\vdot V^2\vdot\left(V-n\vdot b\right)+\cancel{V^2}\vdot\frac{n^2\vdot a}{\cancel{V^2}}\vdot\left(V-n\vdot b\right) = n\vdot R\vdot T\vdot V^2
                $$
            \onslide<2>
                $$
                    P\vdot V^2\vdot\left(V-n\vdot b\right)+n^2\vdot a\vdot\left(V-n\vdot b\right) = n\vdot R\vdot T\vdot V^2
                $$
            \onslide<3->
                $$
                    P\vdot V^2\vdot\left(V-n\vdot b\right) = n\vdot R\vdot T\vdot V^2 - n^2\vdot a\vdot\left(V-n\vdot b\right) 
                $$
    \end{overprint}
    \visible<4->{
        Multiplicamos a ambos lados de la igualdad por $\frac{1}{n\vdot R\vdot T\vdot \left(V-n\vdot b\right)\vdot V}$
        \begin{overprint}
            \onslide<4>
                $$
                    \scriptstyle
                    \frac{1}{n\vdot R\vdot T\vdot \left(V-n\vdot b\right)\vdot V}\vdot
                    P\vdot V^2\vdot\left(V-n\vdot b\right) = 
                    \frac{1}{n\vdot R\vdot T\vdot \left(V-n\vdot b\right)\vdot V}\vdot
                    \left[
                    n\vdot R\vdot T\vdot V^2 - n^2\vdot a\vdot\left(V-n\vdot b\right) 
                    \right]
                $$
            \onslide<5>
                $$
                    \frac{P\vdot V^2\vdot\left(V-n\vdot b\right)}{n\vdot R\vdot T\vdot \left(V-n\vdot b\right)\vdot V}
                    = 
                    \frac{n\vdot R\vdot T\vdot V^2 - n^2\vdot a\vdot\left(V-n\vdot b\right)}{n\vdot R\vdot T\vdot \left(V-n\vdot b\right)\vdot V}
                $$
            \onslide<6>
                $$
                    \frac{P\vdot V^2\vdot\left(V-n\vdot b\right)}{n\vdot R\vdot T\vdot \left(V-n\vdot b\right)\vdot V}
                    = 
                    \frac{n\vdot R\vdot T\vdot V^2}{n\vdot R\vdot T\vdot \left(V-n\vdot b\right)\vdot V}
                    -
                    \frac{n^2\vdot a\vdot\left(V-n\vdot b\right)}{n\vdot R\vdot T\vdot \left(V-n\vdot b\right)\vdot V}
                $$
            \onslide<7>
                $$
                    \frac{P\vdot V\cancel{^2}\vdot\cancel{\left(V-n\vdot b\right)}}{n\vdot R\vdot T\vdot \cancel{\left(V-n\vdot b\right)}\vdot\cancel{V}}
                    = 
                    \frac{\cancel{n\vdot R\vdot T}\vdot V\cancel{^2}}{\cancel{n\vdot R\vdot T}\vdot\left(V-n\vdot b\right)\vdot\cancel{V}}
                    -
                    \frac{n\cancel{^2}\vdot a\vdot\cancel{\left(V-n\vdot b\right)}}{\cancel{n}\vdot R\vdot T\vdot\cancel{\left(V-n\vdot b\right)}\vdot V}
                $$
            \onslide<8>
                $$
                    \frac{P\vdot V}{n\vdot R\vdot T}
                    = 
                    \frac{V}{\left(V-n\vdot b\right)}
                    -
                    \frac{n\vdot a}{R\vdot T\vdot V}
                $$
        \end{overprint}
    }
\end{frame}

