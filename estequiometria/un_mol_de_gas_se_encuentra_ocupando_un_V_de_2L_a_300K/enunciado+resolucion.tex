\begin{frame}
    \frametitle{\ejerciciocmd}
    \framesubtitle{Enunciado}
    \textbf{
		Dadas las siguientes reacciones:
\begin{itemize}
    \item \ce{I2(g) + H2(g) -> 2 HI(g)}~~~$\Delta H_1 = \SI{-0,8}{\kilo\calorie}$
    \item \ce{I2(s) + H2(g) -> 2 HI(g)}~~~$\Delta H_2 = \SI{12}{\kilo\calorie}$
    \item \ce{I2(g) + H2(g) -> 2 HI(ac)}~~~$\Delta H_3 = \SI{-26,8}{\kilo\calorie}$
\end{itemize}
Calcular los parámetros que se indican a continuación:
\begin{description}%[label={\alph*)},font={\color{red!50!black}\bfseries}]
    \item[\texttt{a)}] Calor molar latente de sublimación del yodo.
    \item[\texttt{b)}] Calor molar de disolución del ácido yodhídrico.
    \item[\texttt{c)}] Número de calorías que hay que aportar para disociar en sus componentes el yoduro de hidrógeno gas contenido en un matraz de \SI{750}{\cubic\centi\meter} a \SI{25}{\celsius} y \SI{800}{\torr} de presión.
\end{description}
\resultadocmd{\SI{12,8}{\kilo\calorie}; \SI{-13,0}{\kilo\calorie}; \SI{12,9}{\calorie}}

	}
\end{frame}

\begin{frame}
    \frametitle{\ejerciciocmd}
    \framesubtitle{Datos del problema}
    \begin{center}
        {\large ¿$P$ para gas ideal y para reales?}
        {\large ¿Cuál se desvía más del comportamiento ideal?}
    \end{center}
    $$
        \tcbhighmath[boxrule=0.4pt,arc=4pt,colframe=blue,drop fuzzy shadow=red]{n = \SI{1}{\mol}}\quad
        \tcbhighmath[boxrule=0.4pt,arc=4pt,colframe=blue,drop fuzzy shadow=red]{V = \SI{2}{\liter}}\quad
        \tcbhighmath[boxrule=0.4pt,arc=4pt,colframe=blue,drop fuzzy shadow=red]{T = \SI{300}{\kelvin}}
    $$
    \begin{center}
        \begin{tabular}{lSS}
            \toprule
                     &\textbf{a}&\textbf{b}\\
                     &\si{\square\liter\atm\per\square\mol}&\si{\liter\per\mol}\\
            \midrule
            \ce{Cl2} &   6,49   & 0,0562   \\
            \ce{CO2} &   3,59   & 0,0427   \\
            \ce{CO}  &   1,49   & 0,0399   \\
            \bottomrule
        \end{tabular}
    \end{center}
\end{frame}

\begin{frame}
    \frametitle{\ejerciciocmd}
    \framesubtitle{Resolución (\rom{1}): $P$ de gas ideal}
    \structure{Ecuación de gases ideales:} Recordamos ecuación general de los gases ideales, despejamos y sustituimos.
    \begin{overprint}
        \onslide<1>
            $$
                P\vdot V=n\vdot R\vdot T
            $$
        \onslide<2>
            $$
                P = \frac{n\vdot R\vdot T}{V}
            $$
        \onslide<3->
            $$
                P = \frac{\SI{1}{\cancel\mol}\vdot\SI{0,082}{\atm\cancel\liter\per\cancel\mol\per\cancel\kelvin}\vdot\SI{300}{\cancel\kelvin}}{\SI{2}{\cancel\liter}}
            $$
    \end{overprint}
    \visible<3->{
        $$
            \tcbhighmath[boxrule=0.4pt,arc=4pt,colframe=blue,drop fuzzy shadow=red]{P = \SI{12,3}{\atm}}
        $$
        \structure{Tened en mente:}
        $$
            \frac{P\vdot V}{n\vdot R\vdot T} = 1
        $$
                }
    \visible<4->{
        \structure{Ecuación de Van der Waals:}
        $$
            \frac{P\vdot V}{n\vdot R\vdot T} = \underbrace{\frac{V}{V-n\vdot b}-\frac{n\vdot a}{R\vdot T\vdot V}}_{\tcbhighmath[boxrule=0.4pt,arc=4pt,colframe=blue,drop fuzzy shadow=red]{1\text{ si el gas es ideal}}}
        $$
                }
\end{frame}

\begin{frame}
    \frametitle{\ejerciciocmd}
    \framesubtitle{Resolución (\rom{2}): Gases reales}
    \structure{Operamos:} añadimos columnas calculando los factores que distancian al gas de la idealidad.\\[.4cm]
    \begin{overprint}
        \onslide<2>
            \begin{center}
                \begin{tabular}{lSS}
                    \toprule
                    &\textbf{a}&\textbf{b}\\
                    &\si{\square\liter\atm\per\square\mol}&\si{\liter\per\mol}\\
                    \midrule
                    \ce{Cl2} &   6,49   & 0,0562   \\
                    \ce{CO2} &   3,59   & 0,0427   \\
                    \ce{CO}  &   1,49   & 0,0399   \\
                    \bottomrule
                \end{tabular}
            \end{center}
        \onslide<3>
            \begin{center}
                \begin{tabular}{lSSS}
                    \toprule
                    &\textbf{a}
                    &\textbf{b}
                    &$\frac{V}{V-n\vdot b}$\\
                    &\si{\square\liter\atm\per\square\mol}
                    &\si{\liter\per\mol}
                    & \\
                    \midrule
                    \ce{Cl2} &   6,49   & 0,0562 & 1,0289 \\
                    \ce{CO2} &   3,59   & 0,0427 & 1,0218 \\
                    \ce{CO}  &   1,49   & 0,0399 & 1,0204 \\
                    \bottomrule
                \end{tabular}
            \end{center}
        \onslide<4>
            \begin{center}
                \begin{tabular}{lSSSS}
                    \toprule
                    &\textbf{a}
                    &\textbf{b}
                    &$\frac{V}{V-n\vdot b}$
                    &$\frac{n\vdot a}{R\vdot T\vdot V}$\\
                    &\si{\square\liter\atm\per\square\mol}
                    &\si{\liter\per\mol}
                    &
                    &\\
                    \midrule
                    \ce{Cl2} &   6,49   & 0,0562 & 1,0289 & 0,1319 \\
                    \ce{CO2} &   3,59   & 0,0427 & 1,0218 & 0,0730 \\
                    \ce{CO}  &   1,49   & 0,0399 & 1,0204 & 0,0303 \\
                    \bottomrule
                \end{tabular}
            \end{center}
        \onslide<5>
            \begin{center}
                \begin{tabular}{lSSSSS}
                    \toprule
                    &\textbf{a}
                    &\textbf{b}
                    &$\frac{V}{V-n\vdot b}$
                    &$\frac{n\vdot a}{R\vdot T\vdot V}$
                    &$\frac{V}{V-n\vdot b}\text{--}\frac{n\vdot a}{R\vdot T\vdot V}$\\
                    &\si{\square\liter\atm\per\square\mol}
                    &\si{\liter\per\mol}
                    &
                    &
                    &\\
                    \midrule
                    \ce{Cl2} &   6,49   & 0,0562 & 1,0289 & 0,1319 & 0,8970 \\
                    \ce{CO2} &   3,59   & 0,0427 & 1,0218 & 0,0730 & 0,9488 \\
                    \ce{CO}  &   1,49   & 0,0399 & 1,0204 & 0,0303 & 0,9901 \\
                    \bottomrule
                \end{tabular}
            \end{center}
        \onslide<6->
            \begin{center}
                \begin{tabular}{lSSSSSS}
                    \toprule
                    gas
                    &\textbf{a}
                    &\textbf{b}
                    &$\frac{V}{V-n\vdot b}$
                    &$\frac{n\vdot a}{R\vdot T\vdot V}$
                    &$\frac{V}{V-n\vdot b}\text{--}\frac{n\vdot a}{R\vdot T\vdot V}$
                    &$P\text{ real}$ \\
                    &\si{\square\liter\atm\per\square\mol}
                    &\si{\liter\per\mol}
                    &
                    &
                    &
                    &\si{\atm}\\
                    \midrule
                    \ce{Cl2} &   6,49   & 0,0562 & 1,0289 & 0,1319 & 0,8970 & 11,03 \\
                    \ce{CO2} &   3,59   & 0,0427 & 1,0218 & 0,0730 & 0,9488 & 11,67 \\
                    \ce{CO}  &   1,49   & 0,0399 & 1,0204 & 0,0303 & 0,9901 & 12,18 \\
                    \bottomrule
                \end{tabular}
            \end{center}
    \end{overprint}
    \visible<7->{
        \structure{¿Cuál es el valor que se distancia más de 1 en la columna 6?}\\[.6cm]
               }
    \visible<8->{\huge \centering El \ce{Cl2} ($0,8970$) y tiene una $P(\ce{Cl2})=\SI{11,03}{\atm}$}
\end{frame}

%\subsection[Anexo]{Anexo: relación entre masa molecular y densidad en la ecuación de los gases ideales}
\begin{frame}
    \frametitle{Anexo}
    \framesubtitle{Apartado (\rom{3})}
    En el apartado c se pide la densidad ($d$). Partiendo de la ecuación de los gases ideales se obtiene la densidad:
    $$
        P_T\cdot V = \overbrace{n}^{n=\frac{m}{Mm}}\cdot R\cdot T
    $$
    $$
        P_T=\overbrace{\frac{m}{V}}^{d=\frac{m}{V}}\cdot\frac{R\cdot T}{\overline{Mm}} = d\cdot\frac{R\cdot T}{\overline{Mm}}
        \Rightarrow
        d=\frac{P_T\cdot\overline{Mm}}{R\cdot T}
    $$
    Esta $\overline{Mm}$ viene dada por:
    $$
        \overline{Mm}=\sum_{i=1}^{n}x_i\cdot Mm_i = \sum_{i=1}^{n}\frac{n_i}{n_T}\cdot Mm_i =\frac{1}{n_T}\sum_{i=1}^{n}\overbrace{n_i}^{n_i=\frac{m_i}{Mm_i}}\cdot Mm_i= \frac{1}{\underbrace{n_T}_{\frac{P_T\cdot V}{R\cdot T}}}\sum_{i=1}^{n}\frac{m_i}{\cancel{Mm_i}}\cdot\cancel{Mm_i}
    $$
    $$
        \overline{Mm}=\frac{R\cdot T}{P_T\cdot V}\sum_{i=1}^{n}m_i
    $$
    Sustituyendo en la ecuación de los gases ideales en función de la densidad:
    $$
        d=\frac{P_T\cdot\overbrace{\overline{Mm}}^{\overline{Mm}=\frac{R\cdot T}{P_T\cdot V}\sum_{i=1}^{n}m_i}}{R\cdot T}=
        \frac{\cancel{P_T}\cdot\frac{\cancel{R\cdot T}}{\cancel{P_T}\cdot V}\sum_{i=1}^{n}m_i}{\cancel{R\cdot T}}=
        \frac{\sum_{i=1}^{n}m_i}{V}
    $$
    Es decir, la densidad de una mezcla de gases ideales es la suma de las masas dividido el volumen que ocupan. No es necesario ni $P_T$, ni $T$, ni $\overline{Mm}$.
\end{frame}


