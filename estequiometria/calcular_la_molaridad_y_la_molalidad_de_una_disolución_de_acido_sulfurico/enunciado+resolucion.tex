\begin{frame}
	\frametitle{\ejerciciocmd}
	\framesubtitle{Enunciado}
	\textbf{
		Una reacción tiene una constante de velocidad de \SI{,017}{\per\second} a \SI{298}{\kelvin} y una energía libre de activación del \SI{27,235}{\kilo\joule\per\mol}. La adición de un catalizador disminuye dicha energía de activación hasta un \SI{33}{\percent} de su valor inicial. Calcule la nueva constante de velocidad.
\resultadocmd{ \SI{26,86}{\per\second} }

		}
\end{frame}

\begin{frame}
	\frametitle{\ejerciciocmd}
	\framesubtitle{Datos del problema}
	{\huge
		$$
		[\ce{H2SO4}]\text{ (molaridad de \ce{H2SO4})}?
		$$
		$$
		m(\ce{H2SO4})\text{ (molalidad de \ce{H2SO4})}?
		$$}
	\begin{center}
		\tcbhighmath[boxrule=0.4pt,arc=4pt,colframe=green,drop fuzzy shadow=blue]{d(\ce{H2SO4})=\SI{1,198}{\gram\per\cubic\centi\meter}=\SI{1198}{\gram\per\liter}}
		\tcbhighmath[boxrule=0.4pt,arc=4pt,colframe=green,drop fuzzy shadow=blue]{\SI{27}{\percent} \rfrac{m}{m}\text{ de \ce{H2SO4} puro}}
	\end{center}
	\visible<2-|handout:0>{Pero también\ldots\\
		$$
		\tcbhighmath[boxrule=0.4pt,arc=4pt,colframe=blue,drop fuzzy shadow=red]{Mm(\ce{H2SO4}) = \SI{98,08}{\gram\per\mol}}
		$$
	}
\end{frame}

\begin{frame}
	\frametitle{\ejerciciocmd}
	\framesubtitle{Ejemplo en un laboratorio de Química experimental}
	\begin{tikzpicture}
		\begin{scope}[spy using outlines={rectangle,magnification=3.5,connect spies,size=3.5cm,height=1.8cm}]
			\node[inner sep=0,outer sep=0,anchor=south west] (image) at (0,0)
			{\polaroid{ 4}{.27}{/home/nicux/Documentos/docencia/repositorio/ejercicios/quimica_general-github/estequiometria/calcular_la_molaridad_y_la_molalidad_de_una_disolución_de_acido_sulfurico/figs/sulfurico_comercial}{Un paseo por el laboratorio\ldots}};
			\spy[red!70!black] on (3.95,4.75) in node (zoom) at (-2.00,7.40);
			\spy[blue!70!black] on (2.30,2.72) in node (zoom) at (-2.00,4.40);
		\end{scope}
	\end{tikzpicture}
\end{frame}

\begin{frame}
	\frametitle{\ejerciciocmd}
	\framesubtitle{Resolución (\rom{1}): \underline{molaridad} del ácido sulfúrico}
	\alert{\textbf{Sin importar el volumen ni la masa, las concentraciones se mantienen inalterables}}
	\begin{center}
		Nos interesa ``buscar'' expresar la concentración como \si{\mol\per\liter}
	\end{center}
	\structure{En \SI{1}{\liter} de ácido comercial habrá \SI{1198}{\gram} de ácido comercial (impuro)}
	\visible<2->{
		$$
		\SI{1198}{\gram}\times\overbrace{\num{,27}}^{\SI{27}{\percent}\text{ de ácido puro}} = \SI{323}{\gram}\text{ de ácido puro.}
		$$
	}
	\visible<3->{
		$$
		\underbrace{n_{\text{puro}}(\ce{H2SO4})}_{n=\rfrac{m}{Mm}} = \frac{\SI{323}{\cancel\gram}}{\SI{98,08}{\cancel\gram\per\mol}} = \SI{3,30}{\mol}
		$$
	}
	\visible<4->{
		\structure{Habíamos supuesto \SI{1}{\liter} de disolución:}
		$$
		\underbrace{[\ce{H2SO4}]}_{M=\rfrac{n}{V}} = \frac{\SI{3,30}{\mol}}{\SI{1}{\liter}} = \SI{3,30}{\mol\per\liter} = \SI{3,30}{\Molar}
		$$
	}
\end{frame}

\begin{frame}
	\frametitle{\ejerciciocmd}
	\framesubtitle{Resolución (\rom{2}): \underline{molalidad} del ácido sulfúrico}
	\structure{En \SI{1}{\liter} de ácido comercial habrá \SI{1198}{\gram} de ácido comercial (impuro)}
	$$
	\overbrace{n_{\text{puro}}(\ce{H2SO4}) = \SI{3,30}{\mol}}^{\text{Previamente calculado}}
	$$
	\begin{center}
		Nos interesa ``buscar'' expresar la concentración como \si{\mol\per\kilogram} (de disolvente)
	\end{center}
	\visible<2->{
		$$
		\overbrace{m}^{\small masa}(\text{disolvente}) = \underbrace{\SI{1,198}{\kilogram}}_{\text{densidad es \SI{1,198}{\kilogram\per\liter} en \SI{1}{\liter}}}-\overbrace{\SI{,323}{\kilogram}}^{\text{anterior apartado}} = \SI{,875}{\kilogram}
		$$
	}
	\visible<3->{
		$$
		\overbrace{m}^{\small molalidad}(\ce{H2SO4}) = \frac{\SI{3,30}{\mol}}{\SI{,875}{\kilogram}} = \SI{3,77}{\mol\per\kilogram} = \SI{3,77}{m}
		$$
	}
\end{frame}
