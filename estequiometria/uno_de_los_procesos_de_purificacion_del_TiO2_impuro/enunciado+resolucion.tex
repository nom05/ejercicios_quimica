\begin{frame}
    \frametitle{\ejerciciocmd}
    \framesubtitle{Enunciado}
    \textbf{
		Dadas las siguientes reacciones:
\begin{itemize}
    \item \ce{I2(g) + H2(g) -> 2 HI(g)}~~~$\Delta H_1 = \SI{-0,8}{\kilo\calorie}$
    \item \ce{I2(s) + H2(g) -> 2 HI(g)}~~~$\Delta H_2 = \SI{12}{\kilo\calorie}$
    \item \ce{I2(g) + H2(g) -> 2 HI(ac)}~~~$\Delta H_3 = \SI{-26,8}{\kilo\calorie}$
\end{itemize}
Calcular los parámetros que se indican a continuación:
\begin{description}%[label={\alph*)},font={\color{red!50!black}\bfseries}]
    \item[\texttt{a)}] Calor molar latente de sublimación del yodo.
    \item[\texttt{b)}] Calor molar de disolución del ácido yodhídrico.
    \item[\texttt{c)}] Número de calorías que hay que aportar para disociar en sus componentes el yoduro de hidrógeno gas contenido en un matraz de \SI{750}{\cubic\centi\meter} a \SI{25}{\celsius} y \SI{800}{\torr} de presión.
\end{description}
\resultadocmd{\SI{12,8}{\kilo\calorie}; \SI{-13,0}{\kilo\calorie}; \SI{12,9}{\calorie}}

	}
\end{frame}

\begin{frame}
    \frametitle{\ejerciciocmd}
    \framesubtitle{Datos del problema}
    {\huge $$
        m(\ce{C})?
    $$}
    Reacciones:\\[.3cm]
    \ce{TiO2(impuro) +}
    \tcbhighmath[boxrule=0.4pt,arc=4pt,colframe=blue,drop fuzzy shadow=green]{\ce{C(s)}}
    \ce{+ Cl2(g) -> TiCl4(g) + CO2(g) + CO(g)}\\
    \ce{TiCl4(g) + O2(g) ->}
    \tcbhighmath[boxrule=0.4pt,arc=4pt,colframe=red,drop fuzzy shadow=yellow]{\ce{TiO2}(puro)}
    \ce{+ Cl2(g)}
    \visible<1-|handout:0>{
        $$
            \tcbhighmath[boxrule=0.4pt,arc=4pt,colframe=red,drop fuzzy shadow=yellow]{m_{\text{puro}}(\ce{TiO2}) = \SI{1,00}{\kilogram}}
        $$
                          }
    \visible<2-|handout:0>{Pero también ...\\
        $$
            \tcbhighmath[boxrule=0.4pt,arc=4pt,colframe=blue,drop fuzzy shadow=green]{Mat(\ce{C}) = \SI{12,01}{\gram\per\mol}}
            \quad
            \tcbhighmath[boxrule=0.4pt,arc=4pt,colframe=red,drop fuzzy shadow=yellow]{Mm(\ce{TiO2}) = \SI{79,87}{\gram\per\mol}}
        $$
                          }
\end{frame}

\begin{frame}
    \frametitle{\ejerciciocmd}
    \framesubtitle{Resolución (\rom{1}): ajuste de las ecuaciones de las reacciones químicas}
    \begin{alertblock}{NOTA IMPORTANTE DEL AJUSTE}
        La primera reacción no es válida porque tiene infinitos ajustes.
        $$
            \forall n(\ce{TiO2}) > 1\text{ y }n(CO)=2\Rightarrow\exists\text{ajuste de reacción}
        $$
    \end{alertblock}
    \structure{Reacción 1:}
    \begin{overprint}
        \onslide<1-8>
            {\centering Ajuste de la ecuación.\\[.3cm]}
        \onslide<9->
            \centering\ce{2TiO2 + 3C + 4Cl2 -> 2TiCl4 + CO2 + 2CO}
    \end{overprint}
    \begin{overprint}
        \onslide<1>
            \centering\ce{TiO2 + C + Cl2 -> TiCl4 + CO2 + CO}
        \onslide<2>
            \centering\ce{TiO2 + C + }
            \tcbhighmath[boxrule=0.4pt,arc=4pt,colframe=blue,drop fuzzy shadow=green]{\ce{Cl2}}
            \ce{ -> Ti}
            \tcbhighmath[boxrule=0.4pt,arc=4pt,colframe=blue,drop fuzzy shadow=green]{\ce{Cl4}}
            \ce{ + CO2 + CO}
        \onslide<3>
            \centering\ce{TiO2 + C + }
            \tcbhighmath[boxrule=0.4pt,arc=4pt,colframe=blue,drop fuzzy shadow=green]{2\ce{Cl2}}
            \ce{ ->}
            \tcbhighmath[boxrule=0.4pt,arc=4pt,colframe=blue,drop fuzzy shadow=green]{\ce{TiCl4}}
            \ce{ + CO2 + CO}
        \onslide<4>
            \centering\ce{Ti}
            \tcbhighmath[boxrule=0.4pt,arc=4pt,colframe=green,drop fuzzy shadow=yellow]{\ce{O2}}
            \ce{ + C + 2Cl2 -> TiCl4 + C}
            \tcbhighmath[boxrule=0.4pt,arc=4pt,colframe=green,drop fuzzy shadow=yellow]{\ce{O2}}\ce{+ C}
            \tcbhighmath[boxrule=0.4pt,arc=4pt,colframe=green,drop fuzzy shadow=yellow]{\ce{O}}
        \onslide<5>
            \centering\ce{Ti}
            \tcbhighmath[boxrule=0.4pt,arc=4pt,colframe=green,drop fuzzy shadow=yellow]{\ce{O2}}
            \ce{ + C + 2Cl2 -> TiCl4 + }
            \tcbhighmath[boxrule=0.4pt,arc=4pt,colframe=green,drop fuzzy shadow=yellow]{\frac{1}{2}\ce{CO2}}\ce{+ C}
            \tcbhighmath[boxrule=0.4pt,arc=4pt,colframe=green,drop fuzzy shadow=yellow]{\ce{O}}
        \onslide<6>
            \centering\ce{TiO2 +}
            \tcbhighmath[boxrule=0.4pt,arc=4pt,colframe=red,drop fuzzy shadow=blue]{\ce{C}}
            + 2\ce{Cl2 -> TiCl4 +}
            \tcbhighmath[boxrule=0.4pt,arc=4pt,colframe=red,drop fuzzy shadow=blue]{\frac{1}{2}\ce{C}}\ce{O2 +}
            \tcbhighmath[boxrule=0.4pt,arc=4pt,colframe=red,drop fuzzy shadow=blue]{\ce{C}}\ce{O}
        \onslide<7>
            \centering\ce{TiO2 +}
            \tcbhighmath[boxrule=0.4pt,arc=4pt,colframe=red,drop fuzzy shadow=blue]{\frac{3}{2}\ce{C}}
            + 2\ce{Cl2 -> TiCl4 +}
            \tcbhighmath[boxrule=0.4pt,arc=4pt,colframe=red,drop fuzzy shadow=blue]{\frac{1}{2}\ce{C}}\ce{O2 +}
            \tcbhighmath[boxrule=0.4pt,arc=4pt,colframe=red,drop fuzzy shadow=blue]{\ce{C}}\ce{O}
        \onslide<8>
            \centering
            $$
                2\times\left(\ce{TiO2 + \frac{3}{2} C + 2Cl2 -> TiCl4 + \frac{1}{2}CO2 + CO}\right)
            $$
    \end{overprint}
    \visible<9->{
        \structure{Reacción 2:}
                }            
    \begin{overprint}
        \onslide<9-11>
            {\centering Ajuste de la ecuación.\\[.3cm]}
        \onslide<12->
            \centering\ce{TiCl4 + O2 -> TiO2 + 2Cl2}
    \end{overprint}
    \begin{overprint}
        \onslide<9>
            \centering\ce{TiCl4 + O2 -> TiO2 + Cl2}
        \onslide<10>
            \centering\ce{Ti}
            \tcbhighmath[boxrule=0.4pt,arc=4pt,colframe=orange,drop fuzzy shadow=black]{\ce{Cl4}}
            \ce{ + O2 -> TiO2 + }
            \tcbhighmath[boxrule=0.4pt,arc=4pt,colframe=orange,drop fuzzy shadow=black]{\ce{Cl2}}
        \onslide<11>
            \centering\ce{Ti}
            \tcbhighmath[boxrule=0.4pt,arc=4pt,colframe=orange,drop fuzzy shadow=black]{\ce{Cl4}}
            \ce{ + O2 -> TiO2 + }
            \tcbhighmath[boxrule=0.4pt,arc=4pt,colframe=orange,drop fuzzy shadow=black]{\ce{2Cl2}}
    \end{overprint}
\end{frame}

\begin{frame}
    \frametitle{\ejerciciocmd}
    \framesubtitle{Resolución (\rom{2}): ajuste simultáneo de las ecuaciones}
    \structure{1ª Reacción:} \ce{2TiO2 + 3C + 4Cl2 -> 2TiCl4 + CO2 + 2CO}
    \structure{2ª Reacción:} \ce{TiCl4 + O2 -> TiO2 + 2Cl2}
    \structure{Ajuste de las reacciones secuenciales:} Comprobar que el/los producto(s) de la primera reacción tienen el mismo número de moles que los de la segunda.\\[.5cm]
    \begin{overprint}
        \onslide<2>
            1ª Reacción: \ce{2TiO2 + 3C + 4Cl2 -> 2TiCl4 + CO2 + 2CO}\\
            2ª Reacción: \ce{TiCl4 + O2 -> TiO2 + 2Cl2}
        \onslide<3>
            1ª Reacción: \ce{2TiO2 + 3C + 4Cl2 ->}
            \tcbhighmath[boxrule=0.4pt,arc=4pt,colframe=black,drop fuzzy shadow=orange]{\ce{2TiCl4}}
            \ce{ + CO2 + 2CO}\\
            2ª Reacción: 
            \tcbhighmath[boxrule=0.4pt,arc=4pt,colframe=black,drop fuzzy shadow=orange]{\ce{TiCl4}}
            \ce{+ O2 -> TiO2 + 2Cl2}
        \onslide<4>
            1ª Reacción: \ce{2TiO2 + 3C + 4Cl2 ->}
            \tcbhighmath[boxrule=0.4pt,arc=4pt,colframe=black,drop fuzzy shadow=orange]{\ce{2TiCl4}}
            \ce{ + CO2 + 2CO}\\
            2ª Reacción: 
            $$
                2\times\left(\tcbhighmath[boxrule=0.4pt,arc=4pt,colframe=black,drop fuzzy shadow=orange]{\ce{TiCl4}}\ce{+ O2 -> TiO2 + 2Cl2}\right)
            $$
        \onslide<5>
            1ª Reacción: \ce{2TiO2 + 3C + 4Cl2 ->}
            \tcbhighmath[boxrule=0.4pt,arc=4pt,colframe=black,drop fuzzy shadow=orange]{\ce{2TiCl4}}
            \ce{ + CO2 + 2CO}\\
            2ª Reacción: 
            \tcbhighmath[boxrule=0.4pt,arc=4pt,colframe=black,drop fuzzy shadow=orange]{\ce{2TiCl4}}\ce{+ 2O2 -> 2TiO2 + 4Cl2}
        \onslide<6>
            1ª Reacción: \ce{2TiO2 + 3C + 4Cl2 -> 2TiCl4 + CO2 + 2CO}\\
            2ª Reacción: \ce{2TiCl4 + 2O2 -> 2TiO2 + 4Cl2}
    \end{overprint}
\end{frame}

\begin{frame}
    \frametitle{\ejerciciocmd}
    \framesubtitle{Resolución (\rom{3}): ajuste de las ecuaciones de las reacciones químicas}
    \structure{Paso 1:} Relación estequiométrica entre el \ce{TiO2} puro y el \ce{C}\\[.3cm]
    \visible<2->{
        \begin{center}
            \SI{3}{\mol} de \ce{C}\ce{->}\SI{2}{\mol} de \ce{TiO2} \\[.3cm]
        \end{center}
        $$
            2n(\ce{C}) = 3n(\ce{TiO2}\text{ puro})
        $$
               }
   \visible<3->{
        \structure{Paso 2:} Sustituir el número de moles por las masas ($m$ e $Mm$) empleando $n=\frac{m}{Mm}$
               }
       \begin{overprint}
           \onslide<3>
                $$
                    2\cdot\overbrace{n(\ce{C})}^{\frac{m(C)}{Mm(C)}} = 3\cdot\underbrace{n(\ce{TiO2}\text{ puro})}_{\frac{m(TiO2)}{Mm(TiO2)}}
                $$
           \onslide<4->
                $$
                    2\cdot\frac{m(C)}{Mm(C)} = 3\cdot\frac{m(TiO2)}{Mm(TiO2)}
                $$
       \end{overprint}
   \visible<5->{
        \structure{Paso 3:} Despejar $m(C)$
        $$
            m(C) = \frac{3}{2}m(TiO2)\frac{Mm(C)}{Mm(TiO2)}
        $$
               }
   \visible<6->{
       \structure{Paso 4:} Sustituir por los valores correspondientes:
       $$
           m(C) = \frac{3}{2}\cdot\SI{1000}{\gram}\cdot\frac{\SI{12,01}{\cancel\gram\per\cancel\mol}}{\SI{79,87}{\cancel\gram\per\cancel\mol}}= \tcbhighmath[boxrule=0.4pt,arc=4pt,colframe=green,drop fuzzy shadow=blue]{\SI{226}{\gram} = m(C)}
       $$
               }
\end{frame}
