\begin{frame}
    \frametitle{\ejerciciocmd}
    \framesubtitle{Enunciado}
    \textbf{
	Una reacción tiene una constante de velocidad de \SI{,017}{\per\second} a \SI{298}{\kelvin} y una energía libre de activación del \SI{27,235}{\kilo\joule\per\mol}. La adición de un catalizador disminuye dicha energía de activación hasta un \SI{33}{\percent} de su valor inicial. Calcule la nueva constante de velocidad.
\resultadocmd{ \SI{26,86}{\per\second} }

	}
\end{frame}

\begin{frame}
    \frametitle{\ejerciciocmd}
    \framesubtitle{Datos del problema}
    {\huge
        $$
            m(\ce{PCl3})?\quad\quad
            m_{exc}(\ce{P4})?
        $$
    }
    Reacción:\\[.3cm]
    $$
        \tcbhighmath[boxrule=0.4pt,arc=4pt,colframe=blue,drop fuzzy shadow=green]{\ce{P4(s)}}
        \ce{ + }
        \tcbhighmath[boxrule=0.4pt,arc=4pt,colframe=red,drop fuzzy shadow=yellow]{\ce{Cl2(g)}}
        \ce{ -> }
        \tcbhighmath[boxrule=0.4pt,arc=4pt,colframe=green,drop fuzzy shadow=orange]{\ce{PCl3(l)}}
    $$\\[.5 cm]
    \begin{center}
        \tcbhighmath[boxrule=0.4pt,arc=4pt,colframe=blue,drop fuzzy shadow=green]{m(\ce{P4})=\SI{125}{\gram}}
        \tcbhighmath[boxrule=0.4pt,arc=4pt,colframe=red,drop fuzzy shadow=yellow]{m(\ce{Cl2(g)})=\SI{323}{\gram}}\\[.5cm]
    \end{center}
    \visible<2-|handout:0>{Pero también\ldots\\
        $$
            \tcbhighmath[boxrule=0.4pt,arc=4pt,colframe=blue,drop fuzzy shadow=green]{Mm(\ce{P4}) = \SI{123,90}{\gram\per\mol}}\quad
            \tcbhighmath[boxrule=0.4pt,arc=4pt,colframe=red,drop fuzzy shadow=yellow]{Mm(\ce{Cl2}) = \SI{70,91}{\gram\per\mol}}
        $$
        $$
            \tcbhighmath[boxrule=0.4pt,arc=4pt,colframe=green,drop fuzzy shadow=orange]{Mm(\ce{PCl3})=\SI{137,33}{\gram\per\mol}}
        $$
    }
\end{frame}

\begin{frame}
    \frametitle{\ejerciciocmd}
    \framesubtitle{Resolución (\rom{1}): Ajuste de la reacción y reactivo limitante}
    \begin{overprint}
        \onslide<1>
            $$
                \ce{P4 + Cl2 -> PCl3}
            $$
        \onslide<2>
            $$
                \tcbhighmath[boxrule=0.4pt,arc=4pt,colframe=blue,drop fuzzy shadow=green]{\ce{P4}}
                \ce{ + Cl2 -> }
                \tcbhighmath[boxrule=0.4pt,arc=4pt,colframe=blue,drop fuzzy shadow=green]{\ce{P}}
                \ce{Cl3}
            $$
        \onslide<3>
            $$
                \tcbhighmath[boxrule=0.4pt,arc=4pt,colframe=blue,drop fuzzy shadow=green]{\ce{P4}}
                \ce{ + Cl2 -> }
                \tcbhighmath[boxrule=0.4pt,arc=4pt,colframe=blue,drop fuzzy shadow=green]{\ce{4P}}
                \ce{Cl3}
            $$
        \onslide<4>
            $$
                \ce{P4 + }
                \tcbhighmath[boxrule=0.4pt,arc=4pt,colframe=red,drop fuzzy shadow=blue]{\ce{Cl2}}
                \ce{ -> }
                \tcbhighmath[boxrule=0.4pt,arc=4pt,colframe=red,drop fuzzy shadow=blue]{\ce{4}}
                \ce{P}
                \tcbhighmath[boxrule=0.4pt,arc=4pt,colframe=red,drop fuzzy shadow=blue]{\ce{Cl3}}
            $$
        \onslide<5>
            $$
                \ce{P4 + }
                \tcbhighmath[boxrule=0.4pt,arc=4pt,colframe=red,drop fuzzy shadow=blue]{\ce{6Cl2}}
                \ce{ -> }
                \tcbhighmath[boxrule=0.4pt,arc=4pt,colframe=red,drop fuzzy shadow=blue]{\ce{4}}
                \ce{P}
                \tcbhighmath[boxrule=0.4pt,arc=4pt,colframe=red,drop fuzzy shadow=blue]{\ce{Cl3}}
            $$
        \onslide<6->
            $$
                \ce{P4 + 6Cl2 -> 4PCl3}
            $$
    \end{overprint}
    \visible<6->{
        $$
            n=\frac{m}{Mm}
        $$
        \begin{itemize}
            \item<6-> Según la estequiometría: $\frac{\SI{1}{\mol}\text{ de }\ce{P4}}{\SI{6}{\mol}\text{ de }\ce{Cl2}}$.
            \item<7-> $n(\ce{P4})=\frac{\SI{125}{\cancel\gram}}{\SI{123,90}{\cancel\gram\per\mol}}=\SI{1,01}{\mol}$
            \item<8-> $n(\ce{Cl2})=\frac{\SI{323}{\cancel\gram}}{\SI{70,91}{\cancel\gram\per\mol}}=\SI{4,56}{\mol}$
            \item<9-> Según las cantidades de los reactivos: $\frac{\SI{1,01}{\mol}\text{ de }\ce{P4}}{\SI{4,56}{\mol}\text{ de }\ce{Cl2}}=\frac{\SI{1}{\mol}\text{ de }\ce{P4}}{\SI{4,51}{\mol}\text{ de }\ce{Cl2}}>\frac{\SI{1}{\mol}\text{ de }\ce{P4}}{\SI{6}{\mol}\text{ de }\ce{Cl2}}$.
            \item<10-> \ldots o $\SI{4,51}{\mol}\text{ de }\ce{Cl2}<\SI{6}{\mol}\text{ de }\ce{Cl2}$
        \end{itemize}
                }
    \visible<10->{
        \centering\myovalbox{\textcolor{yellow}{\textbf{El cloro es el reactivo limitante}}}
                }
\end{frame}

\begin{frame}
    \frametitle{\ejerciciocmd}
    \framesubtitle{Resolución (\rom{2}): cálculo de la masa de \ce{PCl3}}
    \begin{center}
        \textbf{\ce{P4(s) + 6Cl2(g) -> 4PCl3(l)}}
    \end{center}
    \structure{Según la estequiometría:}
        $$
            \SI{6}{\mol}\text{ de }\ce{Cl2 ->}\SI{4}{\mol}\text{ de } \ce{PCl3}
        $$
        \begin{overprint}
            \onslide<1>
                $$
                    n(\ce{Cl2})
                    \quad ??\quad 
                    n(\ce{PCl3})
                $$
            \onslide<2>    
                $$
                    n(\ce{Cl2})
                    \quad >\quad 
                    n(\ce{PCl3})
                $$
            \onslide<3>
                $$
                    4\cdot n(\ce{Cl2})
                    \quad =\quad
                    6\cdot n(\ce{PCl3})
                $$
            \onslide<4>
                $$
                    \overbrace{4\cdot n(\ce{Cl2})}^{n=\frac{m}{Mm}} = \underbrace{6\cdot n(\ce{PCl3})}_{n=\frac{m}{Mm}}
                $$
            \onslide<5->
                $$
                    4\cdot\frac{m(\ce{Cl2})}{Mm(\ce{Cl2})} = 6\cdot\frac{m(\ce{PCl3})}{Mm(\ce{PCl3})}
                $$
    \end{overprint}
    \visible<6->{
        \structure{Despejamos:}
        $$
            m(\ce{PCl3})
            =
            \cancelto{\frac{2}{3}}{\frac{4}{6}}
            \cdot m(\ce{Cl2})\frac{Mm(\ce{PCl3})}{Mm(\ce{Cl2})}
        $$
               }
       \visible<7->{
           \structure{Sustituimos:}
           $$
               \tcbhighmath[boxrule=0.4pt,arc=4pt,colframe=blue,drop fuzzy shadow=green]{m(\ce{PCl3})}
               =
               \frac{2}{3}\cdot
               \SI{323}{\gram}
               \cdot
               \frac{\SI{137,33}{\cancel\gram\per\cancel\mol}}{\SI{70,91}{\cancel\gram\per\cancel\mol}}
               =
               \tcbhighmath[boxrule=0.4pt,arc=4pt,colframe=blue,drop fuzzy shadow=green]{\SI{417}{\gram}}
           $$
                  }
\end{frame}

\begin{frame}
    \frametitle{\ejerciciocmd}
    \framesubtitle{Resolución (\rom{3}): masa de \ce{P4} en exceso o sin reaccionar}
    \begin{center}
        \textbf{\ce{P4 + 6Cl2 -> 4PCl3}}
    \end{center}
    $$
        n=\frac{m}{Mm}
    $$
    \begin{itemize}
        \item<1-> Según la estequiometría: $\frac{\SI{1}{\mol}\text{ de }\ce{P4}}{\SI{6}{\mol}\text{ de }\ce{Cl2}}$.
        \item<2-> Según el reactivo limitante: $\frac{x~\si{\mol}\text{ de }\ce{P4}}{\SI{4,51}{\mol}\text{ de }\ce{Cl2}}=\frac{\SI{1}{\mol}\text{ de }\ce{P4}}{\SI{6}{\mol}\text{ de }\ce{Cl2}}\Rightarrow x=\SI{0,752}{\mol}\text{ de }\ce{P4}$.
        \item<3-> Sin reaccionar: $\SI{1,01}{\mol} \text{ (calculado previamente)} - \SI{0,752}{\mol} = \SI{0,248}{\mol}\text{ de \ce{P4} en exceso}$.
        \visible<4->{
            \item<4-> $n=\frac{m}{Mm}\Rightarrow m=n\cdot Mm\Rightarrow
            \tcbhighmath[boxrule=0.4pt,arc=4pt,colframe=blue,drop fuzzy shadow=green]{m_{exceso}(\ce{P4})}
            =\SI{0,248}{\cancel\mol}\cdot\SI{123,90}{\gram\per\cancel\mol}=
            \tcbhighmath[boxrule=0.4pt,arc=4pt,colframe=blue,drop fuzzy shadow=green]{\SI{31}{\gram}}$
                    }
    \end{itemize}
\end{frame}
