\begin{frame}
    \frametitle{\ejerciciocmd}
    \framesubtitle{Enunciado}
    \textbf{
		Una reacción tiene una constante de velocidad de \SI{,017}{\per\second} a \SI{298}{\kelvin} y una energía libre de activación del \SI{27,235}{\kilo\joule\per\mol}. La adición de un catalizador disminuye dicha energía de activación hasta un \SI{33}{\percent} de su valor inicial. Calcule la nueva constante de velocidad.
\resultadocmd{ \SI{26,86}{\per\second} }

	}
\end{frame}

\begin{frame}
    \frametitle{\ejerciciocmd}
    \framesubtitle{Datos del problema}
    {\huge
        $$
            m_{teo}(\ce{C6H12})?\quad\quad\% \text{Rto da reacción}?
        $$
    }
    Reacción:\\[.3cm]
    \tcbhighmath[boxrule=0.4pt,arc=4pt,colframe=blue,drop fuzzy shadow=green]{\ce{C6H12(l)}}
    \ce{ + O2(g) ->}
    \tcbhighmath[boxrule=0.4pt,arc=4pt,colframe=red,drop fuzzy shadow=yellow]{\ce{H2C6H8O4(l)}}
    \ce{ + H2O(g)} \\[.5 cm]
    \tcbhighmath[boxrule=0.4pt,arc=4pt,colframe=blue,drop fuzzy shadow=green]{m(\ce{C6H12}) = \SI{25,0}{\gram}}
    \quad
    \tcbhighmath[boxrule=0.4pt,arc=4pt,colframe=red,drop fuzzy shadow=yellow]{m_{real}(\ce{H2C6H8O4})=\SI{33,5}{\gram}}\\[.5 cm]
    \visible<2-|handout:0>{Pero tamén ...\\
        $$
            \tcbhighmath[boxrule=0.4pt,arc=4pt,colframe=blue,drop fuzzy shadow=green]{Mm(\ce{C6H12}) = \SI{84,16}{\gram\per\mol}}
            \quad
            \tcbhighmath[boxrule=0.4pt,arc=4pt,colframe=red,drop fuzzy shadow=yellow]{Mm(\ce{H2C6H8O4}) = \SI{146,14}{\gram\per\mol}}
        $$
    }
\end{frame}

\begin{frame}
    \frametitle{\ejerciciocmd}
    \framesubtitle{Resolución (\rom{1}): Ajuste de la reacción}
    \begin{overprint}
        \onslide<1>
            \ce{C6H12(l) + O2(g) -> H2C6H8O4(l) + H2O(g)}
        \onslide<2>
            \ce{C6H12(l) + }
            \tcbhighmath[boxrule=0.4pt,arc=4pt,colframe=blue,drop fuzzy shadow=green]{\ce{O2(g)}}
            \ce{-> H2C6H8}
            \tcbhighmath[boxrule=0.4pt,arc=4pt,colframe=blue,drop fuzzy shadow=green]{\ce{O4(l)}}
            \ce{+ H2}
            \tcbhighmath[boxrule=0.4pt,arc=4pt,colframe=blue,drop fuzzy shadow=green]{\ce{O(g)}}
        \onslide<3>
            \ce{C6H12(l) + }
            \tcbhighmath[boxrule=0.4pt,arc=4pt,colframe=blue,drop fuzzy shadow=green]{\frac{5}{2}\ce{O2(g)}}
            \ce{-> H2C6H8}
            \tcbhighmath[boxrule=0.4pt,arc=4pt,colframe=blue,drop fuzzy shadow=green]{\ce{O4(l)}}
            \ce{+ H2}
            \tcbhighmath[boxrule=0.4pt,arc=4pt,colframe=blue,drop fuzzy shadow=green]{\ce{O(g)}}
        \onslide<4->
            \ce{C6H12(l) + \frac{5}{2}\ce{O2(g)} -> H2C6H8O4(l) + H2O(g)}
    \end{overprint}
    \visible<2->{
        \begin{itemize}
            \item<2-> Únicamente los átomos de oxígeno no están ajustados.
            \item<3-> $\rfrac{5}{2}$ es recomendable en este caso en vez de multiplicar toda la reacción $\times 2$ porque simplifica la relación entre los compuestos del problema.
            \item<4-> Según la estequiometría: $\SI{1}{\mol}\text{ de }\ce{C6H12(l)}\Rightarrow\SI{1}{\mol}\text{ de }\ce{H2C6H8O4}$.
        \end{itemize}
                }
    \visible<4->{
        \centering\myovalbox{\textcolor{white}{$n(\ce{C6H12})=n(\ce{H2C6H8O4})$}}
                }
\end{frame}

\begin{frame}
    \frametitle{\ejerciciocmd}
    \framesubtitle{Resolución (\rom{2}): masa teórica de \ce{H2C6H8O4}}
    \structure{Recordando:}
        $$
            n=\frac{m}{Mm}
        $$
     \visible<2->{
         {\centering\myovalbox{\textcolor{white}{$n(\ce{C6H12})=n(\ce{H2C6H8O4})$}}}
                 }
     \visible<3->{
         \structure{Se obtiene:}
         $$
             \frac{m(\ce{C6H12})}{Mm(\ce{C6H12})} = \frac{m(\ce{H2C6H8O4})}{Mm(\ce{H2C6H8O4})}
         $$
                 }
     \visible<4->{
        \structure{Despejando:}
            $$
                m(\ce{H2C6H8O4}) = m(\ce{C6H12})\cdot\frac{Mm(\ce{H2C6H8O4})}{Mm(\ce{C6H12})}
            $$
                 }
     \visible<5->{
        \structure{Sustituyendo:}
            $$
                m(\ce{H2C6H8O4}) = \SI{25,0}{\gram}\cdot\frac{\SI{146,14}{\cancel\gram\per\cancel\mol}}{\SI{84,16}{\cancel\gram\per\cancel\mol}} = \tcbhighmath[boxrule=0.4pt,arc=4pt,colframe=blue,drop fuzzy shadow=green]{\SI{43,4}{\gram}=m(\ce{H2C6H8O4})}
            $$
                 }
\end{frame}

\begin{frame}
    \frametitle{\ejerciciocmd}
    \framesubtitle{Resolución (\rom{3}): \% Rto de la reacción}
    \structure{Masa teórica:} $m_{\text{teo}}(\ce{H2C6H8O4}) = \SI{43,4}{\gram}$
    \structure{Masa real:} $m_{\text{real}}(\ce{H2C6H8O4}) = \SI{33,5}{\gram}$
    \structure{\% Rendimiento:} $\tcbhighmath[boxrule=0.4pt,arc=4pt,colframe=blue,drop fuzzy shadow=green]{Rto (\ce{H2C6H8O4})} = \frac{\SI{33,5}{\cancel\gram}}{\SI{43,4}{\cancel\gram}}\cdot 100 = \tcbhighmath[boxrule=0.4pt,arc=4pt,colframe=blue,drop fuzzy shadow=green]{\SI{77,2}{\percent}}$
\end{frame}
