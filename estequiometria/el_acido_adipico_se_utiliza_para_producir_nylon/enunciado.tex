El ácido adípico, \ce{H2C6H8O4}, se utiliza para producir nylon. El ácido se fabrica comercialmente por medio de una reacción controlada entre ciclohexano (\ce{C6H12}) y oxígeno:
    \ce{C6H12(l) + O2(g) -> H2C6H8O4(l) + H2O(g)}
\begin{enumerate}[label={\alph*)},font={\color{red!50!black}\bfseries}]
	\item Calcular la cantidad de ácido adípico que se podría obtener a partir de \SI{25,0}{\gram} de ciclohexano.
	\item Si realmente sólo se obtienen \SI{33,5}{\gram} de ácido adípico, ¿cuál es el rendimiento porcentual de este?
\end{enumerate}
\resultadocmd{
                \SI{43,4}{\gram};
    		\SI{77,2}{\percent}
}
