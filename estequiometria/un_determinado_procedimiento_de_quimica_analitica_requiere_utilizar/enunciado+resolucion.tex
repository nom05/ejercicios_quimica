\begin{frame}
	\frametitle{\ejerciciocmd}
	\framesubtitle{Enunciado}
	\textbf{
		Una reacción tiene una constante de velocidad de \SI{,017}{\per\second} a \SI{298}{\kelvin} y una energía libre de activación del \SI{27,235}{\kilo\joule\per\mol}. La adición de un catalizador disminuye dicha energía de activación hasta un \SI{33}{\percent} de su valor inicial. Calcule la nueva constante de velocidad.
\resultadocmd{ \SI{26,86}{\per\second} }

	}
\end{frame}

\begin{frame}
	\frametitle{\ejerciciocmd}
	\framesubtitle{Datos del problema}
	\begin{center}
		{\huge¿$V_{\text{conc}}(\ce{K2CrO4})$?} \\[.4cm]
		\tcbhighmath[boxrule=0.4pt,arc=4pt,colframe=yellow,drop fuzzy shadow=orange]{[\ce{K2CrO4}]_{\text{conc}} = \SI{,250}{\Molar}}\\[.4cm]
		\tcbhighmath[boxrule=0.4pt,arc=4pt,colframe=orange,drop fuzzy shadow=green]{[\ce{K2CrO4}]_{\text{dil}} = \SI{,0100}{\Molar}}\quad
		\tcbhighmath[boxrule=0.4pt,arc=4pt,colframe=orange,drop fuzzy shadow=green]{V_{\text{dil}}(\ce{K2CrO4}) = \SI{,250}{\liter}}
	\end{center}
\end{frame}

\begin{frame}
	\frametitle{\ejerciciocmd}
	\framesubtitle{Resolución}
	\structure{El n"o de moles (o número de moléculas de \ce{K2CrO4}) que hay en los dos volúmenes es el mismo.} Usamos la definición de concentración molar 
	$M=\rfrac{n}{V}\Rightarrow n = M\vdot V$ y despejamos
	$$
		n_{\text{conc}} = n_{\text{dil}}\Rightarrow [\ce{K2CrO4}]_{\text{conc}}\vdot V_{\text{conc}} = [\ce{K2CrO4}]_{\text{dil}}\vdot V_{\text{dil}}\Rightarrow
		V_{\text{conc}} = \frac{[\ce{K2CrO4}]_{\text{dil}}\vdot V_{\text{dil}}}{[\ce{K2CrO4}]_{\text{conc}}}
	$$
	$$
		\tcbhighmath[boxrule=0.4pt,arc=4pt,colframe=yellow,drop fuzzy shadow=orange]{
			V_{\text{conc}}(\ce{K2CrO4}) = \frac{\SI{,0100}{\cancel\Molar}\vdot\SI{,250}{\liter}}{\SI{,250}{\cancel\Molar}} = \SI{,0100}{\liter}
		}
	$$
\end{frame}