\begin{frame}
	\frametitle{\ejerciciocmd}
	\framesubtitle{Enunciado}
	\textbf{
		Dadas las siguientes reacciones:
\begin{itemize}
    \item \ce{I2(g) + H2(g) -> 2 HI(g)}~~~$\Delta H_1 = \SI{-0,8}{\kilo\calorie}$
    \item \ce{I2(s) + H2(g) -> 2 HI(g)}~~~$\Delta H_2 = \SI{12}{\kilo\calorie}$
    \item \ce{I2(g) + H2(g) -> 2 HI(ac)}~~~$\Delta H_3 = \SI{-26,8}{\kilo\calorie}$
\end{itemize}
Calcular los parámetros que se indican a continuación:
\begin{description}%[label={\alph*)},font={\color{red!50!black}\bfseries}]
    \item[\texttt{a)}] Calor molar latente de sublimación del yodo.
    \item[\texttt{b)}] Calor molar de disolución del ácido yodhídrico.
    \item[\texttt{c)}] Número de calorías que hay que aportar para disociar en sus componentes el yoduro de hidrógeno gas contenido en un matraz de \SI{750}{\cubic\centi\meter} a \SI{25}{\celsius} y \SI{800}{\torr} de presión.
\end{description}
\resultadocmd{\SI{12,8}{\kilo\calorie}; \SI{-13,0}{\kilo\calorie}; \SI{12,9}{\calorie}}

	}
\end{frame}

\begin{frame}
	\frametitle{\ejerciciocmd}
	\framesubtitle{Datos del problema}
	\begin{center}
		{\huge¿$V_{\text{conc}}(\ce{K2CrO4})$?} \\[.4cm]
		\tcbhighmath[boxrule=0.4pt,arc=4pt,colframe=yellow,drop fuzzy shadow=orange]{[\ce{K2CrO4}]_{\text{conc}} = \SI{,250}{\Molar}}\\[.4cm]
		\tcbhighmath[boxrule=0.4pt,arc=4pt,colframe=orange,drop fuzzy shadow=green]{[\ce{K2CrO4}]_{\text{dil}} = \SI{,0100}{\Molar}}\quad
		\tcbhighmath[boxrule=0.4pt,arc=4pt,colframe=orange,drop fuzzy shadow=green]{V_{\text{dil}}(\ce{K2CrO4}) = \SI{,250}{\liter}}
	\end{center}
\end{frame}

\begin{frame}
	\frametitle{\ejerciciocmd}
	\framesubtitle{Resolución}
	\structure{El n"o de moles (o número de moléculas de \ce{K2CrO4}) que hay en los dos volúmenes es el mismo.} Usamos la definición de concentración molar 
	$M=\rfrac{n}{V}\Rightarrow n = M\vdot V$ y despejamos
	$$
		n_{\text{conc}} = n_{\text{dil}}\Rightarrow [\ce{K2CrO4}]_{\text{conc}}\vdot V_{\text{conc}} = [\ce{K2CrO4}]_{\text{dil}}\vdot V_{\text{dil}}\Rightarrow
		V_{\text{conc}} = \frac{[\ce{K2CrO4}]_{\text{dil}}\vdot V_{\text{dil}}}{[\ce{K2CrO4}]_{\text{conc}}}
	$$
	$$
		\tcbhighmath[boxrule=0.4pt,arc=4pt,colframe=yellow,drop fuzzy shadow=orange]{
			V_{\text{conc}}(\ce{K2CrO4}) = \frac{\SI{,0100}{\cancel\Molar}\vdot\SI{,250}{\liter}}{\SI{,250}{\cancel\Molar}} = \SI{,0100}{\liter}
		}
	$$
\end{frame}