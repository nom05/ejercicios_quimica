\begin{frame}
	\frametitle{\ejerciciocmd}
	\framesubtitle{Enunciado}
	\textbf{
		Una reacción tiene una constante de velocidad de \SI{,017}{\per\second} a \SI{298}{\kelvin} y una energía libre de activación del \SI{27,235}{\kilo\joule\per\mol}. La adición de un catalizador disminuye dicha energía de activación hasta un \SI{33}{\percent} de su valor inicial. Calcule la nueva constante de velocidad.
\resultadocmd{ \SI{26,86}{\per\second} }

	}
\end{frame}

\begin{frame}
	\frametitle{\ejerciciocmd}
	\framesubtitle{Datos del problema}
	{\huge
		$$
			m(\ce{NaOH})?
		$$}
	\begin{center}
		\tcbhighmath[boxrule=0.4pt,arc=4pt,colframe=green,drop fuzzy shadow=blue]{m(\ce{Na2CO3})=\SI{1}{\kilogram}=\SI{1000}{\gram}}
	\end{center}
	\visible<2-|handout:0>{Pero también\ldots\\
		$$
			\tcbhighmath[boxrule=0.4pt,arc=4pt,colframe=green,drop fuzzy shadow=blue]{Mm(\ce{Na2CO3}) = \SI{105,99}{\gram\per\mol}}
			\tcbhighmath[boxrule=0.4pt,arc=4pt,colframe=yellow,drop fuzzy shadow=green]{Mm(\ce{NaOH}) = \SI{40,00}{\gram\per\mol}}
		$$
		  Reacción sin ajustar: \tcbhighmath[boxrule=0.4pt,arc=4pt,colframe=green,drop fuzzy shadow=blue]{\ce{Na2CO3}} + \ce{Ca(OH)2} \ce{->} \tcbhighmath[boxrule=0.4pt,arc=4pt,colframe=yellow,drop fuzzy shadow=green]{\ce{NaOH}} + \ce{CaCO3}
						}
\end{frame}

\begin{frame}
	\frametitle{\ejerciciocmd}
	\framesubtitle{Resolución (\rom{1}): ajuste de la reacción}
	\structure{Reacción:} \textbf{\ce{Na2CO3 + Ca(OH)2 -> NaOH + CaCO3}}
	\begin{block}{Por tanteo:}
		En general es el más rápido. Contabilizar átomos en reactivos y átomos en los productos y ajustar las cantidades en ambos lados.\\[.3cm]
		\begin{overprint}
			\onslide<1>
				\centering\tcbhighmath[boxrule=0.4pt,arc=4pt,colframe=green,drop fuzzy shadow=blue]{\ce{Na2}}\ce{CO3 + Ca(OH)2 -> }
				\tcbhighmath[boxrule=0.4pt,arc=4pt,colframe=green,drop fuzzy shadow=blue]{\ce{Na}}\ce{OH + CaCO3} (\textbf{\underline{sin ajustar}})
			\onslide<2>
				\centering\tcbhighmath[boxrule=0.4pt,arc=4pt,colframe=green,drop fuzzy shadow=blue]{\ce{Na\textbf{2}}}\ce{CO3 + Ca(OH)2 -> }
				\tcbhighmath[boxrule=0.4pt,arc=4pt,colframe=green,drop fuzzy shadow=blue]{\ce{\textbf{2}Na}}\ce{OH + CaCO3} (\textbf{\underline{ajustado}})
			\onslide<3>
				\centering\ce{Na2CO3 + Ca}
				\tcbhighmath[boxrule=0.4pt,arc=4pt,colframe=red,drop fuzzy shadow=blue]{\ce{(OH)\textbf{2}}}
				\ce{ -> }
				\tcbhighmath[boxrule=0.4pt,arc=4pt,colframe=red,drop fuzzy shadow=blue]{\ce{\textbf{2}NaOH}} (\textbf{\underline{ajustado}})
				\ce{ + CaCO3}
			\onslide<4>
				\centering\ce{Na2CO3 + }
				\tcbhighmath[boxrule=0.4pt,arc=4pt,colframe=yellow,drop fuzzy shadow=green]{\ce{Ca}}
				\ce{(OH)2 -> 2NaOH +}
				\tcbhighmath[boxrule=0.4pt,arc=4pt,colframe=yellow,drop fuzzy shadow=green]{\ce{Ca}}
				\ce{CO3} (ajustado)
			\onslide<5>
				\centering\ce{Na2}
				\tcbhighmath[boxrule=0.4pt,arc=4pt,colframe=orange,drop fuzzy shadow=green]{\ce{CO3}}
				\ce{+ Ca(OH)2 -> 2NaOH + Ca}
				\tcbhighmath[boxrule=0.4pt,arc=4pt,colframe=orange,drop fuzzy shadow=green]{\ce{CO3}} (\textbf{\underline{ajustado}})
			\onslide<6->
				\centering\ce{Na2CO3 + Ca(OH)2 -> \textbf{2}NaOH + CaCO3}
		\end{overprint}
	\end{block}
	\visible<7->{
		\begin{alertblock}{Método aritmético:}
			Mediante coeficientes y agrupando por átomos:\\
			\centering\colorbox{red}{\color{white}\textbf{a}}\ce{Na2CO3} + \colorbox{green}{\textbf{b}}\ce{Ca(OH)2} \ce{->} \colorbox{blue}{\color{white}\textbf{c}}\ce{NaOH} +  \colorbox{orange}{\color{white}\textbf{d}}\ce{CaCO3}\\
		    \begin{columns}
	            \visible<8-|handout:0>{
					\column{.45\textwidth}
						\begin{flushleft}
							\ce{Na}:\quad\textbf{2}\colorbox{red}{\color{white}\textbf{a}}=\colorbox{blue}{\color{white}\textbf{c}}\\
							\ce{C}:\quad\colorbox{red}{\color{white}\textbf{a}}=\colorbox{orange}{\color{white}\textbf{d}}\\
							\ce{O}:\quad\textbf{3}\colorbox{red}{\color{white}\textbf{a}} + \textbf{2}\colorbox{green}{\textbf{b}} = \colorbox{blue}{\color{white}\textbf{c}} + \textbf{3}\colorbox{orange}{\color{white}\textbf{d}}\\
							\ce{Ca}:\quad\colorbox{green}{\textbf{b}} = \colorbox{orange}{\color{white}\textbf{d}}\\
							\ce{H}:\quad\textbf{2}\colorbox{green}{\textbf{b}} = \colorbox{blue}{\color{white}\textbf{c}}\\
						\end{flushleft}
									}
				\visible<9-|handout:0>{
					\column{.45\textwidth}
						\underline{Escogemos} \colorbox{red}{\color{white}\textbf{a}} = 1\\
						Entonces:
						\colorbox{blue}{\color{white}\textbf{c}} = \textbf{$2\times 1$}\\
						\colorbox{orange}{\color{white}\textbf{d}} = 1\\
						\colorbox{green}{\textbf{b}} = 1\\
									}
			\end{columns}
		    \visible<10-|handout:0>{
				\centering
				\colorbox{red}{\color{white}\textbf{1}}
				\colorbox{green}{\textbf{1}}
				\colorbox{blue}{\color{white}\textbf{2}}
				\colorbox{orange}{\color{white}\textbf{1}}
				\\[.5cm]
								}
			\visible<10-|handout:0>{
				\centering\colorbox{red}{\color{white}\textbf{1}}\ce{Na2CO3} + \colorbox{green}{\textbf{1}}\ce{Ca(OH)2} \ce{->} \colorbox{blue}{\color{white}\textbf{2}}\ce{NaOH} +  \colorbox{orange}{\color{white}\textbf{1}}\ce{CaCO3}\\
								}
		\end{alertblock}
				}
\end{frame}

\begin{frame}
	\frametitle{\ejerciciocmd}
	\framesubtitle{Resolución (\rom{2}): uso de la estequiometría}
	\centering\textbf{\ce{Na2CO3 + Ca(OH)2 -> 2NaOH + CaCO3}}
	\begin{overprint}
		\onslide<1>
			\centering Según la estequiometría \SI{1}{\mol} de \ce{Na2CO3} produce \SI{2}{\mol} de \ce{NaOH}
			$$
				\ce{\SI{1}{\mol}\text{ de \ce{Na2CO3}} -> \SI{2}{\mol}\text{ de \ce{NaOH}}}
			$$
		\onslide<2-3>
			\centering Numéricamente:
			$$
				\ce{\SI{1}{\mol}\text{ de \ce{Na2CO3}}} < \ce{\SI{2}{\mol}\text{ de \ce{NaOH}}}
			$$
			\visible<3>{
				Si multiplicamos \SI{1}{\mol} de \ce{Na2CO3} por \num{2} (coeficiente de \ce{NaOH})...
						}
		\onslide<4>
			$$
				2\times\overbrace{\ce{\SI{1}{\mol}\text{ de \ce{Na2CO3}}}}^{n(\ce{Na2CO3})} = \underbrace{\ce{\SI{2}{\mol}\text{ de \ce{NaOH}}}}_{n(\ce{NaOH})}
			$$
		\onslide<5->
			$$
				2\times n(\ce{Na2CO3}) = n(\ce{NaOH})
			$$
			\centering Usando $n = \rfrac{m}{Mm}$
	\end{overprint}
	\begin{overprint}
		\onslide<6>
			$$
				2\times\frac{m(\ce{Na2CO3})}{Mm(\ce{Na2CO3})} = \frac{m(\ce{NaOH})}{Mm(\ce{NaOH})}
			$$
		\onslide<7->
			$$
				2\times\frac{m(\ce{Na2CO3})}{Mm(\ce{Na2CO3})} = \frac{m(\ce{NaOH})}{Mm(\ce{NaOH})}\Rightarrow m(\ce{NaOH})= 2\cdot m(\ce{Na2CO3})\cdot\frac{Mm(\ce{NaOH})}{Mm(\ce{Na2CO3})}
			$$
	\end{overprint}
	\visible<8->{
		$$
			m(\ce{NaOH})= 2\cdot\SI{1000}{\gram}\cdot\frac{\SI{40,00}{\cancel\gram\per\cancel\mol}}{\SI{105,99}{\cancel\gram\per\cancel\mol}}
		$$
		$$
			\tcbhighmath[boxrule=0.4pt,arc=4pt,colframe=yellow,drop fuzzy shadow=green]{m(\ce{NaOH}) = \SI{754,79}{\gram}}
		$$
				}
\end{frame}
