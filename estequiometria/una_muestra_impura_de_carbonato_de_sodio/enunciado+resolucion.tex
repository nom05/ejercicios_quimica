\begin{frame}
	\frametitle{\ejerciciocmd}
	\framesubtitle{Enunciado}
	\textbf{
		Una reacción tiene una constante de velocidad de \SI{,017}{\per\second} a \SI{298}{\kelvin} y una energía libre de activación del \SI{27,235}{\kilo\joule\per\mol}. La adición de un catalizador disminuye dicha energía de activación hasta un \SI{33}{\percent} de su valor inicial. Calcule la nueva constante de velocidad.
\resultadocmd{ \SI{26,86}{\per\second} }

		}
\end{frame}

\begin{frame}
	\frametitle{\ejerciciocmd}
	\framesubtitle{Datos del problema}
	\centering{\huge\si{\percent} pureza \ce{Na2CO3}?}
	\begin{center}
		\tcbhighmath[boxrule=0.4pt,arc=4pt,colframe=green,drop fuzzy shadow=blue]{m(\ce{Na2CO3})=\SI{1,2048}{\gram}}
		\tcbhighmath[boxrule=0.4pt,arc=4pt,colframe=orange,drop fuzzy shadow=green]{m(\ce{CaCO3})=\SI{1,0262}{\gram}}
	\end{center}
	\visible<2-|handout:0>{Pero también\ldots\\
		$$
			\tcbhighmath[boxrule=0.4pt,arc=4pt,colframe=green,drop fuzzy shadow=blue]{Mm(\ce{Na2CO3}) = \SI{105,99}{\gram\per\mol}}
			\tcbhighmath[boxrule=0.4pt,arc=4pt,colframe=yellow,drop fuzzy shadow=green]{Mm(\ce{CaCO3}) = \SI{100,09}{\gram\per\mol}}
		$$
		Reacción sin ajustar: \tcbhighmath[boxrule=0.4pt,arc=4pt,colframe=green,drop fuzzy shadow=blue]{\ce{Na2CO3(ac)}} + \ce{CaCl2(ac)} \ce{->} \tcbhighmath[boxrule=0.4pt,arc=4pt,colframe=yellow,drop fuzzy shadow=green]{\ce{CaCO3(ac)}} + \ce{NaCl(ac)}
						}
\end{frame}

\begin{frame}
	\frametitle{\ejerciciocmd}
	\framesubtitle{Resolución (\rom{1}): ajuste de la reacción}
	\structure{Reacción:}
	\begin{overprint}
		\onslide<1>
			\centering\tcbhighmath[boxrule=0.4pt,arc=4pt,colframe=green,drop fuzzy shadow=blue]{\ce{Na2}}\ce{CO3 + CaCl2 -> CaCO3 +}
			\tcbhighmath[boxrule=0.4pt,arc=4pt,colframe=green,drop fuzzy shadow=blue]{\ce{Na}}\ce{Cl} (\textbf{\underline{sin ajustar}})
		\onslide<2>
			\centering\tcbhighmath[boxrule=0.4pt,arc=4pt,colframe=green,drop fuzzy shadow=blue]{\ce{Na2}}\ce{CO3 + CaCl2 -> CaCO3 +}
			\tcbhighmath[boxrule=0.4pt,arc=4pt,colframe=green,drop fuzzy shadow=blue]{\textbf{2}\ce{Na}}\ce{Cl} (\textbf{\underline{ajustado}})
		\onslide<3>
			\centering\ce{Na2}
			\tcbhighmath[boxrule=0.4pt,arc=4pt,colframe=red,drop fuzzy shadow=blue]{\ce{CO3}}
			\ce{ + CaCl2 -> Ca}
			\tcbhighmath[boxrule=0.4pt,arc=4pt,colframe=red,drop fuzzy shadow=blue]{\ce{CO3}}
			\ce{ + \textbf{2}NaCl} (\textbf{\underline{ajustado}})
		\onslide<4>
			\centering\ce{Na2CO3 + }
			\tcbhighmath[boxrule=0.4pt,arc=4pt,colframe=yellow,drop fuzzy shadow=green]{\ce{Ca}}
			\ce{Cl2 -> }
			\tcbhighmath[boxrule=0.4pt,arc=4pt,colframe=yellow,drop fuzzy shadow=green]{\ce{Ca}}
			\ce{CO3 + \textbf{2}NaCl} (\textbf{\underline{ajustado}})
		\onslide<5>
			\centering\ce{Na2CO3 + Ca}
			\tcbhighmath[boxrule=0.4pt,arc=4pt,colframe=orange,drop fuzzy shadow=green]{\ce{\textbf{Cl2}}}
			\ce{ -> CaCO3 + }
			\tcbhighmath[boxrule=0.4pt,arc=4pt,colframe=orange,drop fuzzy shadow=green]{\ce{\textbf{2}Na\textbf{Cl}}} (\textbf{\underline{ajustado}})
		\onslide<6->
			\centering\ce{Na2CO3 + CaCl2 -> CaCO3 + \textbf{2}NaCl} (\textbf{\underline{ajustado}})
	\end{overprint}
\end{frame}

\begin{frame}
	\frametitle{\ejerciciocmd}
	\framesubtitle{Resolución (\rom{2}): masa pura de \ce{Na2CO3}}
	\structure{Reacción ajustada:} \ce{Na2CO3 + CaCl2 -> CaCO3 + 2NaCl}\\[.5cm]
	\begin{overprint}
		\onslide<1>
			Según la estequiometría:
			$$
				\overbrace{n(\ce{Na2CO3})}^{\text{sustancia pura}} = n(\ce{CaCO3})
			$$
		\onslide<2>
			Como $n=\rfrac{m}{Mm}$
			$$
				\frac{m_{\text{puro}}(\ce{Na2CO3})}{Mm(\ce{Na2CO3})} = \frac{m(\ce{CaCO3})}{Mm(\ce{CaCO3})}
			$$
		\onslide<3->
			Despejamos:
			$$
				m_{\text{puro}}(\ce{Na2CO3}) = m(\ce{CaCO3})\cdot\frac{Mm(\ce{Na2CO3})}{Mm(\ce{CaCO3})}
			$$	
	\end{overprint}
	\visible<4->{
		Sustituimos por los valores correspondientes:
		$$
			m_{\text{puro}}(\ce{Na2CO3}) = \SI{1,0262}{\gram}\cdot\frac{\SI{105,99}{\cancel\gram\per\cancel\mol}}{\SI{100,09}{\cancel\gram\per\cancel\mol}}
		$$
		\centering\myovalbox{\textcolor{yellow}{$m_{\text{puro}}(\ce{Na2CO3}) = \SI{1,086}{\gram}$}}
				}
\end{frame}

\begin{frame}
	\frametitle{\ejerciciocmd}
	\framesubtitle{Resolución (\rom{3}): porcentaje de pureza}
	\structure{NOTA:} \textbf{Siempre} $m_{\text{puro}}(\text{sustancia}) < m(\text{sustancia})$
	$$
		\si{\percent}\text{ pureza \ce{Na2CO3}} = \frac{m_{\text{puro}}(\ce{Na2CO3})}{m(\ce{Na2CO3})}\times\num{100}
	$$
	\visible<2->{
		Sustituimos valores:
		$$
			\si{\percent}\text{ pureza \ce{Na2CO3}} = \frac{\SI{1,0876}{\cancel\gram}}{\underbrace{\SI{1,2048}{\cancel\gram}}_{\text{(del enunciado)}}}\times\num{100}
		$$
		$$
			\tcbhighmath[boxrule=0.4pt,arc=4pt,colframe=green,drop fuzzy shadow=blue]{\si{\percent}\text{ pureza \ce{Na2CO3}} = \SI{90,20}{\percent}}
		$$
				}
\end{frame}
