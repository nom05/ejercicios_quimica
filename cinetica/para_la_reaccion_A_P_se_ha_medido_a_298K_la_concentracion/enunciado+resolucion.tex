\begin{frame}
	\frametitle{\ejerciciocmd}
	\framesubtitle{Enunciado}
	\textbf{
		Una reacción tiene una constante de velocidad de \SI{,017}{\per\second} a \SI{298}{\kelvin} y una energía libre de activación del \SI{27,235}{\kilo\joule\per\mol}. La adición de un catalizador disminuye dicha energía de activación hasta un \SI{33}{\percent} de su valor inicial. Calcule la nueva constante de velocidad.
\resultadocmd{ \SI{26,86}{\per\second} }

		}
\end{frame}

\begin{frame}
	\frametitle{\ejerciciocmd}
	\framesubtitle{Datos del problema}
	\begin{center}
		{\huge¿$n$ y $k$?}\\[.3cm]
		\tcbhighmath[boxrule=0.4pt,arc=4pt,colframe=green,drop fuzzy shadow=yellow]{\ce{A -> P}}\\[.3cm]
		\begin{tabular}{SS}
			\toprule
				{$[\ce{A}]~(\si{\Molar})$} & {$t~(\si{\second})$} \\
			\midrule
				1      &  0   \\
				0,9667 &  1,5 \\
				0,9456 &  2,5 \\
				0,9255 &  3,5 \\
				0,9062 &  4,5 \\
				0,8877 &  5,5 \\
				0,8699 &  6,5 \\
				0,8529 &  7,5 \\
				0,8365 &  8,5 \\
				0,8207 &  9,5 \\
				0,8055 & 10,5 \\
				0,7908 & 11,5 \\
				0,7767 & 12,5 \\
				0,7631 & 13,5 \\
				0,7499 & 14,5 \\
				0,7372 & 15,5 \\
				0,7249 & 16,5 \\
			\bottomrule
		\end{tabular}
	\end{center}
\end{frame}

\begin{frame}
	\frametitle{\ejerciciocmd}
	\framesubtitle{Resolución (\rom{1}): comprobando que la reacción es de orden 0}
	\structure{Reacción:}\quad\ce{A -> P}
	\structure{Ecuación cinética general:}
	$$
		v=-\frac{\Delta[\ce{A}]}{\Delta t}=\frac{\Delta[\ce{P}]}{\Delta t}=k[\ce{A}]^n
	$$\\[.4cm]
	\begin{center}
		{\LARGE\textbf{\textit{\underline{Método integral de velocidades}}}}
	\end{center}
	\structure{Orden 0 ($n=0$)}
	$$
		-\dv{[\ce{A}]}{t}=k\overbrace{[\ce{A}]^0}^{1}\Rightarrow
		-\dv{[\ce{A}]}{t}=k\Rightarrow\dd{[\ce{A}]}=-k\dd{t}\Rightarrow\int_{[\ce{A}]_0}^{[\ce{A}]}\dd{[\ce{A}]}=-k\int_{0}^{t}\dd{t}
	$$
	$$
		[\ce{A}]-[\ce{A}]_0=-k\qty(t - 0)\Rightarrow
		\tcbhighmath[boxrule=0.4pt,arc=4pt,colframe=black,drop fuzzy shadow=black]{[\ce{A}]=[\ce{A}]_0 - k\vdot t}
	$$
\end{frame}

\begin{frame}
	\frametitle{\ejerciciocmd}
	\framesubtitle{Resolución (\rom{2}): comprobando que la reacción es de orden 0}
    \begin{center}
        \begin{figure}
            \caption{Representación de la concentración de \ce{A} ($[\ce{A}]$) con respecto al tiempo ($t$). La cinética de la reacción \textbf{\underline{no se ajusta}} bien a una reacción de \textbf{\underline{orden 0}}. $[\ce{A}]_0$ es muy distinto, $r^2$ está alejado y la trayectoria de los puntos parece una curva frente a la recta de regresión lineal.}
            % GNUPLOT: LaTeX picture
\setlength{\unitlength}{0.240900pt}
\ifx\plotpoint\undefined\newsavebox{\plotpoint}\fi
\sbox{\plotpoint}{\rule[-0.200pt]{0.400pt}{0.400pt}}%
\begin{picture}(1299,826)(0,0)
\sbox{\plotpoint}{\rule[-0.200pt]{0.400pt}{0.400pt}}%
\put(191.0,131.0){\rule[-0.200pt]{4.818pt}{0.400pt}}
\put(171,131){\makebox(0,0)[r]{$0.700$}}
\put(1218.0,131.0){\rule[-0.200pt]{4.818pt}{0.400pt}}
\put(191.0,240.0){\rule[-0.200pt]{4.818pt}{0.400pt}}
\put(171,240){\makebox(0,0)[r]{$0.750$}}
\put(1218.0,240.0){\rule[-0.200pt]{4.818pt}{0.400pt}}
\put(191.0,349.0){\rule[-0.200pt]{4.818pt}{0.400pt}}
\put(171,349){\makebox(0,0)[r]{$0.800$}}
\put(1218.0,349.0){\rule[-0.200pt]{4.818pt}{0.400pt}}
\put(191.0,458.0){\rule[-0.200pt]{4.818pt}{0.400pt}}
\put(171,458){\makebox(0,0)[r]{$0.850$}}
\put(1218.0,458.0){\rule[-0.200pt]{4.818pt}{0.400pt}}
\put(191.0,566.0){\rule[-0.200pt]{4.818pt}{0.400pt}}
\put(171,566){\makebox(0,0)[r]{$0.900$}}
\put(1218.0,566.0){\rule[-0.200pt]{4.818pt}{0.400pt}}
\put(191.0,675.0){\rule[-0.200pt]{4.818pt}{0.400pt}}
\put(171,675){\makebox(0,0)[r]{$0.950$}}
\put(1218.0,675.0){\rule[-0.200pt]{4.818pt}{0.400pt}}
\put(191.0,784.0){\rule[-0.200pt]{4.818pt}{0.400pt}}
\put(171,784){\makebox(0,0)[r]{$1.000$}}
\put(1218.0,784.0){\rule[-0.200pt]{4.818pt}{0.400pt}}
\put(191.0,131.0){\rule[-0.200pt]{0.400pt}{4.818pt}}
\put(191,90){\makebox(0,0){$0$}}
\put(191.0,764.0){\rule[-0.200pt]{0.400pt}{4.818pt}}
\put(307.0,131.0){\rule[-0.200pt]{0.400pt}{4.818pt}}
\put(307,90){\makebox(0,0){$2$}}
\put(307.0,764.0){\rule[-0.200pt]{0.400pt}{4.818pt}}
\put(424.0,131.0){\rule[-0.200pt]{0.400pt}{4.818pt}}
\put(424,90){\makebox(0,0){$4$}}
\put(424.0,764.0){\rule[-0.200pt]{0.400pt}{4.818pt}}
\put(540.0,131.0){\rule[-0.200pt]{0.400pt}{4.818pt}}
\put(540,90){\makebox(0,0){$6$}}
\put(540.0,764.0){\rule[-0.200pt]{0.400pt}{4.818pt}}
\put(656.0,131.0){\rule[-0.200pt]{0.400pt}{4.818pt}}
\put(656,90){\makebox(0,0){$8$}}
\put(656.0,764.0){\rule[-0.200pt]{0.400pt}{4.818pt}}
\put(773.0,131.0){\rule[-0.200pt]{0.400pt}{4.818pt}}
\put(773,90){\makebox(0,0){$10$}}
\put(773.0,764.0){\rule[-0.200pt]{0.400pt}{4.818pt}}
\put(889.0,131.0){\rule[-0.200pt]{0.400pt}{4.818pt}}
\put(889,90){\makebox(0,0){$12$}}
\put(889.0,764.0){\rule[-0.200pt]{0.400pt}{4.818pt}}
\put(1005.0,131.0){\rule[-0.200pt]{0.400pt}{4.818pt}}
\put(1005,90){\makebox(0,0){$14$}}
\put(1005.0,764.0){\rule[-0.200pt]{0.400pt}{4.818pt}}
\put(1122.0,131.0){\rule[-0.200pt]{0.400pt}{4.818pt}}
\put(1122,90){\makebox(0,0){$16$}}
\put(1122.0,764.0){\rule[-0.200pt]{0.400pt}{4.818pt}}
\put(1238.0,131.0){\rule[-0.200pt]{0.400pt}{4.818pt}}
\put(1238,90){\makebox(0,0){$18$}}
\put(1238.0,764.0){\rule[-0.200pt]{0.400pt}{4.818pt}}
\put(191.0,131.0){\rule[-0.200pt]{0.400pt}{157.308pt}}
\put(191.0,131.0){\rule[-0.200pt]{252.222pt}{0.400pt}}
\put(1238.0,131.0){\rule[-0.200pt]{0.400pt}{157.308pt}}
\put(191.0,784.0){\rule[-0.200pt]{252.222pt}{0.400pt}}
\put(36,457){\makebox(0,0){$[A]~(\si{\Molar})$\quad}}
\put(714,29){\makebox(0,0){$\text{tiempo}~(\si{\second})$}}
\put(191,784){\makebox(0,0){$\blacksquare$}}
\put(278,712){\makebox(0,0){$\blacksquare$}}
\put(336,666){\makebox(0,0){$\blacksquare$}}
\put(395,622){\makebox(0,0){$\blacksquare$}}
\put(453,580){\makebox(0,0){$\blacksquare$}}
\put(511,540){\makebox(0,0){$\blacksquare$}}
\put(569,501){\makebox(0,0){$\blacksquare$}}
\put(627,464){\makebox(0,0){$\blacksquare$}}
\put(685,428){\makebox(0,0){$\blacksquare$}}
\put(744,394){\makebox(0,0){$\blacksquare$}}
\put(802,361){\makebox(0,0){$\blacksquare$}}
\put(860,329){\makebox(0,0){$\blacksquare$}}
\put(918,298){\makebox(0,0){$\blacksquare$}}
\put(976,268){\makebox(0,0){$\blacksquare$}}
\put(1034,240){\makebox(0,0){$\blacksquare$}}
\put(1093,212){\makebox(0,0){$\blacksquare$}}
\put(1151,185){\makebox(0,0){$\blacksquare$}}
\put(1078,743){\makebox(0,0)[r]{$[A]$ = 0.9839 + -0.0164$t$ ($r^2$ = 0.9922)}}
\put(1098.0,743.0){\rule[-0.200pt]{24.090pt}{0.400pt}}
\put(191,749){\usebox{\plotpoint}}
\multiput(191.00,747.93)(0.852,-0.482){9}{\rule{0.767pt}{0.116pt}}
\multiput(191.00,748.17)(8.409,-6.000){2}{\rule{0.383pt}{0.400pt}}
\multiput(201.00,741.93)(0.762,-0.482){9}{\rule{0.700pt}{0.116pt}}
\multiput(201.00,742.17)(7.547,-6.000){2}{\rule{0.350pt}{0.400pt}}
\multiput(210.00,735.93)(0.852,-0.482){9}{\rule{0.767pt}{0.116pt}}
\multiput(210.00,736.17)(8.409,-6.000){2}{\rule{0.383pt}{0.400pt}}
\multiput(220.00,729.93)(0.852,-0.482){9}{\rule{0.767pt}{0.116pt}}
\multiput(220.00,730.17)(8.409,-6.000){2}{\rule{0.383pt}{0.400pt}}
\multiput(230.00,723.93)(0.762,-0.482){9}{\rule{0.700pt}{0.116pt}}
\multiput(230.00,724.17)(7.547,-6.000){2}{\rule{0.350pt}{0.400pt}}
\multiput(239.00,717.93)(0.852,-0.482){9}{\rule{0.767pt}{0.116pt}}
\multiput(239.00,718.17)(8.409,-6.000){2}{\rule{0.383pt}{0.400pt}}
\multiput(249.00,711.93)(0.852,-0.482){9}{\rule{0.767pt}{0.116pt}}
\multiput(249.00,712.17)(8.409,-6.000){2}{\rule{0.383pt}{0.400pt}}
\multiput(259.00,705.93)(0.852,-0.482){9}{\rule{0.767pt}{0.116pt}}
\multiput(259.00,706.17)(8.409,-6.000){2}{\rule{0.383pt}{0.400pt}}
\multiput(269.00,699.93)(0.762,-0.482){9}{\rule{0.700pt}{0.116pt}}
\multiput(269.00,700.17)(7.547,-6.000){2}{\rule{0.350pt}{0.400pt}}
\multiput(278.00,693.93)(0.852,-0.482){9}{\rule{0.767pt}{0.116pt}}
\multiput(278.00,694.17)(8.409,-6.000){2}{\rule{0.383pt}{0.400pt}}
\multiput(288.00,687.93)(0.852,-0.482){9}{\rule{0.767pt}{0.116pt}}
\multiput(288.00,688.17)(8.409,-6.000){2}{\rule{0.383pt}{0.400pt}}
\multiput(298.00,681.93)(0.762,-0.482){9}{\rule{0.700pt}{0.116pt}}
\multiput(298.00,682.17)(7.547,-6.000){2}{\rule{0.350pt}{0.400pt}}
\multiput(307.00,675.93)(0.852,-0.482){9}{\rule{0.767pt}{0.116pt}}
\multiput(307.00,676.17)(8.409,-6.000){2}{\rule{0.383pt}{0.400pt}}
\multiput(317.00,669.93)(1.044,-0.477){7}{\rule{0.900pt}{0.115pt}}
\multiput(317.00,670.17)(8.132,-5.000){2}{\rule{0.450pt}{0.400pt}}
\multiput(327.00,664.93)(0.762,-0.482){9}{\rule{0.700pt}{0.116pt}}
\multiput(327.00,665.17)(7.547,-6.000){2}{\rule{0.350pt}{0.400pt}}
\multiput(336.00,658.93)(0.852,-0.482){9}{\rule{0.767pt}{0.116pt}}
\multiput(336.00,659.17)(8.409,-6.000){2}{\rule{0.383pt}{0.400pt}}
\multiput(346.00,652.93)(0.852,-0.482){9}{\rule{0.767pt}{0.116pt}}
\multiput(346.00,653.17)(8.409,-6.000){2}{\rule{0.383pt}{0.400pt}}
\multiput(356.00,646.93)(0.852,-0.482){9}{\rule{0.767pt}{0.116pt}}
\multiput(356.00,647.17)(8.409,-6.000){2}{\rule{0.383pt}{0.400pt}}
\multiput(366.00,640.93)(0.762,-0.482){9}{\rule{0.700pt}{0.116pt}}
\multiput(366.00,641.17)(7.547,-6.000){2}{\rule{0.350pt}{0.400pt}}
\multiput(375.00,634.93)(0.852,-0.482){9}{\rule{0.767pt}{0.116pt}}
\multiput(375.00,635.17)(8.409,-6.000){2}{\rule{0.383pt}{0.400pt}}
\multiput(385.00,628.93)(0.852,-0.482){9}{\rule{0.767pt}{0.116pt}}
\multiput(385.00,629.17)(8.409,-6.000){2}{\rule{0.383pt}{0.400pt}}
\multiput(395.00,622.93)(0.762,-0.482){9}{\rule{0.700pt}{0.116pt}}
\multiput(395.00,623.17)(7.547,-6.000){2}{\rule{0.350pt}{0.400pt}}
\multiput(404.00,616.93)(0.852,-0.482){9}{\rule{0.767pt}{0.116pt}}
\multiput(404.00,617.17)(8.409,-6.000){2}{\rule{0.383pt}{0.400pt}}
\multiput(414.00,610.93)(0.852,-0.482){9}{\rule{0.767pt}{0.116pt}}
\multiput(414.00,611.17)(8.409,-6.000){2}{\rule{0.383pt}{0.400pt}}
\multiput(424.00,604.93)(0.762,-0.482){9}{\rule{0.700pt}{0.116pt}}
\multiput(424.00,605.17)(7.547,-6.000){2}{\rule{0.350pt}{0.400pt}}
\multiput(433.00,598.93)(0.852,-0.482){9}{\rule{0.767pt}{0.116pt}}
\multiput(433.00,599.17)(8.409,-6.000){2}{\rule{0.383pt}{0.400pt}}
\multiput(443.00,592.93)(0.852,-0.482){9}{\rule{0.767pt}{0.116pt}}
\multiput(443.00,593.17)(8.409,-6.000){2}{\rule{0.383pt}{0.400pt}}
\multiput(453.00,586.93)(0.762,-0.482){9}{\rule{0.700pt}{0.116pt}}
\multiput(453.00,587.17)(7.547,-6.000){2}{\rule{0.350pt}{0.400pt}}
\multiput(462.00,580.93)(0.852,-0.482){9}{\rule{0.767pt}{0.116pt}}
\multiput(462.00,581.17)(8.409,-6.000){2}{\rule{0.383pt}{0.400pt}}
\multiput(472.00,574.93)(0.852,-0.482){9}{\rule{0.767pt}{0.116pt}}
\multiput(472.00,575.17)(8.409,-6.000){2}{\rule{0.383pt}{0.400pt}}
\multiput(482.00,568.93)(0.852,-0.482){9}{\rule{0.767pt}{0.116pt}}
\multiput(482.00,569.17)(8.409,-6.000){2}{\rule{0.383pt}{0.400pt}}
\multiput(492.00,562.93)(0.762,-0.482){9}{\rule{0.700pt}{0.116pt}}
\multiput(492.00,563.17)(7.547,-6.000){2}{\rule{0.350pt}{0.400pt}}
\multiput(501.00,556.93)(0.852,-0.482){9}{\rule{0.767pt}{0.116pt}}
\multiput(501.00,557.17)(8.409,-6.000){2}{\rule{0.383pt}{0.400pt}}
\multiput(511.00,550.93)(0.852,-0.482){9}{\rule{0.767pt}{0.116pt}}
\multiput(511.00,551.17)(8.409,-6.000){2}{\rule{0.383pt}{0.400pt}}
\multiput(521.00,544.93)(0.762,-0.482){9}{\rule{0.700pt}{0.116pt}}
\multiput(521.00,545.17)(7.547,-6.000){2}{\rule{0.350pt}{0.400pt}}
\multiput(530.00,538.93)(0.852,-0.482){9}{\rule{0.767pt}{0.116pt}}
\multiput(530.00,539.17)(8.409,-6.000){2}{\rule{0.383pt}{0.400pt}}
\multiput(540.00,532.93)(0.852,-0.482){9}{\rule{0.767pt}{0.116pt}}
\multiput(540.00,533.17)(8.409,-6.000){2}{\rule{0.383pt}{0.400pt}}
\multiput(550.00,526.93)(0.762,-0.482){9}{\rule{0.700pt}{0.116pt}}
\multiput(550.00,527.17)(7.547,-6.000){2}{\rule{0.350pt}{0.400pt}}
\multiput(559.00,520.93)(0.852,-0.482){9}{\rule{0.767pt}{0.116pt}}
\multiput(559.00,521.17)(8.409,-6.000){2}{\rule{0.383pt}{0.400pt}}
\multiput(569.00,514.93)(0.852,-0.482){9}{\rule{0.767pt}{0.116pt}}
\multiput(569.00,515.17)(8.409,-6.000){2}{\rule{0.383pt}{0.400pt}}
\multiput(579.00,508.93)(0.933,-0.477){7}{\rule{0.820pt}{0.115pt}}
\multiput(579.00,509.17)(7.298,-5.000){2}{\rule{0.410pt}{0.400pt}}
\multiput(588.00,503.93)(0.852,-0.482){9}{\rule{0.767pt}{0.116pt}}
\multiput(588.00,504.17)(8.409,-6.000){2}{\rule{0.383pt}{0.400pt}}
\multiput(598.00,497.93)(0.852,-0.482){9}{\rule{0.767pt}{0.116pt}}
\multiput(598.00,498.17)(8.409,-6.000){2}{\rule{0.383pt}{0.400pt}}
\multiput(608.00,491.93)(0.852,-0.482){9}{\rule{0.767pt}{0.116pt}}
\multiput(608.00,492.17)(8.409,-6.000){2}{\rule{0.383pt}{0.400pt}}
\multiput(618.00,485.93)(0.762,-0.482){9}{\rule{0.700pt}{0.116pt}}
\multiput(618.00,486.17)(7.547,-6.000){2}{\rule{0.350pt}{0.400pt}}
\multiput(627.00,479.93)(0.852,-0.482){9}{\rule{0.767pt}{0.116pt}}
\multiput(627.00,480.17)(8.409,-6.000){2}{\rule{0.383pt}{0.400pt}}
\multiput(637.00,473.93)(0.852,-0.482){9}{\rule{0.767pt}{0.116pt}}
\multiput(637.00,474.17)(8.409,-6.000){2}{\rule{0.383pt}{0.400pt}}
\multiput(647.00,467.93)(0.762,-0.482){9}{\rule{0.700pt}{0.116pt}}
\multiput(647.00,468.17)(7.547,-6.000){2}{\rule{0.350pt}{0.400pt}}
\multiput(656.00,461.93)(0.852,-0.482){9}{\rule{0.767pt}{0.116pt}}
\multiput(656.00,462.17)(8.409,-6.000){2}{\rule{0.383pt}{0.400pt}}
\multiput(666.00,455.93)(0.852,-0.482){9}{\rule{0.767pt}{0.116pt}}
\multiput(666.00,456.17)(8.409,-6.000){2}{\rule{0.383pt}{0.400pt}}
\multiput(676.00,449.93)(0.762,-0.482){9}{\rule{0.700pt}{0.116pt}}
\multiput(676.00,450.17)(7.547,-6.000){2}{\rule{0.350pt}{0.400pt}}
\multiput(685.00,443.93)(0.852,-0.482){9}{\rule{0.767pt}{0.116pt}}
\multiput(685.00,444.17)(8.409,-6.000){2}{\rule{0.383pt}{0.400pt}}
\multiput(695.00,437.93)(0.852,-0.482){9}{\rule{0.767pt}{0.116pt}}
\multiput(695.00,438.17)(8.409,-6.000){2}{\rule{0.383pt}{0.400pt}}
\multiput(705.00,431.93)(0.852,-0.482){9}{\rule{0.767pt}{0.116pt}}
\multiput(705.00,432.17)(8.409,-6.000){2}{\rule{0.383pt}{0.400pt}}
\multiput(715.00,425.93)(0.762,-0.482){9}{\rule{0.700pt}{0.116pt}}
\multiput(715.00,426.17)(7.547,-6.000){2}{\rule{0.350pt}{0.400pt}}
\multiput(724.00,419.93)(0.852,-0.482){9}{\rule{0.767pt}{0.116pt}}
\multiput(724.00,420.17)(8.409,-6.000){2}{\rule{0.383pt}{0.400pt}}
\multiput(734.00,413.93)(0.852,-0.482){9}{\rule{0.767pt}{0.116pt}}
\multiput(734.00,414.17)(8.409,-6.000){2}{\rule{0.383pt}{0.400pt}}
\multiput(744.00,407.93)(0.762,-0.482){9}{\rule{0.700pt}{0.116pt}}
\multiput(744.00,408.17)(7.547,-6.000){2}{\rule{0.350pt}{0.400pt}}
\multiput(753.00,401.93)(0.852,-0.482){9}{\rule{0.767pt}{0.116pt}}
\multiput(753.00,402.17)(8.409,-6.000){2}{\rule{0.383pt}{0.400pt}}
\multiput(763.00,395.93)(0.852,-0.482){9}{\rule{0.767pt}{0.116pt}}
\multiput(763.00,396.17)(8.409,-6.000){2}{\rule{0.383pt}{0.400pt}}
\multiput(773.00,389.93)(0.762,-0.482){9}{\rule{0.700pt}{0.116pt}}
\multiput(773.00,390.17)(7.547,-6.000){2}{\rule{0.350pt}{0.400pt}}
\multiput(782.00,383.93)(0.852,-0.482){9}{\rule{0.767pt}{0.116pt}}
\multiput(782.00,384.17)(8.409,-6.000){2}{\rule{0.383pt}{0.400pt}}
\multiput(792.00,377.93)(0.852,-0.482){9}{\rule{0.767pt}{0.116pt}}
\multiput(792.00,378.17)(8.409,-6.000){2}{\rule{0.383pt}{0.400pt}}
\multiput(802.00,371.93)(0.762,-0.482){9}{\rule{0.700pt}{0.116pt}}
\multiput(802.00,372.17)(7.547,-6.000){2}{\rule{0.350pt}{0.400pt}}
\multiput(811.00,365.93)(0.852,-0.482){9}{\rule{0.767pt}{0.116pt}}
\multiput(811.00,366.17)(8.409,-6.000){2}{\rule{0.383pt}{0.400pt}}
\multiput(821.00,359.93)(0.852,-0.482){9}{\rule{0.767pt}{0.116pt}}
\multiput(821.00,360.17)(8.409,-6.000){2}{\rule{0.383pt}{0.400pt}}
\multiput(831.00,353.93)(1.044,-0.477){7}{\rule{0.900pt}{0.115pt}}
\multiput(831.00,354.17)(8.132,-5.000){2}{\rule{0.450pt}{0.400pt}}
\multiput(841.00,348.93)(0.762,-0.482){9}{\rule{0.700pt}{0.116pt}}
\multiput(841.00,349.17)(7.547,-6.000){2}{\rule{0.350pt}{0.400pt}}
\multiput(850.00,342.93)(0.852,-0.482){9}{\rule{0.767pt}{0.116pt}}
\multiput(850.00,343.17)(8.409,-6.000){2}{\rule{0.383pt}{0.400pt}}
\multiput(860.00,336.93)(0.852,-0.482){9}{\rule{0.767pt}{0.116pt}}
\multiput(860.00,337.17)(8.409,-6.000){2}{\rule{0.383pt}{0.400pt}}
\multiput(870.00,330.93)(0.762,-0.482){9}{\rule{0.700pt}{0.116pt}}
\multiput(870.00,331.17)(7.547,-6.000){2}{\rule{0.350pt}{0.400pt}}
\multiput(879.00,324.93)(0.852,-0.482){9}{\rule{0.767pt}{0.116pt}}
\multiput(879.00,325.17)(8.409,-6.000){2}{\rule{0.383pt}{0.400pt}}
\multiput(889.00,318.93)(0.852,-0.482){9}{\rule{0.767pt}{0.116pt}}
\multiput(889.00,319.17)(8.409,-6.000){2}{\rule{0.383pt}{0.400pt}}
\multiput(899.00,312.93)(0.762,-0.482){9}{\rule{0.700pt}{0.116pt}}
\multiput(899.00,313.17)(7.547,-6.000){2}{\rule{0.350pt}{0.400pt}}
\multiput(908.00,306.93)(0.852,-0.482){9}{\rule{0.767pt}{0.116pt}}
\multiput(908.00,307.17)(8.409,-6.000){2}{\rule{0.383pt}{0.400pt}}
\multiput(918.00,300.93)(0.852,-0.482){9}{\rule{0.767pt}{0.116pt}}
\multiput(918.00,301.17)(8.409,-6.000){2}{\rule{0.383pt}{0.400pt}}
\multiput(928.00,294.93)(0.762,-0.482){9}{\rule{0.700pt}{0.116pt}}
\multiput(928.00,295.17)(7.547,-6.000){2}{\rule{0.350pt}{0.400pt}}
\multiput(937.00,288.93)(0.852,-0.482){9}{\rule{0.767pt}{0.116pt}}
\multiput(937.00,289.17)(8.409,-6.000){2}{\rule{0.383pt}{0.400pt}}
\multiput(947.00,282.93)(0.852,-0.482){9}{\rule{0.767pt}{0.116pt}}
\multiput(947.00,283.17)(8.409,-6.000){2}{\rule{0.383pt}{0.400pt}}
\multiput(957.00,276.93)(0.852,-0.482){9}{\rule{0.767pt}{0.116pt}}
\multiput(957.00,277.17)(8.409,-6.000){2}{\rule{0.383pt}{0.400pt}}
\multiput(967.00,270.93)(0.762,-0.482){9}{\rule{0.700pt}{0.116pt}}
\multiput(967.00,271.17)(7.547,-6.000){2}{\rule{0.350pt}{0.400pt}}
\multiput(976.00,264.93)(0.852,-0.482){9}{\rule{0.767pt}{0.116pt}}
\multiput(976.00,265.17)(8.409,-6.000){2}{\rule{0.383pt}{0.400pt}}
\multiput(986.00,258.93)(0.852,-0.482){9}{\rule{0.767pt}{0.116pt}}
\multiput(986.00,259.17)(8.409,-6.000){2}{\rule{0.383pt}{0.400pt}}
\multiput(996.00,252.93)(0.762,-0.482){9}{\rule{0.700pt}{0.116pt}}
\multiput(996.00,253.17)(7.547,-6.000){2}{\rule{0.350pt}{0.400pt}}
\multiput(1005.00,246.93)(0.852,-0.482){9}{\rule{0.767pt}{0.116pt}}
\multiput(1005.00,247.17)(8.409,-6.000){2}{\rule{0.383pt}{0.400pt}}
\multiput(1015.00,240.93)(0.852,-0.482){9}{\rule{0.767pt}{0.116pt}}
\multiput(1015.00,241.17)(8.409,-6.000){2}{\rule{0.383pt}{0.400pt}}
\multiput(1025.00,234.93)(0.762,-0.482){9}{\rule{0.700pt}{0.116pt}}
\multiput(1025.00,235.17)(7.547,-6.000){2}{\rule{0.350pt}{0.400pt}}
\multiput(1034.00,228.93)(0.852,-0.482){9}{\rule{0.767pt}{0.116pt}}
\multiput(1034.00,229.17)(8.409,-6.000){2}{\rule{0.383pt}{0.400pt}}
\multiput(1044.00,222.93)(0.852,-0.482){9}{\rule{0.767pt}{0.116pt}}
\multiput(1044.00,223.17)(8.409,-6.000){2}{\rule{0.383pt}{0.400pt}}
\multiput(1054.00,216.93)(0.852,-0.482){9}{\rule{0.767pt}{0.116pt}}
\multiput(1054.00,217.17)(8.409,-6.000){2}{\rule{0.383pt}{0.400pt}}
\multiput(1064.00,210.93)(0.762,-0.482){9}{\rule{0.700pt}{0.116pt}}
\multiput(1064.00,211.17)(7.547,-6.000){2}{\rule{0.350pt}{0.400pt}}
\multiput(1073.00,204.93)(0.852,-0.482){9}{\rule{0.767pt}{0.116pt}}
\multiput(1073.00,205.17)(8.409,-6.000){2}{\rule{0.383pt}{0.400pt}}
\multiput(1083.00,198.93)(0.852,-0.482){9}{\rule{0.767pt}{0.116pt}}
\multiput(1083.00,199.17)(8.409,-6.000){2}{\rule{0.383pt}{0.400pt}}
\multiput(1093.00,192.93)(0.933,-0.477){7}{\rule{0.820pt}{0.115pt}}
\multiput(1093.00,193.17)(7.298,-5.000){2}{\rule{0.410pt}{0.400pt}}
\multiput(1102.00,187.93)(0.852,-0.482){9}{\rule{0.767pt}{0.116pt}}
\multiput(1102.00,188.17)(8.409,-6.000){2}{\rule{0.383pt}{0.400pt}}
\multiput(1112.00,181.93)(0.852,-0.482){9}{\rule{0.767pt}{0.116pt}}
\multiput(1112.00,182.17)(8.409,-6.000){2}{\rule{0.383pt}{0.400pt}}
\multiput(1122.00,175.93)(0.762,-0.482){9}{\rule{0.700pt}{0.116pt}}
\multiput(1122.00,176.17)(7.547,-6.000){2}{\rule{0.350pt}{0.400pt}}
\multiput(1131.00,169.93)(0.852,-0.482){9}{\rule{0.767pt}{0.116pt}}
\multiput(1131.00,170.17)(8.409,-6.000){2}{\rule{0.383pt}{0.400pt}}
\multiput(1141.00,163.93)(0.852,-0.482){9}{\rule{0.767pt}{0.116pt}}
\multiput(1141.00,164.17)(8.409,-6.000){2}{\rule{0.383pt}{0.400pt}}
\put(191.0,131.0){\rule[-0.200pt]{0.400pt}{157.308pt}}
\put(191.0,131.0){\rule[-0.200pt]{252.222pt}{0.400pt}}
\put(1238.0,131.0){\rule[-0.200pt]{0.400pt}{157.308pt}}
\put(191.0,784.0){\rule[-0.200pt]{252.222pt}{0.400pt}}
\end{picture}

        \end{figure}
    \end{center}
\end{frame}

\begin{frame}
	\frametitle{\ejerciciocmd}
	\framesubtitle{Resolución (\rom{3}): comprobando que la reacción es de orden 1}
	\structure{Reacción:}\quad\ce{A -> P}
	\structure{Ecuación cinética general:}
	$$
		v=-\frac{\Delta[\ce{A}]}{\Delta t}=\frac{\Delta[\ce{P}]}{\Delta t}=k[\ce{A}]^n
	$$\\[.4cm]
	\begin{center}
		{\LARGE\textbf{\textit{\underline{Método integral de velocidades}}}}
	\end{center}
	\structure{Orden 1 ($n=1$)}
	$$
		-\dv{[\ce{A}]}{t}=k\vdot[\ce{A}]^1\Rightarrow
		-\dv{[\ce{A}]}{t}=k\vdot[\ce{A}]\Rightarrow\frac{\dd{[\ce{A}]}}{[\ce{A}]}=-k\dd{t}\Rightarrow\int_{[\ce{A}]_0}^{[\ce{A}]}[\ce{A}]^{-1}\dd{[\ce{A}]}=-k\int_{0}^{t}\dd{t}
	$$
	$$
		\ln([\ce{A}])-\ln([\ce{A}]_0)=-k\qty(t - 0)\Rightarrow
		\tcbhighmath[boxrule=0.4pt,arc=4pt,colframe=black,drop fuzzy shadow=black]{\ln([\ce{A}])=\ln([\ce{A}]_0) - k\vdot t}
	$$
\end{frame}

\begin{frame}
	\frametitle{\ejerciciocmd}
	\framesubtitle{Resolución (\rom{4}): comprobando que la reacción es de orden 1}
	\begin{center}
		\begin{figure}
			\caption{Representación del logaritmo neperiano de la concentración de \ce{A} ($\ln([\ce{A}])$) con respecto al tiempo ($t$). La cinética de la reacción \textbf{\underline{no se ajusta}} a una reacción de \textbf{\underline{orden 1}}. $\ln([\ce{A}]_0)$ es distinto, $r^2$ no es 1 y la trayectoria de los puntos todavía parece una curva frente a la recta de regresión lineal.}
			% GNUPLOT: LaTeX picture
\setlength{\unitlength}{0.240900pt}
\ifx\plotpoint\undefined\newsavebox{\plotpoint}\fi
\sbox{\plotpoint}{\rule[-0.200pt]{0.400pt}{0.400pt}}%
\begin{picture}(1299,826)(0,0)
\sbox{\plotpoint}{\rule[-0.200pt]{0.400pt}{0.400pt}}%
\put(211.0,131.0){\rule[-0.200pt]{4.818pt}{0.400pt}}
\put(191,131){\makebox(0,0)[r]{$-0.350$}}
\put(1218.0,131.0){\rule[-0.200pt]{4.818pt}{0.400pt}}
\put(211.0,224.0){\rule[-0.200pt]{4.818pt}{0.400pt}}
\put(191,224){\makebox(0,0)[r]{$-0.300$}}
\put(1218.0,224.0){\rule[-0.200pt]{4.818pt}{0.400pt}}
\put(211.0,318.0){\rule[-0.200pt]{4.818pt}{0.400pt}}
\put(191,318){\makebox(0,0)[r]{$-0.250$}}
\put(1218.0,318.0){\rule[-0.200pt]{4.818pt}{0.400pt}}
\put(211.0,411.0){\rule[-0.200pt]{4.818pt}{0.400pt}}
\put(191,411){\makebox(0,0)[r]{$-0.200$}}
\put(1218.0,411.0){\rule[-0.200pt]{4.818pt}{0.400pt}}
\put(211.0,504.0){\rule[-0.200pt]{4.818pt}{0.400pt}}
\put(191,504){\makebox(0,0)[r]{$-0.150$}}
\put(1218.0,504.0){\rule[-0.200pt]{4.818pt}{0.400pt}}
\put(211.0,597.0){\rule[-0.200pt]{4.818pt}{0.400pt}}
\put(191,597){\makebox(0,0)[r]{$-0.100$}}
\put(1218.0,597.0){\rule[-0.200pt]{4.818pt}{0.400pt}}
\put(211.0,691.0){\rule[-0.200pt]{4.818pt}{0.400pt}}
\put(191,691){\makebox(0,0)[r]{$-0.050$}}
\put(1218.0,691.0){\rule[-0.200pt]{4.818pt}{0.400pt}}
\put(211.0,784.0){\rule[-0.200pt]{4.818pt}{0.400pt}}
\put(191,784){\makebox(0,0)[r]{$0.000$}}
\put(1218.0,784.0){\rule[-0.200pt]{4.818pt}{0.400pt}}
\put(211.0,131.0){\rule[-0.200pt]{0.400pt}{4.818pt}}
\put(211,90){\makebox(0,0){$0$}}
\put(211.0,764.0){\rule[-0.200pt]{0.400pt}{4.818pt}}
\put(325.0,131.0){\rule[-0.200pt]{0.400pt}{4.818pt}}
\put(325,90){\makebox(0,0){$2$}}
\put(325.0,764.0){\rule[-0.200pt]{0.400pt}{4.818pt}}
\put(439.0,131.0){\rule[-0.200pt]{0.400pt}{4.818pt}}
\put(439,90){\makebox(0,0){$4$}}
\put(439.0,764.0){\rule[-0.200pt]{0.400pt}{4.818pt}}
\put(553.0,131.0){\rule[-0.200pt]{0.400pt}{4.818pt}}
\put(553,90){\makebox(0,0){$6$}}
\put(553.0,764.0){\rule[-0.200pt]{0.400pt}{4.818pt}}
\put(667.0,131.0){\rule[-0.200pt]{0.400pt}{4.818pt}}
\put(667,90){\makebox(0,0){$8$}}
\put(667.0,764.0){\rule[-0.200pt]{0.400pt}{4.818pt}}
\put(782.0,131.0){\rule[-0.200pt]{0.400pt}{4.818pt}}
\put(782,90){\makebox(0,0){$10$}}
\put(782.0,764.0){\rule[-0.200pt]{0.400pt}{4.818pt}}
\put(896.0,131.0){\rule[-0.200pt]{0.400pt}{4.818pt}}
\put(896,90){\makebox(0,0){$12$}}
\put(896.0,764.0){\rule[-0.200pt]{0.400pt}{4.818pt}}
\put(1010.0,131.0){\rule[-0.200pt]{0.400pt}{4.818pt}}
\put(1010,90){\makebox(0,0){$14$}}
\put(1010.0,764.0){\rule[-0.200pt]{0.400pt}{4.818pt}}
\put(1124.0,131.0){\rule[-0.200pt]{0.400pt}{4.818pt}}
\put(1124,90){\makebox(0,0){$16$}}
\put(1124.0,764.0){\rule[-0.200pt]{0.400pt}{4.818pt}}
\put(1238.0,131.0){\rule[-0.200pt]{0.400pt}{4.818pt}}
\put(1238,90){\makebox(0,0){$18$}}
\put(1238.0,764.0){\rule[-0.200pt]{0.400pt}{4.818pt}}
\put(211.0,131.0){\rule[-0.200pt]{0.400pt}{157.308pt}}
\put(211.0,131.0){\rule[-0.200pt]{247.404pt}{0.400pt}}
\put(1238.0,131.0){\rule[-0.200pt]{0.400pt}{157.308pt}}
\put(211.0,784.0){\rule[-0.200pt]{247.404pt}{0.400pt}}
\put(36,457){\makebox(0,0){$\ln([A])$\quad}}
\put(724,29){\makebox(0,0){$\text{tiempo}~(\si{\second})$}}
\put(211,784){\makebox(0,0){$\blacksquare$}}
\put(297,721){\makebox(0,0){$\blacksquare$}}
\put(354,680){\makebox(0,0){$\blacksquare$}}
\put(411,640){\makebox(0,0){$\blacksquare$}}
\put(468,600){\makebox(0,0){$\blacksquare$}}
\put(525,562){\makebox(0,0){$\blacksquare$}}
\put(582,524){\makebox(0,0){$\blacksquare$}}
\put(639,487){\makebox(0,0){$\blacksquare$}}
\put(696,451){\makebox(0,0){$\blacksquare$}}
\put(753,415){\makebox(0,0){$\blacksquare$}}
\put(810,380){\makebox(0,0){$\blacksquare$}}
\put(867,346){\makebox(0,0){$\blacksquare$}}
\put(924,313){\makebox(0,0){$\blacksquare$}}
\put(981,280){\makebox(0,0){$\blacksquare$}}
\put(1038,247){\makebox(0,0){$\blacksquare$}}
\put(1095,215){\makebox(0,0){$\blacksquare$}}
\put(1152,184){\makebox(0,0){$\blacksquare$}}
\put(1078,743){\makebox(0,0)[r]{$ln([A])$ = -0.0092 + -0.0194$t$ ($r^2$ = 0.9981)}}
\put(1098.0,743.0){\rule[-0.200pt]{24.090pt}{0.400pt}}
\put(211,767){\usebox{\plotpoint}}
\multiput(211.00,765.93)(0.852,-0.482){9}{\rule{0.767pt}{0.116pt}}
\multiput(211.00,766.17)(8.409,-6.000){2}{\rule{0.383pt}{0.400pt}}
\multiput(221.00,759.93)(0.762,-0.482){9}{\rule{0.700pt}{0.116pt}}
\multiput(221.00,760.17)(7.547,-6.000){2}{\rule{0.350pt}{0.400pt}}
\multiput(230.00,753.93)(0.852,-0.482){9}{\rule{0.767pt}{0.116pt}}
\multiput(230.00,754.17)(8.409,-6.000){2}{\rule{0.383pt}{0.400pt}}
\multiput(240.00,747.93)(0.762,-0.482){9}{\rule{0.700pt}{0.116pt}}
\multiput(240.00,748.17)(7.547,-6.000){2}{\rule{0.350pt}{0.400pt}}
\multiput(249.00,741.93)(0.852,-0.482){9}{\rule{0.767pt}{0.116pt}}
\multiput(249.00,742.17)(8.409,-6.000){2}{\rule{0.383pt}{0.400pt}}
\multiput(259.00,735.93)(0.762,-0.482){9}{\rule{0.700pt}{0.116pt}}
\multiput(259.00,736.17)(7.547,-6.000){2}{\rule{0.350pt}{0.400pt}}
\multiput(268.00,729.93)(0.852,-0.482){9}{\rule{0.767pt}{0.116pt}}
\multiput(268.00,730.17)(8.409,-6.000){2}{\rule{0.383pt}{0.400pt}}
\multiput(278.00,723.93)(0.762,-0.482){9}{\rule{0.700pt}{0.116pt}}
\multiput(278.00,724.17)(7.547,-6.000){2}{\rule{0.350pt}{0.400pt}}
\multiput(287.00,717.93)(0.852,-0.482){9}{\rule{0.767pt}{0.116pt}}
\multiput(287.00,718.17)(8.409,-6.000){2}{\rule{0.383pt}{0.400pt}}
\multiput(297.00,711.93)(0.762,-0.482){9}{\rule{0.700pt}{0.116pt}}
\multiput(297.00,712.17)(7.547,-6.000){2}{\rule{0.350pt}{0.400pt}}
\multiput(306.00,705.93)(0.852,-0.482){9}{\rule{0.767pt}{0.116pt}}
\multiput(306.00,706.17)(8.409,-6.000){2}{\rule{0.383pt}{0.400pt}}
\multiput(316.00,699.93)(0.762,-0.482){9}{\rule{0.700pt}{0.116pt}}
\multiput(316.00,700.17)(7.547,-6.000){2}{\rule{0.350pt}{0.400pt}}
\multiput(325.00,693.93)(0.721,-0.485){11}{\rule{0.671pt}{0.117pt}}
\multiput(325.00,694.17)(8.606,-7.000){2}{\rule{0.336pt}{0.400pt}}
\multiput(335.00,686.93)(0.762,-0.482){9}{\rule{0.700pt}{0.116pt}}
\multiput(335.00,687.17)(7.547,-6.000){2}{\rule{0.350pt}{0.400pt}}
\multiput(344.00,680.93)(0.852,-0.482){9}{\rule{0.767pt}{0.116pt}}
\multiput(344.00,681.17)(8.409,-6.000){2}{\rule{0.383pt}{0.400pt}}
\multiput(354.00,674.93)(0.762,-0.482){9}{\rule{0.700pt}{0.116pt}}
\multiput(354.00,675.17)(7.547,-6.000){2}{\rule{0.350pt}{0.400pt}}
\multiput(363.00,668.93)(0.852,-0.482){9}{\rule{0.767pt}{0.116pt}}
\multiput(363.00,669.17)(8.409,-6.000){2}{\rule{0.383pt}{0.400pt}}
\multiput(373.00,662.93)(0.762,-0.482){9}{\rule{0.700pt}{0.116pt}}
\multiput(373.00,663.17)(7.547,-6.000){2}{\rule{0.350pt}{0.400pt}}
\multiput(382.00,656.93)(0.852,-0.482){9}{\rule{0.767pt}{0.116pt}}
\multiput(382.00,657.17)(8.409,-6.000){2}{\rule{0.383pt}{0.400pt}}
\multiput(392.00,650.93)(0.762,-0.482){9}{\rule{0.700pt}{0.116pt}}
\multiput(392.00,651.17)(7.547,-6.000){2}{\rule{0.350pt}{0.400pt}}
\multiput(401.00,644.93)(0.852,-0.482){9}{\rule{0.767pt}{0.116pt}}
\multiput(401.00,645.17)(8.409,-6.000){2}{\rule{0.383pt}{0.400pt}}
\multiput(411.00,638.93)(0.762,-0.482){9}{\rule{0.700pt}{0.116pt}}
\multiput(411.00,639.17)(7.547,-6.000){2}{\rule{0.350pt}{0.400pt}}
\multiput(420.00,632.93)(0.852,-0.482){9}{\rule{0.767pt}{0.116pt}}
\multiput(420.00,633.17)(8.409,-6.000){2}{\rule{0.383pt}{0.400pt}}
\multiput(430.00,626.93)(0.762,-0.482){9}{\rule{0.700pt}{0.116pt}}
\multiput(430.00,627.17)(7.547,-6.000){2}{\rule{0.350pt}{0.400pt}}
\multiput(439.00,620.93)(0.852,-0.482){9}{\rule{0.767pt}{0.116pt}}
\multiput(439.00,621.17)(8.409,-6.000){2}{\rule{0.383pt}{0.400pt}}
\multiput(449.00,614.93)(0.762,-0.482){9}{\rule{0.700pt}{0.116pt}}
\multiput(449.00,615.17)(7.547,-6.000){2}{\rule{0.350pt}{0.400pt}}
\multiput(458.00,608.93)(0.852,-0.482){9}{\rule{0.767pt}{0.116pt}}
\multiput(458.00,609.17)(8.409,-6.000){2}{\rule{0.383pt}{0.400pt}}
\multiput(468.00,602.93)(0.762,-0.482){9}{\rule{0.700pt}{0.116pt}}
\multiput(468.00,603.17)(7.547,-6.000){2}{\rule{0.350pt}{0.400pt}}
\multiput(477.00,596.93)(0.852,-0.482){9}{\rule{0.767pt}{0.116pt}}
\multiput(477.00,597.17)(8.409,-6.000){2}{\rule{0.383pt}{0.400pt}}
\multiput(487.00,590.93)(0.762,-0.482){9}{\rule{0.700pt}{0.116pt}}
\multiput(487.00,591.17)(7.547,-6.000){2}{\rule{0.350pt}{0.400pt}}
\multiput(496.00,584.93)(0.852,-0.482){9}{\rule{0.767pt}{0.116pt}}
\multiput(496.00,585.17)(8.409,-6.000){2}{\rule{0.383pt}{0.400pt}}
\multiput(506.00,578.93)(0.762,-0.482){9}{\rule{0.700pt}{0.116pt}}
\multiput(506.00,579.17)(7.547,-6.000){2}{\rule{0.350pt}{0.400pt}}
\multiput(515.00,572.93)(0.852,-0.482){9}{\rule{0.767pt}{0.116pt}}
\multiput(515.00,573.17)(8.409,-6.000){2}{\rule{0.383pt}{0.400pt}}
\multiput(525.00,566.93)(0.762,-0.482){9}{\rule{0.700pt}{0.116pt}}
\multiput(525.00,567.17)(7.547,-6.000){2}{\rule{0.350pt}{0.400pt}}
\multiput(534.00,560.93)(0.852,-0.482){9}{\rule{0.767pt}{0.116pt}}
\multiput(534.00,561.17)(8.409,-6.000){2}{\rule{0.383pt}{0.400pt}}
\multiput(544.00,554.93)(0.762,-0.482){9}{\rule{0.700pt}{0.116pt}}
\multiput(544.00,555.17)(7.547,-6.000){2}{\rule{0.350pt}{0.400pt}}
\multiput(553.00,548.93)(0.852,-0.482){9}{\rule{0.767pt}{0.116pt}}
\multiput(553.00,549.17)(8.409,-6.000){2}{\rule{0.383pt}{0.400pt}}
\multiput(563.00,542.93)(0.762,-0.482){9}{\rule{0.700pt}{0.116pt}}
\multiput(563.00,543.17)(7.547,-6.000){2}{\rule{0.350pt}{0.400pt}}
\multiput(572.00,536.93)(0.852,-0.482){9}{\rule{0.767pt}{0.116pt}}
\multiput(572.00,537.17)(8.409,-6.000){2}{\rule{0.383pt}{0.400pt}}
\multiput(582.00,530.93)(0.762,-0.482){9}{\rule{0.700pt}{0.116pt}}
\multiput(582.00,531.17)(7.547,-6.000){2}{\rule{0.350pt}{0.400pt}}
\multiput(591.00,524.93)(0.852,-0.482){9}{\rule{0.767pt}{0.116pt}}
\multiput(591.00,525.17)(8.409,-6.000){2}{\rule{0.383pt}{0.400pt}}
\multiput(601.00,518.93)(0.762,-0.482){9}{\rule{0.700pt}{0.116pt}}
\multiput(601.00,519.17)(7.547,-6.000){2}{\rule{0.350pt}{0.400pt}}
\multiput(610.00,512.93)(0.852,-0.482){9}{\rule{0.767pt}{0.116pt}}
\multiput(610.00,513.17)(8.409,-6.000){2}{\rule{0.383pt}{0.400pt}}
\multiput(620.00,506.93)(0.762,-0.482){9}{\rule{0.700pt}{0.116pt}}
\multiput(620.00,507.17)(7.547,-6.000){2}{\rule{0.350pt}{0.400pt}}
\multiput(629.00,500.93)(0.852,-0.482){9}{\rule{0.767pt}{0.116pt}}
\multiput(629.00,501.17)(8.409,-6.000){2}{\rule{0.383pt}{0.400pt}}
\multiput(639.00,494.93)(0.762,-0.482){9}{\rule{0.700pt}{0.116pt}}
\multiput(639.00,495.17)(7.547,-6.000){2}{\rule{0.350pt}{0.400pt}}
\multiput(648.00,488.93)(0.852,-0.482){9}{\rule{0.767pt}{0.116pt}}
\multiput(648.00,489.17)(8.409,-6.000){2}{\rule{0.383pt}{0.400pt}}
\multiput(658.00,482.93)(0.645,-0.485){11}{\rule{0.614pt}{0.117pt}}
\multiput(658.00,483.17)(7.725,-7.000){2}{\rule{0.307pt}{0.400pt}}
\multiput(667.00,475.93)(0.852,-0.482){9}{\rule{0.767pt}{0.116pt}}
\multiput(667.00,476.17)(8.409,-6.000){2}{\rule{0.383pt}{0.400pt}}
\multiput(677.00,469.93)(0.762,-0.482){9}{\rule{0.700pt}{0.116pt}}
\multiput(677.00,470.17)(7.547,-6.000){2}{\rule{0.350pt}{0.400pt}}
\multiput(686.00,463.93)(0.852,-0.482){9}{\rule{0.767pt}{0.116pt}}
\multiput(686.00,464.17)(8.409,-6.000){2}{\rule{0.383pt}{0.400pt}}
\multiput(696.00,457.93)(0.762,-0.482){9}{\rule{0.700pt}{0.116pt}}
\multiput(696.00,458.17)(7.547,-6.000){2}{\rule{0.350pt}{0.400pt}}
\multiput(705.00,451.93)(0.852,-0.482){9}{\rule{0.767pt}{0.116pt}}
\multiput(705.00,452.17)(8.409,-6.000){2}{\rule{0.383pt}{0.400pt}}
\multiput(715.00,445.93)(0.852,-0.482){9}{\rule{0.767pt}{0.116pt}}
\multiput(715.00,446.17)(8.409,-6.000){2}{\rule{0.383pt}{0.400pt}}
\multiput(725.00,439.93)(0.762,-0.482){9}{\rule{0.700pt}{0.116pt}}
\multiput(725.00,440.17)(7.547,-6.000){2}{\rule{0.350pt}{0.400pt}}
\multiput(734.00,433.93)(0.852,-0.482){9}{\rule{0.767pt}{0.116pt}}
\multiput(734.00,434.17)(8.409,-6.000){2}{\rule{0.383pt}{0.400pt}}
\multiput(744.00,427.93)(0.762,-0.482){9}{\rule{0.700pt}{0.116pt}}
\multiput(744.00,428.17)(7.547,-6.000){2}{\rule{0.350pt}{0.400pt}}
\multiput(753.00,421.93)(0.852,-0.482){9}{\rule{0.767pt}{0.116pt}}
\multiput(753.00,422.17)(8.409,-6.000){2}{\rule{0.383pt}{0.400pt}}
\multiput(763.00,415.93)(0.762,-0.482){9}{\rule{0.700pt}{0.116pt}}
\multiput(763.00,416.17)(7.547,-6.000){2}{\rule{0.350pt}{0.400pt}}
\multiput(772.00,409.93)(0.852,-0.482){9}{\rule{0.767pt}{0.116pt}}
\multiput(772.00,410.17)(8.409,-6.000){2}{\rule{0.383pt}{0.400pt}}
\multiput(782.00,403.93)(0.762,-0.482){9}{\rule{0.700pt}{0.116pt}}
\multiput(782.00,404.17)(7.547,-6.000){2}{\rule{0.350pt}{0.400pt}}
\multiput(791.00,397.93)(0.852,-0.482){9}{\rule{0.767pt}{0.116pt}}
\multiput(791.00,398.17)(8.409,-6.000){2}{\rule{0.383pt}{0.400pt}}
\multiput(801.00,391.93)(0.762,-0.482){9}{\rule{0.700pt}{0.116pt}}
\multiput(801.00,392.17)(7.547,-6.000){2}{\rule{0.350pt}{0.400pt}}
\multiput(810.00,385.93)(0.852,-0.482){9}{\rule{0.767pt}{0.116pt}}
\multiput(810.00,386.17)(8.409,-6.000){2}{\rule{0.383pt}{0.400pt}}
\multiput(820.00,379.93)(0.762,-0.482){9}{\rule{0.700pt}{0.116pt}}
\multiput(820.00,380.17)(7.547,-6.000){2}{\rule{0.350pt}{0.400pt}}
\multiput(829.00,373.93)(0.852,-0.482){9}{\rule{0.767pt}{0.116pt}}
\multiput(829.00,374.17)(8.409,-6.000){2}{\rule{0.383pt}{0.400pt}}
\multiput(839.00,367.93)(0.762,-0.482){9}{\rule{0.700pt}{0.116pt}}
\multiput(839.00,368.17)(7.547,-6.000){2}{\rule{0.350pt}{0.400pt}}
\multiput(848.00,361.93)(0.852,-0.482){9}{\rule{0.767pt}{0.116pt}}
\multiput(848.00,362.17)(8.409,-6.000){2}{\rule{0.383pt}{0.400pt}}
\multiput(858.00,355.93)(0.762,-0.482){9}{\rule{0.700pt}{0.116pt}}
\multiput(858.00,356.17)(7.547,-6.000){2}{\rule{0.350pt}{0.400pt}}
\multiput(867.00,349.93)(0.852,-0.482){9}{\rule{0.767pt}{0.116pt}}
\multiput(867.00,350.17)(8.409,-6.000){2}{\rule{0.383pt}{0.400pt}}
\multiput(877.00,343.93)(0.762,-0.482){9}{\rule{0.700pt}{0.116pt}}
\multiput(877.00,344.17)(7.547,-6.000){2}{\rule{0.350pt}{0.400pt}}
\multiput(886.00,337.93)(0.852,-0.482){9}{\rule{0.767pt}{0.116pt}}
\multiput(886.00,338.17)(8.409,-6.000){2}{\rule{0.383pt}{0.400pt}}
\multiput(896.00,331.93)(0.762,-0.482){9}{\rule{0.700pt}{0.116pt}}
\multiput(896.00,332.17)(7.547,-6.000){2}{\rule{0.350pt}{0.400pt}}
\multiput(905.00,325.93)(0.852,-0.482){9}{\rule{0.767pt}{0.116pt}}
\multiput(905.00,326.17)(8.409,-6.000){2}{\rule{0.383pt}{0.400pt}}
\multiput(915.00,319.93)(0.762,-0.482){9}{\rule{0.700pt}{0.116pt}}
\multiput(915.00,320.17)(7.547,-6.000){2}{\rule{0.350pt}{0.400pt}}
\multiput(924.00,313.93)(0.852,-0.482){9}{\rule{0.767pt}{0.116pt}}
\multiput(924.00,314.17)(8.409,-6.000){2}{\rule{0.383pt}{0.400pt}}
\multiput(934.00,307.93)(0.762,-0.482){9}{\rule{0.700pt}{0.116pt}}
\multiput(934.00,308.17)(7.547,-6.000){2}{\rule{0.350pt}{0.400pt}}
\multiput(943.00,301.93)(0.852,-0.482){9}{\rule{0.767pt}{0.116pt}}
\multiput(943.00,302.17)(8.409,-6.000){2}{\rule{0.383pt}{0.400pt}}
\multiput(953.00,295.93)(0.762,-0.482){9}{\rule{0.700pt}{0.116pt}}
\multiput(953.00,296.17)(7.547,-6.000){2}{\rule{0.350pt}{0.400pt}}
\multiput(962.00,289.93)(0.852,-0.482){9}{\rule{0.767pt}{0.116pt}}
\multiput(962.00,290.17)(8.409,-6.000){2}{\rule{0.383pt}{0.400pt}}
\multiput(972.00,283.93)(0.762,-0.482){9}{\rule{0.700pt}{0.116pt}}
\multiput(972.00,284.17)(7.547,-6.000){2}{\rule{0.350pt}{0.400pt}}
\multiput(981.00,277.93)(0.852,-0.482){9}{\rule{0.767pt}{0.116pt}}
\multiput(981.00,278.17)(8.409,-6.000){2}{\rule{0.383pt}{0.400pt}}
\multiput(991.00,271.93)(0.645,-0.485){11}{\rule{0.614pt}{0.117pt}}
\multiput(991.00,272.17)(7.725,-7.000){2}{\rule{0.307pt}{0.400pt}}
\multiput(1000.00,264.93)(0.852,-0.482){9}{\rule{0.767pt}{0.116pt}}
\multiput(1000.00,265.17)(8.409,-6.000){2}{\rule{0.383pt}{0.400pt}}
\multiput(1010.00,258.93)(0.762,-0.482){9}{\rule{0.700pt}{0.116pt}}
\multiput(1010.00,259.17)(7.547,-6.000){2}{\rule{0.350pt}{0.400pt}}
\multiput(1019.00,252.93)(0.852,-0.482){9}{\rule{0.767pt}{0.116pt}}
\multiput(1019.00,253.17)(8.409,-6.000){2}{\rule{0.383pt}{0.400pt}}
\multiput(1029.00,246.93)(0.762,-0.482){9}{\rule{0.700pt}{0.116pt}}
\multiput(1029.00,247.17)(7.547,-6.000){2}{\rule{0.350pt}{0.400pt}}
\multiput(1038.00,240.93)(0.852,-0.482){9}{\rule{0.767pt}{0.116pt}}
\multiput(1038.00,241.17)(8.409,-6.000){2}{\rule{0.383pt}{0.400pt}}
\multiput(1048.00,234.93)(0.762,-0.482){9}{\rule{0.700pt}{0.116pt}}
\multiput(1048.00,235.17)(7.547,-6.000){2}{\rule{0.350pt}{0.400pt}}
\multiput(1057.00,228.93)(0.852,-0.482){9}{\rule{0.767pt}{0.116pt}}
\multiput(1057.00,229.17)(8.409,-6.000){2}{\rule{0.383pt}{0.400pt}}
\multiput(1067.00,222.93)(0.762,-0.482){9}{\rule{0.700pt}{0.116pt}}
\multiput(1067.00,223.17)(7.547,-6.000){2}{\rule{0.350pt}{0.400pt}}
\multiput(1076.00,216.93)(0.852,-0.482){9}{\rule{0.767pt}{0.116pt}}
\multiput(1076.00,217.17)(8.409,-6.000){2}{\rule{0.383pt}{0.400pt}}
\multiput(1086.00,210.93)(0.762,-0.482){9}{\rule{0.700pt}{0.116pt}}
\multiput(1086.00,211.17)(7.547,-6.000){2}{\rule{0.350pt}{0.400pt}}
\multiput(1095.00,204.93)(0.852,-0.482){9}{\rule{0.767pt}{0.116pt}}
\multiput(1095.00,205.17)(8.409,-6.000){2}{\rule{0.383pt}{0.400pt}}
\multiput(1105.00,198.93)(0.762,-0.482){9}{\rule{0.700pt}{0.116pt}}
\multiput(1105.00,199.17)(7.547,-6.000){2}{\rule{0.350pt}{0.400pt}}
\multiput(1114.00,192.93)(0.852,-0.482){9}{\rule{0.767pt}{0.116pt}}
\multiput(1114.00,193.17)(8.409,-6.000){2}{\rule{0.383pt}{0.400pt}}
\multiput(1124.00,186.93)(0.762,-0.482){9}{\rule{0.700pt}{0.116pt}}
\multiput(1124.00,187.17)(7.547,-6.000){2}{\rule{0.350pt}{0.400pt}}
\multiput(1133.00,180.93)(0.852,-0.482){9}{\rule{0.767pt}{0.116pt}}
\multiput(1133.00,181.17)(8.409,-6.000){2}{\rule{0.383pt}{0.400pt}}
\multiput(1143.00,174.93)(0.762,-0.482){9}{\rule{0.700pt}{0.116pt}}
\multiput(1143.00,175.17)(7.547,-6.000){2}{\rule{0.350pt}{0.400pt}}
\put(211.0,131.0){\rule[-0.200pt]{0.400pt}{157.308pt}}
\put(211.0,131.0){\rule[-0.200pt]{247.404pt}{0.400pt}}
\put(1238.0,131.0){\rule[-0.200pt]{0.400pt}{157.308pt}}
\put(211.0,784.0){\rule[-0.200pt]{247.404pt}{0.400pt}}
\end{picture}

		\end{figure}
	\end{center}
\end{frame}

\begin{frame}
	\frametitle{\ejerciciocmd}
	\framesubtitle{Resolución (\rom{5}): comprobando que la reacción es de orden 2}
	\structure{Reacción:}\quad\ce{A -> P}
	\structure{Ecuación cinética general:}
	$$
		v=-\frac{\Delta[\ce{A}]}{\Delta t}=\frac{\Delta[\ce{P}]}{\Delta t}=k[\ce{A}]^n
	$$\\[.4cm]
	\begin{center}
		{\LARGE\textbf{\textit{\underline{Método integral de velocidades}}}}
	\end{center}
	\structure{Orden 2 ($n=2$)}
	$$
		-\dv{[\ce{A}]}{t}=k\vdot[\ce{A}]^2\Rightarrow
		-\dv{[\ce{A}]}{t}=k\vdot[\ce{A}]^2\Rightarrow\frac{\dd{[\ce{A}]}}{[\ce{A}]^2}=-k\dd{t}\Rightarrow\int_{[\ce{A}]_0}^{[\ce{A}]}\frac{1}{[\ce{A}]^2}\dd{[\ce{A}]}=-k\int_{0}^{t}\dd{t}
	$$
	$$
		-\frac{1}{[\ce{A}]}+\frac{1}{[\ce{A}]_0}=-k\qty(t - 0)\Rightarrow
		\tcbhighmath[boxrule=0.4pt,arc=4pt,colframe=black,drop fuzzy shadow=black]{\frac{1}{[\ce{A}]}=\frac{1}{[\ce{A}]_0}+k\vdot t}
	$$
\end{frame}

\begin{frame}
	\frametitle{\ejerciciocmd}
	\framesubtitle{Resolución (\rom{6}): comprobando que la reacción es de orden 2}
	\begin{center}
		\begin{figure}
			\caption{Representación de la inversa de la concentración de \ce{A} $\left(\rfrac{1}{[\ce{A}]}\right)$ con respecto al tiempo ($t$). La cinética de la reacción \textbf{\underline{SÍ SE AJUSTA}} a una reacción de \textbf{\underline{orden 2}}. \tcbhighmath[boxrule=0.4pt,arc=4pt,colframe=green,drop fuzzy shadow=yellow]{k=\SI{,0230}{\per\second\per\Molar}}.}
			% GNUPLOT: LaTeX picture
\setlength{\unitlength}{0.240900pt}
\ifx\plotpoint\undefined\newsavebox{\plotpoint}\fi
\sbox{\plotpoint}{\rule[-0.200pt]{0.400pt}{0.400pt}}%
\begin{picture}(1299,826)(0,0)
\sbox{\plotpoint}{\rule[-0.200pt]{0.400pt}{0.400pt}}%
\put(191.0,131.0){\rule[-0.200pt]{4.818pt}{0.400pt}}
\put(171,131){\makebox(0,0)[r]{$0.950$}}
\put(1218.0,131.0){\rule[-0.200pt]{4.818pt}{0.400pt}}
\put(191.0,204.0){\rule[-0.200pt]{4.818pt}{0.400pt}}
\put(171,204){\makebox(0,0)[r]{$1.000$}}
\put(1218.0,204.0){\rule[-0.200pt]{4.818pt}{0.400pt}}
\put(191.0,276.0){\rule[-0.200pt]{4.818pt}{0.400pt}}
\put(171,276){\makebox(0,0)[r]{$1.050$}}
\put(1218.0,276.0){\rule[-0.200pt]{4.818pt}{0.400pt}}
\put(191.0,349.0){\rule[-0.200pt]{4.818pt}{0.400pt}}
\put(171,349){\makebox(0,0)[r]{$1.100$}}
\put(1218.0,349.0){\rule[-0.200pt]{4.818pt}{0.400pt}}
\put(191.0,421.0){\rule[-0.200pt]{4.818pt}{0.400pt}}
\put(171,421){\makebox(0,0)[r]{$1.150$}}
\put(1218.0,421.0){\rule[-0.200pt]{4.818pt}{0.400pt}}
\put(191.0,494.0){\rule[-0.200pt]{4.818pt}{0.400pt}}
\put(171,494){\makebox(0,0)[r]{$1.200$}}
\put(1218.0,494.0){\rule[-0.200pt]{4.818pt}{0.400pt}}
\put(191.0,566.0){\rule[-0.200pt]{4.818pt}{0.400pt}}
\put(171,566){\makebox(0,0)[r]{$1.250$}}
\put(1218.0,566.0){\rule[-0.200pt]{4.818pt}{0.400pt}}
\put(191.0,639.0){\rule[-0.200pt]{4.818pt}{0.400pt}}
\put(171,639){\makebox(0,0)[r]{$1.300$}}
\put(1218.0,639.0){\rule[-0.200pt]{4.818pt}{0.400pt}}
\put(191.0,711.0){\rule[-0.200pt]{4.818pt}{0.400pt}}
\put(171,711){\makebox(0,0)[r]{$1.350$}}
\put(1218.0,711.0){\rule[-0.200pt]{4.818pt}{0.400pt}}
\put(191.0,784.0){\rule[-0.200pt]{4.818pt}{0.400pt}}
\put(171,784){\makebox(0,0)[r]{$1.400$}}
\put(1218.0,784.0){\rule[-0.200pt]{4.818pt}{0.400pt}}
\put(191.0,131.0){\rule[-0.200pt]{0.400pt}{4.818pt}}
\put(191,90){\makebox(0,0){$0$}}
\put(191.0,764.0){\rule[-0.200pt]{0.400pt}{4.818pt}}
\put(307.0,131.0){\rule[-0.200pt]{0.400pt}{4.818pt}}
\put(307,90){\makebox(0,0){$2$}}
\put(307.0,764.0){\rule[-0.200pt]{0.400pt}{4.818pt}}
\put(424.0,131.0){\rule[-0.200pt]{0.400pt}{4.818pt}}
\put(424,90){\makebox(0,0){$4$}}
\put(424.0,764.0){\rule[-0.200pt]{0.400pt}{4.818pt}}
\put(540.0,131.0){\rule[-0.200pt]{0.400pt}{4.818pt}}
\put(540,90){\makebox(0,0){$6$}}
\put(540.0,764.0){\rule[-0.200pt]{0.400pt}{4.818pt}}
\put(656.0,131.0){\rule[-0.200pt]{0.400pt}{4.818pt}}
\put(656,90){\makebox(0,0){$8$}}
\put(656.0,764.0){\rule[-0.200pt]{0.400pt}{4.818pt}}
\put(773.0,131.0){\rule[-0.200pt]{0.400pt}{4.818pt}}
\put(773,90){\makebox(0,0){$10$}}
\put(773.0,764.0){\rule[-0.200pt]{0.400pt}{4.818pt}}
\put(889.0,131.0){\rule[-0.200pt]{0.400pt}{4.818pt}}
\put(889,90){\makebox(0,0){$12$}}
\put(889.0,764.0){\rule[-0.200pt]{0.400pt}{4.818pt}}
\put(1005.0,131.0){\rule[-0.200pt]{0.400pt}{4.818pt}}
\put(1005,90){\makebox(0,0){$14$}}
\put(1005.0,764.0){\rule[-0.200pt]{0.400pt}{4.818pt}}
\put(1122.0,131.0){\rule[-0.200pt]{0.400pt}{4.818pt}}
\put(1122,90){\makebox(0,0){$16$}}
\put(1122.0,764.0){\rule[-0.200pt]{0.400pt}{4.818pt}}
\put(1238.0,131.0){\rule[-0.200pt]{0.400pt}{4.818pt}}
\put(1238,90){\makebox(0,0){$18$}}
\put(1238.0,764.0){\rule[-0.200pt]{0.400pt}{4.818pt}}
\put(191.0,131.0){\rule[-0.200pt]{0.400pt}{157.308pt}}
\put(191.0,131.0){\rule[-0.200pt]{252.222pt}{0.400pt}}
\put(1238.0,131.0){\rule[-0.200pt]{0.400pt}{157.308pt}}
\put(191.0,784.0){\rule[-0.200pt]{252.222pt}{0.400pt}}
\put(36,457){\makebox(0,0){$\frac{1}{[A]}~(\si{\per\Molar})$\quad\quad}}
\put(714,29){\makebox(0,0){$\text{tiempo}~(\si{\second})$}}
\put(191,204){\makebox(0,0){$\blacksquare$}}
\put(278,254){\makebox(0,0){$\blacksquare$}}
\put(336,287){\makebox(0,0){$\blacksquare$}}
\put(395,320){\makebox(0,0){$\blacksquare$}}
\put(453,354){\makebox(0,0){$\blacksquare$}}
\put(511,387){\makebox(0,0){$\blacksquare$}}
\put(569,421){\makebox(0,0){$\blacksquare$}}
\put(627,454){\makebox(0,0){$\blacksquare$}}
\put(685,487){\makebox(0,0){$\blacksquare$}}
\put(744,521){\makebox(0,0){$\blacksquare$}}
\put(802,554){\makebox(0,0){$\blacksquare$}}
\put(860,587){\makebox(0,0){$\blacksquare$}}
\put(918,621){\makebox(0,0){$\blacksquare$}}
\put(976,654){\makebox(0,0){$\blacksquare$}}
\put(1034,688){\makebox(0,0){$\blacksquare$}}
\put(1093,721){\makebox(0,0){$\blacksquare$}}
\put(1151,754){\makebox(0,0){$\blacksquare$}}
\put(1078,743){\makebox(0,0)[r]{$[A]^{-1}$ = 1.0000 + 0.0230$t$ ($r^2$ = 1.0000)}}
\put(1098.0,743.0){\rule[-0.200pt]{24.090pt}{0.400pt}}
\put(191,204){\usebox{\plotpoint}}
\multiput(191.00,204.59)(1.044,0.477){7}{\rule{0.900pt}{0.115pt}}
\multiput(191.00,203.17)(8.132,5.000){2}{\rule{0.450pt}{0.400pt}}
\multiput(201.00,209.59)(0.762,0.482){9}{\rule{0.700pt}{0.116pt}}
\multiput(201.00,208.17)(7.547,6.000){2}{\rule{0.350pt}{0.400pt}}
\multiput(210.00,215.59)(1.044,0.477){7}{\rule{0.900pt}{0.115pt}}
\multiput(210.00,214.17)(8.132,5.000){2}{\rule{0.450pt}{0.400pt}}
\multiput(220.00,220.59)(0.852,0.482){9}{\rule{0.767pt}{0.116pt}}
\multiput(220.00,219.17)(8.409,6.000){2}{\rule{0.383pt}{0.400pt}}
\multiput(230.00,226.59)(0.933,0.477){7}{\rule{0.820pt}{0.115pt}}
\multiput(230.00,225.17)(7.298,5.000){2}{\rule{0.410pt}{0.400pt}}
\multiput(239.00,231.59)(0.852,0.482){9}{\rule{0.767pt}{0.116pt}}
\multiput(239.00,230.17)(8.409,6.000){2}{\rule{0.383pt}{0.400pt}}
\multiput(249.00,237.59)(1.044,0.477){7}{\rule{0.900pt}{0.115pt}}
\multiput(249.00,236.17)(8.132,5.000){2}{\rule{0.450pt}{0.400pt}}
\multiput(259.00,242.59)(0.852,0.482){9}{\rule{0.767pt}{0.116pt}}
\multiput(259.00,241.17)(8.409,6.000){2}{\rule{0.383pt}{0.400pt}}
\multiput(269.00,248.59)(0.762,0.482){9}{\rule{0.700pt}{0.116pt}}
\multiput(269.00,247.17)(7.547,6.000){2}{\rule{0.350pt}{0.400pt}}
\multiput(278.00,254.59)(1.044,0.477){7}{\rule{0.900pt}{0.115pt}}
\multiput(278.00,253.17)(8.132,5.000){2}{\rule{0.450pt}{0.400pt}}
\multiput(288.00,259.59)(0.852,0.482){9}{\rule{0.767pt}{0.116pt}}
\multiput(288.00,258.17)(8.409,6.000){2}{\rule{0.383pt}{0.400pt}}
\multiput(298.00,265.59)(0.933,0.477){7}{\rule{0.820pt}{0.115pt}}
\multiput(298.00,264.17)(7.298,5.000){2}{\rule{0.410pt}{0.400pt}}
\multiput(307.00,270.59)(0.852,0.482){9}{\rule{0.767pt}{0.116pt}}
\multiput(307.00,269.17)(8.409,6.000){2}{\rule{0.383pt}{0.400pt}}
\multiput(317.00,276.59)(1.044,0.477){7}{\rule{0.900pt}{0.115pt}}
\multiput(317.00,275.17)(8.132,5.000){2}{\rule{0.450pt}{0.400pt}}
\multiput(327.00,281.59)(0.762,0.482){9}{\rule{0.700pt}{0.116pt}}
\multiput(327.00,280.17)(7.547,6.000){2}{\rule{0.350pt}{0.400pt}}
\multiput(336.00,287.59)(0.852,0.482){9}{\rule{0.767pt}{0.116pt}}
\multiput(336.00,286.17)(8.409,6.000){2}{\rule{0.383pt}{0.400pt}}
\multiput(346.00,293.59)(1.044,0.477){7}{\rule{0.900pt}{0.115pt}}
\multiput(346.00,292.17)(8.132,5.000){2}{\rule{0.450pt}{0.400pt}}
\multiput(356.00,298.59)(0.852,0.482){9}{\rule{0.767pt}{0.116pt}}
\multiput(356.00,297.17)(8.409,6.000){2}{\rule{0.383pt}{0.400pt}}
\multiput(366.00,304.59)(0.933,0.477){7}{\rule{0.820pt}{0.115pt}}
\multiput(366.00,303.17)(7.298,5.000){2}{\rule{0.410pt}{0.400pt}}
\multiput(375.00,309.59)(0.852,0.482){9}{\rule{0.767pt}{0.116pt}}
\multiput(375.00,308.17)(8.409,6.000){2}{\rule{0.383pt}{0.400pt}}
\multiput(385.00,315.59)(1.044,0.477){7}{\rule{0.900pt}{0.115pt}}
\multiput(385.00,314.17)(8.132,5.000){2}{\rule{0.450pt}{0.400pt}}
\multiput(395.00,320.59)(0.762,0.482){9}{\rule{0.700pt}{0.116pt}}
\multiput(395.00,319.17)(7.547,6.000){2}{\rule{0.350pt}{0.400pt}}
\multiput(404.00,326.59)(1.044,0.477){7}{\rule{0.900pt}{0.115pt}}
\multiput(404.00,325.17)(8.132,5.000){2}{\rule{0.450pt}{0.400pt}}
\multiput(414.00,331.59)(0.852,0.482){9}{\rule{0.767pt}{0.116pt}}
\multiput(414.00,330.17)(8.409,6.000){2}{\rule{0.383pt}{0.400pt}}
\multiput(424.00,337.59)(0.762,0.482){9}{\rule{0.700pt}{0.116pt}}
\multiput(424.00,336.17)(7.547,6.000){2}{\rule{0.350pt}{0.400pt}}
\multiput(433.00,343.59)(1.044,0.477){7}{\rule{0.900pt}{0.115pt}}
\multiput(433.00,342.17)(8.132,5.000){2}{\rule{0.450pt}{0.400pt}}
\multiput(443.00,348.59)(0.852,0.482){9}{\rule{0.767pt}{0.116pt}}
\multiput(443.00,347.17)(8.409,6.000){2}{\rule{0.383pt}{0.400pt}}
\multiput(453.00,354.59)(0.933,0.477){7}{\rule{0.820pt}{0.115pt}}
\multiput(453.00,353.17)(7.298,5.000){2}{\rule{0.410pt}{0.400pt}}
\multiput(462.00,359.59)(0.852,0.482){9}{\rule{0.767pt}{0.116pt}}
\multiput(462.00,358.17)(8.409,6.000){2}{\rule{0.383pt}{0.400pt}}
\multiput(472.00,365.59)(1.044,0.477){7}{\rule{0.900pt}{0.115pt}}
\multiput(472.00,364.17)(8.132,5.000){2}{\rule{0.450pt}{0.400pt}}
\multiput(482.00,370.59)(0.852,0.482){9}{\rule{0.767pt}{0.116pt}}
\multiput(482.00,369.17)(8.409,6.000){2}{\rule{0.383pt}{0.400pt}}
\multiput(492.00,376.59)(0.762,0.482){9}{\rule{0.700pt}{0.116pt}}
\multiput(492.00,375.17)(7.547,6.000){2}{\rule{0.350pt}{0.400pt}}
\multiput(501.00,382.59)(1.044,0.477){7}{\rule{0.900pt}{0.115pt}}
\multiput(501.00,381.17)(8.132,5.000){2}{\rule{0.450pt}{0.400pt}}
\multiput(511.00,387.59)(0.852,0.482){9}{\rule{0.767pt}{0.116pt}}
\multiput(511.00,386.17)(8.409,6.000){2}{\rule{0.383pt}{0.400pt}}
\multiput(521.00,393.59)(0.933,0.477){7}{\rule{0.820pt}{0.115pt}}
\multiput(521.00,392.17)(7.298,5.000){2}{\rule{0.410pt}{0.400pt}}
\multiput(530.00,398.59)(0.852,0.482){9}{\rule{0.767pt}{0.116pt}}
\multiput(530.00,397.17)(8.409,6.000){2}{\rule{0.383pt}{0.400pt}}
\multiput(540.00,404.59)(1.044,0.477){7}{\rule{0.900pt}{0.115pt}}
\multiput(540.00,403.17)(8.132,5.000){2}{\rule{0.450pt}{0.400pt}}
\multiput(550.00,409.59)(0.762,0.482){9}{\rule{0.700pt}{0.116pt}}
\multiput(550.00,408.17)(7.547,6.000){2}{\rule{0.350pt}{0.400pt}}
\multiput(559.00,415.59)(1.044,0.477){7}{\rule{0.900pt}{0.115pt}}
\multiput(559.00,414.17)(8.132,5.000){2}{\rule{0.450pt}{0.400pt}}
\multiput(569.00,420.59)(0.852,0.482){9}{\rule{0.767pt}{0.116pt}}
\multiput(569.00,419.17)(8.409,6.000){2}{\rule{0.383pt}{0.400pt}}
\multiput(579.00,426.59)(0.762,0.482){9}{\rule{0.700pt}{0.116pt}}
\multiput(579.00,425.17)(7.547,6.000){2}{\rule{0.350pt}{0.400pt}}
\multiput(588.00,432.59)(1.044,0.477){7}{\rule{0.900pt}{0.115pt}}
\multiput(588.00,431.17)(8.132,5.000){2}{\rule{0.450pt}{0.400pt}}
\multiput(598.00,437.59)(0.852,0.482){9}{\rule{0.767pt}{0.116pt}}
\multiput(598.00,436.17)(8.409,6.000){2}{\rule{0.383pt}{0.400pt}}
\multiput(608.00,443.59)(1.044,0.477){7}{\rule{0.900pt}{0.115pt}}
\multiput(608.00,442.17)(8.132,5.000){2}{\rule{0.450pt}{0.400pt}}
\multiput(618.00,448.59)(0.762,0.482){9}{\rule{0.700pt}{0.116pt}}
\multiput(618.00,447.17)(7.547,6.000){2}{\rule{0.350pt}{0.400pt}}
\multiput(627.00,454.59)(1.044,0.477){7}{\rule{0.900pt}{0.115pt}}
\multiput(627.00,453.17)(8.132,5.000){2}{\rule{0.450pt}{0.400pt}}
\multiput(637.00,459.59)(0.852,0.482){9}{\rule{0.767pt}{0.116pt}}
\multiput(637.00,458.17)(8.409,6.000){2}{\rule{0.383pt}{0.400pt}}
\multiput(647.00,465.59)(0.762,0.482){9}{\rule{0.700pt}{0.116pt}}
\multiput(647.00,464.17)(7.547,6.000){2}{\rule{0.350pt}{0.400pt}}
\multiput(656.00,471.59)(1.044,0.477){7}{\rule{0.900pt}{0.115pt}}
\multiput(656.00,470.17)(8.132,5.000){2}{\rule{0.450pt}{0.400pt}}
\multiput(666.00,476.59)(0.852,0.482){9}{\rule{0.767pt}{0.116pt}}
\multiput(666.00,475.17)(8.409,6.000){2}{\rule{0.383pt}{0.400pt}}
\multiput(676.00,482.59)(0.933,0.477){7}{\rule{0.820pt}{0.115pt}}
\multiput(676.00,481.17)(7.298,5.000){2}{\rule{0.410pt}{0.400pt}}
\multiput(685.00,487.59)(0.852,0.482){9}{\rule{0.767pt}{0.116pt}}
\multiput(685.00,486.17)(8.409,6.000){2}{\rule{0.383pt}{0.400pt}}
\multiput(695.00,493.59)(1.044,0.477){7}{\rule{0.900pt}{0.115pt}}
\multiput(695.00,492.17)(8.132,5.000){2}{\rule{0.450pt}{0.400pt}}
\multiput(705.00,498.59)(0.852,0.482){9}{\rule{0.767pt}{0.116pt}}
\multiput(705.00,497.17)(8.409,6.000){2}{\rule{0.383pt}{0.400pt}}
\multiput(715.00,504.59)(0.933,0.477){7}{\rule{0.820pt}{0.115pt}}
\multiput(715.00,503.17)(7.298,5.000){2}{\rule{0.410pt}{0.400pt}}
\multiput(724.00,509.59)(0.852,0.482){9}{\rule{0.767pt}{0.116pt}}
\multiput(724.00,508.17)(8.409,6.000){2}{\rule{0.383pt}{0.400pt}}
\multiput(734.00,515.59)(0.852,0.482){9}{\rule{0.767pt}{0.116pt}}
\multiput(734.00,514.17)(8.409,6.000){2}{\rule{0.383pt}{0.400pt}}
\multiput(744.00,521.59)(0.933,0.477){7}{\rule{0.820pt}{0.115pt}}
\multiput(744.00,520.17)(7.298,5.000){2}{\rule{0.410pt}{0.400pt}}
\multiput(753.00,526.59)(0.852,0.482){9}{\rule{0.767pt}{0.116pt}}
\multiput(753.00,525.17)(8.409,6.000){2}{\rule{0.383pt}{0.400pt}}
\multiput(763.00,532.59)(1.044,0.477){7}{\rule{0.900pt}{0.115pt}}
\multiput(763.00,531.17)(8.132,5.000){2}{\rule{0.450pt}{0.400pt}}
\multiput(773.00,537.59)(0.762,0.482){9}{\rule{0.700pt}{0.116pt}}
\multiput(773.00,536.17)(7.547,6.000){2}{\rule{0.350pt}{0.400pt}}
\multiput(782.00,543.59)(1.044,0.477){7}{\rule{0.900pt}{0.115pt}}
\multiput(782.00,542.17)(8.132,5.000){2}{\rule{0.450pt}{0.400pt}}
\multiput(792.00,548.59)(0.852,0.482){9}{\rule{0.767pt}{0.116pt}}
\multiput(792.00,547.17)(8.409,6.000){2}{\rule{0.383pt}{0.400pt}}
\multiput(802.00,554.59)(0.762,0.482){9}{\rule{0.700pt}{0.116pt}}
\multiput(802.00,553.17)(7.547,6.000){2}{\rule{0.350pt}{0.400pt}}
\multiput(811.00,560.59)(1.044,0.477){7}{\rule{0.900pt}{0.115pt}}
\multiput(811.00,559.17)(8.132,5.000){2}{\rule{0.450pt}{0.400pt}}
\multiput(821.00,565.59)(0.852,0.482){9}{\rule{0.767pt}{0.116pt}}
\multiput(821.00,564.17)(8.409,6.000){2}{\rule{0.383pt}{0.400pt}}
\multiput(831.00,571.59)(1.044,0.477){7}{\rule{0.900pt}{0.115pt}}
\multiput(831.00,570.17)(8.132,5.000){2}{\rule{0.450pt}{0.400pt}}
\multiput(841.00,576.59)(0.762,0.482){9}{\rule{0.700pt}{0.116pt}}
\multiput(841.00,575.17)(7.547,6.000){2}{\rule{0.350pt}{0.400pt}}
\multiput(850.00,582.59)(1.044,0.477){7}{\rule{0.900pt}{0.115pt}}
\multiput(850.00,581.17)(8.132,5.000){2}{\rule{0.450pt}{0.400pt}}
\multiput(860.00,587.59)(0.852,0.482){9}{\rule{0.767pt}{0.116pt}}
\multiput(860.00,586.17)(8.409,6.000){2}{\rule{0.383pt}{0.400pt}}
\multiput(870.00,593.59)(0.933,0.477){7}{\rule{0.820pt}{0.115pt}}
\multiput(870.00,592.17)(7.298,5.000){2}{\rule{0.410pt}{0.400pt}}
\multiput(879.00,598.59)(0.852,0.482){9}{\rule{0.767pt}{0.116pt}}
\multiput(879.00,597.17)(8.409,6.000){2}{\rule{0.383pt}{0.400pt}}
\multiput(889.00,604.59)(0.852,0.482){9}{\rule{0.767pt}{0.116pt}}
\multiput(889.00,603.17)(8.409,6.000){2}{\rule{0.383pt}{0.400pt}}
\multiput(899.00,610.59)(0.933,0.477){7}{\rule{0.820pt}{0.115pt}}
\multiput(899.00,609.17)(7.298,5.000){2}{\rule{0.410pt}{0.400pt}}
\multiput(908.00,615.59)(0.852,0.482){9}{\rule{0.767pt}{0.116pt}}
\multiput(908.00,614.17)(8.409,6.000){2}{\rule{0.383pt}{0.400pt}}
\multiput(918.00,621.59)(1.044,0.477){7}{\rule{0.900pt}{0.115pt}}
\multiput(918.00,620.17)(8.132,5.000){2}{\rule{0.450pt}{0.400pt}}
\multiput(928.00,626.59)(0.762,0.482){9}{\rule{0.700pt}{0.116pt}}
\multiput(928.00,625.17)(7.547,6.000){2}{\rule{0.350pt}{0.400pt}}
\multiput(937.00,632.59)(1.044,0.477){7}{\rule{0.900pt}{0.115pt}}
\multiput(937.00,631.17)(8.132,5.000){2}{\rule{0.450pt}{0.400pt}}
\multiput(947.00,637.59)(0.852,0.482){9}{\rule{0.767pt}{0.116pt}}
\multiput(947.00,636.17)(8.409,6.000){2}{\rule{0.383pt}{0.400pt}}
\multiput(957.00,643.59)(0.852,0.482){9}{\rule{0.767pt}{0.116pt}}
\multiput(957.00,642.17)(8.409,6.000){2}{\rule{0.383pt}{0.400pt}}
\multiput(967.00,649.59)(0.933,0.477){7}{\rule{0.820pt}{0.115pt}}
\multiput(967.00,648.17)(7.298,5.000){2}{\rule{0.410pt}{0.400pt}}
\multiput(976.00,654.59)(0.852,0.482){9}{\rule{0.767pt}{0.116pt}}
\multiput(976.00,653.17)(8.409,6.000){2}{\rule{0.383pt}{0.400pt}}
\multiput(986.00,660.59)(1.044,0.477){7}{\rule{0.900pt}{0.115pt}}
\multiput(986.00,659.17)(8.132,5.000){2}{\rule{0.450pt}{0.400pt}}
\multiput(996.00,665.59)(0.762,0.482){9}{\rule{0.700pt}{0.116pt}}
\multiput(996.00,664.17)(7.547,6.000){2}{\rule{0.350pt}{0.400pt}}
\multiput(1005.00,671.59)(1.044,0.477){7}{\rule{0.900pt}{0.115pt}}
\multiput(1005.00,670.17)(8.132,5.000){2}{\rule{0.450pt}{0.400pt}}
\multiput(1015.00,676.59)(0.852,0.482){9}{\rule{0.767pt}{0.116pt}}
\multiput(1015.00,675.17)(8.409,6.000){2}{\rule{0.383pt}{0.400pt}}
\multiput(1025.00,682.59)(0.933,0.477){7}{\rule{0.820pt}{0.115pt}}
\multiput(1025.00,681.17)(7.298,5.000){2}{\rule{0.410pt}{0.400pt}}
\multiput(1034.00,687.59)(0.852,0.482){9}{\rule{0.767pt}{0.116pt}}
\multiput(1034.00,686.17)(8.409,6.000){2}{\rule{0.383pt}{0.400pt}}
\multiput(1044.00,693.59)(0.852,0.482){9}{\rule{0.767pt}{0.116pt}}
\multiput(1044.00,692.17)(8.409,6.000){2}{\rule{0.383pt}{0.400pt}}
\multiput(1054.00,699.59)(1.044,0.477){7}{\rule{0.900pt}{0.115pt}}
\multiput(1054.00,698.17)(8.132,5.000){2}{\rule{0.450pt}{0.400pt}}
\multiput(1064.00,704.59)(0.762,0.482){9}{\rule{0.700pt}{0.116pt}}
\multiput(1064.00,703.17)(7.547,6.000){2}{\rule{0.350pt}{0.400pt}}
\multiput(1073.00,710.59)(1.044,0.477){7}{\rule{0.900pt}{0.115pt}}
\multiput(1073.00,709.17)(8.132,5.000){2}{\rule{0.450pt}{0.400pt}}
\multiput(1083.00,715.59)(0.852,0.482){9}{\rule{0.767pt}{0.116pt}}
\multiput(1083.00,714.17)(8.409,6.000){2}{\rule{0.383pt}{0.400pt}}
\multiput(1093.00,721.59)(0.933,0.477){7}{\rule{0.820pt}{0.115pt}}
\multiput(1093.00,720.17)(7.298,5.000){2}{\rule{0.410pt}{0.400pt}}
\multiput(1102.00,726.59)(0.852,0.482){9}{\rule{0.767pt}{0.116pt}}
\multiput(1102.00,725.17)(8.409,6.000){2}{\rule{0.383pt}{0.400pt}}
\multiput(1112.00,732.59)(0.852,0.482){9}{\rule{0.767pt}{0.116pt}}
\multiput(1112.00,731.17)(8.409,6.000){2}{\rule{0.383pt}{0.400pt}}
\multiput(1122.00,738.59)(0.933,0.477){7}{\rule{0.820pt}{0.115pt}}
\multiput(1122.00,737.17)(7.298,5.000){2}{\rule{0.410pt}{0.400pt}}
\multiput(1131.00,743.59)(0.852,0.482){9}{\rule{0.767pt}{0.116pt}}
\multiput(1131.00,742.17)(8.409,6.000){2}{\rule{0.383pt}{0.400pt}}
\multiput(1141.00,749.59)(1.044,0.477){7}{\rule{0.900pt}{0.115pt}}
\multiput(1141.00,748.17)(8.132,5.000){2}{\rule{0.450pt}{0.400pt}}
\put(191.0,131.0){\rule[-0.200pt]{0.400pt}{157.308pt}}
\put(191.0,131.0){\rule[-0.200pt]{252.222pt}{0.400pt}}
\put(1238.0,131.0){\rule[-0.200pt]{0.400pt}{157.308pt}}
\put(191.0,784.0){\rule[-0.200pt]{252.222pt}{0.400pt}}
\end{picture}

		\end{figure}
	\end{center}
\end{frame}
