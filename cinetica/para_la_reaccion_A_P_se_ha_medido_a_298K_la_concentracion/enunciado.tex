Para la reacción \ce{A -> P} se ha medido a \SI{298}{\kelvin} la concentración del reactivo \ce{A} en función del tiempo. Haciendo uso del método de integración y de las representaciones gráficas correspondientes, indique cuál será el orden de la reacción y su constante de velocidad.
\begin{center}
	\begin{tabular}{SS}
		\toprule
			{$[\ce{A}]~(\si{\Molar})$} & {$t~(\si{\second})$} \\
		\midrule
			1      &  0   \\
			0,9667 &  1,5 \\
			0,9456 &  2,5 \\
			0,9255 &  3,5 \\
			0,9062 &  4,5 \\
			0,8877 &  5,5 \\
			0,8699 &  6,5 \\
			0,8529 &  7,5 \\
			0,8365 &  8,5 \\
			0,8207 &  9,5 \\
			0,8055 & 10,5 \\
			0,7908 & 11,5 \\
			0,7767 & 12,5 \\
			0,7631 & 13,5 \\
			0,7499 & 14,5 \\
			0,7372 & 15,5 \\
			0,7249 & 16,5 \\
		\bottomrule
	\end{tabular}
\end{center}
\resultadocmd{ \num{2}, \SI{,0230}{\per\second\per\Molar} }
