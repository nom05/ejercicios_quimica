Se han obtenido los siguientes datos de velocidad iniciales para la reacción \ce{A + 2B + C -> 2D + E}
\begin{center}
	\begin{tabular}{SSSSc}
		\toprule
		{Experimento}	& {$[\ce{A}]~(\si{\Molar})$}	& {$[\ce{B}]~(\si{\Molar})$}	& {$[\ce{C}]~(\si{\Molar})$}	& {Velocidad inicial}		\\
		\midrule
		1		 	 	& 		1,40			   		&		1,40					& 			1,00				& 		$R_1$      			\\
		2 		  		& 		 ,70			   		& 		1,40					& 			1,00				& $R_2=\rfrac{1}{2} R_1$	\\
		3		  		&			 ,70				&		 ,70					& 			1,00				& $R_3=\rfrac{1}{4} R_2$	\\
		4		  		&			1,40				& 		1,40					& 			 ,50				& $R_4 = 16 R_3$		 	\\
		5		  		& 		 ,70			  		&		 ,70					& 			 ,50				& $R_5=?$				 	\\
		\bottomrule
	\end{tabular}
\end{center}
\begin{enumerate}[label={\alph*)},font={\color{red!50!black}\bfseries}]
	\item ¿Cuáles son los órdenes de reacción con respecto a \ce{A}, \ce{B} y \ce{C}?
	\item ¿Cuál es el valor de $R_5$ respecto a $R_1$?
\end{enumerate}
\resultadocmd{ \num{1}, \num{2}, \num{-1}; $\rfrac{1}{4}R_1$ }
