\begin{frame}
	\frametitle{\ejerciciocmd}
	\framesubtitle{Enunciado}
	\textbf{
		Una reacción tiene una constante de velocidad de \SI{,017}{\per\second} a \SI{298}{\kelvin} y una energía libre de activación del \SI{27,235}{\kilo\joule\per\mol}. La adición de un catalizador disminuye dicha energía de activación hasta un \SI{33}{\percent} de su valor inicial. Calcule la nueva constante de velocidad.
\resultadocmd{ \SI{26,86}{\per\second} }

		}
\end{frame}

\begin{frame}
	\frametitle{\ejerciciocmd}
	\framesubtitle{Datos del problema}
	\begin{center}
		{\Large ¿$n$ $R_5$ (en función de $R_1$)?}\\[.3cm]
		\structure{Reacción:}\quad\ce{A + 2B + C -> 2D + E}\\[.3cm]
		\begin{tabular}{SSSSc}
			{Experimento} & {$[\ce{A}]~(\si{\Molar})$} & {$[\ce{B}]~(\si{\Molar})$} & {$[\ce{A}]~(\si{\Molar})$} & {Velocidad inicial}   \\
			1		  & 		1,40			   &		1,40				& 			1,00			 & 		$R_1$            \\
			2 		  & 		 ,70			   & 		1,40				& 			1,00			 & $R_2=\rfrac{1}{2} R_1$\\
			3		  &			 ,70			   &		 ,70				& 			1,00			 & $R_3=\rfrac{1}{4} R_2$\\
			4		  &			1,40			   & 		1,40				& 			 ,50			 & $R_4 = 16 R_3$		 \\
			5		  & 		 ,70			   &		 ,70				& 			 ,50			 & $R_5=?$				 \\
		\end{tabular}
	\end{center}
\end{frame}

\begin{frame}
	\frametitle{\ejerciciocmd}
	\framesubtitle{Resolución (\rom{1}): determinación de órdenes, $k$ y ecuación de velocidad}
	\structure{Reacción:}\quad\ce{A + 2B + C -> 2D + E}\\
	\structure{Ecuación cinética general:}~$v=k[\ce{A}]^x\vdot[\ce{B}]^y\vdot[\ce{C}]^z$
	\begin{overprint}
		\onslide<1>
			\structure{Ponemos $v_0$ en función de únicamente $R_1$:}\\[.3cm]
			\begin{center}
				\begin{tabular}{Scc}
					{Experimento} & {Velocidad inicial}		& {Velocidad inicial ($R_1$)} \\
					1			  & 		$R_1$           & $R_1$ \\
					2 			  & $R_2=\rfrac{1}{2} R_1$  & $\rfrac{1}{2}R_1$\\
					3			  & $R_3=\rfrac{1}{4} R_2$  & $\rfrac{1}{4}\vdot\rfrac{1}{2}R_1=\rfrac{1}{8}R_1$\\
					4			  & $R_4 = 16 R_3$			& $16\vdot\rfrac{1}{8}R_1=2R_1$\\
					5			  & ¿$R_5$?					& \\
				\end{tabular}
			\end{center}
		\onslide<2>
			\structure{Ponemos $v_0$ en función de únicamente $R_1$:}\\[.3cm]
			\begin{center}
				\begin{tabular}{SSSSc}
					{Experimento} & {$[\ce{A}]~(\si{\Molar})$} & {$[\ce{B}]~(\si{\Molar})$} & {$[\ce{A}]~(\si{\Molar})$} & {Velocidad inicial}   \\
					1		  & 		1,40			   &		1,40				& 			1,00			 & 		$R_1$            \\
					2 		  & 		 ,70			   & 		1,40				& 			1,00			 & $\rfrac{1}{2}R_1$\\
					3		  &			 ,70			   &		 ,70				& 			1,00			 & $\rfrac{1}{8}R_1$\\
					4		  &			1,40			   & 		1,40				& 			 ,50			 & $2R_1$		 \\
					5		  & 		 ,70			   &		 ,70				& 			 ,50			 & $R_5=?$				 \\
				\end{tabular}
			\end{center}
		\onslide<3->
			\structure{Con los datos de la tabla del enunciado escribimos las ecuaciones cinéticas:}\\
			$$
				\text{Ecuación cinética general:}\quad v=k\vdot[\ce{A}]^x\vdot[\ce{B}]^y\vdot[\ce{C}]^z
			$$
			\begin{equation}\label{eq:exp1}
				v_1=R_1 = k\vdot(\num{1,40})^x\vdot(\num{1,40})^y\vdot(\num{1,00})^z
			\end{equation}
			\begin{equation}\label{eq:exp2}
				v_2=\frac{1}{2}R_1 = k\vdot(\num{,70})^x\vdot(\num{1,40})^y\vdot(\num{1,00})^z
			\end{equation}
			\begin{equation}\label{eq:exp3}
				v_3=\frac{1}{8}R_1 = k\vdot(\num{,70})^x\vdot(\num{,70})^y\vdot(\num{1,00})^z
			\end{equation}
			\begin{equation}\label{eq:exp4}
				v_4=2R_1 = k\vdot(\num{1,40})^x\vdot(\num{1,40})^y\vdot(\num{,50})^z
			\end{equation}
			\begin{equation}\label{eq:exp5}
				v_5=R_5 = k\vdot(\num{,70})^x\vdot(\num{,70})^y\vdot(\num{,50})^z
			\end{equation}\\
	\end{overprint}
	\begin{overprint}
		\onslide<4>
			\structure{Seleccionamos dos ecuaciones que solo dependan de una incógnita ($x$, $y$ o $z$) y operamos:} Por ejemplo, ec~(1)/ec~(4).
			$$
				\frac{v_1}{v_4}=\frac{\cancel{R_1}}{2\cancel{R_1}}=\frac{\cancel{k}}{\cancel{k}}\vdot\left(\frac{\num{1,40}^x}{\num{1,40}^x}\right)\vdot\left(\frac{\num{1,40}^y}{\num{1,40}^y}\right)\vdot\left(\frac{\num{1,00}^z}{\num{,50}^z}\right)
			$$
		\onslide<5>
			\structure{Operamos usando las propiedades de las potencias:}
			$$
				\frac{1}{2}=\left(\frac{\num{1,40}}{\num{1,40}}\right)^x\vdot\left(\frac{\num{1,40}}{\num{1,40}}\right)^y\vdot\left(\frac{\num{1,00}}{\num{,50}}\right)^z
			$$
		\onslide<6>
			\structure{Operamos usando las propiedades de las potencias:}
			$$
				\frac{1}{2}=1^x\vdot 1^y\vdot 2^z
			$$
		\onslide<7>
			\structure{Operamos usando las propiedades de las potencias:}
			$$
				2^{-1}=2^z
			$$
		\onslide<8>
			\structure{Operamos usando las propiedades de las potencias:}
			$$
				z=-1
			$$
		\onslide<9>
			\structure{Repetimos el proceso con otras dos ecuaciones de velocidad:} Por ejemplo, ec~(1)/ec~(2).
			$$
				\frac{v_1}{v_2}=\frac{\cancel{R_1}}{\frac{1}{2}\cancel{R_1}}=\frac{\cancel{k}}{\cancel{k}}\vdot\left(\frac{\num{1,40}^x}{\num{,70}^x}\right)\vdot\left(\frac{\num{1,40}^y}{\num{1,40}^y}\right)\vdot\left(\frac{\num{1,00}^z}{\num{1,00}^z}\right)
			$$
		\onslide<10>
			\structure{Operamos:}
			$$
				2=\left(\frac{\num{1,40}}{\num{,70}}\right)^x\vdot\left(\frac{\num{1,40}}{\num{1,40}}\right)^y\vdot\left(\frac{\num{1,00}}{\num{1,00}}\right)^z
			$$
		\onslide<11>
			\structure{Operamos:}
			$$
				2=2^x\vdot 1^y\vdot 1^z
			$$
		\onslide<12>
			\structure{Operamos:}
			$$
				x=1
			$$
		\onslide<13>
			\structure{Repetimos el proceso con otras dos ecuaciones de velocidad:} Por ejemplo, ec~(1)/ec~(3).
			$$
				\frac{v_1}{v_3}=\frac{\cancel{R_1}}{\frac{1}{8}\cancel{R_1}}=\frac{\cancel{k}}{\cancel{k}}\vdot\left(\frac{\num{1,40}^x}{\num{,70}^x}\right)\vdot\left(\frac{\num{1,40}^y}{\num{,70}^y}\right)\vdot\left(\frac{\num{1,00}^z}{\num{1,00}^z}\right)
			$$
		\onslide<14>
			\structure{Operamos:}
			$$
				8=\left(\frac{\num{1,40}}{\num{,70}}\right)^x\vdot\left(\frac{\num{1,40}}{\num{,70}}\right)^y\vdot\left(\frac{\num{1,00}}{\num{1,00}}\right)^z
			$$
		\onslide<15>
			\structure{Operamos:}
			$$
				2^x\vdot2^y\vdot 1=8\Rightarrow 2^1\vdot2^y=2^3\Rightarrow 2^y=2^3\vdot 2^{-1}\Rightarrow y=2
			$$
		\onslide<16>
			\structure{Por tanto:}
			$$
				\tcbhighmath[boxrule=0.4pt,arc=4pt,colframe=red,drop fuzzy shadow=orange]{\text{Orden 1 para \ce{A}}}\quad
				\tcbhighmath[boxrule=0.4pt,arc=4pt,colframe=red,drop fuzzy shadow=orange]{\text{Orden 2 para \ce{B}}}\quad
				\tcbhighmath[boxrule=0.4pt,arc=4pt,colframe=red,drop fuzzy shadow=orange]{\text{Orden -1 para \ce{C}}}
			$$

		\onslide<17->
			\structure{Calculamos $R_5$ expresado con $R_1$:}
			$$
				\frac{v_3}{v_5}=\frac{\cancel{k}}{\cancel{k}}\vdot\cancel{\left(\frac{\num{,70}}{\num{,70}}\right)^x}\vdot\cancel{\left(\frac{\num{,70}}{\num{,70}}\right)^y}\vdot\left(\frac{\num{1,00}}{\num{,50}}\right)^z=\frac{\rfrac{1}{8}\vdot R_1}{R_5}=\frac{R_1}{8R_5}\Rightarrow 2^z=\frac{R_1}{8R_5}\Rightarrow\frac{R_1}{R_5}=2^{-1}\vdot 2^3
			$$
			$$
				\tcbhighmath[boxrule=0.4pt,arc=4pt,colframe=red,drop fuzzy shadow=orange]{R_5=\frac{1}{4}R_1}
			$$
 	\end{overprint}
\end{frame}
