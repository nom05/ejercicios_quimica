\begin{frame}
	\frametitle{\ejerciciocmd}
	\framesubtitle{Enunciado}
	\textbf{
		Dadas las siguientes reacciones:
\begin{itemize}
    \item \ce{I2(g) + H2(g) -> 2 HI(g)}~~~$\Delta H_1 = \SI{-0,8}{\kilo\calorie}$
    \item \ce{I2(s) + H2(g) -> 2 HI(g)}~~~$\Delta H_2 = \SI{12}{\kilo\calorie}$
    \item \ce{I2(g) + H2(g) -> 2 HI(ac)}~~~$\Delta H_3 = \SI{-26,8}{\kilo\calorie}$
\end{itemize}
Calcular los parámetros que se indican a continuación:
\begin{description}%[label={\alph*)},font={\color{red!50!black}\bfseries}]
    \item[\texttt{a)}] Calor molar latente de sublimación del yodo.
    \item[\texttt{b)}] Calor molar de disolución del ácido yodhídrico.
    \item[\texttt{c)}] Número de calorías que hay que aportar para disociar en sus componentes el yoduro de hidrógeno gas contenido en un matraz de \SI{750}{\cubic\centi\meter} a \SI{25}{\celsius} y \SI{800}{\torr} de presión.
\end{description}
\resultadocmd{\SI{12,8}{\kilo\calorie}; \SI{-13,0}{\kilo\calorie}; \SI{12,9}{\calorie}}

	}
\end{frame}

\begin{frame}
	\frametitle{\ejerciciocmd}
	\framesubtitle{Resolución (\rom{1}): resumen de ecuaciones para los dos métodos}
	\begin{center}
		\ce{A -> Productos}
	\end{center}
	\structure{Ecuación de velocidad:} $v = -\dv{[\ce{A}]}{t} = k\vdot[\ce{A}]^x$
	\begin{center}
		\begin{tabular}{lccc}
						&	\textbf{M. integraciones}	&	\textbf{M. tiempo de vida media}	&	\textbf{Relación $\mathbf{t_{\rfrac{1}{2}}\Leftrightarrow[\ce{A}]_0}$}	\\
			\midrule
			\underline{\textbf{\color{blue}Orden 0}}							\\
						&	$[\ce{A}] = [\ce{A}]_0 - k\vdot t$					&
							$t_{\rfrac{1}{2}} = \frac{[\ce{A}]_0}{2\vdot k}$	&
							DIRECTA												\\[.3cm]
			\midrule
			\underline{\textbf{\color{blue}Orden 1}}							\\
						&	$\ln[\ce{A}] = \ln[\ce{A}]_0 - k\vdot t$			&
							$t_{\rfrac{1}{2}} = \frac{\ln 2}{k}$					&
							NO TIENE											\\[.3cm]
			\midrule
			\underline{\textbf{\color{blue}Orden 2}}							\\
						&	$[\ce{A}]^{-1} = [\ce{A}]^{-1}_0 + k\vdot t$		&
						$t_{\rfrac{1}{2}} = \frac{1}{[\ce{A}]_0\vdot k}$		&
						INVERSA
		\end{tabular}
	\end{center}
\end{frame}

\begin{frame}
	\frametitle{\ejerciciocmd}
	\framesubtitle{Resolución (\rom{2}): método del tiempo de vida media}
	\begin{overprint}
		\onslide<1>
			\structure{Datos iniciales:}
			\begin{center}
				\begin{tabular}{SS}
					{\textbf{Concentración~(\si{\Molar})}}			&
					{\textbf{Tiempo de reacción~(\si{\second})}}	\\
					1,0								&	  0			\\
					 ,6								&	 40			\\
					 ,5								&	 50			\\
					 ,3								&	 70			\\	
				\end{tabular}
			\end{center}
		\onslide<2>
			\structure{Parejas de datos:}
			\begin{center}
				\begin{tabular}{SS}
					{\textbf{Concentración~(\si{\Molar})}}			&
					{\textbf{Tiempo de reacción~(\si{\second})}}	\\
					\midrule
					1,0								&	  0			\\
					 ,5								&	 50			\\
					\midrule
					 ,6								&	 40			\\
					 ,3								&	 70			\\	
				\end{tabular}
			\end{center}
		\onslide<3->
			\structure{Parejas de datos:}
			\begin{center}
				\begin{tabular}{SSc}
					{\textbf{Concentración~(\si{\Molar})}}			&
					{\textbf{Tiempo de reacción~(\si{\second})}}	&
					{\textbf{$t_{\rfrac{1}{2}}$~(\si{\second})}}	\\
					\midrule
					1,0								&	  0			&	\multirow{2}{*}{{$\num{50}-\num{0}=\num{50}$}}		\\
					 ,5								&	 50			&														\\
					\midrule
					 ,6								&	 40			&	\multirow{2}{*}{{$\num{70}-\num{40}=\num{30}$}}		\\
					 ,3								&	 70			&														\\
				\end{tabular}
			\end{center}
	\end{overprint}
	\begin{enumerate}[label={Paso \arabic*)},font={\color{red!50!black}\bfseries}]
		\item<2-> Emparejamos la concentración inicial con su mitad.
		\item<3-> Calculamos el tiempo relativo necesario para pasar de esa concentración a la mitad.
		\item<4-> Estudiamos la relación que hay entre las concentraciones iniciales de nuestras parejas y $t_{\rfrac{1}{2}}$. En nuestro caso, el tiempo relativo disminuye cuando $[\ce{A}]_0$ decrece. Por tanto, es de \tcbhighmath[boxrule=0.4pt,arc=4pt,colframe=green,drop fuzzy shadow=black]{\text{\visible<4->{\textbf{ORDEN 0}}}}
		\item<5-> Usamos la ecuación correspondiente para averiguar $k$:
					$$
						t_{\rfrac{1}{2}} = \frac{[\ce{A}]_0}{2k}\Rightarrow
						k = \frac{[\ce{A}]_0}{t_{\rfrac{1}{2}}}\Rightarrow
						k = \frac{1}{2\vdot 50}\Rightarrow
						\tcbhighmath[boxrule=0.4pt,arc=4pt,colframe=red,drop fuzzy shadow=black]{\visible<5->{k = \SI{e-2}{\Molar\per\second}}}
					$$
					\textbf{Recordad incluir las \textbf{unidades} de la \textbf{constante cinética}}\\
					\begin{center}
						\tcbhighmath[boxrule=0.4pt,arc=4pt,colframe=blue,drop fuzzy shadow=black]{\visible<5->{v=\SI{e-2}{\Molar\per\second}}}
					\end{center}
	\end{enumerate}
\end{frame}

\begin{frame}
	\frametitle{\ejerciciocmd}
	\framesubtitle{Resolución (\rom{3}): método de integración}
	\structure{Datos iniciales:}
	\begin{center}
		\begin{tabular}{SS}
			{\textbf{Concentración~(\si{\Molar})}}			&
			{\textbf{Tiempo de reacción~(\si{\second})}}	\\
			1,0								&	  0			\\
			,80								&	111,57		\\
			,20								&	804,72		\\
		\end{tabular}
	\end{center}
	\structure{Comprobamos qué ecuación, de los diferentes órdenes, es la válida según nuestros datos:}\\[.2cm]
	\begin{overprint}
		\onslide<1>
			\underline{\textbf{\color{red!50!black}Orden 0:}}
			$$
				[\ce{A}] = [\ce{A}]_0 - k\vdot t\Rightarrow
				k = \frac{[\ce{A}]_0 - [\ce{A}]}{t}\left\{
				\begin{aligned}
					k_1 = \frac{\SI{1,0}{\Molar} - \SI{,8}{\Molar}}{\SI{111,57}{\second}} &= \SI{7,17e-3}{\Molar\per\second}	\\
					k_2 = \frac{\SI{1,0}{\Molar} - \SI{,2}{\Molar}}{\SI{804,72}{\second}} &= \SI{2,49e-3}{\Molar\per\second}
				\end{aligned}
				\right\} k_1\neq k_2
			$$
			\begin{center}
				\underline{\textbf{No es orden 0}}
			\end{center}
		\onslide<2->
			\underline{\textbf{\color{red!50!black}Orden 1:}}
			$$
				\ln[\ce{A}] = \ln[\ce{A}]_0 - k\vdot t\Rightarrow
				k = \frac{\ln(\frac{[\ce{A}]_0}{[\ce{A}]})}{t}\left\{
				\begin{aligned}
					k_1 = \frac{\ln(\frac{\SI{1,0}{\cancel\Molar}}{\SI{,8}{\cancel\Molar}})}{\SI{111,57}{\second}} &= \SI{2,00e-3}{\per\second}\\
					k_2 = \frac{\ln(\frac{\SI{1,0}{\cancel\Molar}}{\SI{,2}{\cancel\Molar}})}{\SI{804,72}{\second}} &= \SI{2,00e-3}{\per\second}
				\end{aligned}
				\right\} k_1 = k_2
			$$
			\begin{center}
				\underline{\textbf{Es orden 1}}\qquad\tcbhighmath[boxrule=0.4pt,arc=4pt,colframe=blue,drop fuzzy shadow=black]{\visible<2->{v=\SI{2,00e-3}{\per\second}\vdot[\ce{A}]}}
			\end{center}
	\end{overprint}
\end{frame}