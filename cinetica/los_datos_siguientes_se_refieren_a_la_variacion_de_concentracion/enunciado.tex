Los datos siguientes se refieren a la variación de concentración con el tiempo para una reacción del tipo \ce{A -> B}. A partir de ellos calcule el orden de reacción y la constante de velocidad haciendo uso del método de los tiempos de vida media.
\begin{center}
	\begin{tabular}{SS}
		\toprule
			{$(\num{e-3})~[\ce{A}]~(\si{\Molar})$} & {$t~(\si{s})$} \\
		\midrule
			1,6 & 0    \\
			1,5 &  ,26 \\
			1,4 &  ,53 \\
			1,3 &  ,83 \\
			1,2 & 1,15 \\
			1,1 & 1,50 \\
			1,0 & 1,88 \\
			0,9 & 2,30 \\
			0,8 & 2,77 \\
			0,7 & 3,31 \\
			0,6 & 3,92 \\
			0,5 & 4,65 \\
			0,4 & 5,55 \\
			0,3 & 6,70 \\
			0,2 & 8,32 \\
		\bottomrule
	\end{tabular}
\end{center}
\resultadocmd{ \SI{,25}{\per\second} }
