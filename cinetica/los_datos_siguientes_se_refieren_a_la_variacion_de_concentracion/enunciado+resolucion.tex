\begin{frame}
	\frametitle{\ejerciciocmd}
	\framesubtitle{Enunciado}
	\textbf{
		Una reacción tiene una constante de velocidad de \SI{,017}{\per\second} a \SI{298}{\kelvin} y una energía libre de activación del \SI{27,235}{\kilo\joule\per\mol}. La adición de un catalizador disminuye dicha energía de activación hasta un \SI{33}{\percent} de su valor inicial. Calcule la nueva constante de velocidad.
\resultadocmd{ \SI{26,86}{\per\second} }

		}
\end{frame}

\begin{frame}
	\frametitle{\ejerciciocmd}
	\framesubtitle{Datos del problema}
	\begin{center}
		{\huge¿$n$ y $k$?}\\[.3cm]
		\tcbhighmath[boxrule=0.4pt,arc=4pt,colframe=green,drop fuzzy shadow=yellow]{\ce{A -> B}}\\[.3cm]
		\begin{tabular}{SS}
			\toprule
				{$(\num{e-3})~[\ce{A}]~(\si{\Molar})$} & {$t~(\si{s})$} \\
			\midrule
				1,6 & 0    \\
				1,5 &  ,26 \\
				1,4 &  ,53 \\
				1,3 &  ,83 \\
				1,2 & 1,15 \\
				1,1 & 1,50 \\
				1,0 & 1,88 \\
				0,9 & 2,30 \\
				0,8 & 2,77 \\
				0,7 & 3,31 \\
				0,6 & 3,92 \\
				0,5 & 4,65 \\
				0,4 & 5,55 \\
				0,3 & 6,70 \\
				0,2 & 8,32 \\
			\bottomrule
		\end{tabular}
	\end{center}
\end{frame}

\begin{frame}
	\frametitle{\ejerciciocmd}
	\framesubtitle{Resolución (\rom{1}): orden de reacción y $k$}
	\structure{A partir de las ecuaciones de velocidad integradas:}\\[.4cm]
	\alert{\textbf{\underline{Orden 0}}}
		{\footnotesize $$
			\overbrace{[\ce{A}]}^{[\ce{A}]=\frac{[\ce{A}]_0}{2}}=[\ce{A}]_0 - k\vdot\overbrace{t}^{t=t_{\rfrac{1}{2}}}\Rightarrow
			\frac{1}{2}\vdot[\ce{A}]_0-\frac{2}{2}\vdot[\ce{A}]_0=-k\vdot t_{\rfrac{1}{2}}\Rightarrow
			-\frac{1}{2}\vdot[\ce{A}]_0=-k\vdot t_{\rfrac{1}{2}}
		$$}
		$$
			t_{\rfrac{1}{2}}=\frac{[\ce{A}]_0}{2k}
		$$
	\alert{\textbf{\underline{Orden 1}}}
		{\footnotesize $$
			\ln(\overbrace{[\ce{A}]}^{[\ce{A}]=\frac{[\ce{A}]_0}{2}})=\ln([\ce{A}]_0) - k\vdot\overbrace{t}^{t=t_{\rfrac{1}{2}}}\Rightarrow
			\overbrace{\ln(\frac{[\ce{A}]_0}{2})}^{\ln([\ce{A}]_0)-\ln(2)}=\ln([\ce{A}]_0) - k\vdot t_{\rfrac{1}{2}}\Rightarrow
			\cancel{\ln([\ce{A}]_0)}-\ln(2)=\cancel{\ln([\ce{A}]_0)} - k\vdot t_{\rfrac{1}{2}}
		$$}
		$$
			t_{\rfrac{1}{2}} = \frac{\ln(2)}{k}
		$$
	\alert{\textbf{\underline{Orden 2}}}
		{\footnotesize $$
			\frac{1}{\underbrace{[\ce{A}]}_{[\ce{A}]=\frac{[\ce{A}]_0}{2}}}=\frac{1}{[\ce{A}]_0}+k\vdot\overbrace{t}^{t=t_{\rfrac{1}{2}}}\Rightarrow
			\frac{2}{[\ce{A}]_0}=\frac{1}{[\ce{A}]_0}+k\vdot t_{\rfrac{1}{2}}\Rightarrow
			\frac{1}{[\ce{A}]_0}=k\vdot t_{\rfrac{1}{2}}
		$$}
		$$
			t_{\rfrac{1}{2}}=\frac{1}{k\vdot[\ce{A}]_0}
		$$
\end{frame}

\begin{frame}
	\frametitle{\ejerciciocmd}
	\framesubtitle{Resolución (\rom{2}): orden de reacción y $k$}
	\structure{De la tabla obtenemos un conjunto de $[\ce{A}]_0$ y $\rfrac{[\ce{A}]_0}{2}$ con $t_{\rfrac{1}{2}}$}
	\begin{overprint}
		\onslide<1>
			Tabla inicial:\\
		\onslide<2>
			Ordenamos por parejas $[\ce{A}]_0/\rfrac{[\ce{A}]_0}{2}$ eliminando concentraciones que no tienen su mitad:\\
		\onslide<3>
			Calculamos $t_{\rfrac{1}{2}}$ en una tercera columna:\\
	\end{overprint}
	\begin{columns}
		\column{.30\textwidth}
			\begin{overprint}
				\onslide<1>
					\begin{tabular}{SS}
						\toprule
							{$(\num{e-3})~[\ce{A}]~(\si{\Molar})$} & {$t~(\si{s})$} \\
						\midrule
							1,6 & 0    \\
							1,5 &  ,26 \\
							1,4 &  ,53 \\
							1,3 &  ,83 \\
							1,2 & 1,15 \\
							1,1 & 1,50 \\
							1,0 & 1,88 \\
							0,9 & 2,30 \\
							0,8 & 2,77 \\
							0,7 & 3,31 \\
							0,6 & 3,92 \\
							0,5 & 4,65 \\
							0,4 & 5,55 \\
							0,3 & 6,70 \\
							0,2 & 8,32 \\
						\bottomrule
					\end{tabular}
				\onslide<2>
					\begin{tabular}{SS}
						\toprule
							{$(\num{e-3})~[\ce{A}]~(\si{\Molar})$} & {$t~(\si{s})$} \\
						\midrule
							1,6 & 0    \\
							0,8 & 2,77 \\[.2cm]
							1,4 &  ,53 \\
							0,7 & 3,31 \\[.2cm]					
							1,2 & 1,15 \\
							0,6 & 3,92 \\[.2cm]
							1,0 & 1,88 \\
							0,5 & 4,65 \\[.2cm]
							0,8 & 2,77 \\
							0,4 & 5,55 \\[.2cm]
							0,6 & 3,92 \\
							0,3 & 6,70 \\[.2cm]
							0,4 & 5,55 \\
							0,2 & 8,32 \\
						\bottomrule
					\end{tabular}
				\onslide<3>
					\begin{tabular}{SSS}
						\toprule
							{$(\num{e-3})~[\ce{A}]~(\si{\Molar})$} & {$t~(\si{s})$} & {$t_{\rfrac{1}{2}}~(\si{s})$}\\
						\midrule
							1,6 & 0    & {\multirow{2}{*}{\num{2,77}}}\\
							0,8 & 2,77 & \\[.2cm]
							1,4 &  ,53 & {\multirow{2}{*}{\num{2,78}}}\\
							0,7 & 3,31 & \\[.2cm]					
							1,2 & 1,15 & {\multirow{2}{*}{\num{2,77}}}\\
							0,6 & 3,92 & \\[.2cm]
							1,0 & 1,88 & {\multirow{2}{*}{\num{2,77}}}\\
							0,5 & 4,65 & \\[.2cm]
							0,8 & 2,77 & {\multirow{2}{*}{\num{2,78}}}\\
							0,4 & 5,55 & \\[.2cm]
							0,6 & 3,92 & {\multirow{2}{*}{\num{2,78}}}\\
							0,3 & 6,70 & \\[.2cm]
							0,4 & 5,55 & {\multirow{2}{*}{\num{2,77}}}\\
							0,2 & 8,32 & \\
						\bottomrule
					\end{tabular}
			\end{overprint}
		\column{.45\textwidth}
			\visible<3->{
				$t_{\rfrac{1}{2}}$ no depende de la concentración inicial ($[\ce{A}]_0$), por lo que la reacción es de orden 1. Aplicando:
				$$
				k = \frac{\ln(2)}{\underbrace{t_{\rfrac{1}{2}}}_{t_{\rfrac{1}{2}}=\SI{2,77}{\second}}}
				$$
				\centering\tcbhighmath[boxrule=0.4pt,arc=4pt,colframe=green,drop fuzzy shadow=yellow]{k=\SI{,25}{\per\second}}
						}
	\end{columns}
\end{frame}
