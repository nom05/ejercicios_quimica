\begin{frame}
    \frametitle{\ejerciciocmd}
    \framesubtitle{Enunciado}
    \textbf{
		Una reacción tiene una constante de velocidad de \SI{,017}{\per\second} a \SI{298}{\kelvin} y una energía libre de activación del \SI{27,235}{\kilo\joule\per\mol}. La adición de un catalizador disminuye dicha energía de activación hasta un \SI{33}{\percent} de su valor inicial. Calcule la nueva constante de velocidad.
\resultadocmd{ \SI{26,86}{\per\second} }

	}
\end{frame}

\begin{frame}
    \frametitle{\ejerciciocmd}
    \framesubtitle{Datos del problema}
    \begin{center}
        {\Large ¿$n$ y $k$?}\\[.5cm]
        \schemestart
            \chemfig{=[1]-[7]=[1]}
            \arrow
            \chemfig{*4(=---)}
        \schemestop\\[.5cm]
        \ce{
            $\underset{\text{butadieno}~(b)}{\ce{C4H6(g)}}$
             -> 
            $\underset{\text{ciclobuteno}~(c)}{\ce{C4H6(g)}}$
           }\\[.5cm]
       \begin{tabular}{SS}
           \toprule
               {$t~(\si{\second})$} & {$[b]~(\si{\mol\per\liter})$}\\
           \midrule
                195 & ,0162 \\
                604 & ,0147 \\
               1246 & ,0129 \\
               2180 & ,0110 \\
               4140 & ,0084 \\
               8135 & ,0057 \\
           \bottomrule
       \end{tabular}
    \end{center}
    $$
        T = \SI{500}{\kelvin}
    $$
\end{frame}

\begin{frame}
    \frametitle{\ejerciciocmd}
    \framesubtitle{Resolución (\rom{1}): comprobando si la reacción es de orden 0  (\rom{1})}
    \structure{Reacción:} \ce{b -> c}
    \structure{Ecuación cinética general:}
    $$
        v = k[b]^x = -\frac{\Delta[b]}{\Delta t}
    $$
    \begin{center}
        {\Large \textbf{\textit{Método integral de velocidad}}}
    \end{center}
    \structure{Orden 0}
    \begin{overprint}
        \onslide<1>
            $$
                -\dv{[b]}{t} = k\Rightarrow\dd{[b]} = -k\dd{t}\Rightarrow\int_{[b]_0}^{[b]}\dd{[b]} = -k\int_{0}^{t}\dd{t}
            $$
        \onslide<2>
            $$
                [b] = [b]_0 - kt
            $$
    \end{overprint}
\end{frame}

\begin{frame}
    \frametitle{\ejerciciocmd}
    \framesubtitle{Resolución (\rom{2}): comprobando si la reacción es de orden 0 (\rom{2})}
    \begin{center}
        \begin{figure}
            \caption{Representación de la concentración de b ($[b]$) con respecto al tiempo ($t$). La reacción \textbf{\underline{no se ajusta}} bien a una reacción de \textbf{\underline{orden 0}}}
            % GNUPLOT: LaTeX picture
\setlength{\unitlength}{0.240900pt}
\ifx\plotpoint\undefined\newsavebox{\plotpoint}\fi
\sbox{\plotpoint}{\rule[-0.200pt]{0.400pt}{0.400pt}}%
\begin{picture}(1299,826)(0,0)
\sbox{\plotpoint}{\rule[-0.200pt]{0.400pt}{0.400pt}}%
\put(191.0,131.0){\rule[-0.200pt]{4.818pt}{0.400pt}}
\put(171,131){\makebox(0,0)[r]{$0.004$}}
\put(1218.0,131.0){\rule[-0.200pt]{4.818pt}{0.400pt}}
\put(191.0,224.0){\rule[-0.200pt]{4.818pt}{0.400pt}}
\put(171,224){\makebox(0,0)[r]{$0.006$}}
\put(1218.0,224.0){\rule[-0.200pt]{4.818pt}{0.400pt}}
\put(191.0,318.0){\rule[-0.200pt]{4.818pt}{0.400pt}}
\put(171,318){\makebox(0,0)[r]{$0.008$}}
\put(1218.0,318.0){\rule[-0.200pt]{4.818pt}{0.400pt}}
\put(191.0,411.0){\rule[-0.200pt]{4.818pt}{0.400pt}}
\put(171,411){\makebox(0,0)[r]{$0.010$}}
\put(1218.0,411.0){\rule[-0.200pt]{4.818pt}{0.400pt}}
\put(191.0,504.0){\rule[-0.200pt]{4.818pt}{0.400pt}}
\put(171,504){\makebox(0,0)[r]{$0.012$}}
\put(1218.0,504.0){\rule[-0.200pt]{4.818pt}{0.400pt}}
\put(191.0,597.0){\rule[-0.200pt]{4.818pt}{0.400pt}}
\put(171,597){\makebox(0,0)[r]{$0.014$}}
\put(1218.0,597.0){\rule[-0.200pt]{4.818pt}{0.400pt}}
\put(191.0,691.0){\rule[-0.200pt]{4.818pt}{0.400pt}}
\put(171,691){\makebox(0,0)[r]{$0.016$}}
\put(1218.0,691.0){\rule[-0.200pt]{4.818pt}{0.400pt}}
\put(191.0,784.0){\rule[-0.200pt]{4.818pt}{0.400pt}}
\put(171,784){\makebox(0,0)[r]{$0.018$}}
\put(1218.0,784.0){\rule[-0.200pt]{4.818pt}{0.400pt}}
\put(191.0,131.0){\rule[-0.200pt]{0.400pt}{4.818pt}}
\put(191,90){\makebox(0,0){$0$}}
\put(191.0,764.0){\rule[-0.200pt]{0.400pt}{4.818pt}}
\put(307.0,131.0){\rule[-0.200pt]{0.400pt}{4.818pt}}
\put(307,90){\makebox(0,0){$1000$}}
\put(307.0,764.0){\rule[-0.200pt]{0.400pt}{4.818pt}}
\put(424.0,131.0){\rule[-0.200pt]{0.400pt}{4.818pt}}
\put(424,90){\makebox(0,0){$2000$}}
\put(424.0,764.0){\rule[-0.200pt]{0.400pt}{4.818pt}}
\put(540.0,131.0){\rule[-0.200pt]{0.400pt}{4.818pt}}
\put(540,90){\makebox(0,0){$3000$}}
\put(540.0,764.0){\rule[-0.200pt]{0.400pt}{4.818pt}}
\put(656.0,131.0){\rule[-0.200pt]{0.400pt}{4.818pt}}
\put(656,90){\makebox(0,0){$4000$}}
\put(656.0,764.0){\rule[-0.200pt]{0.400pt}{4.818pt}}
\put(773.0,131.0){\rule[-0.200pt]{0.400pt}{4.818pt}}
\put(773,90){\makebox(0,0){$5000$}}
\put(773.0,764.0){\rule[-0.200pt]{0.400pt}{4.818pt}}
\put(889.0,131.0){\rule[-0.200pt]{0.400pt}{4.818pt}}
\put(889,90){\makebox(0,0){$6000$}}
\put(889.0,764.0){\rule[-0.200pt]{0.400pt}{4.818pt}}
\put(1005.0,131.0){\rule[-0.200pt]{0.400pt}{4.818pt}}
\put(1005,90){\makebox(0,0){$7000$}}
\put(1005.0,764.0){\rule[-0.200pt]{0.400pt}{4.818pt}}
\put(1122.0,131.0){\rule[-0.200pt]{0.400pt}{4.818pt}}
\put(1122,90){\makebox(0,0){$8000$}}
\put(1122.0,764.0){\rule[-0.200pt]{0.400pt}{4.818pt}}
\put(1238.0,131.0){\rule[-0.200pt]{0.400pt}{4.818pt}}
\put(1238,90){\makebox(0,0){$9000$}}
\put(1238.0,764.0){\rule[-0.200pt]{0.400pt}{4.818pt}}
\put(191.0,131.0){\rule[-0.200pt]{0.400pt}{157.308pt}}
\put(191.0,131.0){\rule[-0.200pt]{252.222pt}{0.400pt}}
\put(1238.0,131.0){\rule[-0.200pt]{0.400pt}{157.308pt}}
\put(191.0,784.0){\rule[-0.200pt]{252.222pt}{0.400pt}}
\put(30,457){\makebox(0,0){$[b]~(\si{\Molar})$}}
\put(714,29){\makebox(0,0){$\text{tiempo}~(\si{\second})$}}
\put(214,700){\makebox(0,0){$\blacksquare$}}
\put(261,630){\makebox(0,0){$\blacksquare$}}
\put(336,546){\makebox(0,0){$\blacksquare$}}
\put(445,457){\makebox(0,0){$\blacksquare$}}
\put(673,336){\makebox(0,0){$\blacksquare$}}
\put(1137,210){\makebox(0,0){$\blacksquare$}}
\put(1078,743){\makebox(0,0)[r]{$[b]$ = \num{1.495e-02} \num{-1.260e-06}$t$ ($r^2$ = \num{.9106})}}
\put(1098.0,743.0){\rule[-0.200pt]{24.090pt}{0.400pt}}
\put(214,630){\usebox{\plotpoint}}
\multiput(214.00,628.93)(0.933,-0.477){7}{\rule{0.820pt}{0.115pt}}
\multiput(214.00,629.17)(7.298,-5.000){2}{\rule{0.410pt}{0.400pt}}
\multiput(223.00,623.94)(1.212,-0.468){5}{\rule{1.000pt}{0.113pt}}
\multiput(223.00,624.17)(6.924,-4.000){2}{\rule{0.500pt}{0.400pt}}
\multiput(232.00,619.93)(1.044,-0.477){7}{\rule{0.900pt}{0.115pt}}
\multiput(232.00,620.17)(8.132,-5.000){2}{\rule{0.450pt}{0.400pt}}
\multiput(242.00,614.93)(0.933,-0.477){7}{\rule{0.820pt}{0.115pt}}
\multiput(242.00,615.17)(7.298,-5.000){2}{\rule{0.410pt}{0.400pt}}
\multiput(251.00,609.94)(1.212,-0.468){5}{\rule{1.000pt}{0.113pt}}
\multiput(251.00,610.17)(6.924,-4.000){2}{\rule{0.500pt}{0.400pt}}
\multiput(260.00,605.93)(1.044,-0.477){7}{\rule{0.900pt}{0.115pt}}
\multiput(260.00,606.17)(8.132,-5.000){2}{\rule{0.450pt}{0.400pt}}
\multiput(270.00,600.93)(0.933,-0.477){7}{\rule{0.820pt}{0.115pt}}
\multiput(270.00,601.17)(7.298,-5.000){2}{\rule{0.410pt}{0.400pt}}
\multiput(279.00,595.93)(0.933,-0.477){7}{\rule{0.820pt}{0.115pt}}
\multiput(279.00,596.17)(7.298,-5.000){2}{\rule{0.410pt}{0.400pt}}
\multiput(288.00,590.94)(1.358,-0.468){5}{\rule{1.100pt}{0.113pt}}
\multiput(288.00,591.17)(7.717,-4.000){2}{\rule{0.550pt}{0.400pt}}
\multiput(298.00,586.93)(0.933,-0.477){7}{\rule{0.820pt}{0.115pt}}
\multiput(298.00,587.17)(7.298,-5.000){2}{\rule{0.410pt}{0.400pt}}
\multiput(307.00,581.93)(0.933,-0.477){7}{\rule{0.820pt}{0.115pt}}
\multiput(307.00,582.17)(7.298,-5.000){2}{\rule{0.410pt}{0.400pt}}
\multiput(316.00,576.94)(1.358,-0.468){5}{\rule{1.100pt}{0.113pt}}
\multiput(316.00,577.17)(7.717,-4.000){2}{\rule{0.550pt}{0.400pt}}
\multiput(326.00,572.93)(0.933,-0.477){7}{\rule{0.820pt}{0.115pt}}
\multiput(326.00,573.17)(7.298,-5.000){2}{\rule{0.410pt}{0.400pt}}
\multiput(335.00,567.93)(0.933,-0.477){7}{\rule{0.820pt}{0.115pt}}
\multiput(335.00,568.17)(7.298,-5.000){2}{\rule{0.410pt}{0.400pt}}
\multiput(344.00,562.93)(1.044,-0.477){7}{\rule{0.900pt}{0.115pt}}
\multiput(344.00,563.17)(8.132,-5.000){2}{\rule{0.450pt}{0.400pt}}
\multiput(354.00,557.94)(1.212,-0.468){5}{\rule{1.000pt}{0.113pt}}
\multiput(354.00,558.17)(6.924,-4.000){2}{\rule{0.500pt}{0.400pt}}
\multiput(363.00,553.93)(0.933,-0.477){7}{\rule{0.820pt}{0.115pt}}
\multiput(363.00,554.17)(7.298,-5.000){2}{\rule{0.410pt}{0.400pt}}
\multiput(372.00,548.93)(1.044,-0.477){7}{\rule{0.900pt}{0.115pt}}
\multiput(372.00,549.17)(8.132,-5.000){2}{\rule{0.450pt}{0.400pt}}
\multiput(382.00,543.94)(1.212,-0.468){5}{\rule{1.000pt}{0.113pt}}
\multiput(382.00,544.17)(6.924,-4.000){2}{\rule{0.500pt}{0.400pt}}
\multiput(391.00,539.93)(0.933,-0.477){7}{\rule{0.820pt}{0.115pt}}
\multiput(391.00,540.17)(7.298,-5.000){2}{\rule{0.410pt}{0.400pt}}
\multiput(400.00,534.93)(1.044,-0.477){7}{\rule{0.900pt}{0.115pt}}
\multiput(400.00,535.17)(8.132,-5.000){2}{\rule{0.450pt}{0.400pt}}
\multiput(410.00,529.93)(0.933,-0.477){7}{\rule{0.820pt}{0.115pt}}
\multiput(410.00,530.17)(7.298,-5.000){2}{\rule{0.410pt}{0.400pt}}
\multiput(419.00,524.94)(1.212,-0.468){5}{\rule{1.000pt}{0.113pt}}
\multiput(419.00,525.17)(6.924,-4.000){2}{\rule{0.500pt}{0.400pt}}
\multiput(428.00,520.93)(1.044,-0.477){7}{\rule{0.900pt}{0.115pt}}
\multiput(428.00,521.17)(8.132,-5.000){2}{\rule{0.450pt}{0.400pt}}
\multiput(438.00,515.93)(0.933,-0.477){7}{\rule{0.820pt}{0.115pt}}
\multiput(438.00,516.17)(7.298,-5.000){2}{\rule{0.410pt}{0.400pt}}
\multiput(447.00,510.94)(1.212,-0.468){5}{\rule{1.000pt}{0.113pt}}
\multiput(447.00,511.17)(6.924,-4.000){2}{\rule{0.500pt}{0.400pt}}
\multiput(456.00,506.93)(1.044,-0.477){7}{\rule{0.900pt}{0.115pt}}
\multiput(456.00,507.17)(8.132,-5.000){2}{\rule{0.450pt}{0.400pt}}
\multiput(466.00,501.93)(0.933,-0.477){7}{\rule{0.820pt}{0.115pt}}
\multiput(466.00,502.17)(7.298,-5.000){2}{\rule{0.410pt}{0.400pt}}
\multiput(475.00,496.94)(1.212,-0.468){5}{\rule{1.000pt}{0.113pt}}
\multiput(475.00,497.17)(6.924,-4.000){2}{\rule{0.500pt}{0.400pt}}
\multiput(484.00,492.93)(1.044,-0.477){7}{\rule{0.900pt}{0.115pt}}
\multiput(484.00,493.17)(8.132,-5.000){2}{\rule{0.450pt}{0.400pt}}
\multiput(494.00,487.93)(0.933,-0.477){7}{\rule{0.820pt}{0.115pt}}
\multiput(494.00,488.17)(7.298,-5.000){2}{\rule{0.410pt}{0.400pt}}
\multiput(503.00,482.93)(0.933,-0.477){7}{\rule{0.820pt}{0.115pt}}
\multiput(503.00,483.17)(7.298,-5.000){2}{\rule{0.410pt}{0.400pt}}
\multiput(512.00,477.94)(1.358,-0.468){5}{\rule{1.100pt}{0.113pt}}
\multiput(512.00,478.17)(7.717,-4.000){2}{\rule{0.550pt}{0.400pt}}
\multiput(522.00,473.93)(0.933,-0.477){7}{\rule{0.820pt}{0.115pt}}
\multiput(522.00,474.17)(7.298,-5.000){2}{\rule{0.410pt}{0.400pt}}
\multiput(531.00,468.93)(0.933,-0.477){7}{\rule{0.820pt}{0.115pt}}
\multiput(531.00,469.17)(7.298,-5.000){2}{\rule{0.410pt}{0.400pt}}
\multiput(540.00,463.94)(1.358,-0.468){5}{\rule{1.100pt}{0.113pt}}
\multiput(540.00,464.17)(7.717,-4.000){2}{\rule{0.550pt}{0.400pt}}
\multiput(550.00,459.93)(0.933,-0.477){7}{\rule{0.820pt}{0.115pt}}
\multiput(550.00,460.17)(7.298,-5.000){2}{\rule{0.410pt}{0.400pt}}
\multiput(559.00,454.93)(0.933,-0.477){7}{\rule{0.820pt}{0.115pt}}
\multiput(559.00,455.17)(7.298,-5.000){2}{\rule{0.410pt}{0.400pt}}
\multiput(568.00,449.93)(1.044,-0.477){7}{\rule{0.900pt}{0.115pt}}
\multiput(568.00,450.17)(8.132,-5.000){2}{\rule{0.450pt}{0.400pt}}
\multiput(578.00,444.94)(1.212,-0.468){5}{\rule{1.000pt}{0.113pt}}
\multiput(578.00,445.17)(6.924,-4.000){2}{\rule{0.500pt}{0.400pt}}
\multiput(587.00,440.93)(0.933,-0.477){7}{\rule{0.820pt}{0.115pt}}
\multiput(587.00,441.17)(7.298,-5.000){2}{\rule{0.410pt}{0.400pt}}
\multiput(596.00,435.93)(1.044,-0.477){7}{\rule{0.900pt}{0.115pt}}
\multiput(596.00,436.17)(8.132,-5.000){2}{\rule{0.450pt}{0.400pt}}
\multiput(606.00,430.94)(1.212,-0.468){5}{\rule{1.000pt}{0.113pt}}
\multiput(606.00,431.17)(6.924,-4.000){2}{\rule{0.500pt}{0.400pt}}
\multiput(615.00,426.93)(0.933,-0.477){7}{\rule{0.820pt}{0.115pt}}
\multiput(615.00,427.17)(7.298,-5.000){2}{\rule{0.410pt}{0.400pt}}
\multiput(624.00,421.93)(1.044,-0.477){7}{\rule{0.900pt}{0.115pt}}
\multiput(624.00,422.17)(8.132,-5.000){2}{\rule{0.450pt}{0.400pt}}
\multiput(634.00,416.93)(0.933,-0.477){7}{\rule{0.820pt}{0.115pt}}
\multiput(634.00,417.17)(7.298,-5.000){2}{\rule{0.410pt}{0.400pt}}
\multiput(643.00,411.94)(1.212,-0.468){5}{\rule{1.000pt}{0.113pt}}
\multiput(643.00,412.17)(6.924,-4.000){2}{\rule{0.500pt}{0.400pt}}
\multiput(652.00,407.93)(1.044,-0.477){7}{\rule{0.900pt}{0.115pt}}
\multiput(652.00,408.17)(8.132,-5.000){2}{\rule{0.450pt}{0.400pt}}
\multiput(662.00,402.93)(0.933,-0.477){7}{\rule{0.820pt}{0.115pt}}
\multiput(662.00,403.17)(7.298,-5.000){2}{\rule{0.410pt}{0.400pt}}
\multiput(671.00,397.94)(1.212,-0.468){5}{\rule{1.000pt}{0.113pt}}
\multiput(671.00,398.17)(6.924,-4.000){2}{\rule{0.500pt}{0.400pt}}
\multiput(680.00,393.93)(1.044,-0.477){7}{\rule{0.900pt}{0.115pt}}
\multiput(680.00,394.17)(8.132,-5.000){2}{\rule{0.450pt}{0.400pt}}
\multiput(690.00,388.93)(0.933,-0.477){7}{\rule{0.820pt}{0.115pt}}
\multiput(690.00,389.17)(7.298,-5.000){2}{\rule{0.410pt}{0.400pt}}
\multiput(699.00,383.93)(0.933,-0.477){7}{\rule{0.820pt}{0.115pt}}
\multiput(699.00,384.17)(7.298,-5.000){2}{\rule{0.410pt}{0.400pt}}
\multiput(708.00,378.94)(1.358,-0.468){5}{\rule{1.100pt}{0.113pt}}
\multiput(708.00,379.17)(7.717,-4.000){2}{\rule{0.550pt}{0.400pt}}
\multiput(718.00,374.93)(0.933,-0.477){7}{\rule{0.820pt}{0.115pt}}
\multiput(718.00,375.17)(7.298,-5.000){2}{\rule{0.410pt}{0.400pt}}
\multiput(727.00,369.93)(0.933,-0.477){7}{\rule{0.820pt}{0.115pt}}
\multiput(727.00,370.17)(7.298,-5.000){2}{\rule{0.410pt}{0.400pt}}
\multiput(736.00,364.94)(1.358,-0.468){5}{\rule{1.100pt}{0.113pt}}
\multiput(736.00,365.17)(7.717,-4.000){2}{\rule{0.550pt}{0.400pt}}
\multiput(746.00,360.93)(0.933,-0.477){7}{\rule{0.820pt}{0.115pt}}
\multiput(746.00,361.17)(7.298,-5.000){2}{\rule{0.410pt}{0.400pt}}
\multiput(755.00,355.93)(0.933,-0.477){7}{\rule{0.820pt}{0.115pt}}
\multiput(755.00,356.17)(7.298,-5.000){2}{\rule{0.410pt}{0.400pt}}
\multiput(764.00,350.93)(0.933,-0.477){7}{\rule{0.820pt}{0.115pt}}
\multiput(764.00,351.17)(7.298,-5.000){2}{\rule{0.410pt}{0.400pt}}
\multiput(773.00,345.94)(1.358,-0.468){5}{\rule{1.100pt}{0.113pt}}
\multiput(773.00,346.17)(7.717,-4.000){2}{\rule{0.550pt}{0.400pt}}
\multiput(783.00,341.93)(0.933,-0.477){7}{\rule{0.820pt}{0.115pt}}
\multiput(783.00,342.17)(7.298,-5.000){2}{\rule{0.410pt}{0.400pt}}
\multiput(792.00,336.93)(0.933,-0.477){7}{\rule{0.820pt}{0.115pt}}
\multiput(792.00,337.17)(7.298,-5.000){2}{\rule{0.410pt}{0.400pt}}
\multiput(801.00,331.94)(1.358,-0.468){5}{\rule{1.100pt}{0.113pt}}
\multiput(801.00,332.17)(7.717,-4.000){2}{\rule{0.550pt}{0.400pt}}
\multiput(811.00,327.93)(0.933,-0.477){7}{\rule{0.820pt}{0.115pt}}
\multiput(811.00,328.17)(7.298,-5.000){2}{\rule{0.410pt}{0.400pt}}
\multiput(820.00,322.93)(0.933,-0.477){7}{\rule{0.820pt}{0.115pt}}
\multiput(820.00,323.17)(7.298,-5.000){2}{\rule{0.410pt}{0.400pt}}
\multiput(829.00,317.93)(1.044,-0.477){7}{\rule{0.900pt}{0.115pt}}
\multiput(829.00,318.17)(8.132,-5.000){2}{\rule{0.450pt}{0.400pt}}
\multiput(839.00,312.94)(1.212,-0.468){5}{\rule{1.000pt}{0.113pt}}
\multiput(839.00,313.17)(6.924,-4.000){2}{\rule{0.500pt}{0.400pt}}
\multiput(848.00,308.93)(0.933,-0.477){7}{\rule{0.820pt}{0.115pt}}
\multiput(848.00,309.17)(7.298,-5.000){2}{\rule{0.410pt}{0.400pt}}
\multiput(857.00,303.93)(1.044,-0.477){7}{\rule{0.900pt}{0.115pt}}
\multiput(857.00,304.17)(8.132,-5.000){2}{\rule{0.450pt}{0.400pt}}
\multiput(867.00,298.94)(1.212,-0.468){5}{\rule{1.000pt}{0.113pt}}
\multiput(867.00,299.17)(6.924,-4.000){2}{\rule{0.500pt}{0.400pt}}
\multiput(876.00,294.93)(0.933,-0.477){7}{\rule{0.820pt}{0.115pt}}
\multiput(876.00,295.17)(7.298,-5.000){2}{\rule{0.410pt}{0.400pt}}
\multiput(885.00,289.93)(1.044,-0.477){7}{\rule{0.900pt}{0.115pt}}
\multiput(885.00,290.17)(8.132,-5.000){2}{\rule{0.450pt}{0.400pt}}
\multiput(895.00,284.93)(0.933,-0.477){7}{\rule{0.820pt}{0.115pt}}
\multiput(895.00,285.17)(7.298,-5.000){2}{\rule{0.410pt}{0.400pt}}
\multiput(904.00,279.94)(1.212,-0.468){5}{\rule{1.000pt}{0.113pt}}
\multiput(904.00,280.17)(6.924,-4.000){2}{\rule{0.500pt}{0.400pt}}
\multiput(913.00,275.93)(1.044,-0.477){7}{\rule{0.900pt}{0.115pt}}
\multiput(913.00,276.17)(8.132,-5.000){2}{\rule{0.450pt}{0.400pt}}
\multiput(923.00,270.93)(0.933,-0.477){7}{\rule{0.820pt}{0.115pt}}
\multiput(923.00,271.17)(7.298,-5.000){2}{\rule{0.410pt}{0.400pt}}
\multiput(932.00,265.94)(1.212,-0.468){5}{\rule{1.000pt}{0.113pt}}
\multiput(932.00,266.17)(6.924,-4.000){2}{\rule{0.500pt}{0.400pt}}
\multiput(941.00,261.93)(1.044,-0.477){7}{\rule{0.900pt}{0.115pt}}
\multiput(941.00,262.17)(8.132,-5.000){2}{\rule{0.450pt}{0.400pt}}
\multiput(951.00,256.93)(0.933,-0.477){7}{\rule{0.820pt}{0.115pt}}
\multiput(951.00,257.17)(7.298,-5.000){2}{\rule{0.410pt}{0.400pt}}
\multiput(960.00,251.93)(0.933,-0.477){7}{\rule{0.820pt}{0.115pt}}
\multiput(960.00,252.17)(7.298,-5.000){2}{\rule{0.410pt}{0.400pt}}
\multiput(969.00,246.94)(1.358,-0.468){5}{\rule{1.100pt}{0.113pt}}
\multiput(969.00,247.17)(7.717,-4.000){2}{\rule{0.550pt}{0.400pt}}
\multiput(979.00,242.93)(0.933,-0.477){7}{\rule{0.820pt}{0.115pt}}
\multiput(979.00,243.17)(7.298,-5.000){2}{\rule{0.410pt}{0.400pt}}
\multiput(988.00,237.93)(0.933,-0.477){7}{\rule{0.820pt}{0.115pt}}
\multiput(988.00,238.17)(7.298,-5.000){2}{\rule{0.410pt}{0.400pt}}
\multiput(997.00,232.94)(1.358,-0.468){5}{\rule{1.100pt}{0.113pt}}
\multiput(997.00,233.17)(7.717,-4.000){2}{\rule{0.550pt}{0.400pt}}
\multiput(1007.00,228.93)(0.933,-0.477){7}{\rule{0.820pt}{0.115pt}}
\multiput(1007.00,229.17)(7.298,-5.000){2}{\rule{0.410pt}{0.400pt}}
\multiput(1016.00,223.93)(0.933,-0.477){7}{\rule{0.820pt}{0.115pt}}
\multiput(1016.00,224.17)(7.298,-5.000){2}{\rule{0.410pt}{0.400pt}}
\multiput(1025.00,218.94)(1.358,-0.468){5}{\rule{1.100pt}{0.113pt}}
\multiput(1025.00,219.17)(7.717,-4.000){2}{\rule{0.550pt}{0.400pt}}
\multiput(1035.00,214.93)(0.933,-0.477){7}{\rule{0.820pt}{0.115pt}}
\multiput(1035.00,215.17)(7.298,-5.000){2}{\rule{0.410pt}{0.400pt}}
\multiput(1044.00,209.93)(0.933,-0.477){7}{\rule{0.820pt}{0.115pt}}
\multiput(1044.00,210.17)(7.298,-5.000){2}{\rule{0.410pt}{0.400pt}}
\multiput(1053.00,204.93)(1.044,-0.477){7}{\rule{0.900pt}{0.115pt}}
\multiput(1053.00,205.17)(8.132,-5.000){2}{\rule{0.450pt}{0.400pt}}
\multiput(1063.00,199.94)(1.212,-0.468){5}{\rule{1.000pt}{0.113pt}}
\multiput(1063.00,200.17)(6.924,-4.000){2}{\rule{0.500pt}{0.400pt}}
\multiput(1072.00,195.93)(0.933,-0.477){7}{\rule{0.820pt}{0.115pt}}
\multiput(1072.00,196.17)(7.298,-5.000){2}{\rule{0.410pt}{0.400pt}}
\multiput(1081.00,190.93)(1.044,-0.477){7}{\rule{0.900pt}{0.115pt}}
\multiput(1081.00,191.17)(8.132,-5.000){2}{\rule{0.450pt}{0.400pt}}
\multiput(1091.00,185.94)(1.212,-0.468){5}{\rule{1.000pt}{0.113pt}}
\multiput(1091.00,186.17)(6.924,-4.000){2}{\rule{0.500pt}{0.400pt}}
\multiput(1100.00,181.93)(0.933,-0.477){7}{\rule{0.820pt}{0.115pt}}
\multiput(1100.00,182.17)(7.298,-5.000){2}{\rule{0.410pt}{0.400pt}}
\multiput(1109.00,176.93)(1.044,-0.477){7}{\rule{0.900pt}{0.115pt}}
\multiput(1109.00,177.17)(8.132,-5.000){2}{\rule{0.450pt}{0.400pt}}
\multiput(1119.00,171.93)(0.933,-0.477){7}{\rule{0.820pt}{0.115pt}}
\multiput(1119.00,172.17)(7.298,-5.000){2}{\rule{0.410pt}{0.400pt}}
\multiput(1128.00,166.94)(1.212,-0.468){5}{\rule{1.000pt}{0.113pt}}
\multiput(1128.00,167.17)(6.924,-4.000){2}{\rule{0.500pt}{0.400pt}}
\put(191.0,131.0){\rule[-0.200pt]{0.400pt}{157.308pt}}
\put(191.0,131.0){\rule[-0.200pt]{252.222pt}{0.400pt}}
\put(1238.0,131.0){\rule[-0.200pt]{0.400pt}{157.308pt}}
\put(191.0,784.0){\rule[-0.200pt]{252.222pt}{0.400pt}}
\end{picture}

        \end{figure}
    \end{center}
\end{frame}

\begin{frame}
    \frametitle{\ejerciciocmd}
    \framesubtitle{Resolución (\rom{3}): comprobando si la reacción es de orden 1 (\rom{1})}
    \structure{Reacción:} \ce{b -> c}
    \structure{Ecuación cinética general:}
    $$
        v = k[b]^x = -\frac{\Delta[b]}{\Delta t}
    $$
    \begin{center}
        {\Large \textbf{\textit{Método integral de velocidad}}}
    \end{center}
    \structure{Orden 1}
    \begin{overprint}
        \onslide<1>
            $$
                -\dv{[b]}{t} = k[b]\Rightarrow\frac{\dd{[b]}}{[b]} = -k\dd{t}\Rightarrow\int_{[b]_0}^{[b]}[b]^{-1}\dd{[b]} = -k\int_{0}^{t}\dd{t}
            $$
        \onslide<2>
        $$
            \ln([b]) = \ln([b]_0) - kt
        $$
    \end{overprint}
\end{frame}

\begin{frame}
    \frametitle{\ejerciciocmd}
    \framesubtitle{Resolución (\rom{4}): comprobando si la reacción es de orden 1 (\rom{2})}
    \begin{center}
        \begin{figure}
            \caption{Representación del logaritmo neperiano de la concentración de b ($\ln([b])$) con respecto al tiempo ($t$). La reacción \textbf{\underline{no se ajusta}} bien a una reacción de \textbf{\underline{orden 1}}}
            % GNUPLOT: LaTeX picture
\setlength{\unitlength}{0.240900pt}
\ifx\plotpoint\undefined\newsavebox{\plotpoint}\fi
\sbox{\plotpoint}{\rule[-0.200pt]{0.400pt}{0.400pt}}%
\begin{picture}(1299,826)(0,0)
\sbox{\plotpoint}{\rule[-0.200pt]{0.400pt}{0.400pt}}%
\put(211.0,131.0){\rule[-0.200pt]{4.818pt}{0.400pt}}
\put(191,131){\makebox(0,0)[r]{$-5.400$}}
\put(1218.0,131.0){\rule[-0.200pt]{4.818pt}{0.400pt}}
\put(211.0,224.0){\rule[-0.200pt]{4.818pt}{0.400pt}}
\put(191,224){\makebox(0,0)[r]{$-5.200$}}
\put(1218.0,224.0){\rule[-0.200pt]{4.818pt}{0.400pt}}
\put(211.0,318.0){\rule[-0.200pt]{4.818pt}{0.400pt}}
\put(191,318){\makebox(0,0)[r]{$-5.000$}}
\put(1218.0,318.0){\rule[-0.200pt]{4.818pt}{0.400pt}}
\put(211.0,411.0){\rule[-0.200pt]{4.818pt}{0.400pt}}
\put(191,411){\makebox(0,0)[r]{$-4.800$}}
\put(1218.0,411.0){\rule[-0.200pt]{4.818pt}{0.400pt}}
\put(211.0,504.0){\rule[-0.200pt]{4.818pt}{0.400pt}}
\put(191,504){\makebox(0,0)[r]{$-4.600$}}
\put(1218.0,504.0){\rule[-0.200pt]{4.818pt}{0.400pt}}
\put(211.0,597.0){\rule[-0.200pt]{4.818pt}{0.400pt}}
\put(191,597){\makebox(0,0)[r]{$-4.400$}}
\put(1218.0,597.0){\rule[-0.200pt]{4.818pt}{0.400pt}}
\put(211.0,691.0){\rule[-0.200pt]{4.818pt}{0.400pt}}
\put(191,691){\makebox(0,0)[r]{$-4.200$}}
\put(1218.0,691.0){\rule[-0.200pt]{4.818pt}{0.400pt}}
\put(211.0,784.0){\rule[-0.200pt]{4.818pt}{0.400pt}}
\put(191,784){\makebox(0,0)[r]{$-4.000$}}
\put(1218.0,784.0){\rule[-0.200pt]{4.818pt}{0.400pt}}
\put(211.0,131.0){\rule[-0.200pt]{0.400pt}{4.818pt}}
\put(211,90){\makebox(0,0){$0$}}
\put(211.0,764.0){\rule[-0.200pt]{0.400pt}{4.818pt}}
\put(325.0,131.0){\rule[-0.200pt]{0.400pt}{4.818pt}}
\put(325,90){\makebox(0,0){$1000$}}
\put(325.0,764.0){\rule[-0.200pt]{0.400pt}{4.818pt}}
\put(439.0,131.0){\rule[-0.200pt]{0.400pt}{4.818pt}}
\put(439,90){\makebox(0,0){$2000$}}
\put(439.0,764.0){\rule[-0.200pt]{0.400pt}{4.818pt}}
\put(553.0,131.0){\rule[-0.200pt]{0.400pt}{4.818pt}}
\put(553,90){\makebox(0,0){$3000$}}
\put(553.0,764.0){\rule[-0.200pt]{0.400pt}{4.818pt}}
\put(667.0,131.0){\rule[-0.200pt]{0.400pt}{4.818pt}}
\put(667,90){\makebox(0,0){$4000$}}
\put(667.0,764.0){\rule[-0.200pt]{0.400pt}{4.818pt}}
\put(782.0,131.0){\rule[-0.200pt]{0.400pt}{4.818pt}}
\put(782,90){\makebox(0,0){$5000$}}
\put(782.0,764.0){\rule[-0.200pt]{0.400pt}{4.818pt}}
\put(896.0,131.0){\rule[-0.200pt]{0.400pt}{4.818pt}}
\put(896,90){\makebox(0,0){$6000$}}
\put(896.0,764.0){\rule[-0.200pt]{0.400pt}{4.818pt}}
\put(1010.0,131.0){\rule[-0.200pt]{0.400pt}{4.818pt}}
\put(1010,90){\makebox(0,0){$7000$}}
\put(1010.0,764.0){\rule[-0.200pt]{0.400pt}{4.818pt}}
\put(1124.0,131.0){\rule[-0.200pt]{0.400pt}{4.818pt}}
\put(1124,90){\makebox(0,0){$8000$}}
\put(1124.0,764.0){\rule[-0.200pt]{0.400pt}{4.818pt}}
\put(1238.0,131.0){\rule[-0.200pt]{0.400pt}{4.818pt}}
\put(1238,90){\makebox(0,0){$9000$}}
\put(1238.0,764.0){\rule[-0.200pt]{0.400pt}{4.818pt}}
\put(211.0,131.0){\rule[-0.200pt]{0.400pt}{157.308pt}}
\put(211.0,131.0){\rule[-0.200pt]{247.404pt}{0.400pt}}
\put(1238.0,131.0){\rule[-0.200pt]{0.400pt}{157.308pt}}
\put(211.0,784.0){\rule[-0.200pt]{247.404pt}{0.400pt}}
\put(30,457){\makebox(0,0){$\ln([b])$}}
\put(724,29){\makebox(0,0){$\text{tiempo}~(\si{\second})$}}
\put(233,727){\makebox(0,0){$\blacksquare$}}
\put(280,681){\makebox(0,0){$\blacksquare$}}
\put(353,620){\makebox(0,0){$\blacksquare$}}
\put(460,546){\makebox(0,0){$\blacksquare$}}
\put(683,420){\makebox(0,0){$\blacksquare$}}
\put(1139,240){\makebox(0,0){$\blacksquare$}}
\put(1078,743){\makebox(0,0)[r]{$ln([b])$ = \num{-4,170e+00} \num{-1,291e-04}$t$ ($r^2$ = \num{,9759})}}
\put(1098.0,743.0){\rule[-0.200pt]{24.090pt}{0.400pt}}
\put(233,693){\usebox{\plotpoint}}
\multiput(233.00,691.93)(0.933,-0.477){7}{\rule{0.820pt}{0.115pt}}
\multiput(233.00,692.17)(7.298,-5.000){2}{\rule{0.410pt}{0.400pt}}
\multiput(242.00,686.93)(1.044,-0.477){7}{\rule{0.900pt}{0.115pt}}
\multiput(242.00,687.17)(8.132,-5.000){2}{\rule{0.450pt}{0.400pt}}
\multiput(252.00,681.93)(0.933,-0.477){7}{\rule{0.820pt}{0.115pt}}
\multiput(252.00,682.17)(7.298,-5.000){2}{\rule{0.410pt}{0.400pt}}
\multiput(261.00,676.94)(1.212,-0.468){5}{\rule{1.000pt}{0.113pt}}
\multiput(261.00,677.17)(6.924,-4.000){2}{\rule{0.500pt}{0.400pt}}
\multiput(270.00,672.93)(0.933,-0.477){7}{\rule{0.820pt}{0.115pt}}
\multiput(270.00,673.17)(7.298,-5.000){2}{\rule{0.410pt}{0.400pt}}
\multiput(279.00,667.93)(0.933,-0.477){7}{\rule{0.820pt}{0.115pt}}
\multiput(279.00,668.17)(7.298,-5.000){2}{\rule{0.410pt}{0.400pt}}
\multiput(288.00,662.93)(0.933,-0.477){7}{\rule{0.820pt}{0.115pt}}
\multiput(288.00,663.17)(7.298,-5.000){2}{\rule{0.410pt}{0.400pt}}
\multiput(297.00,657.93)(0.933,-0.477){7}{\rule{0.820pt}{0.115pt}}
\multiput(297.00,658.17)(7.298,-5.000){2}{\rule{0.410pt}{0.400pt}}
\multiput(306.00,652.93)(1.044,-0.477){7}{\rule{0.900pt}{0.115pt}}
\multiput(306.00,653.17)(8.132,-5.000){2}{\rule{0.450pt}{0.400pt}}
\multiput(316.00,647.94)(1.212,-0.468){5}{\rule{1.000pt}{0.113pt}}
\multiput(316.00,648.17)(6.924,-4.000){2}{\rule{0.500pt}{0.400pt}}
\multiput(325.00,643.93)(0.933,-0.477){7}{\rule{0.820pt}{0.115pt}}
\multiput(325.00,644.17)(7.298,-5.000){2}{\rule{0.410pt}{0.400pt}}
\multiput(334.00,638.93)(0.933,-0.477){7}{\rule{0.820pt}{0.115pt}}
\multiput(334.00,639.17)(7.298,-5.000){2}{\rule{0.410pt}{0.400pt}}
\multiput(343.00,633.93)(0.933,-0.477){7}{\rule{0.820pt}{0.115pt}}
\multiput(343.00,634.17)(7.298,-5.000){2}{\rule{0.410pt}{0.400pt}}
\multiput(352.00,628.93)(0.933,-0.477){7}{\rule{0.820pt}{0.115pt}}
\multiput(352.00,629.17)(7.298,-5.000){2}{\rule{0.410pt}{0.400pt}}
\multiput(361.00,623.93)(1.044,-0.477){7}{\rule{0.900pt}{0.115pt}}
\multiput(361.00,624.17)(8.132,-5.000){2}{\rule{0.450pt}{0.400pt}}
\multiput(371.00,618.94)(1.212,-0.468){5}{\rule{1.000pt}{0.113pt}}
\multiput(371.00,619.17)(6.924,-4.000){2}{\rule{0.500pt}{0.400pt}}
\multiput(380.00,614.93)(0.933,-0.477){7}{\rule{0.820pt}{0.115pt}}
\multiput(380.00,615.17)(7.298,-5.000){2}{\rule{0.410pt}{0.400pt}}
\multiput(389.00,609.93)(0.933,-0.477){7}{\rule{0.820pt}{0.115pt}}
\multiput(389.00,610.17)(7.298,-5.000){2}{\rule{0.410pt}{0.400pt}}
\multiput(398.00,604.93)(0.933,-0.477){7}{\rule{0.820pt}{0.115pt}}
\multiput(398.00,605.17)(7.298,-5.000){2}{\rule{0.410pt}{0.400pt}}
\multiput(407.00,599.93)(0.933,-0.477){7}{\rule{0.820pt}{0.115pt}}
\multiput(407.00,600.17)(7.298,-5.000){2}{\rule{0.410pt}{0.400pt}}
\multiput(416.00,594.93)(0.933,-0.477){7}{\rule{0.820pt}{0.115pt}}
\multiput(416.00,595.17)(7.298,-5.000){2}{\rule{0.410pt}{0.400pt}}
\multiput(425.00,589.94)(1.358,-0.468){5}{\rule{1.100pt}{0.113pt}}
\multiput(425.00,590.17)(7.717,-4.000){2}{\rule{0.550pt}{0.400pt}}
\multiput(435.00,585.93)(0.933,-0.477){7}{\rule{0.820pt}{0.115pt}}
\multiput(435.00,586.17)(7.298,-5.000){2}{\rule{0.410pt}{0.400pt}}
\multiput(444.00,580.93)(0.933,-0.477){7}{\rule{0.820pt}{0.115pt}}
\multiput(444.00,581.17)(7.298,-5.000){2}{\rule{0.410pt}{0.400pt}}
\multiput(453.00,575.93)(0.933,-0.477){7}{\rule{0.820pt}{0.115pt}}
\multiput(453.00,576.17)(7.298,-5.000){2}{\rule{0.410pt}{0.400pt}}
\multiput(462.00,570.93)(0.933,-0.477){7}{\rule{0.820pt}{0.115pt}}
\multiput(462.00,571.17)(7.298,-5.000){2}{\rule{0.410pt}{0.400pt}}
\multiput(471.00,565.94)(1.212,-0.468){5}{\rule{1.000pt}{0.113pt}}
\multiput(471.00,566.17)(6.924,-4.000){2}{\rule{0.500pt}{0.400pt}}
\multiput(480.00,561.93)(1.044,-0.477){7}{\rule{0.900pt}{0.115pt}}
\multiput(480.00,562.17)(8.132,-5.000){2}{\rule{0.450pt}{0.400pt}}
\multiput(490.00,556.93)(0.933,-0.477){7}{\rule{0.820pt}{0.115pt}}
\multiput(490.00,557.17)(7.298,-5.000){2}{\rule{0.410pt}{0.400pt}}
\multiput(499.00,551.93)(0.933,-0.477){7}{\rule{0.820pt}{0.115pt}}
\multiput(499.00,552.17)(7.298,-5.000){2}{\rule{0.410pt}{0.400pt}}
\multiput(508.00,546.93)(0.933,-0.477){7}{\rule{0.820pt}{0.115pt}}
\multiput(508.00,547.17)(7.298,-5.000){2}{\rule{0.410pt}{0.400pt}}
\multiput(517.00,541.93)(0.933,-0.477){7}{\rule{0.820pt}{0.115pt}}
\multiput(517.00,542.17)(7.298,-5.000){2}{\rule{0.410pt}{0.400pt}}
\multiput(526.00,536.94)(1.212,-0.468){5}{\rule{1.000pt}{0.113pt}}
\multiput(526.00,537.17)(6.924,-4.000){2}{\rule{0.500pt}{0.400pt}}
\multiput(535.00,532.93)(0.933,-0.477){7}{\rule{0.820pt}{0.115pt}}
\multiput(535.00,533.17)(7.298,-5.000){2}{\rule{0.410pt}{0.400pt}}
\multiput(544.00,527.93)(1.044,-0.477){7}{\rule{0.900pt}{0.115pt}}
\multiput(544.00,528.17)(8.132,-5.000){2}{\rule{0.450pt}{0.400pt}}
\multiput(554.00,522.93)(0.933,-0.477){7}{\rule{0.820pt}{0.115pt}}
\multiput(554.00,523.17)(7.298,-5.000){2}{\rule{0.410pt}{0.400pt}}
\multiput(563.00,517.93)(0.933,-0.477){7}{\rule{0.820pt}{0.115pt}}
\multiput(563.00,518.17)(7.298,-5.000){2}{\rule{0.410pt}{0.400pt}}
\multiput(572.00,512.93)(0.933,-0.477){7}{\rule{0.820pt}{0.115pt}}
\multiput(572.00,513.17)(7.298,-5.000){2}{\rule{0.410pt}{0.400pt}}
\multiput(581.00,507.94)(1.212,-0.468){5}{\rule{1.000pt}{0.113pt}}
\multiput(581.00,508.17)(6.924,-4.000){2}{\rule{0.500pt}{0.400pt}}
\multiput(590.00,503.93)(0.933,-0.477){7}{\rule{0.820pt}{0.115pt}}
\multiput(590.00,504.17)(7.298,-5.000){2}{\rule{0.410pt}{0.400pt}}
\multiput(599.00,498.93)(0.933,-0.477){7}{\rule{0.820pt}{0.115pt}}
\multiput(599.00,499.17)(7.298,-5.000){2}{\rule{0.410pt}{0.400pt}}
\multiput(608.00,493.93)(1.044,-0.477){7}{\rule{0.900pt}{0.115pt}}
\multiput(608.00,494.17)(8.132,-5.000){2}{\rule{0.450pt}{0.400pt}}
\multiput(618.00,488.93)(0.933,-0.477){7}{\rule{0.820pt}{0.115pt}}
\multiput(618.00,489.17)(7.298,-5.000){2}{\rule{0.410pt}{0.400pt}}
\multiput(627.00,483.93)(0.933,-0.477){7}{\rule{0.820pt}{0.115pt}}
\multiput(627.00,484.17)(7.298,-5.000){2}{\rule{0.410pt}{0.400pt}}
\multiput(636.00,478.94)(1.212,-0.468){5}{\rule{1.000pt}{0.113pt}}
\multiput(636.00,479.17)(6.924,-4.000){2}{\rule{0.500pt}{0.400pt}}
\multiput(645.00,474.93)(0.933,-0.477){7}{\rule{0.820pt}{0.115pt}}
\multiput(645.00,475.17)(7.298,-5.000){2}{\rule{0.410pt}{0.400pt}}
\multiput(654.00,469.93)(0.933,-0.477){7}{\rule{0.820pt}{0.115pt}}
\multiput(654.00,470.17)(7.298,-5.000){2}{\rule{0.410pt}{0.400pt}}
\multiput(663.00,464.93)(1.044,-0.477){7}{\rule{0.900pt}{0.115pt}}
\multiput(663.00,465.17)(8.132,-5.000){2}{\rule{0.450pt}{0.400pt}}
\multiput(673.00,459.93)(0.933,-0.477){7}{\rule{0.820pt}{0.115pt}}
\multiput(673.00,460.17)(7.298,-5.000){2}{\rule{0.410pt}{0.400pt}}
\multiput(682.00,454.93)(0.933,-0.477){7}{\rule{0.820pt}{0.115pt}}
\multiput(682.00,455.17)(7.298,-5.000){2}{\rule{0.410pt}{0.400pt}}
\multiput(691.00,449.94)(1.212,-0.468){5}{\rule{1.000pt}{0.113pt}}
\multiput(691.00,450.17)(6.924,-4.000){2}{\rule{0.500pt}{0.400pt}}
\multiput(700.00,445.93)(0.933,-0.477){7}{\rule{0.820pt}{0.115pt}}
\multiput(700.00,446.17)(7.298,-5.000){2}{\rule{0.410pt}{0.400pt}}
\multiput(709.00,440.93)(0.933,-0.477){7}{\rule{0.820pt}{0.115pt}}
\multiput(709.00,441.17)(7.298,-5.000){2}{\rule{0.410pt}{0.400pt}}
\multiput(718.00,435.93)(0.933,-0.477){7}{\rule{0.820pt}{0.115pt}}
\multiput(718.00,436.17)(7.298,-5.000){2}{\rule{0.410pt}{0.400pt}}
\multiput(727.00,430.93)(1.044,-0.477){7}{\rule{0.900pt}{0.115pt}}
\multiput(727.00,431.17)(8.132,-5.000){2}{\rule{0.450pt}{0.400pt}}
\multiput(737.00,425.93)(0.933,-0.477){7}{\rule{0.820pt}{0.115pt}}
\multiput(737.00,426.17)(7.298,-5.000){2}{\rule{0.410pt}{0.400pt}}
\multiput(746.00,420.94)(1.212,-0.468){5}{\rule{1.000pt}{0.113pt}}
\multiput(746.00,421.17)(6.924,-4.000){2}{\rule{0.500pt}{0.400pt}}
\multiput(755.00,416.93)(0.933,-0.477){7}{\rule{0.820pt}{0.115pt}}
\multiput(755.00,417.17)(7.298,-5.000){2}{\rule{0.410pt}{0.400pt}}
\multiput(764.00,411.93)(0.933,-0.477){7}{\rule{0.820pt}{0.115pt}}
\multiput(764.00,412.17)(7.298,-5.000){2}{\rule{0.410pt}{0.400pt}}
\multiput(773.00,406.93)(0.933,-0.477){7}{\rule{0.820pt}{0.115pt}}
\multiput(773.00,407.17)(7.298,-5.000){2}{\rule{0.410pt}{0.400pt}}
\multiput(782.00,401.93)(1.044,-0.477){7}{\rule{0.900pt}{0.115pt}}
\multiput(782.00,402.17)(8.132,-5.000){2}{\rule{0.450pt}{0.400pt}}
\multiput(792.00,396.93)(0.933,-0.477){7}{\rule{0.820pt}{0.115pt}}
\multiput(792.00,397.17)(7.298,-5.000){2}{\rule{0.410pt}{0.400pt}}
\multiput(801.00,391.94)(1.212,-0.468){5}{\rule{1.000pt}{0.113pt}}
\multiput(801.00,392.17)(6.924,-4.000){2}{\rule{0.500pt}{0.400pt}}
\multiput(810.00,387.93)(0.933,-0.477){7}{\rule{0.820pt}{0.115pt}}
\multiput(810.00,388.17)(7.298,-5.000){2}{\rule{0.410pt}{0.400pt}}
\multiput(819.00,382.93)(0.933,-0.477){7}{\rule{0.820pt}{0.115pt}}
\multiput(819.00,383.17)(7.298,-5.000){2}{\rule{0.410pt}{0.400pt}}
\multiput(828.00,377.93)(0.933,-0.477){7}{\rule{0.820pt}{0.115pt}}
\multiput(828.00,378.17)(7.298,-5.000){2}{\rule{0.410pt}{0.400pt}}
\multiput(837.00,372.93)(0.933,-0.477){7}{\rule{0.820pt}{0.115pt}}
\multiput(837.00,373.17)(7.298,-5.000){2}{\rule{0.410pt}{0.400pt}}
\multiput(846.00,367.94)(1.358,-0.468){5}{\rule{1.100pt}{0.113pt}}
\multiput(846.00,368.17)(7.717,-4.000){2}{\rule{0.550pt}{0.400pt}}
\multiput(856.00,363.93)(0.933,-0.477){7}{\rule{0.820pt}{0.115pt}}
\multiput(856.00,364.17)(7.298,-5.000){2}{\rule{0.410pt}{0.400pt}}
\multiput(865.00,358.93)(0.933,-0.477){7}{\rule{0.820pt}{0.115pt}}
\multiput(865.00,359.17)(7.298,-5.000){2}{\rule{0.410pt}{0.400pt}}
\multiput(874.00,353.93)(0.933,-0.477){7}{\rule{0.820pt}{0.115pt}}
\multiput(874.00,354.17)(7.298,-5.000){2}{\rule{0.410pt}{0.400pt}}
\multiput(883.00,348.93)(0.933,-0.477){7}{\rule{0.820pt}{0.115pt}}
\multiput(883.00,349.17)(7.298,-5.000){2}{\rule{0.410pt}{0.400pt}}
\multiput(892.00,343.93)(0.933,-0.477){7}{\rule{0.820pt}{0.115pt}}
\multiput(892.00,344.17)(7.298,-5.000){2}{\rule{0.410pt}{0.400pt}}
\multiput(901.00,338.94)(1.212,-0.468){5}{\rule{1.000pt}{0.113pt}}
\multiput(901.00,339.17)(6.924,-4.000){2}{\rule{0.500pt}{0.400pt}}
\multiput(910.00,334.93)(1.044,-0.477){7}{\rule{0.900pt}{0.115pt}}
\multiput(910.00,335.17)(8.132,-5.000){2}{\rule{0.450pt}{0.400pt}}
\multiput(920.00,329.93)(0.933,-0.477){7}{\rule{0.820pt}{0.115pt}}
\multiput(920.00,330.17)(7.298,-5.000){2}{\rule{0.410pt}{0.400pt}}
\multiput(929.00,324.93)(0.933,-0.477){7}{\rule{0.820pt}{0.115pt}}
\multiput(929.00,325.17)(7.298,-5.000){2}{\rule{0.410pt}{0.400pt}}
\multiput(938.00,319.93)(0.933,-0.477){7}{\rule{0.820pt}{0.115pt}}
\multiput(938.00,320.17)(7.298,-5.000){2}{\rule{0.410pt}{0.400pt}}
\multiput(947.00,314.93)(0.933,-0.477){7}{\rule{0.820pt}{0.115pt}}
\multiput(947.00,315.17)(7.298,-5.000){2}{\rule{0.410pt}{0.400pt}}
\multiput(956.00,309.94)(1.212,-0.468){5}{\rule{1.000pt}{0.113pt}}
\multiput(956.00,310.17)(6.924,-4.000){2}{\rule{0.500pt}{0.400pt}}
\multiput(965.00,305.93)(1.044,-0.477){7}{\rule{0.900pt}{0.115pt}}
\multiput(965.00,306.17)(8.132,-5.000){2}{\rule{0.450pt}{0.400pt}}
\multiput(975.00,300.93)(0.933,-0.477){7}{\rule{0.820pt}{0.115pt}}
\multiput(975.00,301.17)(7.298,-5.000){2}{\rule{0.410pt}{0.400pt}}
\multiput(984.00,295.93)(0.933,-0.477){7}{\rule{0.820pt}{0.115pt}}
\multiput(984.00,296.17)(7.298,-5.000){2}{\rule{0.410pt}{0.400pt}}
\multiput(993.00,290.93)(0.933,-0.477){7}{\rule{0.820pt}{0.115pt}}
\multiput(993.00,291.17)(7.298,-5.000){2}{\rule{0.410pt}{0.400pt}}
\multiput(1002.00,285.93)(0.933,-0.477){7}{\rule{0.820pt}{0.115pt}}
\multiput(1002.00,286.17)(7.298,-5.000){2}{\rule{0.410pt}{0.400pt}}
\multiput(1011.00,280.94)(1.212,-0.468){5}{\rule{1.000pt}{0.113pt}}
\multiput(1011.00,281.17)(6.924,-4.000){2}{\rule{0.500pt}{0.400pt}}
\multiput(1020.00,276.93)(0.933,-0.477){7}{\rule{0.820pt}{0.115pt}}
\multiput(1020.00,277.17)(7.298,-5.000){2}{\rule{0.410pt}{0.400pt}}
\multiput(1029.00,271.93)(1.044,-0.477){7}{\rule{0.900pt}{0.115pt}}
\multiput(1029.00,272.17)(8.132,-5.000){2}{\rule{0.450pt}{0.400pt}}
\multiput(1039.00,266.93)(0.933,-0.477){7}{\rule{0.820pt}{0.115pt}}
\multiput(1039.00,267.17)(7.298,-5.000){2}{\rule{0.410pt}{0.400pt}}
\multiput(1048.00,261.93)(0.933,-0.477){7}{\rule{0.820pt}{0.115pt}}
\multiput(1048.00,262.17)(7.298,-5.000){2}{\rule{0.410pt}{0.400pt}}
\multiput(1057.00,256.93)(0.933,-0.477){7}{\rule{0.820pt}{0.115pt}}
\multiput(1057.00,257.17)(7.298,-5.000){2}{\rule{0.410pt}{0.400pt}}
\multiput(1066.00,251.94)(1.212,-0.468){5}{\rule{1.000pt}{0.113pt}}
\multiput(1066.00,252.17)(6.924,-4.000){2}{\rule{0.500pt}{0.400pt}}
\multiput(1075.00,247.93)(0.933,-0.477){7}{\rule{0.820pt}{0.115pt}}
\multiput(1075.00,248.17)(7.298,-5.000){2}{\rule{0.410pt}{0.400pt}}
\multiput(1084.00,242.93)(1.044,-0.477){7}{\rule{0.900pt}{0.115pt}}
\multiput(1084.00,243.17)(8.132,-5.000){2}{\rule{0.450pt}{0.400pt}}
\multiput(1094.00,237.93)(0.933,-0.477){7}{\rule{0.820pt}{0.115pt}}
\multiput(1094.00,238.17)(7.298,-5.000){2}{\rule{0.410pt}{0.400pt}}
\multiput(1103.00,232.93)(0.933,-0.477){7}{\rule{0.820pt}{0.115pt}}
\multiput(1103.00,233.17)(7.298,-5.000){2}{\rule{0.410pt}{0.400pt}}
\multiput(1112.00,227.93)(0.933,-0.477){7}{\rule{0.820pt}{0.115pt}}
\multiput(1112.00,228.17)(7.298,-5.000){2}{\rule{0.410pt}{0.400pt}}
\multiput(1121.00,222.94)(1.212,-0.468){5}{\rule{1.000pt}{0.113pt}}
\multiput(1121.00,223.17)(6.924,-4.000){2}{\rule{0.500pt}{0.400pt}}
\multiput(1130.00,218.93)(0.933,-0.477){7}{\rule{0.820pt}{0.115pt}}
\multiput(1130.00,219.17)(7.298,-5.000){2}{\rule{0.410pt}{0.400pt}}
\put(211.0,131.0){\rule[-0.200pt]{0.400pt}{157.308pt}}
\put(211.0,131.0){\rule[-0.200pt]{247.404pt}{0.400pt}}
\put(1238.0,131.0){\rule[-0.200pt]{0.400pt}{157.308pt}}
\put(211.0,784.0){\rule[-0.200pt]{247.404pt}{0.400pt}}
\end{picture}

        \end{figure}
    \end{center}
\end{frame}

\begin{frame}
    \frametitle{\ejerciciocmd}
    \framesubtitle{Resolución (\rom{5}): comprobando si la reacción es de orden 2 (\rom{1})}
    \structure{Reacción:} \ce{b -> c}
    \structure{Ecuación cinética general:}
    $$
        v = k[b]^x = -\frac{\Delta[b]}{\Delta t}
    $$
    \begin{center}
        {\Large \textbf{\textit{Método integral de velocidad}}}
    \end{center}
    \structure{Orden 2}
    \begin{overprint}
        \onslide<1>
            $$
                -\dv{[b]}{t} = k[b]^2\Rightarrow\frac{\dd{[b]}}{[b]^2} = -k\dd{t}\Rightarrow\int_{[b]_0}^{[b]}[b]^{-2}\dd{[b]} = -k\int_{0}^{t}\dd{t}
            $$
        \onslide<2>
            $$
                \frac{1}{[b]} = \frac{1}{[b]_0} + kt
            $$
    \end{overprint}
\end{frame}

\begin{frame}
    \frametitle{\ejerciciocmd}
    \framesubtitle{Resolución (\rom{6}): comprobando si la reacción es de orden 2 (\rom{2})}
    \begin{center}
        \begin{figure}
            \caption{Representación de la inversa de la concentración de b ($\frac{1}{[b]}$) con respecto al tiempo ($t$). La reacción \textbf{\underline{SÍ SE AJUSTA}} perfectamente a una reacción de \textbf{\underline{orden 2}}. La pendiente es $k$ ($k=\SI{,0143}{\per\second\per\Molar}$).}
            % GNUPLOT: LaTeX picture
\setlength{\unitlength}{0.240900pt}
\ifx\plotpoint\undefined\newsavebox{\plotpoint}\fi
\sbox{\plotpoint}{\rule[-0.200pt]{0.400pt}{0.400pt}}%
\begin{picture}(1299,826)(0,0)
\sbox{\plotpoint}{\rule[-0.200pt]{0.400pt}{0.400pt}}%
\put(231.0,131.0){\rule[-0.200pt]{4.818pt}{0.400pt}}
\put(211,131){\makebox(0,0)[r]{$60.000$}}
\put(1218.0,131.0){\rule[-0.200pt]{4.818pt}{0.400pt}}
\put(231.0,240.0){\rule[-0.200pt]{4.818pt}{0.400pt}}
\put(211,240){\makebox(0,0)[r]{$80.000$}}
\put(1218.0,240.0){\rule[-0.200pt]{4.818pt}{0.400pt}}
\put(231.0,349.0){\rule[-0.200pt]{4.818pt}{0.400pt}}
\put(211,349){\makebox(0,0)[r]{$100.000$}}
\put(1218.0,349.0){\rule[-0.200pt]{4.818pt}{0.400pt}}
\put(231.0,458.0){\rule[-0.200pt]{4.818pt}{0.400pt}}
\put(211,458){\makebox(0,0)[r]{$120.000$}}
\put(1218.0,458.0){\rule[-0.200pt]{4.818pt}{0.400pt}}
\put(231.0,566.0){\rule[-0.200pt]{4.818pt}{0.400pt}}
\put(211,566){\makebox(0,0)[r]{$140.000$}}
\put(1218.0,566.0){\rule[-0.200pt]{4.818pt}{0.400pt}}
\put(231.0,675.0){\rule[-0.200pt]{4.818pt}{0.400pt}}
\put(211,675){\makebox(0,0)[r]{$160.000$}}
\put(1218.0,675.0){\rule[-0.200pt]{4.818pt}{0.400pt}}
\put(231.0,784.0){\rule[-0.200pt]{4.818pt}{0.400pt}}
\put(211,784){\makebox(0,0)[r]{$180.000$}}
\put(1218.0,784.0){\rule[-0.200pt]{4.818pt}{0.400pt}}
\put(231.0,131.0){\rule[-0.200pt]{0.400pt}{4.818pt}}
\put(231,90){\makebox(0,0){$0$}}
\put(231.0,764.0){\rule[-0.200pt]{0.400pt}{4.818pt}}
\put(343.0,131.0){\rule[-0.200pt]{0.400pt}{4.818pt}}
\put(343,90){\makebox(0,0){$1000$}}
\put(343.0,764.0){\rule[-0.200pt]{0.400pt}{4.818pt}}
\put(455.0,131.0){\rule[-0.200pt]{0.400pt}{4.818pt}}
\put(455,90){\makebox(0,0){$2000$}}
\put(455.0,764.0){\rule[-0.200pt]{0.400pt}{4.818pt}}
\put(567.0,131.0){\rule[-0.200pt]{0.400pt}{4.818pt}}
\put(567,90){\makebox(0,0){$3000$}}
\put(567.0,764.0){\rule[-0.200pt]{0.400pt}{4.818pt}}
\put(679.0,131.0){\rule[-0.200pt]{0.400pt}{4.818pt}}
\put(679,90){\makebox(0,0){$4000$}}
\put(679.0,764.0){\rule[-0.200pt]{0.400pt}{4.818pt}}
\put(790.0,131.0){\rule[-0.200pt]{0.400pt}{4.818pt}}
\put(790,90){\makebox(0,0){$5000$}}
\put(790.0,764.0){\rule[-0.200pt]{0.400pt}{4.818pt}}
\put(902.0,131.0){\rule[-0.200pt]{0.400pt}{4.818pt}}
\put(902,90){\makebox(0,0){$6000$}}
\put(902.0,764.0){\rule[-0.200pt]{0.400pt}{4.818pt}}
\put(1014.0,131.0){\rule[-0.200pt]{0.400pt}{4.818pt}}
\put(1014,90){\makebox(0,0){$7000$}}
\put(1014.0,764.0){\rule[-0.200pt]{0.400pt}{4.818pt}}
\put(1126.0,131.0){\rule[-0.200pt]{0.400pt}{4.818pt}}
\put(1126,90){\makebox(0,0){$8000$}}
\put(1126.0,764.0){\rule[-0.200pt]{0.400pt}{4.818pt}}
\put(1238.0,131.0){\rule[-0.200pt]{0.400pt}{4.818pt}}
\put(1238,90){\makebox(0,0){$9000$}}
\put(1238.0,764.0){\rule[-0.200pt]{0.400pt}{4.818pt}}
\put(231.0,131.0){\rule[-0.200pt]{0.400pt}{157.308pt}}
\put(231.0,131.0){\rule[-0.200pt]{242.586pt}{0.400pt}}
\put(1238.0,131.0){\rule[-0.200pt]{0.400pt}{157.308pt}}
\put(231.0,784.0){\rule[-0.200pt]{242.586pt}{0.400pt}}
\put(30,457){\makebox(0,0){$\frac{1}{[b]}~(\si{\per\Molar})$}}
\put(734,29){\makebox(0,0){$\text{tiempo}~(\si{\second})$}}
\put(253,140){\makebox(0,0){$\blacksquare$}}
\put(299,175){\makebox(0,0){$\blacksquare$}}
\put(370,226){\makebox(0,0){$\blacksquare$}}
\put(475,299){\makebox(0,0){$\blacksquare$}}
\put(694,452){\makebox(0,0){$\blacksquare$}}
\put(1141,759){\makebox(0,0){$\blacksquare$}}
\put(300,723){\makebox(0,0)[l]{$\frac{1}{[b]}$ = \num{59,48} + \num{1,429e-02}$t$ ($r^2$ = \num{,9999})}}
\put(950.0,723.0){\rule[-0.200pt]{24.090pt}{0.400pt}}
\put(253,143){\usebox{\plotpoint}}
\multiput(253.00,143.59)(0.645,0.485){11}{\rule{0.614pt}{0.117pt}}
\multiput(253.00,142.17)(7.725,7.000){2}{\rule{0.307pt}{0.400pt}}
\multiput(262.00,150.59)(0.762,0.482){9}{\rule{0.700pt}{0.116pt}}
\multiput(262.00,149.17)(7.547,6.000){2}{\rule{0.350pt}{0.400pt}}
\multiput(271.00,156.59)(0.762,0.482){9}{\rule{0.700pt}{0.116pt}}
\multiput(271.00,155.17)(7.547,6.000){2}{\rule{0.350pt}{0.400pt}}
\multiput(280.00,162.59)(0.762,0.482){9}{\rule{0.700pt}{0.116pt}}
\multiput(280.00,161.17)(7.547,6.000){2}{\rule{0.350pt}{0.400pt}}
\multiput(289.00,168.59)(0.645,0.485){11}{\rule{0.614pt}{0.117pt}}
\multiput(289.00,167.17)(7.725,7.000){2}{\rule{0.307pt}{0.400pt}}
\multiput(298.00,175.59)(0.762,0.482){9}{\rule{0.700pt}{0.116pt}}
\multiput(298.00,174.17)(7.547,6.000){2}{\rule{0.350pt}{0.400pt}}
\multiput(307.00,181.59)(0.762,0.482){9}{\rule{0.700pt}{0.116pt}}
\multiput(307.00,180.17)(7.547,6.000){2}{\rule{0.350pt}{0.400pt}}
\multiput(316.00,187.59)(0.762,0.482){9}{\rule{0.700pt}{0.116pt}}
\multiput(316.00,186.17)(7.547,6.000){2}{\rule{0.350pt}{0.400pt}}
\multiput(325.00,193.59)(0.762,0.482){9}{\rule{0.700pt}{0.116pt}}
\multiput(325.00,192.17)(7.547,6.000){2}{\rule{0.350pt}{0.400pt}}
\multiput(334.00,199.59)(0.645,0.485){11}{\rule{0.614pt}{0.117pt}}
\multiput(334.00,198.17)(7.725,7.000){2}{\rule{0.307pt}{0.400pt}}
\multiput(343.00,206.59)(0.762,0.482){9}{\rule{0.700pt}{0.116pt}}
\multiput(343.00,205.17)(7.547,6.000){2}{\rule{0.350pt}{0.400pt}}
\multiput(352.00,212.59)(0.762,0.482){9}{\rule{0.700pt}{0.116pt}}
\multiput(352.00,211.17)(7.547,6.000){2}{\rule{0.350pt}{0.400pt}}
\multiput(361.00,218.59)(0.671,0.482){9}{\rule{0.633pt}{0.116pt}}
\multiput(361.00,217.17)(6.685,6.000){2}{\rule{0.317pt}{0.400pt}}
\multiput(369.00,224.59)(0.645,0.485){11}{\rule{0.614pt}{0.117pt}}
\multiput(369.00,223.17)(7.725,7.000){2}{\rule{0.307pt}{0.400pt}}
\multiput(378.00,231.59)(0.762,0.482){9}{\rule{0.700pt}{0.116pt}}
\multiput(378.00,230.17)(7.547,6.000){2}{\rule{0.350pt}{0.400pt}}
\multiput(387.00,237.59)(0.762,0.482){9}{\rule{0.700pt}{0.116pt}}
\multiput(387.00,236.17)(7.547,6.000){2}{\rule{0.350pt}{0.400pt}}
\multiput(396.00,243.59)(0.762,0.482){9}{\rule{0.700pt}{0.116pt}}
\multiput(396.00,242.17)(7.547,6.000){2}{\rule{0.350pt}{0.400pt}}
\multiput(405.00,249.59)(0.645,0.485){11}{\rule{0.614pt}{0.117pt}}
\multiput(405.00,248.17)(7.725,7.000){2}{\rule{0.307pt}{0.400pt}}
\multiput(414.00,256.59)(0.762,0.482){9}{\rule{0.700pt}{0.116pt}}
\multiput(414.00,255.17)(7.547,6.000){2}{\rule{0.350pt}{0.400pt}}
\multiput(423.00,262.59)(0.762,0.482){9}{\rule{0.700pt}{0.116pt}}
\multiput(423.00,261.17)(7.547,6.000){2}{\rule{0.350pt}{0.400pt}}
\multiput(432.00,268.59)(0.762,0.482){9}{\rule{0.700pt}{0.116pt}}
\multiput(432.00,267.17)(7.547,6.000){2}{\rule{0.350pt}{0.400pt}}
\multiput(441.00,274.59)(0.645,0.485){11}{\rule{0.614pt}{0.117pt}}
\multiput(441.00,273.17)(7.725,7.000){2}{\rule{0.307pt}{0.400pt}}
\multiput(450.00,281.59)(0.762,0.482){9}{\rule{0.700pt}{0.116pt}}
\multiput(450.00,280.17)(7.547,6.000){2}{\rule{0.350pt}{0.400pt}}
\multiput(459.00,287.59)(0.762,0.482){9}{\rule{0.700pt}{0.116pt}}
\multiput(459.00,286.17)(7.547,6.000){2}{\rule{0.350pt}{0.400pt}}
\multiput(468.00,293.59)(0.762,0.482){9}{\rule{0.700pt}{0.116pt}}
\multiput(468.00,292.17)(7.547,6.000){2}{\rule{0.350pt}{0.400pt}}
\multiput(477.00,299.59)(0.762,0.482){9}{\rule{0.700pt}{0.116pt}}
\multiput(477.00,298.17)(7.547,6.000){2}{\rule{0.350pt}{0.400pt}}
\multiput(486.00,305.59)(0.645,0.485){11}{\rule{0.614pt}{0.117pt}}
\multiput(486.00,304.17)(7.725,7.000){2}{\rule{0.307pt}{0.400pt}}
\multiput(495.00,312.59)(0.762,0.482){9}{\rule{0.700pt}{0.116pt}}
\multiput(495.00,311.17)(7.547,6.000){2}{\rule{0.350pt}{0.400pt}}
\multiput(504.00,318.59)(0.762,0.482){9}{\rule{0.700pt}{0.116pt}}
\multiput(504.00,317.17)(7.547,6.000){2}{\rule{0.350pt}{0.400pt}}
\multiput(513.00,324.59)(0.762,0.482){9}{\rule{0.700pt}{0.116pt}}
\multiput(513.00,323.17)(7.547,6.000){2}{\rule{0.350pt}{0.400pt}}
\multiput(522.00,330.59)(0.645,0.485){11}{\rule{0.614pt}{0.117pt}}
\multiput(522.00,329.17)(7.725,7.000){2}{\rule{0.307pt}{0.400pt}}
\multiput(531.00,337.59)(0.762,0.482){9}{\rule{0.700pt}{0.116pt}}
\multiput(531.00,336.17)(7.547,6.000){2}{\rule{0.350pt}{0.400pt}}
\multiput(540.00,343.59)(0.762,0.482){9}{\rule{0.700pt}{0.116pt}}
\multiput(540.00,342.17)(7.547,6.000){2}{\rule{0.350pt}{0.400pt}}
\multiput(549.00,349.59)(0.762,0.482){9}{\rule{0.700pt}{0.116pt}}
\multiput(549.00,348.17)(7.547,6.000){2}{\rule{0.350pt}{0.400pt}}
\multiput(558.00,355.59)(0.645,0.485){11}{\rule{0.614pt}{0.117pt}}
\multiput(558.00,354.17)(7.725,7.000){2}{\rule{0.307pt}{0.400pt}}
\multiput(567.00,362.59)(0.762,0.482){9}{\rule{0.700pt}{0.116pt}}
\multiput(567.00,361.17)(7.547,6.000){2}{\rule{0.350pt}{0.400pt}}
\multiput(576.00,368.59)(0.762,0.482){9}{\rule{0.700pt}{0.116pt}}
\multiput(576.00,367.17)(7.547,6.000){2}{\rule{0.350pt}{0.400pt}}
\multiput(585.00,374.59)(0.762,0.482){9}{\rule{0.700pt}{0.116pt}}
\multiput(585.00,373.17)(7.547,6.000){2}{\rule{0.350pt}{0.400pt}}
\multiput(594.00,380.59)(0.645,0.485){11}{\rule{0.614pt}{0.117pt}}
\multiput(594.00,379.17)(7.725,7.000){2}{\rule{0.307pt}{0.400pt}}
\multiput(603.00,387.59)(0.762,0.482){9}{\rule{0.700pt}{0.116pt}}
\multiput(603.00,386.17)(7.547,6.000){2}{\rule{0.350pt}{0.400pt}}
\multiput(612.00,393.59)(0.762,0.482){9}{\rule{0.700pt}{0.116pt}}
\multiput(612.00,392.17)(7.547,6.000){2}{\rule{0.350pt}{0.400pt}}
\multiput(621.00,399.59)(0.762,0.482){9}{\rule{0.700pt}{0.116pt}}
\multiput(621.00,398.17)(7.547,6.000){2}{\rule{0.350pt}{0.400pt}}
\multiput(630.00,405.59)(0.645,0.485){11}{\rule{0.614pt}{0.117pt}}
\multiput(630.00,404.17)(7.725,7.000){2}{\rule{0.307pt}{0.400pt}}
\multiput(639.00,412.59)(0.762,0.482){9}{\rule{0.700pt}{0.116pt}}
\multiput(639.00,411.17)(7.547,6.000){2}{\rule{0.350pt}{0.400pt}}
\multiput(648.00,418.59)(0.762,0.482){9}{\rule{0.700pt}{0.116pt}}
\multiput(648.00,417.17)(7.547,6.000){2}{\rule{0.350pt}{0.400pt}}
\multiput(657.00,424.59)(0.762,0.482){9}{\rule{0.700pt}{0.116pt}}
\multiput(657.00,423.17)(7.547,6.000){2}{\rule{0.350pt}{0.400pt}}
\multiput(666.00,430.59)(0.762,0.482){9}{\rule{0.700pt}{0.116pt}}
\multiput(666.00,429.17)(7.547,6.000){2}{\rule{0.350pt}{0.400pt}}
\multiput(675.00,436.59)(0.645,0.485){11}{\rule{0.614pt}{0.117pt}}
\multiput(675.00,435.17)(7.725,7.000){2}{\rule{0.307pt}{0.400pt}}
\multiput(684.00,443.59)(0.762,0.482){9}{\rule{0.700pt}{0.116pt}}
\multiput(684.00,442.17)(7.547,6.000){2}{\rule{0.350pt}{0.400pt}}
\multiput(693.00,449.59)(0.762,0.482){9}{\rule{0.700pt}{0.116pt}}
\multiput(693.00,448.17)(7.547,6.000){2}{\rule{0.350pt}{0.400pt}}
\multiput(702.00,455.59)(0.671,0.482){9}{\rule{0.633pt}{0.116pt}}
\multiput(702.00,454.17)(6.685,6.000){2}{\rule{0.317pt}{0.400pt}}
\multiput(710.00,461.59)(0.645,0.485){11}{\rule{0.614pt}{0.117pt}}
\multiput(710.00,460.17)(7.725,7.000){2}{\rule{0.307pt}{0.400pt}}
\multiput(719.00,468.59)(0.762,0.482){9}{\rule{0.700pt}{0.116pt}}
\multiput(719.00,467.17)(7.547,6.000){2}{\rule{0.350pt}{0.400pt}}
\multiput(728.00,474.59)(0.762,0.482){9}{\rule{0.700pt}{0.116pt}}
\multiput(728.00,473.17)(7.547,6.000){2}{\rule{0.350pt}{0.400pt}}
\multiput(737.00,480.59)(0.762,0.482){9}{\rule{0.700pt}{0.116pt}}
\multiput(737.00,479.17)(7.547,6.000){2}{\rule{0.350pt}{0.400pt}}
\multiput(746.00,486.59)(0.645,0.485){11}{\rule{0.614pt}{0.117pt}}
\multiput(746.00,485.17)(7.725,7.000){2}{\rule{0.307pt}{0.400pt}}
\multiput(755.00,493.59)(0.762,0.482){9}{\rule{0.700pt}{0.116pt}}
\multiput(755.00,492.17)(7.547,6.000){2}{\rule{0.350pt}{0.400pt}}
\multiput(764.00,499.59)(0.762,0.482){9}{\rule{0.700pt}{0.116pt}}
\multiput(764.00,498.17)(7.547,6.000){2}{\rule{0.350pt}{0.400pt}}
\multiput(773.00,505.59)(0.762,0.482){9}{\rule{0.700pt}{0.116pt}}
\multiput(773.00,504.17)(7.547,6.000){2}{\rule{0.350pt}{0.400pt}}
\multiput(782.00,511.59)(0.645,0.485){11}{\rule{0.614pt}{0.117pt}}
\multiput(782.00,510.17)(7.725,7.000){2}{\rule{0.307pt}{0.400pt}}
\multiput(791.00,518.59)(0.762,0.482){9}{\rule{0.700pt}{0.116pt}}
\multiput(791.00,517.17)(7.547,6.000){2}{\rule{0.350pt}{0.400pt}}
\multiput(800.00,524.59)(0.762,0.482){9}{\rule{0.700pt}{0.116pt}}
\multiput(800.00,523.17)(7.547,6.000){2}{\rule{0.350pt}{0.400pt}}
\multiput(809.00,530.59)(0.762,0.482){9}{\rule{0.700pt}{0.116pt}}
\multiput(809.00,529.17)(7.547,6.000){2}{\rule{0.350pt}{0.400pt}}
\multiput(818.00,536.59)(0.645,0.485){11}{\rule{0.614pt}{0.117pt}}
\multiput(818.00,535.17)(7.725,7.000){2}{\rule{0.307pt}{0.400pt}}
\multiput(827.00,543.59)(0.762,0.482){9}{\rule{0.700pt}{0.116pt}}
\multiput(827.00,542.17)(7.547,6.000){2}{\rule{0.350pt}{0.400pt}}
\multiput(836.00,549.59)(0.762,0.482){9}{\rule{0.700pt}{0.116pt}}
\multiput(836.00,548.17)(7.547,6.000){2}{\rule{0.350pt}{0.400pt}}
\multiput(845.00,555.59)(0.762,0.482){9}{\rule{0.700pt}{0.116pt}}
\multiput(845.00,554.17)(7.547,6.000){2}{\rule{0.350pt}{0.400pt}}
\multiput(854.00,561.59)(0.762,0.482){9}{\rule{0.700pt}{0.116pt}}
\multiput(854.00,560.17)(7.547,6.000){2}{\rule{0.350pt}{0.400pt}}
\multiput(863.00,567.59)(0.645,0.485){11}{\rule{0.614pt}{0.117pt}}
\multiput(863.00,566.17)(7.725,7.000){2}{\rule{0.307pt}{0.400pt}}
\multiput(872.00,574.59)(0.762,0.482){9}{\rule{0.700pt}{0.116pt}}
\multiput(872.00,573.17)(7.547,6.000){2}{\rule{0.350pt}{0.400pt}}
\multiput(881.00,580.59)(0.762,0.482){9}{\rule{0.700pt}{0.116pt}}
\multiput(881.00,579.17)(7.547,6.000){2}{\rule{0.350pt}{0.400pt}}
\multiput(890.00,586.59)(0.762,0.482){9}{\rule{0.700pt}{0.116pt}}
\multiput(890.00,585.17)(7.547,6.000){2}{\rule{0.350pt}{0.400pt}}
\multiput(899.00,592.59)(0.645,0.485){11}{\rule{0.614pt}{0.117pt}}
\multiput(899.00,591.17)(7.725,7.000){2}{\rule{0.307pt}{0.400pt}}
\multiput(908.00,599.59)(0.762,0.482){9}{\rule{0.700pt}{0.116pt}}
\multiput(908.00,598.17)(7.547,6.000){2}{\rule{0.350pt}{0.400pt}}
\multiput(917.00,605.59)(0.762,0.482){9}{\rule{0.700pt}{0.116pt}}
\multiput(917.00,604.17)(7.547,6.000){2}{\rule{0.350pt}{0.400pt}}
\multiput(926.00,611.59)(0.762,0.482){9}{\rule{0.700pt}{0.116pt}}
\multiput(926.00,610.17)(7.547,6.000){2}{\rule{0.350pt}{0.400pt}}
\multiput(935.00,617.59)(0.645,0.485){11}{\rule{0.614pt}{0.117pt}}
\multiput(935.00,616.17)(7.725,7.000){2}{\rule{0.307pt}{0.400pt}}
\multiput(944.00,624.59)(0.762,0.482){9}{\rule{0.700pt}{0.116pt}}
\multiput(944.00,623.17)(7.547,6.000){2}{\rule{0.350pt}{0.400pt}}
\multiput(953.00,630.59)(0.762,0.482){9}{\rule{0.700pt}{0.116pt}}
\multiput(953.00,629.17)(7.547,6.000){2}{\rule{0.350pt}{0.400pt}}
\multiput(962.00,636.59)(0.762,0.482){9}{\rule{0.700pt}{0.116pt}}
\multiput(962.00,635.17)(7.547,6.000){2}{\rule{0.350pt}{0.400pt}}
\multiput(971.00,642.59)(0.645,0.485){11}{\rule{0.614pt}{0.117pt}}
\multiput(971.00,641.17)(7.725,7.000){2}{\rule{0.307pt}{0.400pt}}
\multiput(980.00,649.59)(0.762,0.482){9}{\rule{0.700pt}{0.116pt}}
\multiput(980.00,648.17)(7.547,6.000){2}{\rule{0.350pt}{0.400pt}}
\multiput(989.00,655.59)(0.762,0.482){9}{\rule{0.700pt}{0.116pt}}
\multiput(989.00,654.17)(7.547,6.000){2}{\rule{0.350pt}{0.400pt}}
\multiput(998.00,661.59)(0.762,0.482){9}{\rule{0.700pt}{0.116pt}}
\multiput(998.00,660.17)(7.547,6.000){2}{\rule{0.350pt}{0.400pt}}
\multiput(1007.00,667.59)(0.762,0.482){9}{\rule{0.700pt}{0.116pt}}
\multiput(1007.00,666.17)(7.547,6.000){2}{\rule{0.350pt}{0.400pt}}
\multiput(1016.00,673.59)(0.645,0.485){11}{\rule{0.614pt}{0.117pt}}
\multiput(1016.00,672.17)(7.725,7.000){2}{\rule{0.307pt}{0.400pt}}
\multiput(1025.00,680.59)(0.762,0.482){9}{\rule{0.700pt}{0.116pt}}
\multiput(1025.00,679.17)(7.547,6.000){2}{\rule{0.350pt}{0.400pt}}
\multiput(1034.00,686.59)(0.762,0.482){9}{\rule{0.700pt}{0.116pt}}
\multiput(1034.00,685.17)(7.547,6.000){2}{\rule{0.350pt}{0.400pt}}
\multiput(1043.00,692.59)(0.671,0.482){9}{\rule{0.633pt}{0.116pt}}
\multiput(1043.00,691.17)(6.685,6.000){2}{\rule{0.317pt}{0.400pt}}
\multiput(1051.00,698.59)(0.645,0.485){11}{\rule{0.614pt}{0.117pt}}
\multiput(1051.00,697.17)(7.725,7.000){2}{\rule{0.307pt}{0.400pt}}
\multiput(1060.00,705.59)(0.762,0.482){9}{\rule{0.700pt}{0.116pt}}
\multiput(1060.00,704.17)(7.547,6.000){2}{\rule{0.350pt}{0.400pt}}
\multiput(1069.00,711.59)(0.762,0.482){9}{\rule{0.700pt}{0.116pt}}
\multiput(1069.00,710.17)(7.547,6.000){2}{\rule{0.350pt}{0.400pt}}
\multiput(1078.00,717.59)(0.762,0.482){9}{\rule{0.700pt}{0.116pt}}
\multiput(1078.00,716.17)(7.547,6.000){2}{\rule{0.350pt}{0.400pt}}
\multiput(1087.00,723.59)(0.645,0.485){11}{\rule{0.614pt}{0.117pt}}
\multiput(1087.00,722.17)(7.725,7.000){2}{\rule{0.307pt}{0.400pt}}
\multiput(1096.00,730.59)(0.762,0.482){9}{\rule{0.700pt}{0.116pt}}
\multiput(1096.00,729.17)(7.547,6.000){2}{\rule{0.350pt}{0.400pt}}
\multiput(1105.00,736.59)(0.762,0.482){9}{\rule{0.700pt}{0.116pt}}
\multiput(1105.00,735.17)(7.547,6.000){2}{\rule{0.350pt}{0.400pt}}
\multiput(1114.00,742.59)(0.762,0.482){9}{\rule{0.700pt}{0.116pt}}
\multiput(1114.00,741.17)(7.547,6.000){2}{\rule{0.350pt}{0.400pt}}
\multiput(1123.00,748.59)(0.645,0.485){11}{\rule{0.614pt}{0.117pt}}
\multiput(1123.00,747.17)(7.725,7.000){2}{\rule{0.307pt}{0.400pt}}
\multiput(1132.00,755.59)(0.762,0.482){9}{\rule{0.700pt}{0.116pt}}
\multiput(1132.00,754.17)(7.547,6.000){2}{\rule{0.350pt}{0.400pt}}
\put(231.0,131.0){\rule[-0.200pt]{0.400pt}{157.308pt}}
\put(231.0,131.0){\rule[-0.200pt]{242.586pt}{0.400pt}}
\put(1238.0,131.0){\rule[-0.200pt]{0.400pt}{157.308pt}}
\put(231.0,784.0){\rule[-0.200pt]{242.586pt}{0.400pt}}
\end{picture}

        \end{figure}
    \end{center}
\end{frame}
