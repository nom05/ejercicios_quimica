\begin{frame}
	\frametitle{\ejerciciocmd}
	\framesubtitle{Enunciado}
	\textbf{
		Una reacción tiene una constante de velocidad de \SI{,017}{\per\second} a \SI{298}{\kelvin} y una energía libre de activación del \SI{27,235}{\kilo\joule\per\mol}. La adición de un catalizador disminuye dicha energía de activación hasta un \SI{33}{\percent} de su valor inicial. Calcule la nueva constante de velocidad.
\resultadocmd{ \SI{26,86}{\per\second} }

	}
 \end{frame}

\newcommand{\tmedio}{t_{\rfrac{1}{2}}}

\begin{frame}
	\frametitle{\ejerciciocmd}
	\framesubtitle{Resolución (\rom{1}): resumen de ecuaciones para métodos de integración y $\tmedio$}
	\begin{center}
		\ce{A -> Productos}
	\end{center}
	\structure{Ecuación de velocidad:} $v = -\dv{[\ce{A}]}{t} = k\vdot[\ce{A}]^x$
	\begin{center}
		\begin{tabular}{lccc}
						&	\textbf{M. integración}		&	\textbf{M. tiempo de vida media}		&	\textbf{Relación $\mathbf{\tmedio\Leftrightarrow[\ce{A}]_{\text{REF}}}$}	\\
						&	$t$ desde el inicio			&	$t$ con respecto a REF					&																					\\
			\midrule
			\underline{\textbf{\color{blue}Orden 0}}												\\
						&	$[\ce{A}] = [\ce{A}]_0 - k\vdot t$										&
							$\tmedio  = \frac{[\ce{A}]_{\text{REF}}}{2\vdot k}$						&
							DIRECTA	$\left(\tmedio\propto[\ce{A}]_{\text{REF}}\right)$				\\[.3cm]
			\midrule
			\underline{\textbf{\color{blue}Orden 1}}												\\
						&	$\ln[\ce{A}] = \ln[\ce{A}]_0 - k\vdot t$								&
							$\tmedio = \frac{\ln 2}{k}$												&
							NO TIENE																\\[.3cm]
			\midrule
			\underline{\textbf{\color{blue}Orden 2}}												\\
						&	$[\ce{A}]^{-1} = [\ce{A}]^{-1}_0 + k\vdot t$							&
						$\tmedio = \frac{1}{[\ce{A}]_{\text{REF}}\vdot k}$							&
						INVERSA $\left(\tmedio\propto\rfrac{1}{[\ce{A}]_{\text{REF}}}\right)$
		\end{tabular}
	\end{center}
\end{frame}

\begin{frame}
	\frametitle{\ejerciciocmd}
	\framesubtitle{Resolución (\rom{2}): método del tiempo de vida media}
	\begin{overprint}
		\onslide<1>
			\structure{Datos iniciales:}
			\begin{center}
				\begin{tabular}{SS}
						{\textbf{[\ce{A}]~(\unit{\Molar})}}			&
						{\textbf{$t$ reacción~(\unit{\second})}}	\\
					\midrule
						1,0	&	0	\\
						,6	&	37	\\				
						,5	&	50	\\
						,3	&	87
				\end{tabular}
			\end{center}
		\onslide<2>
			\structure{Parejas de datos:}
			\begin{center}
				\begin{tabular}{SS}
						{\textbf{[\ce{A}]~(\unit{\Molar})}}			&
						{\textbf{$t$ reacción~(\unit{\second})}}	\\
					\midrule
						1,0		&	  0		\\
						 ,5		&	 50		\\
					\midrule
						 ,6		&	 37		\\
						 ,3		&	 87		\\	
				\end{tabular}
			\end{center}
		\onslide<3->
			\structure{Parejas de datos:}
			\begin{center}
				\begin{tabular}{lSSc}
						&
						{\textbf{[\ce{A}]~(\unit{\Molar})}}			&
						{\textbf{$t$ reacción~(\unit{\second})}}	&
						{\textbf{$\tmedio$~(\unit{\second})}}		\\
					\midrule
						REF$\rightarrow$	&
						1,0		&	   0	&	\multirow{2}{*}{{$\num{50}-\num{0}=\num{50}$}}	\\
						&
						 ,6		&	  50	&													\\
					\midrule
						REF$\rightarrow$	&
						 ,6		&	 37		&	\multirow{2}{*}{{$\num{87}-\num{37}=\num{50}$}}	\\
						&
						 ,3		&	 87		&													\\
				\end{tabular}
			\end{center}
	\end{overprint}
	\begin{enumerate}
		\item<2-> Emparejamos la concentración inicial de referencia con su mitad.
		\item<3-> Calculamos el tiempo relativo necesario para pasar de esa concentración a la mitad.
		\item<4-> Estudiamos la relación que hay entre las concentraciones de referencia de nuestras parejas y $\tmedio$. En nuestro caso, el tiempo relativo no cambia aunque $[\ce{A}]_{\text{REF}}$ decrece (pasamos de \SI{1,0}{\Molar} en la 1"a pareja a \SI{,6}{\Molar} en la 2"a). Por tanto, tenemos que comprobar que es de \tcbhighmath[boxrule=0.4pt,arc=4pt,colframe=green,drop fuzzy shadow=black]{\text{\visible<4->{\textbf{ORDEN 1}}}}
		\item<5-> Usamos la ecuación correspondiente para averiguar $k$:
					$$
						\tmedio = \frac{\ln(2)}{k}\Rightarrow
						k = \frac{\ln(2)}{\tmedio}\Rightarrow
						\left.
							\begin{cases}
								k_1 = \frac{\ln(2)}{\num{50}}	\\
								k_2 = \frac{\ln(2)}{\num{50}}
							\end{cases}
						\right\}\Rightarrow
						k_1 = k_2
					$$
					\textbf{Recordad incluir las \textbf{unidades} de la \textbf{constante cinética}:}\\
					\begin{center}
						\tcbhighmath[boxrule=0.4pt,arc=4pt,colframe=red,drop fuzzy shadow=black]{\visible<5->{k = \SI{,0138}{\per\second}}}\qquad
						\tcbhighmath[boxrule=0.4pt,arc=4pt,colframe=blue,drop fuzzy shadow=black]{\visible<5->{v=\SI{,0138}{\per\second}[\ce{A}]^1}}
					\end{center}
	\end{enumerate}
\end{frame}

\begin{frame}
	\frametitle{\ejerciciocmd}
	\framesubtitle{Resolución (\rom{3}): método de integración}
	\structure{Datos iniciales:}
	\begin{center}
		\begin{tabular}{SS}
			{\textbf{[\ce{A}]~(\unit{\Molar})}}			&
			{\textbf{$t$ reacción~(\unit{\second})}}	\\
			1,0		&	 0			\\
			 ,6		&	37			\\
			 ,5		&	50			\\
			 ,3		&	87			\\
		\end{tabular}
	\end{center}
	\structure{Comprobamos qué ecuación, de los diferentes órdenes, es la válida según nuestros datos:}\\[.2cm]
	\begin{overprint}
		\onslide<1>
			\underline{\textbf{\color{red!50!black}Orden 0:}}
			$$
				[\ce{A}] = [\ce{A}]_0 - k\vdot t\Rightarrow
				k = \frac{[\ce{A}]_0 - [\ce{A}]}{t}\left\{
				\begin{aligned}
					k_1 = \frac{\SI{1,0}{\Molar} - \SI{,6}{\Molar}}{\SI{37}{\second}} &= \SI{,0108}{\Molar\per\second}	\\
					k_2 = \frac{\SI{1,0}{\Molar} - \SI{,5}{\Molar}}{\SI{50}{\second}} &= \SI{,0100}{\Molar\per\second}	\\
					k_3 = \frac{\SI{1,0}{\Molar} - \SI{,3}{\Molar}}{\SI{87}{\second}} &= \SI{,0080}{\Molar\per\second}
				\end{aligned}
				\right\} k_1\neq k_2 \neq k_3
			$$
			\begin{center}
				\underline{\textbf{No es de orden 0}}
			\end{center}
		\onslide<2>
			\underline{\textbf{\color{red!50!black}Orden 1:}}
			$$
				\ln[\ce{A}] = \ln[\ce{A}]_0 - k\vdot t\Rightarrow
				k = \frac{\ln(\frac{[\ce{A}]_0}{[\ce{A}]})}{t}\left\{
				\begin{aligned}
					k_1 = \frac{\ln(\frac{\SI{1,0}{\cancel\Molar}}{\SI{,6}{\cancel\Molar}})}{\SI{37}{\second}} &= \SI{,0138}{\per\second}	\\
					k_2 = \frac{\ln(\frac{\SI{1,0}{\cancel\Molar}}{\SI{,5}{\cancel\Molar}})}{\SI{50}{\second}} &= \SI{,0138}{\per\second}	\\
					k_3 = \frac{\ln(\frac{\SI{1,0}{\cancel\Molar}}{\SI{,3}{\cancel\Molar}})}{\SI{87}{\second}} &= \SI{,0138}{\per\second}
				\end{aligned}
				\right.
			$$
			\begin{center}
				Como $k_1 = k_2 = k_3$. Recordad incluir las \textbf{unidades} de la \textbf{constante cinética}.\\
				\tcbhighmath[boxrule=0.4pt,arc=4pt,colframe=red,drop fuzzy shadow=black]{\underline{\textbf{ES DE ORDEN 1}}}\quad
				\tcbhighmath[boxrule=0.4pt,arc=4pt,colframe=red,drop fuzzy shadow=black]{\visible<2->{k = \SI{,0138}{\per\second}}}\quad
				\tcbhighmath[boxrule=0.4pt,arc=4pt,colframe=blue,drop fuzzy shadow=black]{\visible<2->{v=\SI{,0138}{\per\second}[\ce{A}]^1}}
			\end{center}
	\end{overprint}
\end{frame}

\newcommand{\tmed}[1]{\tmedio^{(#1)}}

\begin{frame}
	\frametitle{\ejerciciocmd}
	\framesubtitle{Resolución (\rom{4}): energía de activación (\rom{1})}
	\structure{Ecuación de Arrhenius:} para relacionar $k$ a distintas temperaturas. Hay varias formas, la más práctica es la siguiente:
	$$
		\ln(\frac{k_1}{k_2}) = \frac{E_a}{R}\left(\frac{T_1-T_2}{T_1\vdot T_2}\right)
	$$
	Según el enunciado, la reacción es de orden 2, pero es un dato innecesario. Reemplacemos cada $k$ en el cociente $\rfrac{k_1}{k_2}$ por el $\tmedio$ correspondiente para cada orden.\\
	\begin{center}
		\begin{tabular}{ccc}
			\underline{\textbf{\color{red!50!black}Orden 0:}}	&
			\underline{\textbf{\color{red!50!black}Orden 1:}}	&
			\underline{\textbf{\color{red!50!black}Orden 2:}}	\\
			$
				k = \frac{[\ce{A}]_{\text{REF}}}{2\vdot\tmedio}\Rightarrow\frac{k_1}{k_2}=\frac{
					\frac{\cancel{[\ce{A}]_{\text{REF}}}}{\cancel{2}\vdot\tmed{1}}
				}{
					\frac{\cancel{[\ce{A}]_{\text{REF}}}}{\cancel{2}\vdot\tmed{2}}
				}
%				= \frac{\tmed{2}}{\tmed{1}}
			$													&
			$
				k = \frac{\ln(2)}{\tmedio}\Rightarrow\frac{k_1}{k_2}=\frac{
					\frac{\cancel{\ln(2)}}{\tmed{1}}
				}{
					\frac{\cancel{\ln(2)}}{\tmed{2}}
				}
%				= \frac{\tmed{2}}{\tmed{1}}
			$													&
			$
				k = \frac{1}{[\ce{A}]_{\text{REF}}\vdot\tmedio}\Rightarrow\frac{k_1}{k_2}=\frac{
					\frac{1}{\cancel{[\ce{A}]_{\text{REF}}}\vdot\tmed{1}}
				}{
					\frac{1}{\cancel{[\ce{A}]_{\text{REF}}}\vdot\tmed{2}}
				}
%				= \frac{\tmed{2}}{\tmed{1}}
			$													\\
		\end{tabular}
	\end{center}
	Sin depender del orden, el resultado es siempre:
	$$
		\frac{k_1}{k_2}=\frac{\tmed{2}}{\tmed{1}}
	$$
\end{frame}

\begin{frame}
	\frametitle{\ejerciciocmd}
	\framesubtitle{Resolución (\rom{5}): energía de activación (\rom{2})}
	\structure{Ecuación de Arrhenius:} para relacionar $k$ a distintas temperaturas. Hay varias formas, la más práctica es la siguiente:
	$$
		\ln(\frac{k_1}{k_2}) = \frac{E_a}{R}\vdot\left(\frac{T_1-T_2}{T_1\vdot T_2}\right)\Rightarrow
		\ln(\frac{\tmed{2}}{\tmed{1}}) = \frac{E_a}{R}\vdot\left(\frac{T_1-T_2}{T_1\vdot T_2}\right)\Rightarrow
		E_a = R\vdot\ln(\frac{\tmed{2}}{\tmed{1}})\vdot\left(\frac{T_1\vdot T_2}{T_1-T_2}\right)
	$$
	$$
		\tmed{1} = \SI{70}{\minute} = \SI{4200}{\second}\qquad T_1 = \SI{298,15}{\kelvin}
	$$
	$$
		\tmed{2} = \SI{90}{\minute} = \SI{5400}{\second}\qquad T_1 = \SI{293,15}{\kelvin}
	$$
	Sustituimos por los correspondientes valores:
	$$
		E_a = \SI{8,314}{\joule\per\mol\per\cancel\kelvin}\vdot\ln(\frac{\SI{5400}{\cancel\second}}{\SI{4200}{\cancel\second}})\vdot\left(\frac{\SI{298,15}{\cancel\kelvin}\vdot\SI{293,15}{\cancel\kelvin}}{\SI{298,15}{\cancel\kelvin}-\SI{293,15}{\cancel\kelvin}}\right)
	$$
	$$
		\tcbhighmath[boxrule=0.4pt,arc=4pt,colframe=blue,drop fuzzy shadow=black]{E_a = \SI{36524,32}{\joule\per\mol} = \SI{36,52}{\kilo\joule\per\mol}}
	$$
\end{frame}