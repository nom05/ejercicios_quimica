\begin{enumerate}
	\item Determinar mediante el método del tiempo de vida media y el método de integración el orden de reacción y la constante de velocidad (cinética) de la reacción ``\ce{A -> Productos}'' teniendo en cuenta la siguiente tabla de datos:
	\begin{center}
		\begin{tabular}{l|SSSS}
				$[\ce{A}]~(\unit{\Molar}$)			&	1,0		&	  ,6	&	  ,5	&	  ,3	\\
			\midrule
				$t$ de reacción (\unit{\second})	&	0		&	37		&	50		&	87
		\end{tabular}
	\end{center}
	\item En la reacción ``\ce{A -> P}'' se ha seguido la concentración del reactivo \ce{A} con el tiempo. En todos los experimentos se parte de una concentración inicial de \SI{,1}{\Molar} de \ce{A}. Se observa que realizando la reacción a \SI{25}{\celsius} la concentración inicial se reduce a la mitad cuando han pasado \SI{1}{hora} y \SI{10}{minutos}. Cuando se realiza la misma reacción a \SI{20}{\celsius} esta tarda \SI{90}{minutos} en alcanzar la mitad de la concentración inicial. Calcular la energía de activación de la reacción si esta es de orden 2.
\end{enumerate}
\resultadocmd{
	1, \SI{,0138}{\per\second};
	\SI{36,52}{\kilo\joule\per\mol}
}