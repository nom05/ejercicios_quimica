\begin{frame}
    \frametitle{\ejerciciocmd}
    \framesubtitle{Enunciado}
    \textbf{
	Dadas las siguientes reacciones:
\begin{itemize}
    \item \ce{I2(g) + H2(g) -> 2 HI(g)}~~~$\Delta H_1 = \SI{-0,8}{\kilo\calorie}$
    \item \ce{I2(s) + H2(g) -> 2 HI(g)}~~~$\Delta H_2 = \SI{12}{\kilo\calorie}$
    \item \ce{I2(g) + H2(g) -> 2 HI(ac)}~~~$\Delta H_3 = \SI{-26,8}{\kilo\calorie}$
\end{itemize}
Calcular los parámetros que se indican a continuación:
\begin{description}%[label={\alph*)},font={\color{red!50!black}\bfseries}]
    \item[\texttt{a)}] Calor molar latente de sublimación del yodo.
    \item[\texttt{b)}] Calor molar de disolución del ácido yodhídrico.
    \item[\texttt{c)}] Número de calorías que hay que aportar para disociar en sus componentes el yoduro de hidrógeno gas contenido en un matraz de \SI{750}{\cubic\centi\meter} a \SI{25}{\celsius} y \SI{800}{\torr} de presión.
\end{description}
\resultadocmd{\SI{12,8}{\kilo\calorie}; \SI{-13,0}{\kilo\calorie}; \SI{12,9}{\calorie}}

        }
\end{frame}

\begin{frame}
    \frametitle{\ejerciciocmd}
    \framesubtitle{Datos del enunciado}
    \begin{enumerate}[label={\alph*)},font={\color{red!50!black}\bfseries}]
        \item ¿Ecuación de $v$?
        \item ¿$k$?
        \item ¿$v$ cuando $[\ce{NH3}]=\SI{0,015}{\Molar}$?
    \end{enumerate}
    $$
        \tcbhighmath[boxrule=0.4pt,arc=4pt,colframe=green,drop fuzzy shadow=yellow]{\text{descomposición de \ce{NH3}}}\quad
        \tcbhighmath[boxrule=0.4pt,arc=4pt,colframe=green,drop fuzzy shadow=yellow]{\text{orden cero (0)}}\quad
    $$
    $$
        \tcbhighmath[boxrule=0.4pt,arc=4pt,colframe=green,drop fuzzy shadow=yellow]{T=\SI{1100}{\kelvin}}\quad
        \tcbhighmath[boxrule=0.4pt,arc=4pt,colframe=green,drop fuzzy shadow=yellow]{[\ce{NH3}]=\SI{,04}{\Molar}}\quad
        \tcbhighmath[boxrule=0.4pt,arc=4pt,colframe=green,drop fuzzy shadow=yellow]{v(\ce{NH3})=\SI{2,5e-4}{\Molar\per\minute}}\quad
    $$
\end{frame}

\begin{frame}
    \frametitle{\ejerciciocmd}
    \framesubtitle{Resolución (\rom{1}): ecuación de velocidad}
    \structure{En general:}
    $$
        \ce{aA + bB -> cC + dD}
    $$
    \textbf{\textit{Ecuación de velocidad:}}
    $$
        -\frac{1}{a}\frac{\Delta[\ce{A}]}{\Delta t} = 
        -\frac{1}{b}\frac{\Delta[\ce{B}]}{\Delta t} =
         \frac{1}{c}\frac{\Delta[\ce{C}]}{\Delta t} =
         \frac{1}{d}\frac{\Delta[\ce{D}]}{\Delta t} =
         k[\ce{A}]^x[\ce{B}]^y
    $$
    \visible<2>{
        \structure{Descomposición de \ce{NH3}:}
        $$
            \ce{2NH3 ->[W] N2 + 3H2}
        $$
        \begin{center}
            \alert{Orden 0 (exponente de amoníaco es cero) y $v(\ce{NH3})$ no depende de concentración}
        \end{center}
        $$
            \tcbhighmath[boxrule=0.4pt,arc=4pt,colframe=green,drop fuzzy shadow=yellow]{v(\ce{NH3}) = -\frac{1}{2}\frac{\Delta[\ce{NH3}]}{\Delta t} = k\vdot\cancelto{1}{[\ce{NH3}]^0}}
        $$
                }
\end{frame}

\begin{frame}
    \frametitle{\ejerciciocmd}
    \framesubtitle{Resolución (\rom{2}): valor de la constante de velocidad}
    \structure{Ecuación de velocidad:}
    $$
        v(\ce{NH3}) = k
    $$
    Pero ya sabíamos la velocidad de antemano en un cierto instante $v(\ce{NH3}) = \SI{2,5e-4}{\Molar\per\minute}$, por tanto:
    $$
        \tcbhighmath[boxrule=0.4pt,arc=4pt,colframe=green,drop fuzzy shadow=yellow]{k = \SI{2,5e-4}{\Molar\per\minute}}
    $$
\end{frame}

\begin{frame}
    \frametitle{\ejerciciocmd}
    \framesubtitle{Resolución (\rom{3}): velocidad cuando $[\ce{NH3}]=\SI{0,015}{\Molar}$}
    \structure{Ecuación de velocidad:}
    $$
        v(\ce{NH3}) = \SI{2,5e-4}{\Molar\per\minute}
    $$
    Esta ecuación no depende de la concentración porque es de orden 0, por tanto:
    $$
        \tcbhighmath[boxrule=0.4pt,arc=4pt,colframe=green,drop fuzzy shadow=yellow]{v(\ce{NH3}) = \SI{2,5e-4}{\Molar\per\minute}}
    $$
\end{frame}
