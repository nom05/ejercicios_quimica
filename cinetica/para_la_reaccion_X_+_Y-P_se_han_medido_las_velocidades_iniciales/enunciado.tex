Para la reacción \ce{X + Y -> P} se han medido las velocidades iniciales de la misma para diversas concentraciones de ambos reactivos, obteniéndose la tabla siguiente:
\begin{center}
	\begin{tabular}{SSS}
		\toprule
			{$[\ce{X}]~(\si{\Molar})$} & {$[\ce{Y}]~(\si{\Molar})$} & {$v~(\si{\Molar\per\second})$} \\
		\midrule
			,01  & ,01  & 3,0e-6 \\
			,005 & ,01  & 1,5e-6 \\
			,001 & ,020 & 1,2e-6 \\
			,001 & ,010 & 3,0e-7 \\
		\bottomrule
	\end{tabular}
\end{center}
A partir de estos datos indique cuál es el orden de reacción y estime su constante de velocidad.
\resultadocmd{\num{3};\SI{3,0}{\per\square\Molar\per\second}}
