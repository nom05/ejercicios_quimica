\begin{frame}
	\frametitle{\ejerciciocmd}
	\framesubtitle{Enunciado}
	\textbf{
		Dadas las siguientes reacciones:
\begin{itemize}
    \item \ce{I2(g) + H2(g) -> 2 HI(g)}~~~$\Delta H_1 = \SI{-0,8}{\kilo\calorie}$
    \item \ce{I2(s) + H2(g) -> 2 HI(g)}~~~$\Delta H_2 = \SI{12}{\kilo\calorie}$
    \item \ce{I2(g) + H2(g) -> 2 HI(ac)}~~~$\Delta H_3 = \SI{-26,8}{\kilo\calorie}$
\end{itemize}
Calcular los parámetros que se indican a continuación:
\begin{description}%[label={\alph*)},font={\color{red!50!black}\bfseries}]
    \item[\texttt{a)}] Calor molar latente de sublimación del yodo.
    \item[\texttt{b)}] Calor molar de disolución del ácido yodhídrico.
    \item[\texttt{c)}] Número de calorías que hay que aportar para disociar en sus componentes el yoduro de hidrógeno gas contenido en un matraz de \SI{750}{\cubic\centi\meter} a \SI{25}{\celsius} y \SI{800}{\torr} de presión.
\end{description}
\resultadocmd{\SI{12,8}{\kilo\calorie}; \SI{-13,0}{\kilo\calorie}; \SI{12,9}{\calorie}}

		}
\end{frame}

\begin{frame}
	\frametitle{\ejerciciocmd}
	\framesubtitle{Datos del problema}
	\begin{center}
		{\huge¿orden de reacción ($n$)? ¿constante cinética ($k$)?}\\[.3cm]
		\tcbhighmath[boxrule=0.4pt,arc=4pt,colframe=green,drop fuzzy shadow=yellow]{\ce{X + Y -> P}}\\[.3cm]
		\begin{tabular}{SSS}
			\toprule
				{$[\ce{X}]~(\si{\Molar})$} & {$[\ce{Y}]~(\si{\Molar})$} & {$\overbrace{v_0}^{\text{velocidad inicial}}~(\si{\Molar\per\second})$} \\
			\midrule
				,01  & ,01  & 3,0e-6 \\
				,005 & ,01  & 1,5e-6 \\
				,001 & ,020 & 1,2e-6 \\
				,001 & ,010 & 3,0e-7 \\
			\bottomrule
		\end{tabular}
	\end{center}
\end{frame}

\begin{frame}
	\frametitle{\ejerciciocmd}
	\framesubtitle{Resolución (\rom{1}): determinar orden global de reacción}
	\structure{Método de las velocidades iniciales:}\\
	Partimos de la ecuación de la reacción y escribimos la ecuación cinética genérica:
	$$
		\ce{X + Y -> P}\qquad v=k\vdot[\ce{X}]^a\vdot[\ce{Y}]^b
	$$
	Sustituimos la ecuación genérica por los datos de la tabla en cada caso:
	\begin{center}
		\begin{tabular}{SSScS}
			\toprule
				{$[\ce{X}]~(\si{\Molar})$} & {$[\ce{Y}]~(\si{\Molar})$} & {$v_0~(\si{\Molar\per\second})$} & {Ecuación} & {N"o ecuación}\\
			\midrule
				,010 & ,010 & 3,0e-6 & $\num{3,0e-6}=k\cdot\num{,010}^a\cdot\num{,010}^b$ & 1 \\
				,005 & ,010 & 1,5e-6 & $\num{1,5e-6}=k\cdot\num{,005}^a\cdot\num{,010}^b$ & 2 \\
				,001 & ,020 & 1,2e-6 & $\num{1,2e-6}=k\cdot\num{,001}^a\cdot\num{,020}^b$ & 3 \\
				,001 & ,010 & 3,0e-7 & $\num{3,0e-7}=k\cdot\num{,001}^a\cdot\num{,010}^b$ & 4 \\
			\bottomrule
		\end{tabular}
	\end{center}
	Dividimos una ecuación por otra para determinar el orden de cada reactivo (buscar estrategias que nos faciliten el cálculo como eliminar una de las incógnitas):
	$$
		\frac{\text{Ecuación}~1}{\text{Ecuación}~2}:\quad\frac{\num{3,0}\times\cancel{\num{e-6}}}{\num{1,5}\times\cancel{\num{e-6}}}=\frac{\cancel{k}}{\cancel{k}}\vdot\frac{\num{,01}^a}{\num{,005}^a}\vdot\cancelto{1}{\frac{\num{,01}^b}{\num{,01}^b}}\Rightarrow
		\num{2}=\left(\frac{\num{10}}{\num{5}}\right)^a\Rightarrow
		\num{2}^1=\num{2}^a\Rightarrow\tcbhighmath[boxrule=0.4pt,arc=4pt,colframe=green,drop fuzzy shadow=yellow]{a=1}
	$$
	$$
		\frac{\text{Ecuación}~3}{\text{Ecuación}~4}:\quad\frac{\num{12}\times\cancel{\num{e-7}}}{\num{3,0}\times\cancel{\num{e-7}}}=\frac{\cancel{k}}{\cancel{k}}\cdot\cancelto{1}{\frac{\num{,001}^a}{\num{,001}^a}}\cdot\frac{\num{,02}^b}{\num{,01}^b}\Rightarrow
		\num{4}=\left(\frac{\num{2}}{\num{1}}\right)^b\Rightarrow
		\num{2}^2=\num{2}^b\Rightarrow\tcbhighmath[boxrule=0.4pt,arc=4pt,colframe=yellow,drop fuzzy shadow=green]{b=2}
	$$
	\begin{center}
		\tcbhighmath[boxrule=0.4pt,arc=4pt,colframe=green,drop fuzzy shadow=yellow,colback=red!30]{\text{\textbf{Orden 1 para \ce{X}}}}\quad
		\tcbhighmath[boxrule=0.4pt,arc=4pt,colframe=yellow,drop fuzzy shadow=green,colback=red!30]{\text{\textbf{Orden 2 para \ce{Y}}}}\quad
		\tcbhighmath[boxrule=0.4pt,arc=4pt,colframe=green,drop fuzzy shadow=green,colback=orange!30]{\text{\textbf{\underline{Orden 3 global}}}}
	\end{center}

\end{frame}

\begin{frame}
	\frametitle{\ejerciciocmd}
	\framesubtitle{Resolución (\rom{2}): determinación de la constante cinética}
	Sustituimos los exponentes (órdenes) ya calculados en una de las ecuaciones:
	$$
		\text{Ecuación}~1:\quad\num{3,0e-6}=k\cdot\num{,010}^1\cdot\num{,010}^2\Rightarrow
		\num{3,0e-6}=k\cdot\num{e-2}\cdot\overbrace{(\num{e-2})^2}^{\num{e-4}}
	$$
	$$
		k=\frac{\num{3,0}\times\cancel{\num{e-6}}}{\cancel{\num{e-6}}}=\num{3,0}
	$$
	\structure{Unidades de $k$:} Las unidades dependen del orden global de la reacción.
	$$
		\frac{\si{\Molar}}{\si{\second}}=\frac{1}{\si{\second\square\Molar}}\cdot\si{\cubic\Molar}
	$$
	\begin{center}
		\centering\tcbhighmath[boxrule=0.4pt,arc=4pt,colframe=green,drop fuzzy shadow=green]{k=\SI{3,0}{\per\square\Molar\per\second}}\\[.3cm]
		\centering\tcbhighmath[boxrule=0.4pt,arc=4pt,colframe=red,drop fuzzy shadow=black]{v=\num{3,0}\vdot[\ce{X}]\vdot[\ce{Y}]^2 }\\[.3cm]
	\end{center}

\end{frame}
