\begin{frame}
	\frametitle{\ejerciciocmd}
	\framesubtitle{Enunciado}
	\textbf{
		Dadas las siguientes reacciones:
\begin{itemize}
    \item \ce{I2(g) + H2(g) -> 2 HI(g)}~~~$\Delta H_1 = \SI{-0,8}{\kilo\calorie}$
    \item \ce{I2(s) + H2(g) -> 2 HI(g)}~~~$\Delta H_2 = \SI{12}{\kilo\calorie}$
    \item \ce{I2(g) + H2(g) -> 2 HI(ac)}~~~$\Delta H_3 = \SI{-26,8}{\kilo\calorie}$
\end{itemize}
Calcular los parámetros que se indican a continuación:
\begin{description}%[label={\alph*)},font={\color{red!50!black}\bfseries}]
    \item[\texttt{a)}] Calor molar latente de sublimación del yodo.
    \item[\texttt{b)}] Calor molar de disolución del ácido yodhídrico.
    \item[\texttt{c)}] Número de calorías que hay que aportar para disociar en sus componentes el yoduro de hidrógeno gas contenido en un matraz de \SI{750}{\cubic\centi\meter} a \SI{25}{\celsius} y \SI{800}{\torr} de presión.
\end{description}
\resultadocmd{\SI{12,8}{\kilo\calorie}; \SI{-13,0}{\kilo\calorie}; \SI{12,9}{\calorie}}

		}
\end{frame}

\begin{frame}
	\frametitle{\ejerciciocmd}
	\framesubtitle{Datos del problema}
	\begin{center}
		{\huge¿$E_a$?}\\[.3cm]
		\tcbhighmath[boxrule=0.4pt,arc=4pt,colframe=blue,drop fuzzy shadow=red]{\text{Orden 1 (}n=1\text{)}}\\[.3cm]
		\begin{tabular}{SS}
			\toprule
				{$T~(\si{\kelvin})$} & {$k~(\si{\per\second})$}\\
			\midrule
				278 & 9,065e-3 \\
				288 & 1,470e-2 \\
				298 & 2,306e-2 \\
				308 & 3,515e-2 \\
				318 & 5,217e-2 \\
			\bottomrule
		\end{tabular}
	\end{center}
\end{frame}

\begin{frame}
	\frametitle{\ejerciciocmd}
	\framesubtitle{Resolución (\rom{1}): determinación de la energía de activación}
	\structure{Partimos de la ecuación de Arrhenius:}
	$$
		\ln(k)=\ln(A) -\frac{E_a}{R\vdot T}
	$$
	Por analogía a la ecuación de una recta ($y = mx + n$) si representamos $\ln k$ frente a $\rfrac{1}{T}$ deberíamos poder usar el método de regresión lineal para obtener la ecuación de una recta.
	\structure{Expandimos tabla con las columnas que necesitamos para la representación gráfica:}
	\begin{center}
		\begin{tabular}{SSSS}
			\toprule
				{$T~(\si{\kelvin})$} & {$k~(\si{\per\second})$} & {$\frac{1}{T}~(\si{\per\kelvin})$} & {$\ln(k)$}\\
			\midrule
				278 & 9,065e-3 & 3,597e-3 & -4,7033 \\
				288 & 1,470e-2 & 3,472e-3 & -4,2199 \\
				298 & 2,306e-2 & 3,356e-3 & -3,7697 \\
				308 & 3,515e-2 & 3,247e-3 & -3,3481 \\
				318 & 5,217e-2 & 3,145e-3 & -2,9532 \\
			\bottomrule
		\end{tabular}
	\end{center}
\end{frame}

\begin{frame}
	\frametitle{\ejerciciocmd}
	\framesubtitle{Resolución (\rom{2}): determinación de la energía de activación}
	\begin{center}
		\begin{figure}
			\caption{Representación del logaritmo neperiano de $k$ ($\ln(k)$) con respecto a la inversa de la temperatura ($T^{-1}$).}
			% GNUPLOT: LaTeX picture
\setlength{\unitlength}{0.240900pt}
\ifx\plotpoint\undefined\newsavebox{\plotpoint}\fi
\sbox{\plotpoint}{\rule[-0.200pt]{0.400pt}{0.400pt}}%
\begin{picture}(1299,826)(0,0)
\sbox{\plotpoint}{\rule[-0.200pt]{0.400pt}{0.400pt}}%
\put(231.0,131.0){\rule[-0.200pt]{4.818pt}{0.400pt}}
\put(211,131){\makebox(0,0)[r]{$-4.8000$}}
\put(1218.0,131.0){\rule[-0.200pt]{4.818pt}{0.400pt}}
\put(231.0,196.0){\rule[-0.200pt]{4.818pt}{0.400pt}}
\put(211,196){\makebox(0,0)[r]{$-4.6000$}}
\put(1218.0,196.0){\rule[-0.200pt]{4.818pt}{0.400pt}}
\put(231.0,262.0){\rule[-0.200pt]{4.818pt}{0.400pt}}
\put(211,262){\makebox(0,0)[r]{$-4.4000$}}
\put(1218.0,262.0){\rule[-0.200pt]{4.818pt}{0.400pt}}
\put(231.0,327.0){\rule[-0.200pt]{4.818pt}{0.400pt}}
\put(211,327){\makebox(0,0)[r]{$-4.2000$}}
\put(1218.0,327.0){\rule[-0.200pt]{4.818pt}{0.400pt}}
\put(231.0,392.0){\rule[-0.200pt]{4.818pt}{0.400pt}}
\put(211,392){\makebox(0,0)[r]{$-4.0000$}}
\put(1218.0,392.0){\rule[-0.200pt]{4.818pt}{0.400pt}}
\put(231.0,458.0){\rule[-0.200pt]{4.818pt}{0.400pt}}
\put(211,458){\makebox(0,0)[r]{$-3.8000$}}
\put(1218.0,458.0){\rule[-0.200pt]{4.818pt}{0.400pt}}
\put(231.0,523.0){\rule[-0.200pt]{4.818pt}{0.400pt}}
\put(211,523){\makebox(0,0)[r]{$-3.6000$}}
\put(1218.0,523.0){\rule[-0.200pt]{4.818pt}{0.400pt}}
\put(231.0,588.0){\rule[-0.200pt]{4.818pt}{0.400pt}}
\put(211,588){\makebox(0,0)[r]{$-3.4000$}}
\put(1218.0,588.0){\rule[-0.200pt]{4.818pt}{0.400pt}}
\put(231.0,653.0){\rule[-0.200pt]{4.818pt}{0.400pt}}
\put(211,653){\makebox(0,0)[r]{$-3.2000$}}
\put(1218.0,653.0){\rule[-0.200pt]{4.818pt}{0.400pt}}
\put(231.0,719.0){\rule[-0.200pt]{4.818pt}{0.400pt}}
\put(211,719){\makebox(0,0)[r]{$-3.0000$}}
\put(1218.0,719.0){\rule[-0.200pt]{4.818pt}{0.400pt}}
\put(231.0,784.0){\rule[-0.200pt]{4.818pt}{0.400pt}}
\put(211,784){\makebox(0,0)[r]{$-2.8000$}}
\put(1218.0,784.0){\rule[-0.200pt]{4.818pt}{0.400pt}}
\put(231.0,131.0){\rule[-0.200pt]{0.400pt}{4.818pt}}
\put(231,90){\makebox(0,0){$3.10$}}
\put(231.0,764.0){\rule[-0.200pt]{0.400pt}{4.818pt}}
\put(332.0,131.0){\rule[-0.200pt]{0.400pt}{4.818pt}}
\put(332,90){\makebox(0,0){$3.15$}}
\put(332.0,764.0){\rule[-0.200pt]{0.400pt}{4.818pt}}
\put(432.0,131.0){\rule[-0.200pt]{0.400pt}{4.818pt}}
\put(432,90){\makebox(0,0){$3.20$}}
\put(432.0,764.0){\rule[-0.200pt]{0.400pt}{4.818pt}}
\put(533.0,131.0){\rule[-0.200pt]{0.400pt}{4.818pt}}
\put(533,90){\makebox(0,0){$3.25$}}
\put(533.0,764.0){\rule[-0.200pt]{0.400pt}{4.818pt}}
\put(634.0,131.0){\rule[-0.200pt]{0.400pt}{4.818pt}}
\put(634,90){\makebox(0,0){$3.30$}}
\put(634.0,764.0){\rule[-0.200pt]{0.400pt}{4.818pt}}
\put(734.0,131.0){\rule[-0.200pt]{0.400pt}{4.818pt}}
\put(734,90){\makebox(0,0){$3.35$}}
\put(734.0,764.0){\rule[-0.200pt]{0.400pt}{4.818pt}}
\put(835.0,131.0){\rule[-0.200pt]{0.400pt}{4.818pt}}
\put(835,90){\makebox(0,0){$3.40$}}
\put(835.0,764.0){\rule[-0.200pt]{0.400pt}{4.818pt}}
\put(936.0,131.0){\rule[-0.200pt]{0.400pt}{4.818pt}}
\put(936,90){\makebox(0,0){$3.45$}}
\put(936.0,764.0){\rule[-0.200pt]{0.400pt}{4.818pt}}
\put(1037.0,131.0){\rule[-0.200pt]{0.400pt}{4.818pt}}
\put(1037,90){\makebox(0,0){$3.50$}}
\put(1037.0,764.0){\rule[-0.200pt]{0.400pt}{4.818pt}}
\put(1137.0,131.0){\rule[-0.200pt]{0.400pt}{4.818pt}}
\put(1137,90){\makebox(0,0){$3.55$}}
\put(1137.0,764.0){\rule[-0.200pt]{0.400pt}{4.818pt}}
\put(1238.0,131.0){\rule[-0.200pt]{0.400pt}{4.818pt}}
\put(1238,90){\makebox(0,0){$3.60$}}
\put(1238.0,764.0){\rule[-0.200pt]{0.400pt}{4.818pt}}
\put(231.0,131.0){\rule[-0.200pt]{0.400pt}{157.308pt}}
\put(231.0,131.0){\rule[-0.200pt]{242.586pt}{0.400pt}}
\put(1238.0,131.0){\rule[-0.200pt]{0.400pt}{157.308pt}}
\put(231.0,784.0){\rule[-0.200pt]{242.586pt}{0.400pt}}
\put(36,457){\makebox(0,0){$\ln(k)$\quad}}
\put(734,29){\makebox(0,0){$T^{-1}\times\num{e3}~(\si{\per\kelvin})$}}
\put(1232,163){\makebox(0,0){$\blacksquare$}}
\put(980,320){\makebox(0,0){$\blacksquare$}}
\put(747,467){\makebox(0,0){$\blacksquare$}}
\put(527,605){\makebox(0,0){$\blacksquare$}}
\put(322,734){\makebox(0,0){$\blacksquare$}}
\put(1078,743){\makebox(0,0)[r]{$ln(k)$ = 9.23 + (-3872.46)$T^{-1}$ ($r^2$ = 1.0000)}}
\put(1098.0,743.0){\rule[-0.200pt]{24.090pt}{0.400pt}}
\put(322,734){\usebox{\plotpoint}}
\multiput(322.00,732.93)(0.762,-0.482){9}{\rule{0.700pt}{0.116pt}}
\multiput(322.00,733.17)(7.547,-6.000){2}{\rule{0.350pt}{0.400pt}}
\multiput(331.00,726.93)(0.762,-0.482){9}{\rule{0.700pt}{0.116pt}}
\multiput(331.00,727.17)(7.547,-6.000){2}{\rule{0.350pt}{0.400pt}}
\multiput(340.00,720.93)(0.933,-0.477){7}{\rule{0.820pt}{0.115pt}}
\multiput(340.00,721.17)(7.298,-5.000){2}{\rule{0.410pt}{0.400pt}}
\multiput(349.00,715.93)(0.762,-0.482){9}{\rule{0.700pt}{0.116pt}}
\multiput(349.00,716.17)(7.547,-6.000){2}{\rule{0.350pt}{0.400pt}}
\multiput(358.00,709.93)(0.852,-0.482){9}{\rule{0.767pt}{0.116pt}}
\multiput(358.00,710.17)(8.409,-6.000){2}{\rule{0.383pt}{0.400pt}}
\multiput(368.00,703.93)(0.762,-0.482){9}{\rule{0.700pt}{0.116pt}}
\multiput(368.00,704.17)(7.547,-6.000){2}{\rule{0.350pt}{0.400pt}}
\multiput(377.00,697.93)(0.933,-0.477){7}{\rule{0.820pt}{0.115pt}}
\multiput(377.00,698.17)(7.298,-5.000){2}{\rule{0.410pt}{0.400pt}}
\multiput(386.00,692.93)(0.762,-0.482){9}{\rule{0.700pt}{0.116pt}}
\multiput(386.00,693.17)(7.547,-6.000){2}{\rule{0.350pt}{0.400pt}}
\multiput(395.00,686.93)(0.762,-0.482){9}{\rule{0.700pt}{0.116pt}}
\multiput(395.00,687.17)(7.547,-6.000){2}{\rule{0.350pt}{0.400pt}}
\multiput(404.00,680.93)(0.852,-0.482){9}{\rule{0.767pt}{0.116pt}}
\multiput(404.00,681.17)(8.409,-6.000){2}{\rule{0.383pt}{0.400pt}}
\multiput(414.00,674.93)(0.933,-0.477){7}{\rule{0.820pt}{0.115pt}}
\multiput(414.00,675.17)(7.298,-5.000){2}{\rule{0.410pt}{0.400pt}}
\multiput(423.00,669.93)(0.762,-0.482){9}{\rule{0.700pt}{0.116pt}}
\multiput(423.00,670.17)(7.547,-6.000){2}{\rule{0.350pt}{0.400pt}}
\multiput(432.00,663.93)(0.762,-0.482){9}{\rule{0.700pt}{0.116pt}}
\multiput(432.00,664.17)(7.547,-6.000){2}{\rule{0.350pt}{0.400pt}}
\multiput(441.00,657.93)(0.762,-0.482){9}{\rule{0.700pt}{0.116pt}}
\multiput(441.00,658.17)(7.547,-6.000){2}{\rule{0.350pt}{0.400pt}}
\multiput(450.00,651.93)(0.852,-0.482){9}{\rule{0.767pt}{0.116pt}}
\multiput(450.00,652.17)(8.409,-6.000){2}{\rule{0.383pt}{0.400pt}}
\multiput(460.00,645.93)(0.933,-0.477){7}{\rule{0.820pt}{0.115pt}}
\multiput(460.00,646.17)(7.298,-5.000){2}{\rule{0.410pt}{0.400pt}}
\multiput(469.00,640.93)(0.762,-0.482){9}{\rule{0.700pt}{0.116pt}}
\multiput(469.00,641.17)(7.547,-6.000){2}{\rule{0.350pt}{0.400pt}}
\multiput(478.00,634.93)(0.762,-0.482){9}{\rule{0.700pt}{0.116pt}}
\multiput(478.00,635.17)(7.547,-6.000){2}{\rule{0.350pt}{0.400pt}}
\multiput(487.00,628.93)(0.762,-0.482){9}{\rule{0.700pt}{0.116pt}}
\multiput(487.00,629.17)(7.547,-6.000){2}{\rule{0.350pt}{0.400pt}}
\multiput(496.00,622.93)(1.044,-0.477){7}{\rule{0.900pt}{0.115pt}}
\multiput(496.00,623.17)(8.132,-5.000){2}{\rule{0.450pt}{0.400pt}}
\multiput(506.00,617.93)(0.762,-0.482){9}{\rule{0.700pt}{0.116pt}}
\multiput(506.00,618.17)(7.547,-6.000){2}{\rule{0.350pt}{0.400pt}}
\multiput(515.00,611.93)(0.762,-0.482){9}{\rule{0.700pt}{0.116pt}}
\multiput(515.00,612.17)(7.547,-6.000){2}{\rule{0.350pt}{0.400pt}}
\multiput(524.00,605.93)(0.762,-0.482){9}{\rule{0.700pt}{0.116pt}}
\multiput(524.00,606.17)(7.547,-6.000){2}{\rule{0.350pt}{0.400pt}}
\multiput(533.00,599.93)(0.762,-0.482){9}{\rule{0.700pt}{0.116pt}}
\multiput(533.00,600.17)(7.547,-6.000){2}{\rule{0.350pt}{0.400pt}}
\multiput(542.00,593.93)(1.044,-0.477){7}{\rule{0.900pt}{0.115pt}}
\multiput(542.00,594.17)(8.132,-5.000){2}{\rule{0.450pt}{0.400pt}}
\multiput(552.00,588.93)(0.762,-0.482){9}{\rule{0.700pt}{0.116pt}}
\multiput(552.00,589.17)(7.547,-6.000){2}{\rule{0.350pt}{0.400pt}}
\multiput(561.00,582.93)(0.762,-0.482){9}{\rule{0.700pt}{0.116pt}}
\multiput(561.00,583.17)(7.547,-6.000){2}{\rule{0.350pt}{0.400pt}}
\multiput(570.00,576.93)(0.762,-0.482){9}{\rule{0.700pt}{0.116pt}}
\multiput(570.00,577.17)(7.547,-6.000){2}{\rule{0.350pt}{0.400pt}}
\multiput(579.00,570.93)(0.933,-0.477){7}{\rule{0.820pt}{0.115pt}}
\multiput(579.00,571.17)(7.298,-5.000){2}{\rule{0.410pt}{0.400pt}}
\multiput(588.00,565.93)(0.762,-0.482){9}{\rule{0.700pt}{0.116pt}}
\multiput(588.00,566.17)(7.547,-6.000){2}{\rule{0.350pt}{0.400pt}}
\multiput(597.00,559.93)(0.852,-0.482){9}{\rule{0.767pt}{0.116pt}}
\multiput(597.00,560.17)(8.409,-6.000){2}{\rule{0.383pt}{0.400pt}}
\multiput(607.00,553.93)(0.762,-0.482){9}{\rule{0.700pt}{0.116pt}}
\multiput(607.00,554.17)(7.547,-6.000){2}{\rule{0.350pt}{0.400pt}}
\multiput(616.00,547.93)(0.933,-0.477){7}{\rule{0.820pt}{0.115pt}}
\multiput(616.00,548.17)(7.298,-5.000){2}{\rule{0.410pt}{0.400pt}}
\multiput(625.00,542.93)(0.762,-0.482){9}{\rule{0.700pt}{0.116pt}}
\multiput(625.00,543.17)(7.547,-6.000){2}{\rule{0.350pt}{0.400pt}}
\multiput(634.00,536.93)(0.762,-0.482){9}{\rule{0.700pt}{0.116pt}}
\multiput(634.00,537.17)(7.547,-6.000){2}{\rule{0.350pt}{0.400pt}}
\multiput(643.00,530.93)(0.852,-0.482){9}{\rule{0.767pt}{0.116pt}}
\multiput(643.00,531.17)(8.409,-6.000){2}{\rule{0.383pt}{0.400pt}}
\multiput(653.00,524.93)(0.762,-0.482){9}{\rule{0.700pt}{0.116pt}}
\multiput(653.00,525.17)(7.547,-6.000){2}{\rule{0.350pt}{0.400pt}}
\multiput(662.00,518.93)(0.933,-0.477){7}{\rule{0.820pt}{0.115pt}}
\multiput(662.00,519.17)(7.298,-5.000){2}{\rule{0.410pt}{0.400pt}}
\multiput(671.00,513.93)(0.762,-0.482){9}{\rule{0.700pt}{0.116pt}}
\multiput(671.00,514.17)(7.547,-6.000){2}{\rule{0.350pt}{0.400pt}}
\multiput(680.00,507.93)(0.762,-0.482){9}{\rule{0.700pt}{0.116pt}}
\multiput(680.00,508.17)(7.547,-6.000){2}{\rule{0.350pt}{0.400pt}}
\multiput(689.00,501.93)(0.852,-0.482){9}{\rule{0.767pt}{0.116pt}}
\multiput(689.00,502.17)(8.409,-6.000){2}{\rule{0.383pt}{0.400pt}}
\multiput(699.00,495.93)(0.933,-0.477){7}{\rule{0.820pt}{0.115pt}}
\multiput(699.00,496.17)(7.298,-5.000){2}{\rule{0.410pt}{0.400pt}}
\multiput(708.00,490.93)(0.762,-0.482){9}{\rule{0.700pt}{0.116pt}}
\multiput(708.00,491.17)(7.547,-6.000){2}{\rule{0.350pt}{0.400pt}}
\multiput(717.00,484.93)(0.762,-0.482){9}{\rule{0.700pt}{0.116pt}}
\multiput(717.00,485.17)(7.547,-6.000){2}{\rule{0.350pt}{0.400pt}}
\multiput(726.00,478.93)(0.762,-0.482){9}{\rule{0.700pt}{0.116pt}}
\multiput(726.00,479.17)(7.547,-6.000){2}{\rule{0.350pt}{0.400pt}}
\multiput(735.00,472.93)(0.852,-0.482){9}{\rule{0.767pt}{0.116pt}}
\multiput(735.00,473.17)(8.409,-6.000){2}{\rule{0.383pt}{0.400pt}}
\multiput(745.00,466.93)(0.933,-0.477){7}{\rule{0.820pt}{0.115pt}}
\multiput(745.00,467.17)(7.298,-5.000){2}{\rule{0.410pt}{0.400pt}}
\multiput(754.00,461.93)(0.762,-0.482){9}{\rule{0.700pt}{0.116pt}}
\multiput(754.00,462.17)(7.547,-6.000){2}{\rule{0.350pt}{0.400pt}}
\multiput(763.00,455.93)(0.762,-0.482){9}{\rule{0.700pt}{0.116pt}}
\multiput(763.00,456.17)(7.547,-6.000){2}{\rule{0.350pt}{0.400pt}}
\multiput(772.00,449.93)(0.762,-0.482){9}{\rule{0.700pt}{0.116pt}}
\multiput(772.00,450.17)(7.547,-6.000){2}{\rule{0.350pt}{0.400pt}}
\multiput(781.00,443.93)(1.044,-0.477){7}{\rule{0.900pt}{0.115pt}}
\multiput(781.00,444.17)(8.132,-5.000){2}{\rule{0.450pt}{0.400pt}}
\multiput(791.00,438.93)(0.762,-0.482){9}{\rule{0.700pt}{0.116pt}}
\multiput(791.00,439.17)(7.547,-6.000){2}{\rule{0.350pt}{0.400pt}}
\multiput(800.00,432.93)(0.762,-0.482){9}{\rule{0.700pt}{0.116pt}}
\multiput(800.00,433.17)(7.547,-6.000){2}{\rule{0.350pt}{0.400pt}}
\multiput(809.00,426.93)(0.762,-0.482){9}{\rule{0.700pt}{0.116pt}}
\multiput(809.00,427.17)(7.547,-6.000){2}{\rule{0.350pt}{0.400pt}}
\multiput(818.00,420.93)(0.933,-0.477){7}{\rule{0.820pt}{0.115pt}}
\multiput(818.00,421.17)(7.298,-5.000){2}{\rule{0.410pt}{0.400pt}}
\multiput(827.00,415.93)(0.852,-0.482){9}{\rule{0.767pt}{0.116pt}}
\multiput(827.00,416.17)(8.409,-6.000){2}{\rule{0.383pt}{0.400pt}}
\multiput(837.00,409.93)(0.762,-0.482){9}{\rule{0.700pt}{0.116pt}}
\multiput(837.00,410.17)(7.547,-6.000){2}{\rule{0.350pt}{0.400pt}}
\multiput(846.00,403.93)(0.762,-0.482){9}{\rule{0.700pt}{0.116pt}}
\multiput(846.00,404.17)(7.547,-6.000){2}{\rule{0.350pt}{0.400pt}}
\multiput(855.00,397.93)(0.762,-0.482){9}{\rule{0.700pt}{0.116pt}}
\multiput(855.00,398.17)(7.547,-6.000){2}{\rule{0.350pt}{0.400pt}}
\multiput(864.00,391.93)(0.933,-0.477){7}{\rule{0.820pt}{0.115pt}}
\multiput(864.00,392.17)(7.298,-5.000){2}{\rule{0.410pt}{0.400pt}}
\multiput(873.00,386.93)(0.852,-0.482){9}{\rule{0.767pt}{0.116pt}}
\multiput(873.00,387.17)(8.409,-6.000){2}{\rule{0.383pt}{0.400pt}}
\multiput(883.00,380.93)(0.762,-0.482){9}{\rule{0.700pt}{0.116pt}}
\multiput(883.00,381.17)(7.547,-6.000){2}{\rule{0.350pt}{0.400pt}}
\multiput(892.00,374.93)(0.762,-0.482){9}{\rule{0.700pt}{0.116pt}}
\multiput(892.00,375.17)(7.547,-6.000){2}{\rule{0.350pt}{0.400pt}}
\multiput(901.00,368.93)(0.933,-0.477){7}{\rule{0.820pt}{0.115pt}}
\multiput(901.00,369.17)(7.298,-5.000){2}{\rule{0.410pt}{0.400pt}}
\multiput(910.00,363.93)(0.762,-0.482){9}{\rule{0.700pt}{0.116pt}}
\multiput(910.00,364.17)(7.547,-6.000){2}{\rule{0.350pt}{0.400pt}}
\multiput(919.00,357.93)(0.852,-0.482){9}{\rule{0.767pt}{0.116pt}}
\multiput(919.00,358.17)(8.409,-6.000){2}{\rule{0.383pt}{0.400pt}}
\multiput(929.00,351.93)(0.762,-0.482){9}{\rule{0.700pt}{0.116pt}}
\multiput(929.00,352.17)(7.547,-6.000){2}{\rule{0.350pt}{0.400pt}}
\multiput(938.00,345.93)(0.762,-0.482){9}{\rule{0.700pt}{0.116pt}}
\multiput(938.00,346.17)(7.547,-6.000){2}{\rule{0.350pt}{0.400pt}}
\multiput(947.00,339.93)(0.933,-0.477){7}{\rule{0.820pt}{0.115pt}}
\multiput(947.00,340.17)(7.298,-5.000){2}{\rule{0.410pt}{0.400pt}}
\multiput(956.00,334.93)(0.762,-0.482){9}{\rule{0.700pt}{0.116pt}}
\multiput(956.00,335.17)(7.547,-6.000){2}{\rule{0.350pt}{0.400pt}}
\multiput(965.00,328.93)(0.762,-0.482){9}{\rule{0.700pt}{0.116pt}}
\multiput(965.00,329.17)(7.547,-6.000){2}{\rule{0.350pt}{0.400pt}}
\multiput(974.00,322.93)(0.852,-0.482){9}{\rule{0.767pt}{0.116pt}}
\multiput(974.00,323.17)(8.409,-6.000){2}{\rule{0.383pt}{0.400pt}}
\multiput(984.00,316.93)(0.933,-0.477){7}{\rule{0.820pt}{0.115pt}}
\multiput(984.00,317.17)(7.298,-5.000){2}{\rule{0.410pt}{0.400pt}}
\multiput(993.00,311.93)(0.762,-0.482){9}{\rule{0.700pt}{0.116pt}}
\multiput(993.00,312.17)(7.547,-6.000){2}{\rule{0.350pt}{0.400pt}}
\multiput(1002.00,305.93)(0.762,-0.482){9}{\rule{0.700pt}{0.116pt}}
\multiput(1002.00,306.17)(7.547,-6.000){2}{\rule{0.350pt}{0.400pt}}
\multiput(1011.00,299.93)(0.762,-0.482){9}{\rule{0.700pt}{0.116pt}}
\multiput(1011.00,300.17)(7.547,-6.000){2}{\rule{0.350pt}{0.400pt}}
\multiput(1020.00,293.93)(1.044,-0.477){7}{\rule{0.900pt}{0.115pt}}
\multiput(1020.00,294.17)(8.132,-5.000){2}{\rule{0.450pt}{0.400pt}}
\multiput(1030.00,288.93)(0.762,-0.482){9}{\rule{0.700pt}{0.116pt}}
\multiput(1030.00,289.17)(7.547,-6.000){2}{\rule{0.350pt}{0.400pt}}
\multiput(1039.00,282.93)(0.762,-0.482){9}{\rule{0.700pt}{0.116pt}}
\multiput(1039.00,283.17)(7.547,-6.000){2}{\rule{0.350pt}{0.400pt}}
\multiput(1048.00,276.93)(0.762,-0.482){9}{\rule{0.700pt}{0.116pt}}
\multiput(1048.00,277.17)(7.547,-6.000){2}{\rule{0.350pt}{0.400pt}}
\multiput(1057.00,270.93)(0.762,-0.482){9}{\rule{0.700pt}{0.116pt}}
\multiput(1057.00,271.17)(7.547,-6.000){2}{\rule{0.350pt}{0.400pt}}
\multiput(1066.00,264.93)(1.044,-0.477){7}{\rule{0.900pt}{0.115pt}}
\multiput(1066.00,265.17)(8.132,-5.000){2}{\rule{0.450pt}{0.400pt}}
\multiput(1076.00,259.93)(0.762,-0.482){9}{\rule{0.700pt}{0.116pt}}
\multiput(1076.00,260.17)(7.547,-6.000){2}{\rule{0.350pt}{0.400pt}}
\multiput(1085.00,253.93)(0.762,-0.482){9}{\rule{0.700pt}{0.116pt}}
\multiput(1085.00,254.17)(7.547,-6.000){2}{\rule{0.350pt}{0.400pt}}
\multiput(1094.00,247.93)(0.762,-0.482){9}{\rule{0.700pt}{0.116pt}}
\multiput(1094.00,248.17)(7.547,-6.000){2}{\rule{0.350pt}{0.400pt}}
\multiput(1103.00,241.93)(0.933,-0.477){7}{\rule{0.820pt}{0.115pt}}
\multiput(1103.00,242.17)(7.298,-5.000){2}{\rule{0.410pt}{0.400pt}}
\multiput(1112.00,236.93)(0.852,-0.482){9}{\rule{0.767pt}{0.116pt}}
\multiput(1112.00,237.17)(8.409,-6.000){2}{\rule{0.383pt}{0.400pt}}
\multiput(1122.00,230.93)(0.762,-0.482){9}{\rule{0.700pt}{0.116pt}}
\multiput(1122.00,231.17)(7.547,-6.000){2}{\rule{0.350pt}{0.400pt}}
\multiput(1131.00,224.93)(0.762,-0.482){9}{\rule{0.700pt}{0.116pt}}
\multiput(1131.00,225.17)(7.547,-6.000){2}{\rule{0.350pt}{0.400pt}}
\multiput(1140.00,218.93)(0.762,-0.482){9}{\rule{0.700pt}{0.116pt}}
\multiput(1140.00,219.17)(7.547,-6.000){2}{\rule{0.350pt}{0.400pt}}
\multiput(1149.00,212.93)(0.933,-0.477){7}{\rule{0.820pt}{0.115pt}}
\multiput(1149.00,213.17)(7.298,-5.000){2}{\rule{0.410pt}{0.400pt}}
\multiput(1158.00,207.93)(0.852,-0.482){9}{\rule{0.767pt}{0.116pt}}
\multiput(1158.00,208.17)(8.409,-6.000){2}{\rule{0.383pt}{0.400pt}}
\multiput(1168.00,201.93)(0.762,-0.482){9}{\rule{0.700pt}{0.116pt}}
\multiput(1168.00,202.17)(7.547,-6.000){2}{\rule{0.350pt}{0.400pt}}
\multiput(1177.00,195.93)(0.762,-0.482){9}{\rule{0.700pt}{0.116pt}}
\multiput(1177.00,196.17)(7.547,-6.000){2}{\rule{0.350pt}{0.400pt}}
\multiput(1186.00,189.93)(0.933,-0.477){7}{\rule{0.820pt}{0.115pt}}
\multiput(1186.00,190.17)(7.298,-5.000){2}{\rule{0.410pt}{0.400pt}}
\multiput(1195.00,184.93)(0.762,-0.482){9}{\rule{0.700pt}{0.116pt}}
\multiput(1195.00,185.17)(7.547,-6.000){2}{\rule{0.350pt}{0.400pt}}
\multiput(1204.00,178.93)(0.852,-0.482){9}{\rule{0.767pt}{0.116pt}}
\multiput(1204.00,179.17)(8.409,-6.000){2}{\rule{0.383pt}{0.400pt}}
\multiput(1214.00,172.93)(0.762,-0.482){9}{\rule{0.700pt}{0.116pt}}
\multiput(1214.00,173.17)(7.547,-6.000){2}{\rule{0.350pt}{0.400pt}}
\multiput(1223.00,166.93)(0.933,-0.477){7}{\rule{0.820pt}{0.115pt}}
\multiput(1223.00,167.17)(7.298,-5.000){2}{\rule{0.410pt}{0.400pt}}
\put(231.0,131.0){\rule[-0.200pt]{0.400pt}{157.308pt}}
\put(231.0,131.0){\rule[-0.200pt]{242.586pt}{0.400pt}}
\put(1238.0,131.0){\rule[-0.200pt]{0.400pt}{157.308pt}}
\put(231.0,784.0){\rule[-0.200pt]{242.586pt}{0.400pt}}
\end{picture}

		\end{figure}
	\end{center}
\end{frame}

\begin{frame}
	\frametitle{\ejerciciocmd}
	\framesubtitle{Resolución (\rom{3}): determinación de la energía de activación}
	\structure{A partir de la pendiente determinamos la energía de activación:}
	$$
		\underbrace{-\frac{E_a}{R}}_{\SI{8,314}{\joule\per\mol\per\kelvin}}=\SI{-3872,46}{\kelvin}
	$$
	$$
		E_a=\SI{3872,46}{\cancel\kelvin}\vdot\SI{8,314}{\joule\per\mol\per\cancel\kelvin}
	$$
	\begin{center}
		\tcbhighmath[boxrule=0.4pt,arc=4pt,colframe=blue,drop fuzzy shadow=red]{E_a=\SI{32195,63}{\joule\per\mol}=\SI{32,20}{\kilo\joule\per\mol}}
	\end{center}
\end{frame}
