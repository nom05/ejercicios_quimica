\begin{frame}
    \frametitle{\ejerciciocmd}
    \framesubtitle{Enunciado}
    \textbf{
	Dadas las siguientes reacciones:
\begin{itemize}
    \item \ce{I2(g) + H2(g) -> 2 HI(g)}~~~$\Delta H_1 = \SI{-0,8}{\kilo\calorie}$
    \item \ce{I2(s) + H2(g) -> 2 HI(g)}~~~$\Delta H_2 = \SI{12}{\kilo\calorie}$
    \item \ce{I2(g) + H2(g) -> 2 HI(ac)}~~~$\Delta H_3 = \SI{-26,8}{\kilo\calorie}$
\end{itemize}
Calcular los parámetros que se indican a continuación:
\begin{description}%[label={\alph*)},font={\color{red!50!black}\bfseries}]
    \item[\texttt{a)}] Calor molar latente de sublimación del yodo.
    \item[\texttt{b)}] Calor molar de disolución del ácido yodhídrico.
    \item[\texttt{c)}] Número de calorías que hay que aportar para disociar en sus componentes el yoduro de hidrógeno gas contenido en un matraz de \SI{750}{\cubic\centi\meter} a \SI{25}{\celsius} y \SI{800}{\torr} de presión.
\end{description}
\resultadocmd{\SI{12,8}{\kilo\calorie}; \SI{-13,0}{\kilo\calorie}; \SI{12,9}{\calorie}}

	}
\end{frame}

\begin{frame}
    \frametitle{\ejerciciocmd}
    \framesubtitle{Resolución del problema}
    \structure{Reacción:}
    $$
        \ce{2NO(g) + 2H2(g) -> N2(g) + 2H2O(g)}
    $$
    \begin{overprint}
        \onslide<1>
            \structure{Ecuación de velocidad sin determinar los coeficientes y $k$:}
            $$
                v = k\cdot[\ce{NO}]^x\cdot[\ce{H2}]^y
            $$
        \onslide<2->
            \structure{Usando la ecuación de los gases ideales:} $PV=nRT\Rightarrow \frac{n}{V} = \frac{P}{RT}\Rightarrow M=\rfrac{P}{RT}$
            $$
                v = k\cdot\left(\frac{P(\ce{NO})}{RT}\right)^x\left(\frac{P(\ce{H2})}{RT}\right)^y
            $$
    \end{overprint}
    \structure{Método de velocidades iniciales (se omite $RT$ para que las expresiones se entiendan más fácilmente):}
     \visible<3->{
         \begin{equation}\label{eq:4exp1}
             \SI{3,0e-5}{} = k\cdot(\SI{0,451}{})^x\cdot(\SI{,2255}{})^y
         \end{equation}
         \begin{equation}\label{eq:4exp2}
             \SI{9,0e-5}{} = k\cdot(\SI{1,353}{})^x\cdot(\SI{,2255}{})^y
         \end{equation}
         \begin{equation}\label{eq:4exp3}
             \SI{36e-5}{} = k\cdot(\SI{1,353}{})^x\cdot(\SI{,9020}{})^y
         \end{equation}
                  }
    \visible<4-10>{
        \begin{overprint}
            \onslide<4>
                \structure{Dividiendo la ecuación~\eqref{eq:4exp2} por ecuación~\eqref{eq:4exp1}:}
                $$
                    \frac{\SI{9}{}}{\SI{3}{}} = \frac{(\SI{1353}{})^x}{(\SI{451}{})^x}\cdot\frac{(\SI{2255}{})^y}{(\SI{2255}{})^y}               
                $$
            \onslide<5>
                \structure{Dividiendo la ecuación~\eqref{eq:4exp2} por ecuación~\eqref{eq:4exp1}:}
                $$
                    3 = 3^x\Rightarrow x=1
                $$
            \onslide<6>
                \structure{Dividiendo la ecuación~\eqref{eq:4exp3} por ecuación~\eqref{eq:4exp2}:}
                $$
                    \frac{\SI{36}{}}{\SI{9}{}} = \left(\frac{\SI{1353}{}}{\SI{1353}{}}\right)^1\cdot\left(\frac{\SI{902}{}}{\SI{225,5}{}}\right)^y
                $$
            \onslide<7>
                \structure{Dividiendo la ecuación~\eqref{eq:4exp3} por ecuación~\eqref{eq:4exp2}:}
                $$
                    4 = 4^y\Rightarrow y=1
                $$
            \onslide<8>
                \structure{Una vez calculados los exponentes, determinamos $k$ con una de las ecuaciones (p.ej. ecuación~\eqref{eq:4exp1}):}
                $$
                    \SI{3,0e-5}{} = k\cdot\left(\frac{\SI{,451}{}}{R\cdot\SI{110}{\kelvin}}\right)\cdot\left(\frac{\SI{,2255}{}}{R\cdot\SI{110}{\kelvin}}\right)
                $$
            \onslide<9>
                \structure{Una vez calculados los exponentes, determinamos $k$ con una de las ecuaciones (p.ej. ecuación~\eqref{eq:4exp1}):}
                $$
                    k = \frac{\SI{3,0e-5}{}\cdot R^2\cdot\SI{110}{}^2}{\SI{,451}{}\cdot\SI{,2255}{}} = \SI{,024}{\second\per\Molar}
                $$
            \onslide<10>
                $$
                    \tcbhighmath[boxrule=0.4pt,arc=4pt,colframe=green,drop fuzzy shadow=yellow]{v = \SI{,024}{\per\Molar\per\second}\cdot\left(\frac{P(\ce{NO})}{R\cdot\SI{110}{}}\right)\cdot\left(\frac{P(\ce{H2})}{R\cdot\SI{110}{}}\right)}
                $$
        \end{overprint}
                }
\end{frame}
