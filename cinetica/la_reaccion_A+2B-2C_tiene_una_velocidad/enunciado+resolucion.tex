\begin{frame}
	\frametitle{\ejerciciocmd}
	\framesubtitle{Enunciado}
	\textbf{
		Una reacción tiene una constante de velocidad de \SI{,017}{\per\second} a \SI{298}{\kelvin} y una energía libre de activación del \SI{27,235}{\kilo\joule\per\mol}. La adición de un catalizador disminuye dicha energía de activación hasta un \SI{33}{\percent} de su valor inicial. Calcule la nueva constante de velocidad.
\resultadocmd{ \SI{26,86}{\per\second} }

		}
\end{frame}

\begin{frame}
	\frametitle{\ejerciciocmd}
	\framesubtitle{Datos del problema}
	\begin{center}
		\textbf{\Large \begin{enumerate}[label={\alph*)},font={\color{red!50!black}\bfseries}]
			\item ¿$v(\ce{C})$?
			\item ¿$[\ce{A}]$ si \SI{1,00}{\minute}?
			\item ¿$t$ si $[\ce{A}]$ pasa de \num{,3580} a \SI{,3500}{\Molar}?
		\end{enumerate}}
		\structure{Reacción:}
		\ce{A + 2B -> 2C}\\[.3cm]
		\tcbhighmath[boxrule=0.4pt,arc=4pt,colframe=green,drop fuzzy shadow=blue]{v=\SI{1,76e-5}{\Molar\per\second}}\quad
		\tcbhighmath[boxrule=0.4pt,arc=4pt,colframe=green,drop fuzzy shadow=blue]{[\ce{A}]=\SI{,3580}{\Molar}}\\[.3cm]
	\end{center}
\end{frame}

\begin{frame}
	\frametitle{\ejerciciocmd}
	\framesubtitle{Resolución (\rom{1}): determinación de la velocidad de formación de \ce{C}}
	\structure{En general:}
	$$
		\ce{aA + bB -> cC + dD}
	$$
	\textbf{\textit{Ecuación de velocidad general:}}
	$$
		v = -\frac{1}{a}\dv{[\ce{A}]}{t} =
		    -\frac{1}{b}\dv{[\ce{B}]}{t} =
		     \frac{1}{c}\dv{[\ce{C}]}{t} =
		     \frac{1}{d}\dv{[\ce{D}]}{t}
		  =  k[\ce{A}]^x[\ce{B}]^y
	$$
	\visible<2->{
		\structure{En nuestro caso:}
		$$
			\ce{A + 2B -> 2C}
		$$
		\textbf{\textit{Ecuación de velocidad:}}
		$$
		   v = -\frac{1}{1}\dv{[\ce{A}]}{t} = 
			   -\frac{1}{2}\dv{[\ce{B}]}{t} =
			    \frac{1}{2}\dv{[\ce{C}]}{t}
			 =  k\vdot[\ce{A}]^x\vdot[\ce{B}]^y
		$$
		Como se ve en la expresión, la velocidad ($v$) en un instante es independiente de la estequiometría de reactivos y productos. El signo negativo y dividir por el coeficiente estequiométrico hace que esté ``normalizada'' a todos los compuestos de la reacción. Por tanto:
		\begin{center}
			\tcbhighmath[boxrule=0.4pt,arc=4pt,colframe=red,drop fuzzy shadow=green]{v=\SI{1,76e-5}{\Molar\per\second}}
		\end{center}
	
				}
\end{frame}

\begin{frame}
	\frametitle{\ejerciciocmd}
	\framesubtitle{Resolución (\rom{2}): determinación de la concentración de \ce{A} un minuto más tarde}
	Si $v=\text{cte}$ (como es en este apartado) podemos suponer que la expresión infinitesimal se puede aproximar de forma finita en función de variaciones o incrementos:
	$$
		v = -\dv{[\ce{A}]}{t} \approx
		 	-\frac{\Delta[\ce{A}]}{\Delta t}
		  \Rightarrow
			-\frac{\Delta[\ce{A}]}{\Delta t} = -\frac{[\ce{A}]-[\ce{A}]_0}{t-t_0}=v
	$$
	\structure{Datos:}
	$t_0 =  \SI{0}{\second}\Rightarrow[\ce{A}]_0=\SI{,3580}{\Molar}$\quad\quad\quad
	$t   = \SI{60}{\second}\Rightarrow$¿$[\ce{A}]$?
	\begin{overprint}
		\onslide<1>
			$$
				-\frac{[\ce{A}]-\overbrace{[\ce{A}]_0}^{\SI{,3580}{\Molar}}}{\underbrace{t}_{\SI{60}{\second}}-\underbrace{t_0}_{\SI{0}{\second}}}=\underbrace{v}_{\SI{1,76e-5}{\Molar\per\second}}
			$$
		\onslide<2>
			$$
				-\frac{[\ce{A}]-\SI{,3580}{\Molar}}{\SI{60}{\second}}=\SI{1,76e-5}{\Molar\per\second}
			$$
		\onslide<3>
			$$
				-([\ce{A}]-\SI{,3580}{\Molar})=\SI{1,76e-5}{\Molar\per\cancel\second}\vdot\SI{60}{\cancel\second}
			$$
		\onslide<4->
			$$
				[\ce{A}]=\SI{-1,76e-5}{\Molar\per\cancel\second}\vdot\SI{60}{\cancel\second}+\SI{,3580}{\Molar}
			$$
	\end{overprint}
	\visible<4->{
		\begin{center}
			\tcbhighmath[boxrule=0.4pt,arc=4pt,colframe=green,drop fuzzy shadow=blue]{[\ce{A}]=\SI{,3569}{\Molar}}
		\end{center}
				}
\end{frame}

\begin{frame}
	\frametitle{\ejerciciocmd}
	\framesubtitle{Resolución (\rom{3}): determinación del tiempo transcurrido si [\ce{A}] es \SI{,3500}{\Molar}}
	Como $v=\text{cte}$:
	$$
	v = -\frac{[\ce{A}]-[\ce{A}]_0}{t-t_0}=v
	$$
	\structure{Datos:}
	$t_0 =  \SI{0}{\second}\Rightarrow[\ce{A}]_0=\SI{,3580}{\Molar}$\quad\quad\quad
	¿$t$?				  $\Rightarrow$$[\ce{A}]=\SI{,3500}{\Molar}$
	\begin{overprint}
		\onslide<1>
		$$
			-\frac{\overbrace{[\ce{A}]}^{\SI{,3500}{\Molar}}-\overbrace{[\ce{A}]_0}^{\SI{,3580}{\Molar}}}{t-\underbrace{t_0}_{\SI{0}{\second}}}=\underbrace{v}_{\SI{1,76e-5}{\Molar\per\second}}
		$$
		\onslide<2>
		$$
			-\frac{\SI{,3500}{\Molar}-\SI{,3580}{\Molar}}{t}=\SI{1,76e-5}{\Molar\per\second}
		$$
		\onslide<3->
		$$
			t=\frac{\SI{,3580}{\cancel\Molar}-\SI{,3500}{\cancel\Molar}}{\SI{1,76e-5}{\cancel\Molar\per\second}}
		$$
	\end{overprint}
	\visible<3->{
		\begin{center}
			\tcbhighmath[boxrule=0.4pt,arc=4pt,colframe=orange,drop fuzzy shadow=blue]{t=\SI{454,54}{\second}}\quad o\quad
			\tcbhighmath[boxrule=0.4pt,arc=4pt,colframe=orange,drop fuzzy shadow=blue]{t=\SI{7}{\minute}~\SI{34,5}{\second}}
		\end{center}
	}
\end{frame}
