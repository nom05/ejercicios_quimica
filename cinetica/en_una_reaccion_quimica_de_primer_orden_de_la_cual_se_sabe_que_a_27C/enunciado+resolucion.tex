\begin{frame}
	\frametitle{\ejerciciocmd}
	\framesubtitle{Enunciado}
	\textbf{
		Dadas las siguientes reacciones:
\begin{itemize}
    \item \ce{I2(g) + H2(g) -> 2 HI(g)}~~~$\Delta H_1 = \SI{-0,8}{\kilo\calorie}$
    \item \ce{I2(s) + H2(g) -> 2 HI(g)}~~~$\Delta H_2 = \SI{12}{\kilo\calorie}$
    \item \ce{I2(g) + H2(g) -> 2 HI(ac)}~~~$\Delta H_3 = \SI{-26,8}{\kilo\calorie}$
\end{itemize}
Calcular los parámetros que se indican a continuación:
\begin{description}%[label={\alph*)},font={\color{red!50!black}\bfseries}]
    \item[\texttt{a)}] Calor molar latente de sublimación del yodo.
    \item[\texttt{b)}] Calor molar de disolución del ácido yodhídrico.
    \item[\texttt{c)}] Número de calorías que hay que aportar para disociar en sus componentes el yoduro de hidrógeno gas contenido en un matraz de \SI{750}{\cubic\centi\meter} a \SI{25}{\celsius} y \SI{800}{\torr} de presión.
\end{description}
\resultadocmd{\SI{12,8}{\kilo\calorie}; \SI{-13,0}{\kilo\calorie}; \SI{12,9}{\calorie}}

		}
\end{frame}

\begin{frame}
	\frametitle{\ejerciciocmd}
	\framesubtitle{Datos del problema}
	\begin{center}
		{\huge¿$E_a$ y $k(\SI{47}{\celsius})$?}\\[.3cm]
		\tcbhighmath[boxrule=0.4pt,arc=4pt,colframe=green,drop fuzzy shadow=yellow]{\text{Orden 1 (}n=1\text{)}}\quad
		\tcbhighmath[boxrule=0.4pt,arc=4pt,colframe=green,drop fuzzy shadow=yellow]{t_{\rfrac{1}{2}}(\SI{27}{\celsius})=\SI{5000}{\second}}\quad
		\tcbhighmath[boxrule=0.4pt,arc=4pt,colframe=blue,drop fuzzy shadow=green]{t_{\rfrac{1}{2}}(\SI{37}{\celsius})=\SI{1000}{\second}}
	\end{center}
\end{frame}

\begin{frame}
	\frametitle{\ejerciciocmd}
	\framesubtitle{Resolución (\rom{1}): determinación de la energía de activación y $k$ a \SI{47}{\celsius}}
	\structure{Usando la ecuación del tiempo de vida media:}
	$$
		t_{\rfrac{1}{2}}=\frac{\ln(2)}{k}\Rightarrow k=\frac{\ln(2)}{t_{\rfrac{1}{2}}}
	$$
	\structure{y empleando la ecuación de Arrhenius:}
	\begin{overprint}
		\onslide<1>
			$$
				k=A\vdot\exp(-\frac{E_a}{RT})
			$$
		\onslide<2>
			$$
				\frac{k_2}{k_1}=\frac{\cancel{A}}{\cancel{A}}\vdot\frac{\exp(-\frac{E_a}{RT_2})}{\exp(-\frac{E_a}{RT_1})}
			$$
		\onslide<3>
			$$
				\ln(\frac{\overbrace{k_2}^{k_2=\frac{\ln(2)}{t^{(2)}_{\rfrac{1}{2}}}}}{\underbrace{k_1}_{k_1=\frac{\ln(2)}{t^{(1)}_{\rfrac{1}{2}}}}})=\frac{E_a}{R}\left(\frac{1}{T_1}-\frac{1}{T_2}\right)
			$$
		\onslide<4>
			$$
				\ln(\frac{\frac{\cancel{\ln(2)}}{t^{(2)}_{\rfrac{1}{2}}}}{\frac{\cancel{\ln(2)}}{t^{(1)}_{\rfrac{1}{2}}}})=\frac{E_a}{R}\left(\frac{1}{T_1}-\frac{1}{T_2}\right)
			$$
		\onslide<5>
			$$
				\ln(\frac{t^{(1)}_{\rfrac{1}{2}}}{t^{(2)}_{\rfrac{1}{2}}})=\frac{E_a}{R}\left(\frac{1}{T_1}-\frac{1}{T_2}\right)
			$$
		\onslide<6->
			$$
				\ln(\frac{\SI{5000}{\cancel\second}}{\SI{1000}{\cancel\second}})=\frac{E_a}{\SI{8,314}{\joule\per\mol\per\cancel\kelvin}}\left(\frac{1}{\SI{300,15}{\cancel\kelvin}}-\frac{1}{\SI{310,15}{\cancel\kelvin}}\right)
			$$
	\end{overprint}
	\visible<6->{
		\centering\tcbhighmath[boxrule=0.4pt,arc=4pt,colframe=green,drop fuzzy shadow=yellow]{E_a=\SI{124564,52}{\joule\per\mol}=\SI{124,56}{\kilo\joule\per\mol}}\\[.3cm]
		\structure{Cálculo de $k_1$ (por ejemplo $k(\SI{27}{\celsius})$):}\quad$k_1=\frac{\ln(2)}{t^{(1)}_{\rfrac{1}{2}}}\Rightarrow k_1=\SI{1,386e-4}{\per\second}$
		\structure{Para $k$ a \SI{47}{\celsius} volvemos a usar la ecuación de Arrhenius en las nuevas condiciones:}
		\begin{overprint}
			\onslide<6>
				$$
					\ln(\frac{k_2}{\SI{1,386e-4}{\per\second}})=\frac{\SI{124564,52}{\cancel\joule\per\cancel\mol}}{\SI{8,314}{\cancel\joule\per\cancel\mol\per\cancel\kelvin}}\left(\frac{1}{\SI{300,15}{\cancel\kelvin}}-\frac{1}{\SI{320,15}{\cancel\kelvin}}\right)
				$$
			\onslide<7>
				\centering\tcbhighmath[boxrule=0.4pt,arc=4pt,colframe=red,drop fuzzy shadow=green]{k(\SI{47}{\celsius})=\SI{3,134e-3}{\per\second}}
		\end{overprint}
				}
\end{frame}
