En una reacción química de primer orden de la cual se sabe que a \SI{27}{\celsius} la concentración inicial de reactivo se reduce a la mitad después de \SI{5000}{\second} y que a \SI{37}{\celsius} la concentración se reduce a la mitad después de \SI{1000}{\second}, calcular:
\begin{enumerate}[label={\alph*)},font={\color{red!50!black}\bfseries}]
	\item La energía de activación de la reacción.
	\item El valor de la constante de velocidad ($k$) a \SI{47}{\celsius}.
\end{enumerate}
\resultadocmd{
		\SI{124,56}{\kilo\joule\per\mol};
		\SI{3,134e-3}{\per\second}
}
