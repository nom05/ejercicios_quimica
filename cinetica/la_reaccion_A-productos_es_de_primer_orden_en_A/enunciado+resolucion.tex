\begin{frame}
	\frametitle{\ejerciciocmd}
	\framesubtitle{Enunciado}
	\textbf{
		Dadas las siguientes reacciones:
\begin{itemize}
    \item \ce{I2(g) + H2(g) -> 2 HI(g)}~~~$\Delta H_1 = \SI{-0,8}{\kilo\calorie}$
    \item \ce{I2(s) + H2(g) -> 2 HI(g)}~~~$\Delta H_2 = \SI{12}{\kilo\calorie}$
    \item \ce{I2(g) + H2(g) -> 2 HI(ac)}~~~$\Delta H_3 = \SI{-26,8}{\kilo\calorie}$
\end{itemize}
Calcular los parámetros que se indican a continuación:
\begin{description}%[label={\alph*)},font={\color{red!50!black}\bfseries}]
    \item[\texttt{a)}] Calor molar latente de sublimación del yodo.
    \item[\texttt{b)}] Calor molar de disolución del ácido yodhídrico.
    \item[\texttt{c)}] Número de calorías que hay que aportar para disociar en sus componentes el yoduro de hidrógeno gas contenido en un matraz de \SI{750}{\cubic\centi\meter} a \SI{25}{\celsius} y \SI{800}{\torr} de presión.
\end{description}
\resultadocmd{\SI{12,8}{\kilo\calorie}; \SI{-13,0}{\kilo\calorie}; \SI{12,9}{\calorie}}

		}
\end{frame}

\begin{frame}
	\frametitle{\ejerciciocmd}
	\framesubtitle{Datos del problema}
	\begin{center}
		{\Large ¿$t_{\rfrac{1}{2}}$? y ¿$m(\ce{A})$ después de \SI{1}{\hour}?}\\[.3cm]
		\tcbhighmath[boxrule=0.4pt,arc=4pt,colframe=green,drop fuzzy shadow=blue]{\text{Orden 1}}\quad
		\tcbhighmath[boxrule=0.4pt,arc=4pt,colframe=green,drop fuzzy shadow=blue]{m_0(\ce{A})=\SI{1,60}{\gram}}\quad
		\tcbhighmath[boxrule=0.4pt,arc=4pt,colframe=green,drop fuzzy shadow=blue]{t=\SI{38}{\minute}}\quad
		\tcbhighmath[boxrule=0.4pt,arc=4pt,colframe=green,drop fuzzy shadow=blue]{m(\ce{A})=\SI{,40}{\gram}}
	\end{center}
\end{frame}

\begin{frame}
	\frametitle{\ejerciciocmd}
	\framesubtitle{Resolución (\rom{1}): determinación del tiempo de vida media}
	\structure{Si la reacción es de orden 1 con respecto a \ce{A} implica:} {\tiny (revisar ejercicios anteriores para observar cómo se obtienen las expresiones)}
	$$
		v=k\vdot[\ce{A}];\quad \ln(\frac{[\ce{A}]}{[\ce{A}]_0})=-k\vdot t;\quad k=\frac{\ln(2)}{t_{\rfrac{1}{2}}}
	$$
	\visible<2->{
		\begin{overprint}
			\onslide<2>
				\structure{Despejamos y sustituimos $k$ en la ecuación integrada:}
				$$
					\ln(\frac{[\ce{A}]}{[\ce{A}]_0})=-\overbrace{k}^{k=\frac{\ln(2)}{t_{\rfrac{1}{2}}}}\vdot t
				$$
			\onslide<3>
				\structure{Usamos la definición de molaridad:} $M=\frac{n}{V}$
				$$
					\ln\frac{\overbrace{[\ce{A}]}^{[\ce{A}]=\frac{n(\ce{A})}{V}}}{\underbrace{[\ce{A}]_0}_{[\ce{A}]_0=\frac{n_0(\ce{A})}{V}}}=-\frac{\ln(2)}{t_{\rfrac{1}{2}}}\vdot t
				$$
			\onslide<4>
				\structure{Usamos la definición de molaridad:} $M=\frac{n}{V}$
				$$
					\ln(\frac{\frac{n(\ce{A})}{\cancel{V}}}{\frac{n_0(\ce{A})}{\cancel{V}}})=-\frac{\ln(2)}{t_{\rfrac{1}{2}}}\vdot t
				$$
			\onslide<5>
				\structure{Empleamos la relación entre el número de moles $n$ y la masa $m$:} $n=\frac{m}{Mm}$
				$$
					\ln\frac{\overbrace{n(\ce{A})}^{n(\ce{A})=\frac{m(\ce{A})}{Mm(\ce{A})}}}{\underbrace{n_0(\ce{A})}_{n_0(\ce{A})=\frac{m_0(\ce{A})}{Mm(\ce{A})}}}=-\frac{\ln(2)}{t_{\rfrac{1}{2}}}\vdot t
				$$
			\onslide<6>
				\structure{Empleamos la relación entre el número de moles $n$ y la masa $m$:} $n=\frac{m}{Mm}$
				$$
					\ln(\frac{\frac{m(\ce{A})}{\cancel{Mm(\ce{A})}}}{\frac{m_0(\ce{A})}{\cancel{Mm(\ce{A})}}})=-\frac{\ln(2)}{t_{\rfrac{1}{2}}}\vdot t
				$$
			\onslide<7>
				\structure{Despejamos $t_{\rfrac{1}{2}}$:}
				$$
					\ln(\frac{m_0(\ce{A})}{m(\ce{A})})=\frac{\ln(2)}{t_{\rfrac{1}{2}}}\vdot t
				$$
			\onslide<8->
				\structure{Despejamos $t_{\rfrac{1}{2}}$:}
				$$
					t_{\rfrac{1}{2}}=\frac{\ln(2)}{\ln(\frac{m_0(\ce{A})}{m(\ce{A})})}\vdot t
				$$
		\end{overprint}
				}
		\visible<8->{
			\structure{Sustituimos por los datos del problema:}
			$$
				t_{\rfrac{1}{2}}=\frac{\ln(2)}{\ln(\frac{\SI{1,60}{\cancel\gram}}{\SI{,40}{\cancel\gram}})}\vdot\SI{38}{\minute}
			$$
			$$
				\tcbhighmath[boxrule=0.4pt,arc=4pt,colframe=green,drop fuzzy shadow=blue]{t_{\rfrac{1}{2}}=\SI{19}{\minute}}
			$$
					}
\end{frame}

\begin{frame}
	\frametitle{\ejerciciocmd}
	\framesubtitle{Resolución (\rom{2}): determinación de la masa de \ce{A} después de \SI{1}{\hour} de reacción}
	\structure{Con el $t_{\rfrac{1}{2}}=\SI{19}{\minute}$ y la expresión del anterior apartado solo hay que sustituir y operar:}
	\begin{overprint}
		\onslide<1>
			$$
				\ln(\frac{m_0(\ce{A})}{m(\ce{A})})=\frac{\ln(2)}{t_{\rfrac{1}{2}}}\vdot t\Rightarrow
				\ln(\frac{\SI{1,60}{\gram}}{m(\ce{A})})=\frac{\ln(2)}{\SI{19}{\cancel\minute}}\vdot\SI{60}{\cancel\minute}
			$$
		\onslide<2->
			$$
				\ln(\frac{\SI{1,60}{\gram}}{m(\ce{A})})=\ln(2^{\frac{\num{19}}{\num{60}}})\Rightarrow
				\frac{\SI{1,60}{\gram}}{m(\ce{A})}=2^{\frac{\num{19}}{\num{60}}}\Rightarrow
				m(\ce{A})=\frac{\SI{1,60}{\gram}}{2^{\frac{\num{19}}{\num{60}}}}
			$$
	\end{overprint}
	\visible<2->{
		$$
			\tcbhighmath[boxrule=0.4pt,arc=4pt,colframe=green,drop fuzzy shadow=blue]{m(\ce{A})=\SI{,18}{\gram}}
		$$
				}
\end{frame}
