\begin{frame}
    \frametitle{\ejerciciocmd}
    \framesubtitle{Enunciado}
    \textbf{
		Una reacción tiene una constante de velocidad de \SI{,017}{\per\second} a \SI{298}{\kelvin} y una energía libre de activación del \SI{27,235}{\kilo\joule\per\mol}. La adición de un catalizador disminuye dicha energía de activación hasta un \SI{33}{\percent} de su valor inicial. Calcule la nueva constante de velocidad.
\resultadocmd{ \SI{26,86}{\per\second} }

    	}
\end{frame}

\begin{frame}
    \frametitle{\ejerciciocmd}
    \framesubtitle{Datos del problema}
    \begin{center}
        {\Large ¿$v$ y $k$?}
    \end{center}
    $$
        \ce{A + 2B -> P}
    $$
    \begin{center}
        \begin{tabular}{lSSS}
            $v~\text{inicial}~\si{\Molar\per\second}$ &
            5,7e-7 &
            11,4e-7 &
            22,8e-7\\
            $[\ce{A}]_0$ &
            2,0e-3 &
            2,0e-3 &
            4,0e-3 \\
            $[\ce{B}]_0$ &
            4,0e-3 &
            8,0e-3 &
            4,0e-3
        \end{tabular}
    \end{center}
\end{frame}

\begin{frame}
    \frametitle{\ejerciciocmd}
    \framesubtitle{Resolución (\rom{1}): determinación de ecuación de velocidad y $k$}
    \structure{Método de velocidades iniciales:} A partir de ecuación de velocidad, construimos sistemas de ecuaciones.
    \visible<2->{
        \structure{Ecuación general:} $v = k\cdot[\ce{A}]^x\cdot[\ce{B}]^y$        
                }
    \visible<3->{
         \structure{Con los datos de la tabla buscar ecuaciones que solo dejen una de las incógnitas por calcular:}
         \begin{equation}\label{eq:exp1}
             \SI{5,7e-7}{} = k\cdot(\SI{2,0e-3}{})^x\cdot(\SI{4,0e-3}{})^y
         \end{equation}
         \begin{equation}\label{eq:exp2}
             \SI{11,4e-7}{} = k\cdot(\SI{2,0e-3}{})^x\cdot(\SI{8,0e-3}{})^y
         \end{equation}
         \begin{equation}\label{eq:exp3}
             \SI{22,8e-7}{} = k\cdot(\SI{4,0e-3}{})^x\cdot(\SI{4,0e-3}{})^y
         \end{equation}
                }
    \visible<4-17>{
        \begin{overprint}
            \onslide<4>
                \structure{Dividiendo la ecuación~\eqref{eq:exp2} por ecuación~\eqref{eq:exp1}:}
                $$
                    \frac{\SI{11,4e-7}{}}{\SI{5,7e-7}{}} = \frac{k}{k}\cdot\frac{(\SI{2,0e-3}{})^x}{(\SI{2,0e-3}{})^x}\cdot\frac{(\SI{8,0e-3}{})^y}{(\SI{4,0e-3}{})^y}               
                $$
            \onslide<5>
                $$
                    \overbrace{\frac{\SI{11,4}{}}{\SI{5,7}{}}}^2 =
                         \cancelto{1}{\frac{k}{k}}
                         \cdot\left(\overbrace{\frac{\SI{2,0}{}}{\SI{2,0}{}}}^1\right)^x
                         \cdot\left(\overbrace{\frac{\SI{8,0}{}}{\SI{4,0}{}}}^2\right)^y
                $$
            \onslide<6>
                $$
                    2 = 1^x\cdot2^y
                $$
            \onslide<7>
                $$
                    2^1 = 2^y
                $$
            \onslide<8>
                $$
                    y=1
                $$
            \onslide<9>
                \structure{Dividiendo la ecuación~\eqref{eq:exp3} por ecuación~\eqref{eq:exp2}:}
                $$
                    \frac{\SI{22,8e-7}{}}{\SI{11,4e-7}{}} = \frac{k}{k}\cdot\frac{(\SI{4,0e-3}{})^x}{(\SI{2,0e-3}{})^x}\cdot\frac{(\SI{4,0e-3}{})^y}{(\SI{8,0e-3}{})^y}               
                $$
            \onslide<10>
                $$
                    \frac{\SI{22,8}{}}{\SI{11,4}{}} = \left(\frac{\SI{4,0}{}}{\SI{2,0}{}}\right)^x\cdot\left(\frac{\SI{4,0}{}}{\SI{8,0}{}}\right)^y               
                $$
            \onslide<11>
                $$
                    2 = 2^x\cdot\left(\frac{1}{2}\right)^1               
                $$
            \onslide<12>
                $$
                    2 = 2^x\cdot\frac{1}{2}               
                $$
            \onslide<13>
                $$
                    2\cdot 2 = 2^x
                $$
            \onslide<14>
                $$
                    x=2
                $$
            \onslide<15->
                \structure{Ecuación de velocidad con coeficientes:}
                $$
                    v = k\cdot[\ce{A}]^2\cdot[\ce{B}]
                $$
        \end{overprint}
                }
    \visible<16->{
        \structure{Despejamos $k$ y sustituimos por uno de los experimentos:}
        \begin{overprint}
            \onslide<16>
                $$
                    k = \frac{v}{[\ce{A}]^2\cdot[\ce{B}]}
                $$
            \onslide<17>
                $$
                    k = \frac{\SI{5,7e-7}{}}{(\SI{2,0e-3}{})^2\cdot\SI{4,0e-3}{}} = \SI{35,6}{}
                $$
            \onslide<18->
                $$
                    \tcbhighmath[boxrule=0.4pt,arc=4pt,colframe=green,drop fuzzy shadow=yellow]{v = \SI{35,6}{}\cdot[\ce{A}]^2\cdot[\ce{B}]}
                $$
        \end{overprint}
                }
\end{frame}

\begin{frame}
    \frametitle{\ejerciciocmd}
    \framesubtitle{Resolución (\rom{2}): $v$ y $k$ si el volumen se duplica}
    \structure{Ecuación de Arrhenius:}
    $$
        k = A\cdot\exp^{\rfrac{-E_a}{RT}}
    $$
    \begin{center}
        {\large \textbf{\textit{Como veis, $k$ no depende del volumen}}}
    \end{center}
    \visible<2->{
        \structure{Si los reactivos son gases o líquidos:} $V=2\cdot V_0$ ($V_0$ es el volumen inicial)
        \begin{overprint}
            \onslide<2>
                $$
                    v = \SI{35,6}{}\cdot[\ce{A}]^2\cdot[\ce{B}]
                $$
            \onslide<3>
                $$
                    v = \SI{35,6}{}\cdot\left(\frac{n(\ce{A})}{V}\right)^2\cdot\frac{n(\ce{B})}{V}
                $$
            \onslide<4>
                $$
                    v = \SI{35,6}{}\cdot\left(\frac{n(\ce{A})}{2\cdot V_0}\right)^2\cdot\frac{n(\ce{B})}{2\cdot V_0}
                $$
            \onslide<5>
                $$
                    v = \SI{35,6}{}\cdot\frac{1}{2^2}\left(\frac{n(\ce{A})}{V_0}\right)^2\cdot\frac{1}{2}\frac{n(\ce{B})}{V_0}
                $$
            \onslide<6>
                $$
                    v = \frac{1}{8}\cdot\SI{35,6}{}\cdot\left(\frac{n(\ce{A})}{V_0}\right)^2\cdot\frac{n(\ce{B})}{V_0}
                $$
            \onslide<7->
                $$
                    \tcbhighmath[boxrule=0.4pt,arc=4pt,colframe=green,drop fuzzy shadow=yellow]{v = \frac{1}{8}\cdot v_0}
                $$
        \end{overprint}
                }
    \visible<7->{
        \\$v_0$ es la velocidad inicial antes de cambiar el volumen.
                }
\end{frame}
