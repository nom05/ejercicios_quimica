\begin{frame}
    \frametitle{\ejerciciocmd}
    \framesubtitle{Enunciado}
    \textbf{
		Una reacción tiene una constante de velocidad de \SI{,017}{\per\second} a \SI{298}{\kelvin} y una energía libre de activación del \SI{27,235}{\kilo\joule\per\mol}. La adición de un catalizador disminuye dicha energía de activación hasta un \SI{33}{\percent} de su valor inicial. Calcule la nueva constante de velocidad.
\resultadocmd{ \SI{26,86}{\per\second} }

	}
\end{frame}

\begin{frame}
    \frametitle{\ejerciciocmd}
    \framesubtitle{Datos del enunciado}
    \begin{center}
        {\huge \textbf{¿$\frac{[\ce{HCOO-}]}{[\ce{HCOOH}]}$?}}
    \end{center}
    $$
        \tcbhighmath[boxrule=0.4pt,arc=4pt,colframe=red,drop fuzzy shadow=blue]{pH = \SI{3}{}}\quad
        \tcbhighmath[boxrule=0.4pt,arc=4pt,colframe=red,drop fuzzy shadow=blue]{K_a(\ce{HCOOH}) = \SI{2,1e-4}{}}
    $$
\end{frame}

\begin{frame}
    \frametitle{\ejerciciocmd}
    \framesubtitle{Resolución (\rom{1}): Determinación de $\frac{[\ce{HCOO-}]}{[\ce{HCOOH}]}$}
    \structure{La reacción de disociación será:}
    $$
        \ce{HCOOH(ac) <=> H+(ac) + HCOO-(ac)}\quad\quad K_a(\ce{HCOOH}) = \frac{[\ce{HCOO-}][\ce{H+}]}{[\ce{HCOOH}]}
    $$
    \visible<2->{
        \structure{Despejando el cociente:}
        $$
            \frac{[\ce{HCOO-}]}{[\ce{HCOOH}]} = \frac{K_a(\ce{HCOOH})}{[\ce{H+}]}
        $$
                }
    \visible<3->{
        \structure{Aplicando logaritmos y despejando:}
        \begin{overprint}
            \onslide<3>
                $$
                    \log\left(\frac{[\ce{HCOO-}]}{[\ce{HCOOH}]}\right) = \overbrace{\log\left(\frac{K_a(\ce{HCOOH})}{[\ce{H+}]}\right)}^{\log{K_a(\ce{HCOOH})}-\log{[\ce{H+}]}}
                $$
            \onslide<4>
                $$
                    \log\left(\frac{[\ce{HCOO-}]}{[\ce{HCOOH}]}\right) = -pK_a(\ce{HCOOH})+pH
                $$
            \onslide<5>
                $$
                    \frac{[\ce{HCOO-}]}{[\ce{HCOOH}]} = 10^{pH-pK_a(\ce{HCOOH})}
                $$
            \onslide<6>
                $$
                    \frac{[\ce{HCOO-}]}{[\ce{HCOOH}]} = 10^{\SI{3}{}-\SI{3,38}{}}
                $$
            \onslide<7->
                $$
                    \tcbhighmath[boxrule=0.4pt,arc=4pt,colframe=red,drop fuzzy shadow=blue]{\frac{[\ce{HCOO-}]}{[\ce{HCOOH}]} = \SI{,21}{}}
                $$\\[.3cm]
        \end{overprint}
                }
    \visible<7>{
        \textbf{Toda mezcla que siga este cociente. Por ejemplo: $[\ce{HCOO-}] = \SI{1}{\Molar}$ y $[\ce{HCOOH}] = \SI{4,76}{\Molar}$.}
                }
\end{frame}
