\begin{frame}
    \frametitle{\ejerciciocmd}
    \framesubtitle{Enunciado}
    \textbf{
		Dadas las siguientes reacciones:
\begin{itemize}
    \item \ce{I2(g) + H2(g) -> 2 HI(g)}~~~$\Delta H_1 = \SI{-0,8}{\kilo\calorie}$
    \item \ce{I2(s) + H2(g) -> 2 HI(g)}~~~$\Delta H_2 = \SI{12}{\kilo\calorie}$
    \item \ce{I2(g) + H2(g) -> 2 HI(ac)}~~~$\Delta H_3 = \SI{-26,8}{\kilo\calorie}$
\end{itemize}
Calcular los parámetros que se indican a continuación:
\begin{description}%[label={\alph*)},font={\color{red!50!black}\bfseries}]
    \item[\texttt{a)}] Calor molar latente de sublimación del yodo.
    \item[\texttt{b)}] Calor molar de disolución del ácido yodhídrico.
    \item[\texttt{c)}] Número de calorías que hay que aportar para disociar en sus componentes el yoduro de hidrógeno gas contenido en un matraz de \SI{750}{\cubic\centi\meter} a \SI{25}{\celsius} y \SI{800}{\torr} de presión.
\end{description}
\resultadocmd{\SI{12,8}{\kilo\calorie}; \SI{-13,0}{\kilo\calorie}; \SI{12,9}{\calorie}}

	}
\end{frame}

\begin{frame}
    \frametitle{\ejerciciocmd}
    \framesubtitle{Datos del enunciado}
    \begin{center}
        {\huge \textbf{¿$\frac{[\ce{HCOO-}]}{[\ce{HCOOH}]}$?}}
    \end{center}
    $$
        \tcbhighmath[boxrule=0.4pt,arc=4pt,colframe=red,drop fuzzy shadow=blue]{pH = \SI{3}{}}\quad
        \tcbhighmath[boxrule=0.4pt,arc=4pt,colframe=red,drop fuzzy shadow=blue]{K_a(\ce{HCOOH}) = \SI{2,1e-4}{}}
    $$
\end{frame}

\begin{frame}
    \frametitle{\ejerciciocmd}
    \framesubtitle{Resolución (\rom{1}): Determinación de $\frac{[\ce{HCOO-}]}{[\ce{HCOOH}]}$}
    \structure{La reacción de disociación será:}
    $$
        \ce{HCOOH(ac) <=> H+(ac) + HCOO-(ac)}\quad\quad K_a(\ce{HCOOH}) = \frac{[\ce{HCOO-}][\ce{H+}]}{[\ce{HCOOH}]}
    $$
    \visible<2->{
        \structure{Despejando el cociente:}
        $$
            \frac{[\ce{HCOO-}]}{[\ce{HCOOH}]} = \frac{K_a(\ce{HCOOH})}{[\ce{H+}]}
        $$
                }
    \visible<3->{
        \structure{Aplicando logaritmos y despejando:}
        \begin{overprint}
            \onslide<3>
                $$
                    \log\left(\frac{[\ce{HCOO-}]}{[\ce{HCOOH}]}\right) = \overbrace{\log\left(\frac{K_a(\ce{HCOOH})}{[\ce{H+}]}\right)}^{\log{K_a(\ce{HCOOH})}-\log{[\ce{H+}]}}
                $$
            \onslide<4>
                $$
                    \log\left(\frac{[\ce{HCOO-}]}{[\ce{HCOOH}]}\right) = -pK_a(\ce{HCOOH})+pH
                $$
            \onslide<5>
                $$
                    \frac{[\ce{HCOO-}]}{[\ce{HCOOH}]} = 10^{pH-pK_a(\ce{HCOOH})}
                $$
            \onslide<6>
                $$
                    \frac{[\ce{HCOO-}]}{[\ce{HCOOH}]} = 10^{\SI{3}{}-\SI{3,38}{}}
                $$
            \onslide<7->
                $$
                    \tcbhighmath[boxrule=0.4pt,arc=4pt,colframe=red,drop fuzzy shadow=blue]{\frac{[\ce{HCOO-}]}{[\ce{HCOOH}]} = \SI{,21}{}}
                $$\\[.3cm]
        \end{overprint}
                }
    \visible<7>{
        \textbf{Toda mezcla que siga este cociente. Por ejemplo: $[\ce{HCOO-}] = \SI{1}{\Molar}$ y $[\ce{HCOOH}] = \SI{4,76}{\Molar}$.}
                }
\end{frame}
