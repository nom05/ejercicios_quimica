\begin{frame}
	\frametitle{\ejerciciocmd}
	\framesubtitle{Enunciado}
	\textbf{
			Una reacción tiene una constante de velocidad de \SI{,017}{\per\second} a \SI{298}{\kelvin} y una energía libre de activación del \SI{27,235}{\kilo\joule\per\mol}. La adición de un catalizador disminuye dicha energía de activación hasta un \SI{33}{\percent} de su valor inicial. Calcule la nueva constante de velocidad.
\resultadocmd{ \SI{26,86}{\per\second} }

		}
\end{frame}

\begin{frame}
	\frametitle{\ejerciciocmd}
	\framesubtitle{Datos del problema}
	\begin{center}
		{\huge\textbf{¿[\ce{BOH}]? ¿pH?}}\\[.2cm]
	\end{center}
	\structure{Base débil (\ce{BOH}):}\quad
	\tcbhighmath[boxrule=0.4pt,arc=4pt,colframe=blue,drop fuzzy shadow=red]{V(\ce{BOH})   = \SI{40}{\milli\liter}}				\quad
	\tcbhighmath[boxrule=0.4pt,arc=4pt,colframe=blue,drop fuzzy shadow=red]{K_b(\ce{BOH}) = \num{4e-5}}							\\[.2cm]
	\structure{Ácido fuerte (\ce{HNO3}):}\quad
	\tcbhighmath[boxrule=0.4pt,arc=4pt,colframe=red,drop fuzzy shadow=blue]{[\ce{HCl}]=\SI{,020}{\Molar}}						\quad
	\tcbhighmath[boxrule=0.4pt,arc=4pt,colframe=red,drop fuzzy shadow=blue]{V_{\text{PE}}(\ce{HCl})=\SI{20}{\milli\liter}}		\\[.2cm]
	{\Large
		\begin{center}
			\begin{tabular}{cSSSS}
				\toprule
				$V(\ce{HCl})$ añadido (\si{\milli\liter})	 & 0	& 10	& 20	&	23	\\
				pH 											 & 		&		& 		&		\\
				\bottomrule
			\end{tabular}
		\end{center}
	}
	PE: punto de equivalencia
\end{frame}

\begin{frame}
	\frametitle{\ejerciciocmd}
	\framesubtitle{Resolución (\rom{1}): Cálculo de $[\ce{BOH}]_0$ y su pH}
	\structure{En el punto de equivalencia:}\quad\tcbhighmath[boxrule=0.4pt,arc=4pt,colframe=black,drop fuzzy shadow=black]{n(\ce{BOH}) = n(\ce{HCl})}
	\structure{Usamos el $V_{\text{PE}}(\ce{HCl})$ añadido para alcanzar el punto de equivalencia y su concentración:}
	$M=\rfrac{n}{V}\Rightarrow n=M\vdot V\Rightarrow n(\ce{HCl}) = \SI{,020}{\mol\per\cancel\liter}\vdot\SI{20e-3}{\cancel\liter} = \SI{4e-4}{\mol} = n(\ce{BOH})$
	\structure{Dividimos por $V(\ce{BOH})=\SI{40}{\milli\liter}$ para obtener $[\ce{BOH}]$:}
	$[\ce{BOH}] = \frac{\SI{4e-4}{\mol}}{\SI{40e-3}{\liter}} = \SI{,010}{\Molar}$
	\structure{Cuando $V(\ce{HCl}) = \SI{0}{\milli\liter}$ solo tenemos \ce{BOH} y su pH. Por tanto, podemos plantear el equilibrio de disociación de la base y calculamos el pH con su $K_b$:}
    \begin{center}
		{\small \begin{tabular}{clll}
				\toprule
							& \multicolumn{3}{c}{\ce{BOH(ac) <=> B+(ac) + OH-(ac)}}	\\
				Estado  	& [\ce{BOH}] (\si{\Molar})	& [\ce{B+}] (\si{\Molar})	& [\ce{OH-}] (\si{\Molar})	\\
				\midrule
				Inicial 	& \num{,01}					& 0							& 0							\\
				Equilibrio  & $\num{,01}-x$ 			& $x$						& $x$						\\
				\bottomrule
		\end{tabular}}
		$$
			K_b = \frac{[\ce{B+}][\ce{OH-}]}{[\ce{BOH}]} =\frac{x^2}{[\ce{BOH}]_{0}-x}\approx
			\frac{x^2}{[\ce{BOH}]_{0}}\Rightarrow
			\num{4e-5} = \frac{x^2}{\num{,010}}\Rightarrow
		$$
		$$
			x = [\ce{OH-}] = \sqrt{\num{4e-5}\vdot\num{,010}} = \SI{6,325e-4}{\Molar}\Rightarrow
			\tcbhighmath[boxrule=0.4pt,arc=4pt,colframe=blue,drop fuzzy shadow=red]{\text{pH} = 14 + \log[\ce{OH-}] = \num{10,80}}
		$$
	\end{center}
		{\small Como $K_b < \num{1e-4}$ la disociación de la base débil es muy pequeña con respecto a su concentración inicial. Aproximamos: $[\ce{BOH}]_0 - x \approx[\ce{BOH}]_0$. Podéis usar $\text{pH} = 14-\underbrace{\text{pOH}}_{-\log[\ce{OH-}]} = 14+\log[\ce{OH-}]$.
		}
\end{frame}

\begin{frame}
	\frametitle{\ejerciciocmd}
	\framesubtitle{Resolución (\rom{2}): pH si añadimos \SI{10}{\milli\liter} de \ce{HCl} (\rom{1})}
	\begin{enumerate}
		\item\structure{Averiguamos quién es el reactivo limitante}
			$$
				n(\ce{BOH})=\SI{,010}{\mol\per\cancel\liter}\times\SI{40e-3}{\cancel\liter}=\SI{4,00e-4}{\mol}
			$$
			$$
				n(\ce{HCl})=\SI{,020}{\mol\per\cancel\liter}\times\SI{10e-3}{\cancel\liter}=\SI{2,00e-4}{\mol}
			$$\\[.2cm]
		\begin{center}
			\ce{BOH(ac) + HCl(ac) -> BCl3(ac) + H2O(l)}
		\end{center}
		\alert{Relación estequiométrica:} $n_{\text{reaccionan}}(\ce{BOH})=n_{\text{reaccionan}}(\ce{HCl})=n_{\text{producen}}(\ce{BCl})$
		\begin{center}
			\begin{tabular}{cccc}
				\toprule
					{Estado}		& n(\ce{BOH})~(\si{\mol})						&  n(\ce{HNO3})~(\si{\mol})		& n(\ce{BNO3})~(\si{\mol})	\\
					{Inicial}		& \num{4e-4}     							&   \num{2e-4}						&  0						\\
					{Final}			&$\num{4e-4}-\num{2e-4}=\num{2e-4}$			&		0							& \num{2e-4}				\\
				\bottomrule
			\end{tabular}
			\myovalbox{\textcolor{white}{
					\ce{HCl} es el reactivo limitante, \ce{BOH} es el reactivo en exceso
			}}
		\end{center}
		\item\structure{Obtenemos el volumen total:}
		$$
			V_T = V_{\text{inicial}}(\ce{BOH}) + V_{\text{añadido}}(\ce{HCl})\Rightarrow V_T = \SI{40e-3}{\liter} + \SI{10e-3}{\liter} = \SI{50e-3}{\liter}
		$$
		\item\structure{Concentración de nuestro reactivo en exceso (\ce{BOH}) y de nuestro producto (\ce{B+}):} tenemos una base débil y su ion conjugado sin contar con la disociación. \ce{B+} proviene de la disociación \ce{BCl(ac) -> B+(ac) + Cl-(ac)}
		\begin{center}
			\myovalbox{\textcolor{red}{\textbf{DISOLUCIÓN AMORTIGUADORA}}}
		\end{center}
		$$
			[\ce{BOH}]	=\frac{\SI{2e-4}{\mol}}{\SI{50e-3}{\liter}}=\SI{,004}{\Molar};\qquad
			[\ce{B+}]	=\frac{\SI{2e-4}{\mol}}{\SI{50e-3}{\liter}}=\SI{,004}{\Molar}
		$$
	\end{enumerate}
\end{frame}

\begin{frame}
	\frametitle{\ejerciciocmd}
	\framesubtitle{Resolución (\rom{3}): pH si añadimos \SI{10}{\milli\liter} de \ce{HCl} (\rom{2})}
	\begin{enumerate}
		\setcounter{enumi}{3}
		\item\structure{Cálculo del pH:} tenemos que usar $K_b$.
		\begin{center}
			\begin{tabular}{cccc}
											& \multicolumn{3}{c}{\ce{BOH(ac) <=> B+(ac) + OH-(ac)}}	\\
				\midrule
				Estado		& [\ce{BOH}]~(\si{\Molar})	&  [\ce{B+}]~(\si{\Molar})	&	[\ce{OH-}]~(\si{\Molar})	\\
				Inicial		& \num{,004}				&	\num{,004}				&	\num{0}						\\
				Equilibrio	&$\num{,004}-x$				& 	$\num{,004}+x$			&	$x$						 	\\
				\bottomrule
			\end{tabular}
		\end{center}
		Podemos resolverlo de dos formas:
		\begin{enumerate}
			\item\structure{Sustituimos valores en la ecuación de la constante de equilibrio:} Podemos considerar $x$ despreciable si $K_b < \num{1e-4}$.
			$$
				\overbrace{K_b(\ce{BOH})}^{\num{4e-5}}=\frac{\overbrace{[\ce{B+}]}^{\num{,004}+x}\overbrace{[\ce{OH-}]}^{x}}{\underbrace{[\ce{BOH}]}_{\num{,004}-x}}=
				\frac{(\num{,004}+x)\vdot x}{\num{,004}-x}\approx\frac{\num{,004}\vdot x}{\num{,004}}\Rightarrow x=[\ce{OH-}]=\SI{4,00e-5}{\Molar}
			$$
			\item\structure{Como segunda opción, podemos tomar logaritmos decimales y operar:} Ecuación de HENDERSON -- HASSELBALCH
			$$
				\log K_b(\ce{BOH})=\log\left(\frac{[\ce{B+}][\ce{OH-}]}{[\ce{BOH}]}\right)\Rightarrow\overbrace{-\log[\ce{OH-}]}^{\text{pOH}}=\underbrace{-\log K_b(\ce{BOH})}_{\text{pK}_b}+\log\left(\frac{[\ce{B+}]}{[\ce{BOH}]}\right)
			$$
			$$
				\text{pOH} = \text{pK}_b + \log(\frac{[\ce{B+}]}{[\ce{BOH}]})\Rightarrow
				\text{pOH} = \text{pK}_b + \log(\frac{[\ce{B+}]_0}{[\ce{BOH}]_0})\quad\text{(despreciamos $x$)}%\Rightarrow
%				\pOH = \underbrace{\pKb}_{\num{4,52}} + \log(\frac{\rfrac{n_0(\ce{B+})}{\cancel{V}}}{\rfrac{n_0(\ce{BOH})}{\cancel{V}}})
			$$
			\begin{center}
				\tcbhighmath[boxrule=0.4pt,arc=4pt,colframe=green,drop fuzzy shadow=blue]{
					\text{pOH}=-\log[\ce{OH-}]=\num{4,40}
						\qquad
					\text{pH}=-\log[\ce{H+}]=\num{9,60}
				}
			\end{center}
		\end{enumerate}
	\end{enumerate}
\end{frame}

\begin{frame}
	\frametitle{\ejerciciocmd}
	\framesubtitle{Resolución (\rom{4}): pH si añadimos \SI{20}{\milli\liter} de \ce{HCl} (\rom{1})}
	\begin{enumerate}
		\item\structure{Averiguamos si hay reactivo limitante y en exceso}
		$$
			n(\ce{BOH})=\SI{,01}{\mol\per\cancel\liter}\times\SI{40e-3}{\cancel\liter}=\SI{4e-4}{\mol}
		$$
		$$
			n(\ce{HNO3})=\SI{,02}{\mol\per\cancel\liter}\times\SI{20e-3}{\cancel\liter}=\SI{4e-4}{\mol}
		$$\\[.2cm]
		\begin{center}
			\ce{BOH(ac) + HCl(ac) -> BCl(ac) + H2O(l)}
		\end{center}
		\alert{Relación estequiométrica:} $n_{\text{reaccionan}}(\ce{BOH})=n_{\text{reaccionan}}(\ce{HCl})=n_{\text{producen}}(\ce{BCl})$
		\begin{center}
			\begin{tabular}{cccc}
				\toprule
				Estado	&	n(\ce{BOH})~(\si{\mol})					&	n(\ce{HCl})~(\si{\mol})		&	n(\ce{BCl})~(\si{\mol})		\\
				Inicial	&	\num{4e-4}								&	\num{4e-4}					&	0							\\
				Final	&	$\num{4e-4}-\num{4e-4}=\num{0}$			&    	0						&	\num{4e-4}					\\
				\bottomrule
			\end{tabular}
			\myovalbox{\textcolor{white}{
					No hay r. limitante ni r. en exceso, \underline{PUNTO DE EQUIVALENCIA}
			}}
			\myovalbox{\textcolor{red}{pH controlado por \ce{B+}, \textbf{pH ÁCIDO}}}
		\end{center}
		\item\structure{Obtenemos el volumen total:}
		$$
			V_T = V_{\text{inicial}}(\ce{BOH}) + V_{\text{añadido}}(\ce{HCl})\Rightarrow V_T = \SI{40e-3}{\liter} + \SI{20e-3}{\liter} = \SI{60e-3}{\liter}
		$$
		\item\structure{Disociación: \ce{BCl(ac) -> B+(ac) + Cl-(ac)}. Concentración de nuestro producto (\ce{B+}):} tenemos el ion conjugado de un ácido débil.
		$$
			[\ce{B+}]=\frac{\SI{4e-4}{\mol}}{\SI{60e-3}{\liter}}=\SI{,00667}{\Molar}
		$$
	\end{enumerate}
\end{frame}	

\begin{frame}
	\frametitle{\ejerciciocmd}
	\framesubtitle{Resolución (\rom{4}): pH si añadimos \SI{20}{\milli\liter} de \ce{HCl} (\rom{2})}
	\begin{enumerate}
		\setcounter{enumi}{3}
		\item\structure{Cálculo del pH:} tenemos que usar $K_h=\rfrac{K_w}{K_b}$, $\text{pH}<7$.
		\begin{center}
			\begin{tabular}{cccc}
				& \multicolumn{3}{c}{\ce{B+(ac) + H2O(l) <=> BOH(ac) + H+(ac)}}	\\
				\midrule
					Estado 		& 	[\ce{B+}]~(\si{\Molar})	&  [\ce{BOH}]~(\si{\Molar})	& [\ce{H+}]~(\si{\Molar})			\\
					Inicial		& 	 \num{,0067}			&	0						&  0				\\
					Equilibrio	&	$\num{,0067}-x$			& 	$x$						& $x$ 				\\
				\bottomrule
			\end{tabular}
		\end{center}
		\structure{Sustituimos valores en la ecuación de la constante de equilibrio:} Podemos considerar $x$ despreciable si $K_h<\num{1e-4}$.
		$$
			\overbrace{K_h(\ce{B+})}^{\num{2,5e-10}}=\frac{\overbrace{[\ce{BOH}]}^{x}\overbrace{[\ce{H+}]}^{x}}{\underbrace{[\ce{BOH}]}_{\num{,0067}-x}}=
			\frac{x^2}{\num{,0067}-x}\approx\frac{x^2}{\num{,0067}}\Rightarrow x=[\ce{H+}]=\SI{1,29e-06}{\Molar}
		$$
		\begin{center}
			\tcbhighmath[boxrule=0.4pt,arc=4pt,colframe=green,drop fuzzy shadow=blue]{\text{pH}=\num{5,88}}
		\end{center}
	\end{enumerate}
\end{frame}

\begin{frame}
	\frametitle{\ejerciciocmd}
	\framesubtitle{Resolución (\rom{5}): pH si añadimos \SI{23}{\milli\liter} de \ce{HCl}}
	\begin{enumerate}
		\item\structure{Averiguamos si hay reactivo limitante y en exceso}
		$$
			n(\ce{BOH})=\SI{,010}{\mol\per\cancel\liter}\times\SI{40e-3}{\cancel\liter}=\SI{4e-4}{\mol}
		$$
		$$
			n(\ce{HCl})=\SI{,020}{\mol\per\cancel\liter}\times\SI{23e-3}{\cancel\liter}=\SI{4,6e-4}{\mol}
		$$\\[.2cm]
		\begin{center}
			\ce{BOH(ac) + HCl(ac) -> BCl(ac) + H2O(l)}
		\end{center}
		\alert{Relación estequiométrica:} $n_{\text{reaccionan}}(\ce{BOH})=n_{\text{reaccionan}}(\ce{HCl})=n_{\text{producen}}(\ce{BCl})$
		\begin{center}
			\begin{tabular}{cccc}
				\toprule
					Estado	&	n(\ce{BOH})~(\si{\mol})					&	n(\ce{HCl})~(\si{\mol})							&	n(\ce{BCl})~(\si{\mol})		\\
					Inicial	&	\num{4e-4}								&	\num{4,6e-4}									&	0							\\
					Final	&	0										&   $\num{4,6e-4}-\num{4e-4}=\num{,6e-4}$			&	\num{4e-4}					\\
				\bottomrule
			\end{tabular}
			\myovalbox{\textcolor{white}{
					\ce{BOH} es el reactivo limitante, \ce{HCl} es el reactivo en exceso
			}}
			\myovalbox{\textcolor{red}{pH controlado por \ce{HCl}, \textbf{pH muy ÁCIDO}}}
		\end{center}
		\item\structure{Obtenemos el volumen total:}
		$$
			V_T = V_{\text{inicial}}(\ce{BOH}) + V_{\text{añadido}}(\ce{HCl})\Rightarrow V_T = \SI{40e-3}{\liter} + \SI{23e-3}{\liter} = \SI{63e-3}{\liter}
		$$
		\item\structure{Concentración del exceso de \ce{HCl}:} recordemos que es un ácido fuerte (\ce{HCl(ac) -> H+(ac) + Cl-(ac)})
		$$
			[\ce{HCl}]_{\text{exceso}} = [\ce{H+}]_{\text{exceso \ce{HCl}}} = \frac{\SI{,6e-4}{\mol}}{\SI{63e-3}{\liter}} = \SI{9,52e-4}{\Molar}
		$$
		\begin{center}
			\tcbhighmath[boxrule=0.4pt,arc=4pt,colframe=green,drop fuzzy shadow=blue]{\text{pH}=\num{3,02}}
		\end{center}
	\end{enumerate}
\end{frame}