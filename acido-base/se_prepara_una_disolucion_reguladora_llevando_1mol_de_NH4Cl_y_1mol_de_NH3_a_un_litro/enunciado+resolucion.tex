\begin{frame}
    \frametitle{\ejerciciocmd}
    \framesubtitle{Enunciado}
    \textbf{
		Dadas las siguientes reacciones:
\begin{itemize}
    \item \ce{I2(g) + H2(g) -> 2 HI(g)}~~~$\Delta H_1 = \SI{-0,8}{\kilo\calorie}$
    \item \ce{I2(s) + H2(g) -> 2 HI(g)}~~~$\Delta H_2 = \SI{12}{\kilo\calorie}$
    \item \ce{I2(g) + H2(g) -> 2 HI(ac)}~~~$\Delta H_3 = \SI{-26,8}{\kilo\calorie}$
\end{itemize}
Calcular los parámetros que se indican a continuación:
\begin{description}%[label={\alph*)},font={\color{red!50!black}\bfseries}]
    \item[\texttt{a)}] Calor molar latente de sublimación del yodo.
    \item[\texttt{b)}] Calor molar de disolución del ácido yodhídrico.
    \item[\texttt{c)}] Número de calorías que hay que aportar para disociar en sus componentes el yoduro de hidrógeno gas contenido en un matraz de \SI{750}{\cubic\centi\meter} a \SI{25}{\celsius} y \SI{800}{\torr} de presión.
\end{description}
\resultadocmd{\SI{12,8}{\kilo\calorie}; \SI{-13,0}{\kilo\calorie}; \SI{12,9}{\calorie}}

	}
\end{frame}

\begin{frame}
    \frametitle{\ejerciciocmd}
    \framesubtitle{Datos del apartado (a)}
    \begin{center}
        {\huge \textbf{¿pH?}}
    \end{center}
    $$
        \tcbhighmath[boxrule=0.4pt,arc=4pt,colframe=black,drop fuzzy shadow=yellow]{V_{dis} = \SI{1}{\liter}}
    $$
    $$
        \tcbhighmath[boxrule=0.4pt,arc=4pt,colframe=green,drop fuzzy shadow=blue]{n(\ce{NH3}) = \SI{1}{\mol}}\quad
        \tcbhighmath[boxrule=0.4pt,arc=4pt,colframe=green,drop fuzzy shadow=blue]{K_b(\ce{NH3}) = \SI{1,8e-5}{}}
    $$
    $$
        \tcbhighmath[boxrule=0.4pt,arc=4pt,colframe=red,drop fuzzy shadow=blue]{n(\ce{NH4Cl}) = \SI{1}{\mol}}
    $$
\end{frame}

\begin{frame}
    \frametitle{\ejerciciocmd}
    \framesubtitle{Resolución (\rom{1}): pH inicial}
    \structure{A partir de la expresión derivada de la constante de equilibrio y suponiendo la disociación despreciable frente a las concentraciones iniciales:}
    \begin{overprint}
        \onslide<1>
            $$
                \overbrace{pOH}^{pOH=14-pH} = pK_b + \log\frac{[\ce{NH4+}]}{[\ce{NH3}]}
            $$
        \onslide<2->
            $$
                pH = 14-pK_b + \log\frac{[\ce{NH3}]}{[\ce{NH4+}]}
            $$
    \end{overprint}
    \visible<3->{
        \structure{Como el volumen se mantiene constante, las concentraciones serán \SI{1}{\Molar} en ambos casos:}
        $$
            \tcbhighmath[boxrule=0.4pt,arc=4pt,colframe=black,drop fuzzy shadow=yellow]{pH = 14-pK_b + \overbrace{\log\frac{[\ce{NH3}]}{[\ce{NH4+}]}}^{\log 1 = 0} = \SI{9,3}{}}
        $$
                }
\end{frame}

\begin{frame}
    \frametitle{\ejerciciocmd}
    \framesubtitle{Datos del apartado (b)}
    \begin{center}
        {\huge \textbf{¿pH?}}
    \end{center}
    $$
        \tcbhighmath[boxrule=0.4pt,arc=4pt,colframe=black,drop fuzzy shadow=yellow]{V_{dis} = \SI{1}{\liter}}\quad
        \tcbhighmath[boxrule=0.4pt,arc=4pt,colframe=black,drop fuzzy shadow=yellow]{V_{dis} = \text{constante}}
    $$
    $$
        \tcbhighmath[boxrule=0.4pt,arc=4pt,colframe=green,drop fuzzy shadow=blue]{n(\ce{NH3}) = \SI{1}{\mol}}\quad
        \tcbhighmath[boxrule=0.4pt,arc=4pt,colframe=green,drop fuzzy shadow=blue]{K_b(\ce{NH3}) = \SI{1,8e-5}{}}
    $$
    $$
        \tcbhighmath[boxrule=0.4pt,arc=4pt,colframe=red,drop fuzzy shadow=blue]{n(\ce{NH4Cl}) = \SI{1}{\mol}}
    $$
    $$
        \tcbhighmath[boxrule=0.4pt,arc=4pt,colframe=blue,drop fuzzy shadow=green]{n(\ce{NaOH}) = \SI{,2}{\mol}}
    $$
\end{frame}

\begin{frame}
    \frametitle{\ejerciciocmd}
    \framesubtitle{Resolución (\rom{2}): pH después de añadir \SI{,2}{\mol} de \ce{NaOH}}
    \structure{Reacción de neutralización:}
    $$
        \ce{NaOH(ac) + NH4+(ac) + Cl-(ac) -> NH3(ac) + NaCl(ac) + H2O(l)}
    $$
    \textbf{Se \underline{CONSUMEN} \SI{,2}{\mol} de \ce{NH4+}.}
    \visible<2>{
        \structure{Por tanto se consumen cationes de \ce{NH4+}. Suponemos que las disociaciones son despreciables frente a las concentraciones de \ce{NH3} y \ce{NH4+}:}
        $$
            \tcbhighmath[boxrule=0.4pt,arc=4pt,colframe=black,drop fuzzy shadow=yellow]{pH = 14-pK_b + \log\overbrace{\frac{1+0,2}{1-0,2}}^{\frac{3}{2}} = \num{9,4}}
        $$
        \begin{center}
            {\Large \textbf{¿Por qué prácticamente no cambia el pH?}}
        \end{center}
                }
\end{frame}

\begin{frame}
    \frametitle{\ejerciciocmd}
    \framesubtitle{Datos del apartado (c)}
    \begin{center}
        {\huge \textbf{¿pH?}}
    \end{center}
    $$
        \tcbhighmath[boxrule=0.4pt,arc=4pt,colframe=black,drop fuzzy shadow=yellow]{V_{dis} = \SI{1}{\liter}}\quad
        \tcbhighmath[boxrule=0.4pt,arc=4pt,colframe=black,drop fuzzy shadow=yellow]{V_{dis} = \text{constante}}
    $$
    $$
        \tcbhighmath[boxrule=0.4pt,arc=4pt,colframe=green,drop fuzzy shadow=blue]{n(\ce{NH3}) = \SI{1}{\mol}}\quad
        \tcbhighmath[boxrule=0.4pt,arc=4pt,colframe=green,drop fuzzy shadow=blue]{K_b(\ce{NH3}) = \SI{1,8e-5}{}}
    $$
    $$
        \tcbhighmath[boxrule=0.4pt,arc=4pt,colframe=red,drop fuzzy shadow=blue]{n(\ce{NH4Cl}) = \SI{1}{\mol}}
    $$
    $$
        \tcbhighmath[boxrule=0.4pt,arc=4pt,colframe=blue,drop fuzzy shadow=green]{n(\ce{HCl}) = \SI{,5}{\mol}}
    $$
\end{frame}

\begin{frame}
\frametitle{\ejerciciocmd}
\framesubtitle{Resolución (\rom{3}): pH después de añadir \SI{,5}{\mol} de \ce{HCl}}
\structure{Reacción de neutralización:}
$$
\ce{HCl(ac) + NH3(ac) -> NH4+(ac) + Cl-(ac)}
$$
\textbf{Se \underline{CONSUMEN} \SI{,5}{\mol} de \ce{NH3}.}
\visible<2>{
    \structure{Por tanto se consumen moléculas de \ce{NH3}. Suponemos que las disociaciones son despreciables frente a las concentraciones de \ce{NH3} y \ce{NH4+}:}
    $$
        \tcbhighmath[boxrule=0.4pt,arc=4pt,colframe=black,drop fuzzy shadow=yellow]{pH = 14-pK_b + \log\overbrace{\frac{1-0,5}{1+0,5}}^{\frac{1}{3}} = \num{8,8}}
    $$
}
\end{frame}
