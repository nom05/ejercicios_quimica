Se ha realizado una valoración de \SI{20}{\milli\liter} de una base débil \ce{BOH} con una disolución de ácido nítrico (ácido fuerte, \ce{HNO3}) \SI{,15}{\Molar}. En el punto de equivalencia se han empleado \SI{16,5}{\milli\liter} de este ácido. Responda a las siguientes preguntas:
\begin{enumerate}[label={\alph*)},font={\color{red!50!black}\bfseries}]
	\item Concentración inicial de la base y su pH.
	\item pH de la disolución cuando añadimos a los \SI{20}{\milli\liter} de la base \SI{10}{\milli\liter} de la disolución de \ce{HNO3}.
	\item pH en el punto de equivalencia.
	\item pH de la disolución cuando añadimos a los \SI{20}{\milli\liter} de la base \SI{18}{\milli\liter} de la disolución de \ce{HNO3}.
\end{enumerate}
DATOS: $K_b(\ce{BOH}) = \num{3e-5}$.
\resultadocmd{
				\SI{,1238}{\Molar},\num{11,28};
				\num{9,66};
				\num{5,32};
				\num{2,23}
			}