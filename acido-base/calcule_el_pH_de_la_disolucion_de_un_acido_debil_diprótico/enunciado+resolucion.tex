\begin{frame}
	\frametitle{\ejerciciocmd}
	\framesubtitle{Enunciado}
	\textbf{
		Una reacción tiene una constante de velocidad de \SI{,017}{\per\second} a \SI{298}{\kelvin} y una energía libre de activación del \SI{27,235}{\kilo\joule\per\mol}. La adición de un catalizador disminuye dicha energía de activación hasta un \SI{33}{\percent} de su valor inicial. Calcule la nueva constante de velocidad.
\resultadocmd{ \SI{26,86}{\per\second} }

		}
\end{frame}

\begin{frame}
	\frametitle{\ejerciciocmd}
	\framesubtitle{Datos del problema}
	\begin{center}
		{\huge$\mathrm{pH}$}\\[.3cm]
		\tcbhighmath[boxrule=0.4pt,arc=4pt,colframe=green,drop fuzzy shadow=blue]{n(\ce{H2A})=\SI{,02}{\mol}}\quad
		\tcbhighmath[boxrule=0.4pt,arc=4pt,colframe=green,drop fuzzy shadow=blue]{V(\ce{H2A})=\SI{,5}{\liter}}\\[.3cm]
		\tcbhighmath[boxrule=0.4pt,arc=4pt,colframe=black,drop fuzzy shadow=green]{K_{a_1}=\num{4,3e-3}}\quad
		\tcbhighmath[boxrule=0.4pt,arc=4pt,colframe=black,drop fuzzy shadow=red]{K_{a_2}=\num{5,6e-7}}
	\end{center}
\end{frame}

\begin{frame}
	\frametitle{\ejerciciocmd}
	\framesubtitle{Resolución (\rom{1}): primera disociación}
	\structure{Concentración del ácido sin disociar:} $M=\rfrac{n}{V}\Rightarrow[\ce{H2A}]_0=\rfrac{\SI{,02}{\mol}}{\SI{,5}{\liter}}=\SI{,04}{\Molar}$
	\structure{Reacción de la 1"a disociación:}\\
	\centering\ce{H2A(ac) <=> H+(ac) + HA-(ac)}\quad\quad$K_{a_1}=\frac{[\ce{H+}]_1[\ce{HA-}]_1}{[\ce{H2A}]}$
	\begin{center}
		\begin{tabular}{cccc}
			\toprule
			Concentración(\si{\Molar}) & [\ce{H2A}]       &         [\ce{H+}]              & [\ce{HA-}] \\
			{[Inicial]}                & $[\ce{H2A}]_0$   &              0                 &         0  \\
			{[Equilibrio]}             & $[\ce{H2A}]_0-x$ &  			$x$ 			   &        $x$ \\
			\bottomrule
		\end{tabular}%\\[.5cm]
	\end{center}
	\structure{Constante de equilibrio de la primera disociación ($K_{a_1}>\num{1e-4}$ (\num{4,3e-3}), $x$ no es despreciable):}
	$$
		K_{a_1}=\frac{\overbrace{[\ce{H+}]_1}^{x}\overbrace{[\ce{HA-}]_1}^{x}}{\underbrace{[\ce{H2A}]}_{[\ce{H2A}]_0-x}}\Rightarrow
		\overbrace{K_{a_1}}^{\num{4,3e-3}}=\frac{x^2}{\underbrace{[\ce{H2A}]_0}_{\SI{,04}{\Molar}}-x}
	$$
	\begin{multicols}{2}
		\structure{Sin aproximación (forma correcta de resolver en este caso):}
		{\small
			$$
				\frac{x^2}{\num{,04}-x}=\num{4,3e-3}\Rightarrow
			$$
			$$
				x^2+\num{4,3e-3}x-\num{1,62e-4}=0
			$$
		}
		\begin{center}
			\myovalbox{\textcolor{yellow}{$x=[\ce{H+}]_1=[\ce{HA-}]_1=\SI{,0129}{\Molar}$}}
		\end{center}
		\alert{\textbf{Con aproximación (forma incorrecta) $[\ce{H2A}]_0-x\approx[\ce{H2A}]$:}}
		{\small $$
			\frac{x^2}{\num{,04}}=\num{4,3e-3}\Rightarrow x=\sqrt{\num{,04}\cdot\num{4,3e-3}}
		$$}
		\centering$x=[\ce{H+}]_1=\cancel{\SI{,0131}{\Molar}}$
	\end{multicols}
\end{frame}

\begin{frame}
	\frametitle{\ejerciciocmd}
	\framesubtitle{Resolución (\rom{2}): segunda disociación y $\mathrm{pH}$}
	\structure{Reacción de la 2"a disociación:}\\
	\begin{center}
		\ce{HA-(ac) <=> H+(ac) + A2-(ac)}\quad\quad$K_{a_2}=\frac{[\ce{H+}][\ce{A^{2-}}]}{[\ce{HA-}]}$
	\end{center}

	\begin{center}
		\begin{tabular}{cccc}
			\toprule
			Concentración(\si{\Molar}) & [\ce{HA-}]       &         [\ce{H+}]              					  & [\ce{A^{2-}}] \\
			{[Inicial]}                & $[\ce{HA-}]_1$   &        $[\ce{H+}]_1$           					  &         0     \\
			{[Equilibrio]}             & $[\ce{HA-}]_1-y$ &  	   $[\ce{H+}]_1+\underbrace{y}_{[\ce{H+}]_2}$ &        $y$    \\
			\bottomrule
		\end{tabular}\\[.5cm]
	\end{center}
%	\begin{overprint}
%		\onslide<2>
	\structure{Concentración de \ce{H+} de la segunda disociación ({\footnotesize $K_{a_2}=\num{5,6e-7}<<\num{1e-4}$ , $y$ es despreciable}):}
	$$
		K_{a_2}=\frac{\overbrace{[\ce{H+}]}^{[\ce{H+}]_1+y}\overbrace{[\ce{A^{2-}}]_1}^{y}}{\underbrace{[\ce{HA-}]}_{[\ce{HA-}]_1-y}}\Rightarrow
		K_{a_2}=\frac{([\ce{H+}]_1+\cancel{y})y}{[\ce{HA-}]_1-\cancel{y}}\Rightarrow
		K_{a_2}=\frac{(\overbrace{[\ce{H+}]_1}^{x})y}{\underbrace{[\ce{HA-}]_1}_{x}}\Rightarrow
		K_{a_2}=\frac{\cancel{x}y}{\cancel{x}}=[\ce{H+}]_2
	$$
	\begin{center}
		{\small ($[\ce{H+}]_1$ y $[\ce{HA-}]_1$ se conocen de antes y valían $x=\SI{,0129}{\Molar}$:)}\\[.2cm]
		\myovalbox{\textcolor{yellow}{$[\ce{H+}]_2=\SI{5,6e-7}{\Molar}$}}
	\end{center}
	$$
		[\ce{H+}]_{\text{TOTAL}}=\SI{,0129}{\Molar}+\SI{5,6e-7}{\Molar}=\SI{,01290056}{\Molar}\approx\SI{,0129}{\Molar}
	$$
	\begin{center}
		\tcbhighmath[boxrule=0.4pt,arc=4pt,colframe=green,drop fuzzy shadow=blue]{\mathrm{pH}=-\log[\ce{H+}]_{\text{TOTAL}}=\num{1,89}}
	\end{center}
\end{frame}
