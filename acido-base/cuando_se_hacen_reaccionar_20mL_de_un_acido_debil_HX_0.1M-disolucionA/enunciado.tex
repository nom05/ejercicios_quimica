Cuando se hacen reaccionar \SI{20}{\milli\liter} de un ácido débil \ce{HX} \SI{,1}{\Molar} (disolución A) con \SI{10}{\milli\liter} de \ce{NaOH} \SI{,1}{\Molar} (disolución B) se obtiene una disolución reguladora de $\pH = \num{5,00}$. Calcule el pH de la disolución A (sin mezclar con B) y de una disolución C formada al añadir \SI{30}{\milli\liter} de la disolución A a \SI{30}{\milli\liter} de la disolución B.
\resultadocmd{\num{3,00}; \num{8,85}}


