Se desea preparar una disolución amortiguadora con un pH \num{9} a partir de \SI{15}{\milli\liter} de una disolución de amoníaco \SI{,6}{\Molar} (\ce{NH3}, $K_b = \num{1,8e-5}$) mezclada con una disolución de ácido clorhídrico (\ce{HCl}, ácido fuerte) \SI{,3}{\Molar}.
\begin{enumerate}[label={\alph*)},font={\color{red!50!black}\bfseries}]
	\item ¿Qué volumen de \ce{HCl} hay que añadir a la base?
	\item ¿Qué pH se alcanza en la disolución amortiguadora si añadimos una alícuota de \SI{5}{\milli\liter} de \ce{HClO4} \SI{,5}{\Molar} (ácido fuerte) a las condiciones iniciales?
	\item ¿Qué pH se alcanza en la disolución amortiguadora si añadimos una alícuota de \SI{5}{\milli\liter} de \ce{KOH} \SI{,5}{\Molar} a las condiciones iniciales?
\end{enumerate}
\resultadocmd{
				\SI{19,36}{\milli\liter};
				\num{8,18};
				\num{9,49}
			}