Se desea preparar una disolución amortiguadora con un pH 5 a partir de \SI{20}{\milli\liter} de una disolución de ácido acético \SI{,5}{\Molar} (\ce{CH3COOH}, pero podéis usar la abreviatura \ce{HAc} si lo creéis oportuno, $K_a = \num{1,8e-5}$) mezclada con una disolución de hidróxido de sodio (\ce{NaOH}) \SI{,3}{\Molar}.
\begin{enumerate}%[label={\alph*)},font={\color{red!50!black}\bfseries}]
	\item ¿Qué volumen de \ce{NaOH} hay que añadir al ácido?
	\item ¿Qué pH se alcanza en la disolución amortiguadora si añadimos una alícuota de \SI{5}{\milli\liter} de \ce{HCl} \SI{,5}{\Molar} a las condiciones iniciales?
	\item ¿Qué pH se alcanza en la disolución amortiguadora si añadimos una alícuota de \SI{3}{\milli\liter} de \ce{KOH} \SI{,4}{\Molar} a las condiciones iniciales?
\end{enumerate}
\resultadocmd{
				\SI{21,5}{\milli\liter};
				\num{4,56};
				\num{5,26}
			}