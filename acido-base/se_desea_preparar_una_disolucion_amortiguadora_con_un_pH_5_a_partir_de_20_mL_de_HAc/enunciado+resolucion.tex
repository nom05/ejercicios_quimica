\begin{frame}
	\frametitle{\ejerciciocmd}
	\framesubtitle{Enunciado}
	\textbf{
			Dadas las siguientes reacciones:
\begin{itemize}
    \item \ce{I2(g) + H2(g) -> 2 HI(g)}~~~$\Delta H_1 = \SI{-0,8}{\kilo\calorie}$
    \item \ce{I2(s) + H2(g) -> 2 HI(g)}~~~$\Delta H_2 = \SI{12}{\kilo\calorie}$
    \item \ce{I2(g) + H2(g) -> 2 HI(ac)}~~~$\Delta H_3 = \SI{-26,8}{\kilo\calorie}$
\end{itemize}
Calcular los parámetros que se indican a continuación:
\begin{description}%[label={\alph*)},font={\color{red!50!black}\bfseries}]
    \item[\texttt{a)}] Calor molar latente de sublimación del yodo.
    \item[\texttt{b)}] Calor molar de disolución del ácido yodhídrico.
    \item[\texttt{c)}] Número de calorías que hay que aportar para disociar en sus componentes el yoduro de hidrógeno gas contenido en un matraz de \SI{750}{\cubic\centi\meter} a \SI{25}{\celsius} y \SI{800}{\torr} de presión.
\end{description}
\resultadocmd{\SI{12,8}{\kilo\calorie}; \SI{-13,0}{\kilo\calorie}; \SI{12,9}{\calorie}}

		}
\end{frame}

\begin{frame}
	\frametitle{\ejerciciocmd}
	\framesubtitle{Datos del problema}
	\begin{center}
		{\huge\textbf{¿$V(\ce{NaOH})$? ¿pH?}}\\[.2cm]
	\end{center}
	\structure{Disolución amortiguadora formada por:}
	\begin{itemize}
		\item	Ácido (\ce{CH3COOH} o \ce{HAc}):\quad
			\tcbhighmath[boxrule=0.4pt,arc=4pt,colframe=blue,drop fuzzy shadow=red]{V(\ce{HAc}) = \SI{20}{\milli\liter}}\quad
			\tcbhighmath[boxrule=0.4pt,arc=4pt,colframe=blue,drop fuzzy shadow=red]{[\ce{HAc}] = \SI{,5}{\Molar}}\\[.2cm]
			\qquad\qquad\qquad\qquad\qquad\qquad\qquad\tcbhighmath[boxrule=0.4pt,arc=4pt,colframe=blue,drop fuzzy shadow=red]{K_a(\ce{HAc}) = \num{1,8e-5}}\\[.2cm]
		\item	Base (\ce{NaOH}):\qquad\qquad\qquad\quad
			\tcbhighmath[boxrule=0.4pt,arc=4pt,colframe=red,drop fuzzy shadow=blue]{[\ce{NaOH}]=\SI{,3}{\Molar}}\quad
	\end{itemize}
	\begin{enumerate}[label={Apartado \alph*)},font={\color{red!50!black}\bfseries}]
		\setcounter{enumi}{1}
		\item	\tcbhighmath[boxrule=0.4pt,arc=4pt,colframe=green,drop fuzzy shadow=blue]{[\ce{HCl}]=\SI{,5}{\Molar}}\quad
				\tcbhighmath[boxrule=0.4pt,arc=4pt,colframe=green,drop fuzzy shadow=blue]{V(\ce{HCl})=\SI{5}{\milli\liter}}
		\item	\tcbhighmath[boxrule=0.4pt,arc=4pt,colframe=orange,drop fuzzy shadow=green]{[\ce{KOH}]=\SI{,4}{\Molar}}\quad
				\tcbhighmath[boxrule=0.4pt,arc=4pt,colframe=orange,drop fuzzy shadow=green]{V(\ce{KOH})=\SI{3}{\milli\liter}}\quad
	\end{enumerate}
\end{frame}

\begin{frame}
	\frametitle{\ejerciciocmd}
	\framesubtitle{Resolución (\rom{1}): Consideraciones iniciales}
	El ejercicio se puede hacer usando la ec. de Henderson--Hasselbalch o directamente expresando la constante de equilibrio y operando. Las dos formas son equivalentes pero La primera quizá tenga alguna operación menos. Esa es la razón por la que la hemos escogido.
	$$
		\pH = \underbrace{\pKa}_{\pKa = -\log K_a} + \log(\frac{[\ce{A-}]_0}{[\ce{HA}]_0})
	$$
	
	\structure{Disolución amortiguadora:} existen las dos especies, ácido o base débil y su contraión conjugado. En nuestro caso \ce{HAc} y \ce{Ac-}. Hay dos formas de conseguir esa situación:
	\begin{itemize}
		\item Mezclar una \textbf{disolución del ácido con} otra de \textbf{su sal conjugada}.
		\item Mezclar una \textbf{disolución del ácido con} otra de una \textbf{base fuerte antes del punto de equivalencia}.
	\end{itemize}
	En este ejercicio consideramos la segunda opción. Las reacciones de neutralización y de equilibrio de disociación del ácido serán:
	\begin{center}
		\ce{HAc(ac) + NaOH(ac) -> $\underset{\ce{NaAc(ac) -> Na+(ac) + Ac-(ac)}}{\ce{NaAc(ac)}}$ + H2O(l)}\\
		$$
			\ce{HA(ac) <=> H+(ac) + Ac-(ac)}\qquad K_a=\frac{[\ce{H+}][\ce{Ac-}]}{[\ce{HA}]}
		$$
	\end{center}
\end{frame}

\begin{frame}
	\frametitle{\ejerciciocmd}
	\framesubtitle{Resolución (\rom{2}): Volumen necesario de \ce{NaOH} \SI{,3}{\Molar}}
	\structure{Ecuación de Henderson--Hasselbalch:} recuerda que $\pKa = -\log K_a\Rightarrow\pKa = -\log\num{1,8e-5}$
	$$
		\overbrace{\pH}^5 = \overbrace{\pKa}^{\num{4,74}} + \log(\frac{[\ce{Ac-}]}{[\ce{HAc}]})\Rightarrow
		\frac{[\ce{Ac-}]}{[\ce{HAc}]} = \overbrace{\frac{\rfrac{n(\ce{Ac-})}{\cancel{V}}}{\rfrac{n(\ce{HAc})}{\cancel{V}}}}^{M=\rfrac{n}{V}} = 10^{(5-4,74)} = \num{1,82}
	$$
	{\small (Despreciamos las disociaciones frente a las concentraciones iniciales, $K_a < \num{4e-4}$)}
	\structure{N"o de moles sin reaccionar de \ce{HAc}:} $n(\ce{HAc}) = \SI{,5}{\mol\per\cancel\liter}\vdot\SI{20e-3}{\cancel\liter} = \SI{10e-3}{\mol}$
	\alert{Relación estequiométrica:} $n_{\text{reaccionan}}(\ce{HA}) = n_{\text{reaccionan}}(\ce{NaOH}) = n_{\text{producen}}(\ce{A-})$
	\begin{center}
		\begin{tabular}{lccc}
					&	\multicolumn{3}{c}{\ce{HAc(ac) + NaOH(ac) -> Na+(ac) + Ac-(ac) + H2O(l)}}					\\
			\midrule
			Estado	&	$n(\ce{HAc})~(\si{\mol})$	&	$n(\ce{NaOH})~(\si{\mol})$	&	$n(\ce{Ac-})~(\si{\mol})$	\\
			\midrule
			Inicial	&	\num{10e-3}					&	$x$							&	0							\\
			Final	&	$\num{10e-3}-x$				&	0							&	$x$							\\
			\bottomrule
		\end{tabular}
	\end{center}
	$$
		\frac{n(\ce{Ac-})}{n(\ce{HAc})} = \frac{x}{\num{10e-3}-x} = \num{1,82}\Rightarrow
		\num{2,82}x = \num{1,82e-2}\Rightarrow
		x = n(\ce{A-}) =\SI{6,454e-3}{\mol}
	$$
	\structure{Según la relación estequiométrica anterior:} $n(\ce{NaOH}) = \SI{6,454e-3}{\mol}$
	$$
		M = \frac{n}{V}\Rightarrow V = \frac{n}{M}\Rightarrow
		\tcbhighmath[boxrule=0.4pt,arc=4pt,colframe=red,drop fuzzy shadow=blue]{V(\ce{NaOH}) = \frac{\SI{6,454e-3}{\cancel\mol}}{\SI{,3}{\cancel\mol\per\liter}} = \SI{,0215}{\liter}=\SI{21,5}{\milli\liter}}
	$$
\end{frame}

\begin{frame}
	\frametitle{\ejerciciocmd}
	\framesubtitle{Resolución (\rom{3}): pH al añadir \SI{5}{\milli\liter} de \ce{HCl} \SI{,5}{\Molar}}
	\structure{Condiciones del anterior apartado:} Calculamos volumen total y concentración de \ce{HAc} y \ce{Ac-} después de añadir el \ce{NaOH}. $V_T = \SI{20}{\milli\liter} + \SI{21,5}{\milli\liter} = \SI{41,5}{\milli\liter}$
	$$
		[\ce{HA}] = \frac{\overbrace{\SI{10e-3}{\mol}-\SI{6,454e-3}{\mol}}^{\SI{3,546e-3}{\mol}}}{\SI{41,5e-3}{\liter}} = \SI{,08545}{\Molar};
		\qquad
		[\ce{A-}] = \frac{\SI{6,454e-3}{\mol}}{\SI{41,5e-3}{\liter}} = \SI{,1555}{\Molar}
	$$
	\structure{N"o de moles de \ce{HCl} añadidos:} $n(\ce{HCl}) = \SI{,5}{\mol\per\cancel\liter}\vdot\SI{5e-3}{\cancel\liter} = \SI{2,5e-3}{\mol}$
	\structure{Volumen después de añadir el \ce{HCl}:} $V_T = \SI{41,5}{\milli\liter} + \SI{5}{\milli\liter} = \SI{46,5}{\milli\liter}$
	\structure{\ce{HCl} (ácido fuerte) reaccionará con la \amarillo{especie básica}:}\qquad
	$\ce{$\underset{\text{Ácido}}{\ce{HA}}$ <=> H+ + $\underset{\text{\textbf{Base}}}{\amarillo{\ce{Ac-}}}$}$
	\begin{center}
		\begin{tabular}{lccc}
					&	\multicolumn{3}{c}{\ce{Ac-(ac) + HCl(ac) -> HAc(ac) + Cl-(ac)}}										\\
			\midrule
			Estado	&	$n(\ce{Ac-})$~(\si{\mol})		&	$n(\ce{HCl})$~(\si{\mol})	&	$n(\ce{HAc})$~(\si{\mol})		\\
			\midrule
			Inicial	&	\num{6,454e-3}					&	\num{2,5e-3}				&	\num{3,546e-3}					\\
			Final	&	$\num{6,454e-3}-\num{2,5e-3}$	&	\num{0}						&	$\num{3,546e-3}+\num{2,5e-3}$	\\
					&		=\num{3,954e-3}				&								&		=\num{6,046e-3}				\\
			\bottomrule
		\end{tabular}
	\end{center}
	$$
		\pH = \overbrace{\pKa}^{\num{4,74}} + \log(\frac{\rfrac{n(\ce{Ac-})}{\cancel{V}}}{\rfrac{n(\ce{HAc})}{\cancel{V}}})\Rightarrow
		\pH = \num{4,74} + \log(\frac{\num{3,954e-3}}{\num{6,046e-3}})\Rightarrow
		\tcbhighmath[boxrule=0.4pt,arc=4pt,colframe=green,drop fuzzy shadow=blue]{\pH = \num{4,56}}
	$$
\end{frame}

\begin{frame}
	\frametitle{\ejerciciocmd}
	\framesubtitle{Resolución (\rom{4}): pH al añadir \SI{3}{\milli\liter} de \ce{KOH} \SI{,4}{\Molar}}
	\structure{Calculado anteriormente:} $V_T = \SI{41,5}{\milli\liter}$
	$$
		[\ce{HA}] = \SI{,08545}{\Molar};
			\qquad
		[\ce{A-}] = \SI{,1555}{\Molar}
	$$
	\structure{N"o de moles de \ce{KOH} añadidos:} $n(\ce{KOH}) = \SI{,4}{\mol\per\cancel\liter}\vdot\SI{3e-3}{\cancel\liter} = \SI{1,2e-3}{\mol}$
	\structure{Volumen después de añadir el \ce{KOH}:} $V_T = \SI{41,5}{\milli\liter} + \SI{3}{\milli\liter} = \SI{44,5}{\milli\liter}$
	\structure{\ce{KOH} (base fuerte) reaccionará con la \amarillo{especie ácida}:}\qquad
	$\ce{$\underset{\text{\textbf{Ácido}}}{\amarillo{\ce{HA}}}$ <=> H+ + $\underset{\text{Base}}{\ce{Ac-}}$}$
	\begin{center}
		\begin{tabular}{lccc}
			&	\multicolumn{3}{c}{\ce{HAc(ac) + KOH(ac) -> K+(ac) + Ac-(ac) + H2O(ac)}}									\\
			\midrule
			Estado	&	$n(\ce{HAc})$~(\si{\mol})		&	$n(\ce{KOH})$~(\si{\mol})	&	$n(\ce{Ac-})$~(\si{\mol})		\\
			\midrule
			Inicial	&	\num{3,546e-3}					&	\num{1,2e-3}				&	\num{6,454e-3}					\\
			Final	&	$\num{3,546e-3}-\num{1,2e-3}$	&	\num{0}						&	$\num{6,454e-3}+\num{1,2e-3}$	\\
					&		=\num{2,346e-3}				&								&		=\num{7,654e-3}				\\
			\bottomrule
		\end{tabular}
	\end{center}
	$$
		\pH = \overbrace{\pKa}^{\num{4,74}} + \log(\frac{\rfrac{n(\ce{Ac-})}{\cancel{V}}}{\rfrac{n(\ce{HAc})}{\cancel{V}}})\Rightarrow
		\pH = \num{4,74} + \log(\frac{\num{7,654e-3}}{\num{2,346e-3}})\Rightarrow
		\tcbhighmath[boxrule=0.4pt,arc=4pt,colframe=orange,drop fuzzy shadow=green]{\pH = \num{5,26}}
	$$
\end{frame}