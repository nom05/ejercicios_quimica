\begin{frame}
    \frametitle{\ejerciciocmd}
    \framesubtitle{Enunciado}
    \textbf{
		Una reacción tiene una constante de velocidad de \SI{,017}{\per\second} a \SI{298}{\kelvin} y una energía libre de activación del \SI{27,235}{\kilo\joule\per\mol}. La adición de un catalizador disminuye dicha energía de activación hasta un \SI{33}{\percent} de su valor inicial. Calcule la nueva constante de velocidad.
\resultadocmd{ \SI{26,86}{\per\second} }

	}
\end{frame}

\begin{frame}
    \frametitle{\ejerciciocmd}
    \framesubtitle{Datos del enunciado}
    {\Large \begin{enumerate}[label={\alph*)},font={\color{red!50!black}\bfseries}]
        \item ¿$K_h(\ce{NH4+}) = K_a(\ce{NH4+})$?
        \item ¿$\alpha(\ce{NH4+})$?
        \item ¿pH?
    \end{enumerate}}
    $$
        \tcbhighmath[boxrule=0.4pt,arc=4pt,colframe=red,drop fuzzy shadow=blue]{[\ce{NH4Cl}]_0 = \SI{,1}{\Molar}}\quad
        \tcbhighmath[boxrule=0.4pt,arc=4pt,colframe=red,drop fuzzy shadow=blue]{K_b(\ce{NH3}) = \SI{1,8e-5}{}}
    $$
\end{frame}

\begin{frame}
    \frametitle{\ejerciciocmd}
    \framesubtitle{Resolución (\rom{1}): determinación de $K_h(\ce{NH4+})$}
    \structure{Disociación de ácido débil:} \underline{PARCIAL} implicando estas reacciones:\\
    \begin{equation}\label{re:dis_NH4+}
	    \ce{NH4+(ac) + H2O(l) <=> NH3(ac)  + H3O+(ac)}
    \end{equation}
	\begin{equation}\label{re:dis_NH3}
    	\ce{NH3(ac) + H2O(l) <=> NH4+(ac) + OH-(ac)}
	\end{equation}
	\begin{equation}\label{re:dis_agua}
		\ce{2H2O(l) <=> H3O+(ac) + OH-(ac)}
	\end{equation}
    \visible<2->{
        \structure{Combinando:} la reacción opuesta de la segunda $-$\eqref{re:dis_NH3} sumada a la disociación del agua \eqref{re:dis_agua} nos da la primera ecuación \eqref{re:dis_NH4+}
        $$
            \ce{NH4+(ac) + \cancel{\ce{OH-(ac)}} -> NH3(ac) + \cancel{\ce{H2O(l)}}}\quad K^{-1}_b(\ce{NH3}) = \frac{[\ce{NH3}]}{[\ce{NH4+}][\ce{OH-}]}
        $$
        $$
                \ce{\cancel{2}H2O(l) -> H3O+(ac) + \cancel{\ce{OH-(ac)}}}\quad K_w = [\ce{H3O+}][\ce{OH-}]
        $$
        \hrulefill
        $$
            \ce{NH4+(ac) + H2O(l) <=> NH3(ac)  + H3O+(ac)}\quad K_h = \frac{[\ce{NH3}]}{[\ce{NH4+}]\cancel{[\ce{OH-}]}}\cdot [\ce{H3O+}]\cancel{[\ce{OH-}]}
        $$
                }
    \visible<3->{
        Que se corresponde con la constante de equilibrio de la reacción \eqref{re:dis_NH4+}. \textbf{Por tanto:}
        $$
            \tcbhighmath[boxrule=0.4pt,arc=4pt,colframe=red,drop fuzzy shadow=blue]{K_h = \frac{\SI{1e-14}{}}{\SI{1,8e-5}{}} = \SI{5,6e-10}{}}
        $$
                }
\end{frame}

\begin{frame}
    \frametitle{\ejerciciocmd}
    \framesubtitle{Resolución (\rom{2}): determinación de $\alpha(\ce{NH4+})$}
    \structure{Reacción:} \ce{NH4+(ac) + H2O(l) <=> NH3(ac)  + H3O+(ac)}~~~~$[\ce{NH4+}]_0 = \SI{,1}{\Molar}$
    \visible<2->{
        \begin{center}
            \begin{tabular}{cccc}
                \toprule
                Estado  & [\ce{NH4+}] (\si{\Molar}) & [\ce{NH3}] (\si{\Molar}) & [\ce{H3O+}] (\si{\Molar}) \\
                \midrule
                Inicial & $[\ce{NH4+}]_0$                  & 0                            & 0                \\
                Final 1  & $[\ce{NH4+}]_0-\overbrace{x}^{[\ce{NH4+}]_0\cdot\alpha(\ce{NH4+})}$                & $x$ & $x$ \\
                Final 2  & $[\ce{NH4+}]_0(1-\alpha(\ce{NH4+}))$                & $[\ce{NH4+}]_0\cdot\alpha(\ce{NH4+})$ & $[\ce{NH4+}]_0\cdot\alpha(\ce{NH4+})$ \\
                \bottomrule
            \end{tabular}
        \end{center}
                }
    \visible<3->{
        \begin{overprint}
            \onslide<3>
                $$
                    K_h = \frac{[\ce{NH3}][\ce{H3O+}]}{[\ce{NH4+}]}
                $$
            \onslide<4>
                $$
                    K_h = \frac{[\ce{NH3}][\ce{H3O+}]}{[\ce{NH4+}]} = 
                    \frac{[\ce{NH4+}]^{\cancel{2}}_0\cdot\alpha(\ce{NH4+})^2}{\cancel{[\ce{NH4+}]_0}(\underbrace{1-\alpha(\ce{NH4+}))}_{\approx 1}}
                $$
            \onslide<5>
                $$
                    K_h = \frac{[\ce{NH3}][\ce{H3O+}]}{[\ce{NH4+}]} = 
                    \frac{[\ce{NH4+}]_0\cdot\alpha(\ce{NH4+})^2}{1}
                $$
            \onslide<6->
                $$
                    \alpha(\ce{NH4+}) = +\sqrt{\frac{K_w}{K_b(\ce{NH3})\cdot[\ce{NH3}]_0}}\Rightarrow
                    \tcbhighmath[boxrule=0.4pt,arc=4pt,colframe=red,drop fuzzy shadow=blue]{\alpha(\ce{NH4+}) = +\sqrt{\frac{\SI{1e-14}{}}{\SI{1,8e-5}{}\cdot\SI{,1}{\Molar}}} = \SI{7,5e-5}{}}
                $$
        \end{overprint}
                }
\end{frame}

\begin{frame}
    \frametitle{\ejerciciocmd}
    \framesubtitle{Resolución (\rom{3}): determinación de $pH$}
    \structure{Con la definición de $pH$:}
    \begin{overprint}
        \onslide<1>
            $$
                pH = -\log[\ce{H+}] = -\log\left(\alpha(\ce{NH4+})\cdot[\ce{NH4+}]_0\right)
            $$
        \onslide<2>
            $$
                pH = -\log[\ce{H+}] = -\log\left(\alpha(\ce{NH4+})\cdot[\ce{NH4+}]_0\right) = -\log\left(\alpha(\ce{NH4+})\right)-\log\left([\ce{NH4+}]_0\right)
            $$
        \onslide<3>
            $$
                pH = -\log[\ce{H+}] = -\log\left(\alpha(\ce{NH4+})\cdot[\ce{NH4+}]_0\right) = -\log\left(\alpha(\ce{NH4+})\right)-\log\left(\SI{,1}{}\right)
            $$
        \onslide<4>
            $$
                pH = -\log[\ce{H+}] = -\log\left(\alpha(\ce{NH4+})\cdot[\ce{NH4+}]_0\right) = -\log\left(\alpha(\ce{NH4+})\right)+1
            $$
    \end{overprint}
    \visible<4>{
        $$
            \tcbhighmath[boxrule=0.4pt,arc=4pt,colframe=red,drop fuzzy shadow=blue]{pH = \SI{5,13}{}}
        $$
                }
\end{frame}

