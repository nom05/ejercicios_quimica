\begin{frame}
    \frametitle{\ejerciciocmd}
    \framesubtitle{Enunciado}
    \textbf{
		Dadas las siguientes reacciones:
\begin{itemize}
    \item \ce{I2(g) + H2(g) -> 2 HI(g)}~~~$\Delta H_1 = \SI{-0,8}{\kilo\calorie}$
    \item \ce{I2(s) + H2(g) -> 2 HI(g)}~~~$\Delta H_2 = \SI{12}{\kilo\calorie}$
    \item \ce{I2(g) + H2(g) -> 2 HI(ac)}~~~$\Delta H_3 = \SI{-26,8}{\kilo\calorie}$
\end{itemize}
Calcular los parámetros que se indican a continuación:
\begin{description}%[label={\alph*)},font={\color{red!50!black}\bfseries}]
    \item[\texttt{a)}] Calor molar latente de sublimación del yodo.
    \item[\texttt{b)}] Calor molar de disolución del ácido yodhídrico.
    \item[\texttt{c)}] Número de calorías que hay que aportar para disociar en sus componentes el yoduro de hidrógeno gas contenido en un matraz de \SI{750}{\cubic\centi\meter} a \SI{25}{\celsius} y \SI{800}{\torr} de presión.
\end{description}
\resultadocmd{\SI{12,8}{\kilo\calorie}; \SI{-13,0}{\kilo\calorie}; \SI{12,9}{\calorie}}

	}
\end{frame}

\begin{frame}
    \frametitle{\ejerciciocmd}
    \framesubtitle{Datos del ejercicio}
    \begin{center}
        {\huge \textbf{¿pH?}}
    \end{center}
    $$
        \tcbhighmath[boxrule=0.4pt,arc=4pt,colframe=green,drop fuzzy shadow=yellow]{V(\ce{HCN}) = \SI{500}{\milli\liter}}\quad
        \tcbhighmath[boxrule=0.4pt,arc=4pt,colframe=green,drop fuzzy shadow=yellow]{[\ce{HCN}] = \SI{,5}{\Molar}}\quad
        \tcbhighmath[boxrule=0.4pt,arc=4pt,colframe=green,drop fuzzy shadow=yellow]{K_a(\ce{HCN})= \SI{4,9e-10}{}}
    $$
    $$
        \tcbhighmath[boxrule=0.4pt,arc=4pt,colframe=blue,drop fuzzy shadow=green]{V(\ce{NaOH}) = \SI{500}{\milli\liter}}\quad
        \tcbhighmath[boxrule=0.4pt,arc=4pt,colframe=blue,drop fuzzy shadow=green]{[NaOH] = \SI{0,5}{\Molar}}
    $$
\end{frame}

\begin{frame}
    \frametitle{\ejerciciocmd}
    \framesubtitle{Resolución (\rom{1}): pH de disolución en el punto de equivalencia}
    \structure{Tipo de reacción:} base fuerte + ácido débil
    $$
        n(HCN) = \SI{,5}{\mol\per\cancel\liter}\vdot\SI{,5}{\cancel\liter} = \SI{2,5e-1}{\mol}
    $$
    $$
        n(NaOH) = \SI{,5}{\mol\per\cancel\liter}\vdot\SI{,5}{\cancel\liter} = \SI{2,5e-1}{\mol}
    $$
    \structure{Reacción de neutralización:} \ce{HCN(ac) + NaOH(ac) -> NaCN(ac) + H2O(l)}\\[.3cm]
    \visible<2->{
        Como $n(HCN) = n(NaOH)$, como la reacción es \textbf{\underline{mol a mol}} y estamos en el \textbf{\underline{punto de equivalencia}}.
                }
    \visible<3->{
        El volumen total será: $V_T = \SI{,5}{\liter} + \SI{,5}{\liter} = \SI{1,0}{\liter}$
        \structure{Al final de la reacción:} $n(\ce{CN-}) = \SI{2,5e-1}{\mol}\Rightarrow[\ce{CN-}] = \SI{2,5e-1}{\Molar}$
                }
    \visible<4->{
        \structure{Por tanto tendremos:} \ce{CN-(ac) + H2O(l) <=> HCN(ac) + OH-(ac)}
        \begin{overprint}
            \onslide<4>
                $$
                    K_h(CN-) = \frac{K_w}{K_a(HCN)} = \frac{[\ce{HCN}][\ce{OH-}]}{[\ce{CN-}]}
                $$
            \onslide<5>
                \structure{Despreciamos la disociación frente a la concentración inicial de sal conjugada:}
                    $$
                        K_h(CN-) = \frac{K_w}{K_a(HCN)} = \frac{[\ce{HCN}][\ce{OH-}]}{[\ce{CN-}]_0}
                    $$
            \onslide<6>
                \structure{Usando logaritmos:}
                    $$
                        \log\frac{K_w}{K_a(\ce{HCN})} = \log\frac{\overbrace{[\ce{HCN}]}^{[\ce{HCN}]=[\ce{OH-}]}[\ce{OH-}]}{[\ce{CN-}]_0}
                    $$
            \onslide<7>
                    $$
                        -14 -\log{K_a(\ce{HCN})} = \log\frac{[\ce{OH-}]^2}{[\ce{CN-}]_0}
                    $$
            \onslide<8>
                    $$
                        -14 -\log{K_a(\ce{HCN})} -2\log{[\ce{OH-}]} = -\log{[\ce{CN-}]_0}
                    $$
            \onslide<9>
                    $$
                        -14 +pK_a(\ce{HCN}) + 2\overbrace{pOH}^{pOH = 14-pH} = -\log{[\ce{CN-}]_0}
                    $$
            \onslide<10>
                    $$
                        -14 +pK_a(\ce{HCN}) + 2\vdot(14-pH) = -\log{[\ce{CN-}]_0}
                    $$
            \onslide<11>
                    $$
                        pK_a(\ce{HCN}) + 14 - 2\vdot pH = -\log{[\ce{CN-}]_0}
                    $$
            \onslide<12>
                    $$
                        pH = 7 + \frac{1}{2}\left( pK_a(\ce{HCN}) +\log{[\ce{CN-}]_0}\right)
                    $$
        \end{overprint}
                }
    \visible<12>{
        $$
            \tcbhighmath[boxrule=0.4pt,arc=4pt,colframe=green,drop fuzzy shadow=yellow]{pH = \SI{11,35}{}}
        $$
                }
\end{frame}
