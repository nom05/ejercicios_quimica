\begin{frame}
	\frametitle{\ejerciciocmd}
	\framesubtitle{Enunciado}
	\textbf{
			Una reacción tiene una constante de velocidad de \SI{,017}{\per\second} a \SI{298}{\kelvin} y una energía libre de activación del \SI{27,235}{\kilo\joule\per\mol}. La adición de un catalizador disminuye dicha energía de activación hasta un \SI{33}{\percent} de su valor inicial. Calcule la nueva constante de velocidad.
\resultadocmd{ \SI{26,86}{\per\second} }

		}
\end{frame}

\begin{frame}
	\frametitle{\ejerciciocmd}
	\framesubtitle{Datos del problema}
	\begin{center}
		{\huge ¿\% disociación?}\\[.4cm]
		\tcbhighmath[boxrule=0.4pt,arc=4pt,colframe=green,drop fuzzy shadow=blue]{[\ce{NH4Cl}]=\SI{,0100}{\Molar}}\quad
		\tcbhighmath[boxrule=0.4pt,arc=4pt,colframe=red,drop fuzzy shadow=black]{K_b(\ce{NH3})=\num{1,75e-5}}
	\end{center}
\end{frame}

\begin{frame}
	\frametitle{\ejerciciocmd}
	\framesubtitle{Resolución (\rom{1}): obtención de $K_h(\ce{NH4+})$}
	\structure{Reacciones a tener en cuenta:}
	\begin{center}
		El \ce{NH4Cl} se disuelve en agua:
		\ce{NH4Cl(ac) -> NH4+(ac) + Cl-}
	\end{center}
	\begin{itemize}
		\item Ion \ce{Cl-} viene de un ácido fuerte. No tiene carácter básico.
		\item Ion \ce{NH4+} proviene del \ce{NH3}, base débil. En el enunciado nos dan su $K_b(\ce{NH3})$. \ce{NH4+} se comporta como ácido débil.
	\end{itemize}
	\structure{Equilibrio de disociación del \ce{NH4+}:}
		$$
			\ce{NH4+(ac) + H2O(l) <=> NH3(ac) + H3O+(ac)}\quad
			K_h(\ce{NH4+}) = \frac{[\ce{NH3}][\ce{H3O+}]}{[\ce{NH4+}]}
		$$
\end{frame}

\begin{frame}
	\frametitle{\ejerciciocmd}
	\framesubtitle{Resolución (\rom{2}): obtención de $K_h(\ce{NH4+})$}
	\structure{Cómo obtener la constante de hidrólisis del \ce{NH4+}:}\\[.2cm]
	\begin{enumerate}[label={Paso \arabic*.},font=\bfseries]
		\item Partimos de la disociación de \ce{NH3}:
			$$
				\ce{NH3(ac) + H2O(l) <=> NH4+(ac) + OH-(ac)}\quad
				K_b(\ce{NH3}) = \frac{[\ce{NH4+}][\ce{OH-}]}{[\ce{NH3}]}
			$$
		\item Invertimos la reacción de disociación de \ce{NH3}:
			$$
				\ce{NH4+(ac) + OH-(ac) <=> NH3(ac) + H2O(l)}\quad
				\frac{1}{K_b(\ce{NH3})} = \frac{[\ce{NH3}]}{[\ce{NH4+}][\ce{OH-}]}
			$$
		\item Sumamos disociación de \ce{H2O}:
			$$
				\text{\textcolor{blue}{\ce{2H2O(l) <=> H3O+(ac) + OH-(ac)}}}\quad
				K_w(\ce{H2O}) = [\ce{H3O+}][\ce{OH-}]
			$$
			$$
				\ce{NH4+(ac) + \cancel{\ce{OH-(ac)}} + \textcolor{blue}{\ce{\cancel{2}H2O(l)}} <=> NH3(ac) + \cancel{\ce{H2O(l)}} + \textcolor{blue}{\ce{H3O+(ac)}} + \textcolor{blue}{\cancel{\ce{OH-(ac)}}}}
			$$
			$$
				K_h(\ce{NH4+}) = \frac{1}{K_b(\ce{NH3})}\cdot K_w = \frac{[\ce{NH3}][\ce{H3O+}]\cancel{[\ce{OH-}]}}{[\ce{NH4+}]\cancel{[\ce{OH-}]}}
			$$
		\item Resultado final:
			$$
				\centering\ce{NH4+(ac) + H2O(l) <=> NH3(ac) + H3O+(ac)}\quad
				K_h(\ce{NH4+}) = \frac{\overbrace{K_w}^{\num{1e-14}}}{\underbrace{K_b(\ce{NH3})}_{\num{1,75e-5}}} = \frac{[\ce{NH3}][\ce{H3O+}]}{[\ce{NH4+}]}
			$$
			\begin{center}
				\myovalbox{\textcolor{yellow}{$K_h(\ce{NH4+}) = \frac{4}{7}\times\num{e-9}$}}
			\end{center}
	\end{enumerate}
\end{frame}

\begin{frame}
	\frametitle{\ejerciciocmd}
	\framesubtitle{Resolución (\rom{2}): determinación de \% de disociación}
	\begin{center}
		\begin{tabular}{cccc}
			\toprule
				Concentración(\si{\Molar}) & [\ce{NH4+}]               & [\ce{NH3}] & [\ce{H3O+}]\\
				{[Inicial]}                & $[\ce{NH4+}]_0=\num{,01}$ &       0	&     0      \\
				{[Equilibrio]}             & $[\ce{NH4+}]_0-x$         &      $x$   &    $x$     \\
			\bottomrule
		\end{tabular}
	\end{center}
	\structure{Habíamos visto en ejercicios anteriores:}
		$$
			\alpha = \frac{[\ce{H3O+}]}{[\ce{NH4+}]_0}\quad\text{correspondiéndose }x=[\ce{H3O+}]
		$$
	\structure{Sustituyendo en la expresión de la constante de equilibrio:}
		$$
			K_h(\ce{NH4+})=\frac{[\ce{NH3}][\ce{H3O+}]}{[\ce{NH4+}]}\Rightarrow
			K_h(\ce{NH4+})=\frac{\overbrace{x^2}^{x^2=[\ce{NH4+}]_0^2\cdot\alpha^2}}{[\ce{NH4+}]_0-\underbrace{x}_{x=[\ce{NH4+}]_0\cdot\alpha}}
		$$
		$$
			K_h(\ce{NH4+})=\frac{\cancel{[\ce{NH4+}]_0}\vdot[\ce{NH4+}]_0\vdot\alpha^2}{\cancel{[\ce{NH4+}]_0}\vdot(\underbrace{1-\alpha}_{\text{suponemos }1>>\alpha})}\approx
			\frac{\overbrace{[\ce{NH4+}]_0}^{[\ce{NH4+}]_0=\SI{1e-2}{\Molar}}\cdot\alpha^2}{1}\Rightarrow
            \alpha=\sqrt{\underbrace{\frac{4}{7}\times\num{e-9}}_{K_h}\times\underbrace{\num{1e-2}}_{[\ce{NH4+}]_0}}
		$$

			\centering\tcbhighmath[boxrule=0.4pt,arc=4pt,colframe=green,drop fuzzy shadow=blue]{\alpha=\num{2,4e-4}\Rightarrow\text{\% disociación}=\SI{,024}{\percent}}
\end{frame}
