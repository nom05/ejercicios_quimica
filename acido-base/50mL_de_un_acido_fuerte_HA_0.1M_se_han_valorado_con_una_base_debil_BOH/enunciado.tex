\SI{50}{\milli\liter} de un ácido fuerte \ce{HA} \SI{,1000}{\Molar} se han valorado con una base débil \ce{BOH} ($K_b=\num{3e-5}$), obteniéndose la correspondiente curva de valoración. De ella se seleccionan tres puntos que corresponden a volúmenes de base de:
\begin{itemize}
	\item\SI{65}{\milli\liter}
	\item\SI{75}{\milli\liter} (punto de equivalencia)
	\item\SI{85}{\milli\liter}
	\item Además, al volumen del apartado anterior se diluye con \SI{10}{\milli\liter} de agua a mayores.
\end{itemize}
Calcule en cada caso el pH. Suponga que los volúmenes son aditivos.
\resultadocmd{
			\num{2,24};
			\num{5,33};
			\num{9,86};
			\num{9,86}
		}
