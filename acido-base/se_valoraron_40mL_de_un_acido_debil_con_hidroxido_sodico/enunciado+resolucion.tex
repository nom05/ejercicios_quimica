\begin{frame}
	\frametitle{\ejerciciocmd}
	\framesubtitle{Enunciado}
	\textbf{
		Dadas las siguientes reacciones:
\begin{itemize}
    \item \ce{I2(g) + H2(g) -> 2 HI(g)}~~~$\Delta H_1 = \SI{-0,8}{\kilo\calorie}$
    \item \ce{I2(s) + H2(g) -> 2 HI(g)}~~~$\Delta H_2 = \SI{12}{\kilo\calorie}$
    \item \ce{I2(g) + H2(g) -> 2 HI(ac)}~~~$\Delta H_3 = \SI{-26,8}{\kilo\calorie}$
\end{itemize}
Calcular los parámetros que se indican a continuación:
\begin{description}%[label={\alph*)},font={\color{red!50!black}\bfseries}]
    \item[\texttt{a)}] Calor molar latente de sublimación del yodo.
    \item[\texttt{b)}] Calor molar de disolución del ácido yodhídrico.
    \item[\texttt{c)}] Número de calorías que hay que aportar para disociar en sus componentes el yoduro de hidrógeno gas contenido en un matraz de \SI{750}{\cubic\centi\meter} a \SI{25}{\celsius} y \SI{800}{\torr} de presión.
\end{description}
\resultadocmd{\SI{12,8}{\kilo\calorie}; \SI{-13,0}{\kilo\calorie}; \SI{12,9}{\calorie}}

		}
\end{frame}

\begin{frame}
	\frametitle{\ejerciciocmd}
	\framesubtitle{Datos del problema}
	\begin{center}
		{\huge ¿$\mathrm{pH}$ a \SI{10}{\milli\liter}, \SI{20}{\milli\liter} y \SI{23}{\milli\liter}?}\\[.3cm]
		\tcbhighmath[boxrule=0.4pt,arc=4pt,colframe=green,drop fuzzy shadow=blue]{V(\ce{HA})=\SI{40e-3}{\liter}}
		\tcbhighmath[boxrule=0.4pt,arc=4pt,colframe=green,drop fuzzy shadow=blue]{[\ce{HA}]_0=\SI{1e-2}{\Molar}}
		\tcbhighmath[boxrule=0.4pt,arc=4pt,colframe=green,drop fuzzy shadow=blue]{K_a(\ce{HA})=\num{1,75e-5}}\\[.3cm]
		\tcbhighmath[boxrule=0.4pt,arc=4pt,colframe=blue,drop fuzzy shadow=red]{[\ce{NaOH}]=\SI{2e-2}{\Molar}}
	\end{center}
\end{frame}

\begin{frame}
	\frametitle{\ejerciciocmd}
	\framesubtitle{Resolución (\rom{1}): reacciones implicadas}
	\alert{\textbf{Las FLECHAS son importantes. Hay reacciones irreversibles y equilibrios que influyen en la resolución del ejercicio.}}
	\structure{Equilibrio del ácido débil:}\\
	\begin{center}
		\ce{HA(ac) <=> H+(ac) + A-(ac)}
	\end{center}
	\structure{Al añadir \ce{NaOH} ocurre la reacción de neutralización (ácido+base dan sal+agua):}\\
	\begin{center}
		\ce{HA(ac) + NaOH(ac) -> NaA(ac) + H2O(l)}
	\end{center}
	\structure{Disolución de la sal. No aclaran que sea insoluble. Por tanto:}
	\begin{center}
		\ce{NaA(ac) -> Na+(ac) + A-(ac)}
	\end{center}
	\begin{itemize}
		\item\ce{Na+} proviene de una base fuerte. No tiene actividad ácido-base.
		\item\ce{A-}, que está en el equilibrio del ácido débil, tiene carácter básico débil.
	\end{itemize}
	\alert{\textbf{Antes de calcular el pH (usando concentraciones) los ejercicios de volumetría se resuelven como los ejercicios de estequiometría (reactivo limitante y reactivo en exceso):}} tenemos que trabajar con \underline{número de moles}.
	\structure{A partir del punto de equivalencia:} impera el equilibrio del ion conjugado (básico) porque no queda \ce{HA}.
	\begin{center}
		\ce{A-(ac) + H2O(l) <=> HA(ac) + OH-(ac)}
	\end{center}
\end{frame}

\begin{frame}
	\frametitle{\ejerciciocmd}
	\framesubtitle{Resolución (\rom{2}): $\mathrm{pH}$ si añadimos \SI{10}{\milli\liter} de \ce{NaOH} (\rom{1})}
		\begin{enumerate}[label={Paso \arabic*.},font=\bfseries]
			\item\structure{Averiguamos quién es el reactivo limitante}
				$$
					n(\ce{HA})=\SI{e-2}{\mol\per\cancel\liter}\times\SI{40e-3}{\cancel\liter}=\SI{4e-4}{\mol}
				$$
				$$
					n(\ce{NaOH})=\SI{2e-2}{\mol\per\cancel\liter}\times\SI{10e-3}{\cancel\liter}=\SI{2e-4}{\mol}
				$$\\[.2cm]
				\begin{center}
					\ce{HA(ac) + NaOH(ac) -> NaA(ac) + H2O(l)}
				\end{center}
				\alert{Relación estequiométrica:} $n_{\text{reaccionan}}(\ce{HA})=n_{\text{reaccionan}}(\ce{NaOH})=n_{\text{producen}}(\ce{NaA})$
				\begin{center}
					\begin{tabular}{cccc}
						\toprule
						N"o de moles~(\si{\mol}) & n(\ce{HA})      					   &  n(\ce{NaOH})    & n(\ce{NaA})  \\
						{Inicial}               & \num{4e-4}     					   &   \num{2e-4}     &  0           \\
						{Final}                 &$\num{4e-4}-\num{2e-4}=\num{2e-4}$    &    	0		  & \num{2e-4}   \\
						\bottomrule
					\end{tabular}
					\myovalbox{\textcolor{white}{
							\ce{NaOH} es el reactivo limitante, \ce{HA} es el reactivo en exceso
					}}
				\end{center}
			\item\structure{Obtenemos el volumen total:}
				$$
					V_T = V_{\text{inicial}}(\ce{HA}) + V_{\text{añadido}}(\ce{NaOH})\Rightarrow V_T = \SI{40e-3}{\liter} + \SI{10e-3}{\liter} = \SI{50e-3}{\liter}
				$$
			\item\structure{Concentración de nuestro reactivo en exceso (\ce{HA}) y de nuestro producto (\ce{A-}):} tenemos un ácido débil y su ion conjugado sin contar con la disociación.
				\begin{center}
					\myovalbox{\textcolor{red}{\textbf{DISOLUCIÓN AMORTIGUADORA}}}
				\end{center}
				$$
					[\ce{HA}]=\frac{\SI{2e-4}{\mol}}{\SI{50e-3}{\liter}}=\SI{4e-3}{\Molar};\qquad
					[\ce{A-}]=\frac{\SI{2e-4}{\mol}}{\SI{50e-3}{\liter}}=\SI{4e-3}{\Molar}
				$$
	\end{enumerate}
\end{frame}	
	
\begin{frame}
	\frametitle{\ejerciciocmd}
	\framesubtitle{Resolución (\rom{3}): $\mathrm{pH}$ si añadimos \SI{10}{\milli\liter} de \ce{NaOH} (\rom{2})}
	\begin{enumerate}[label={Paso \arabic*.},font=\bfseries]
		\setcounter{enumi}{3}
		\item\structure{Cálculo del pH:} tenemos que usar $K_a$.
			\begin{center}
				\begin{tabular}{cccc}
													& \multicolumn{3}{c}{\ce{HA(ac) <=> H+(ac) + A-(ac)}}	\\
					\midrule
						Concentración~(\si{\Molar}) & [\ce{HA}]			&  [\ce{H+}] 	& [\ce{A-}]			\\
						{[Inicial]}					& \num{4e-3}		&	$x$			&  \num{4e-3}		\\
						{[Equilibrio]}				&$\num{4e-3}-x$ 	& 	$x$			& $\num{4e-3}+x$ 	\\
					\bottomrule
				\end{tabular}
			\end{center}
			Podemos resolverlo de dos formas:
			\begin{enumerate}[label={\alph*)},font=\bfseries]
				\item\structure{Sustituimos valores en la ecuación de la constante de equilibrio:} Podemos considerar $x$ despreciable si $K_a<\num{e-4}$.
					$$
						\overbrace{K_a(\ce{HA})}^{\num{1,75e-5}}=\frac{\overbrace{[\ce{H+}]}^{x}\overbrace{[\ce{A-}]}^{\num{4e-3}+x}}{\underbrace{[\ce{HA}]}_{\num{4e-3}-x}}=
						\frac{x\vdot(\num{4e-3}+x)}{\num{4e-3}-x}\approx\frac{x\vdot\cancel{\num{4e-3}}}{\cancel{\num{4e-3}}}\Rightarrow x=\SI{1,75e-5}{\Molar}
					$$
				\item\structure{Como segunda opción, podemos tomar logaritmos decimales y operar:} Ecuación de HENDERSON -- HASSELBALCH
					$$
						\log K_a(\ce{HA})=\log\left(\frac{[\ce{H+}][\ce{A-}]}{[\ce{HA}]}\right)\Rightarrow\log K_a(\ce{HA})\overbrace{-\log[\ce{H+}]}^{\mathrm{pH}}=\underbrace{-\log K_a(\ce{HA})}_{\mathrm{p}K_a}+\log\left(\frac{[\ce{A-}]}{[\ce{HA}]}\right)
					$$
					$$
						\pH = \pKa + \log(\frac{[\ce{A-}]}{[\ce{HA}]})\Rightarrow
						\pH = \pKa + \log(\frac{[\ce{A-}]_0}{[\ce{HA}]_0})\Rightarrow
						\pH = \underbrace{\pKa}_{\num{4,76}} + \log(\frac{\rfrac{n_0(\ce{A-})}{\cancel{V}}}{\rfrac{n_0(\ce{HA})}{\cancel{V}}})
					$$
					\begin{center}
						\tcbhighmath[boxrule=0.4pt,arc=4pt,colframe=green,drop fuzzy shadow=blue]{\pH=\num{4,76}}
					\end{center}
			\end{enumerate}
	\end{enumerate}
\end{frame}

\begin{frame}
	\frametitle{\ejerciciocmd}
	\framesubtitle{Resolución (\rom{4}): $\mathrm{pH}$ si añadimos \SI{20}{\milli\liter} de \ce{NaOH} (\rom{1})}
	\begin{enumerate}[label={Paso \arabic*.},font=\bfseries]
		\item\structure{Averiguamos quién es el reactivo limitante}
		$$
			n(\ce{HA})=\SI{e-2}{\mol\per\cancel\liter}\times\SI{40e-3}{\cancel\liter}=\SI{4e-4}{\mol}
		$$
		$$
			n(\ce{NaOH})=\SI{2e-2}{\mol\per\cancel\liter}\times\SI{20e-3}{\cancel\liter}=\SI{4e-4}{\mol}
		$$\\[.2cm]
		\begin{center}
			\ce{HA(ac) + NaOH(ac) -> NaA(ac) + H2O(l)}
		\end{center}
		\alert{Relación estequiométrica:} $n_{\text{reaccionan}}(\ce{HA})=n_{\text{reaccionan}}(\ce{NaOH})=n_{\text{producen}}(\ce{NaA})$
		\begin{center}
			\begin{tabular}{cccc}
				\toprule
				N"o de moles~(\si{\mol}) & n(\ce{HA})      						&  n(\ce{NaOH})    & n(\ce{NaA})  \\
				{Inicial}               & \num{4e-4}     						&   \num{4e-4}     &  0           \\
				{Final}                 &$\num{4e-4}-\num{4e-4}=\num{0}$		&    	0		  & \num{4e-4}   \\
				\bottomrule
			\end{tabular}
			\myovalbox{\textcolor{white}{
					No hay r. limitante ni r. en exceso, \underline{PUNTO DE EQUIVALENCIA}
			}}
			\myovalbox{\textcolor{red}{pH controlado por \ce{A-}, \textbf{pH BÁSICO}}}
		\end{center}
		\item\structure{Obtenemos el volumen total:}
		$$
			V_T = V_{\text{inicial}}(\ce{HA}) + V_{\text{añadido}}(\ce{NaOH})\Rightarrow V_T = \SI{40e-3}{\liter} + \SI{20e-3}{\liter} = \SI{60e-3}{\liter}
		$$
		\item\structure{Concentración de nuestro producto (\ce{A-}):} tenemos el ion conjugado de un ácido débil.
		$$
			[\ce{A-}]=\frac{\SI{4e-4}{\mol}}{\SI{60e-3}{\liter}}=\SI{6,67e-3}{\Molar}
		$$
	\end{enumerate}
\end{frame}	

\begin{frame}
	\frametitle{\ejerciciocmd}
	\framesubtitle{Resolución (\rom{5}): $\mathrm{pH}$ si añadimos \SI{20}{\milli\liter} de \ce{NaOH} (\rom{2})}
	\begin{enumerate}[label={Paso \arabic*.},font=\bfseries]
		\setcounter{enumi}{3}
		\item\structure{Cálculo del pH:} tenemos que usar $K_h=\rfrac{K_w}{K_a}$, $\pH>7$.
		\begin{center}
			\begin{tabular}{cccc}
				& \multicolumn{3}{c}{\ce{A-(ac) + H2O(l) <=> OH-(ac) + HA(ac)}}	\\
				\midrule
				Concentración~(\si{\Molar}) & [\ce{A-}]			&  [\ce{OH-}] 	& [\ce{HA}]			\\
				{[Inicial]}					& \num{6,67e-3}		&	$x$			&  $x$				\\
				{[Equilibrio]}				&$\num{6,67e-3}-x$ 	& 	$x$			& $x$ 				\\
				\bottomrule
			\end{tabular}
		\end{center}
		\structure{Sustituimos valores en la ecuación de la constante de equilibrio:} Podemos considerar $x$ despreciable si $K_h<\num{e-4}$.
			$$
				\overbrace{K_h(\ce{A-})}^{\num{5,56e-10}}=\frac{\overbrace{[\ce{OH-}]}^{x}\overbrace{[\ce{HA}]}^{x}}{\underbrace{[\ce{A-}]}_{\num{6,67e-3}-x}}=
				\frac{x^2}{\num{6,67e-3}-x}\approx\frac{x^2}{\num{6,67e-3}}\Rightarrow x=\SI{1,95e-6}{\Molar}
			$$
			\begin{center}
				\tcbhighmath[boxrule=0.4pt,arc=4pt,colframe=green,drop fuzzy shadow=blue]{\mathrm{pH}=\num{8,29}}
			\end{center}
	\end{enumerate}
\end{frame}

\begin{frame}
	\frametitle{\ejerciciocmd}
	\framesubtitle{Resolución (\rom{6}): $\mathrm{pH}$ si añadimos \SI{23}{\milli\liter} de \ce{NaOH} (\rom{1})}
	\begin{enumerate}[label={Paso \arabic*.},font=\bfseries]
		\item\structure{Averiguamos quién es el reactivo limitante}
		$$
			n(\ce{HA})=\SI{e-2}{\mol\per\cancel\liter}\times\SI{40e-3}{\cancel\liter}=\SI{4e-4}{\mol}
		$$
		$$
			n(\ce{NaOH})=\SI{2e-2}{\mol\per\cancel\liter}\times\SI{23e-3}{\cancel\liter}=\SI{4,6e-4}{\mol}
		$$\\[.2cm]
		\begin{center}
			\ce{HA(ac) + NaOH(ac) -> NaA(ac) + H2O(l)}
		\end{center}
		\alert{Relación estequiométrica:} $n_{\text{reaccionan}}(\ce{HA})=n_{\text{reaccionan}}(\ce{NaOH})=n_{\text{producen}}(\ce{NaA})$
		\begin{center}
			\begin{tabular}{cccc}
				\toprule
				N"o de moles~(\si{\mol}) & n(\ce{HA})	&  n(\ce{NaOH})								& n(\ce{NaA})  \\
				{Inicial}               & \num{4e-4}	&   \num{4,6e-4}							&  0           \\
				{Final}                 & 0				& $\num{4,6e-4}-\num{4e-4}=\num{,6e-4}$ 	& \num{4e-4}   \\
				\bottomrule
			\end{tabular}
			\myovalbox{\textcolor{white}{
							\ce{HA} es el reactivo limitante, \ce{NaOH} es el reactivo en exceso
			}}
			\myovalbox{\textcolor{red}{pH controlado por \ce{NaOH}, \textbf{pH BÁSICO}}}
		\end{center}
		\item\structure{Obtenemos el volumen total:}
			$$
				V_T = V_{\text{inicial}}(\ce{HA}) + V_{\text{añadido}}(\ce{NaOH})\Rightarrow V_T = \SI{40e-3}{\liter} + \SI{23e-3}{\liter} = \SI{63e-3}{\liter}
			$$
		\item\structure{Concentración de las dos especies básicas (\ce{A-} y \ce{NaOH}):}
			$$
				[\ce{A-}]=\frac{\SI{4e-4}{\mol}}{\SI{63e-3}{\liter}}=\SI{6,35e-3}{\Molar};\quad
				[\ce{NaOH}]=\frac{\SI{,6e-4}{\mol}}{\SI{63e-3}{\liter}}=\SI{9,52e-4}{\Molar}
			$$
		\item\structure{Cálculo del pH:} $\pOH = -\log([\ce{OH-}])$; $\pH = 14 - \pOH$
			$$
				\tcbhighmath[boxrule=0.4pt,arc=4pt,colframe=black,drop fuzzy shadow=green]{\pH=\num{10,98}}
			$$
	\end{enumerate}
\end{frame}

\begin{frame}
	\frametitle{\ejerciciocmd}
	\framesubtitle{Resolución (\rom{7}): $\mathrm{pH}$ si añadimos \SI{23}{\milli\liter} de \ce{NaOH} (\rom{2})}
	\begin{block}{ANEXO del apartado anterior}
		Alguien podría decir que nos hemos olvidado de la otra especie básica (\ce{A-}). Está en lo cierto: $[\ce{OH-}]_T = [\ce{OH-}]_{\ce{NaOH}} + [\ce{OH-}]_{\ce{A-}}$. Sin embargo, la contribución $[\ce{OH-}]_{\ce{NaOH}}$ es tan grande que afecta al equilibrio de \ce{A-} y $[\ce{OH-}]_{\ce{A-}}<<<[\ce{OH-}]_{\ce{NaOH}}$. Si [\ce{NaOH}] es del orden de [\ce{A-}] la resolución del ejercicio anterior podría parar aquí. $[\ce{OH-}]_{\ce{A-}}$ es tan baja que tendríamos que tener en cuenta la disociación del agua, que complicaría mucho el ejercicio. Para ver de qué orden numérico hablamos vamos a suponer que el agua no influye en \ce{A-} y obtener $[\ce{OH-}]_{\ce{A-}}$.
		\begin{center}
			\begin{tabular}{cccc}
				& \multicolumn{3}{c}{\ce{A-(ac) + H2O(l) <=> OH-(ac) + HA(ac)}}	\\
				\midrule
				Concentración~(\si{\Molar}) & [\ce{A-}]			&  [\ce{OH-}] 		& [\ce{HA}]			\\
				{[Inicial]}					& \num{6,35e-3}		&	\num{9,52e-4}	& \num{0}			\\
				{[Equilibrio]}				&$\num{6,35e-3}-x$ 	& $\num{9,52e-4}+x$	& $x$ 				\\
				\bottomrule
			\end{tabular}
		\end{center}
		\structure{Sustituimos valores en la ecuación de la constante de equilibrio:} Podemos considerar $x$ despreciable si $K_h<\num{e-4}$.
		$$
			\overbrace{K_h(\ce{A-})}^{\num{5,56e-10}}=\frac{\overbrace{[\ce{OH-}]}^{\num{9,52e-4}+x}\overbrace{[\ce{HA}]}^{x}}{\underbrace{[\ce{A-}]}_{\num{6,35e-3}-x}}=
			\frac{x\vdot(\num{9,52e-4}+x)}{\num{6,35e-3}-x}\approx\frac{x\vdot\num{9,52e-4}}{\num{6,35e-3}}\Rightarrow x=\SI{3,71e-9}{\Molar}
		$$
		$$
			[\ce{OH-}]_T = [\ce{OH-}]_{\ce{NaOH}} + \cancel{[\ce{OH-}]_{\ce{A-}}}\approx[\ce{OH-}]_{\ce{NaOH}}
		$$
	\end{block}
\end{frame}
