\begin{frame}
    \frametitle{\ejerciciocmd}
    \framesubtitle{Enunciado}
    \textbf{
		Una reacción tiene una constante de velocidad de \SI{,017}{\per\second} a \SI{298}{\kelvin} y una energía libre de activación del \SI{27,235}{\kilo\joule\per\mol}. La adición de un catalizador disminuye dicha energía de activación hasta un \SI{33}{\percent} de su valor inicial. Calcule la nueva constante de velocidad.
\resultadocmd{ \SI{26,86}{\per\second} }

	}
\end{frame}

\begin{frame}
    \frametitle{\ejerciciocmd}
    \framesubtitle{Datos del apartado (a): determinación de pH de disolución de \ce{H2CO3}}
    \begin{center}
        {\Large \textbf{¿pH?}}
    \end{center}
    $$
        \tcbhighmath[boxrule=0.4pt,arc=4pt,colframe=blue,drop fuzzy shadow=red]{n(\ce{HCl}) = \SI{,02}{\mol}}\quad
        \tcbhighmath[boxrule=0.4pt,arc=4pt,colframe=blue,drop fuzzy shadow=red]{V(\ce{CH3COOH}) = \SI{1}{\liter}}\quad
    $$
    $$
        \tcbhighmath[boxrule=0.4pt,arc=4pt,colframe=blue,drop fuzzy shadow=red]{K_{a_1} = \SI{4,3e-7}{}}\quad
        \tcbhighmath[boxrule=0.4pt,arc=4pt,colframe=blue,drop fuzzy shadow=red]{K_{a_2} = \SI{5,6e-11}{}}
    $$
\end{frame}

\begin{frame}
    \frametitle{\ejerciciocmd}
    \framesubtitle{Resolución (\rom{1}): determinación de pH de disolución de \ce{H2CO3}}
    {\small \structure{Concentración:} $[\ce{H2CO3}]_0 = \frac{n(\ce{H2CO3})}{V(\ce{H2CO3})} = \SI{,02}{\Molar}$}
    {\small \structure{\ce{H2CO3} es un \underline{ácido poliprótico}}:}\\
    \begin{center}
	    \ce{H2CO3(ac) <=>[$K_{a_1}$] H+(ac) + HCO3-(ac)}\\
    	\ce{HCO3-(ac) <=>[$K_{a_2}$] H+(ac) + CO3^{2-}(ac)}
    \end{center}
    \visible<2->{
        \structure{Primera disociación:}
        \begin{overprint}
            \onslide<2-4>
                \begin{center}
                    {\small \begin{tabular}{clll}
                            \toprule
                            Estado  & [\ce{H2CO3}] (\si{\Molar}) & [\ce{H+}] (\si{\Molar})& [\ce{HCO3-}] (\si{\Molar}) \\
                            \midrule
                            Inicial & $[\ce{H2CO3}]_{0}$   & 0                & 0            \\
                            Final   & $[\ce{H2CO3}]_{0}-x$ & $x$              & $x$          \\
                            \bottomrule
                    \end{tabular}}
                    $$
                        K_{a_1} = \frac{[\ce{H+}][\ce{HCO3-}]}{[\ce{H2CO3}]} =\frac{x^2}{[\ce{H2CO3}]_{0}-x}
                    $$
                \end{center}
                \visible<3-4>{
                    {\small \structure{Resolver ecuación de 2"o grado:}}
                    $$
                        x^2 -\SI{4,3e-7}{}\vdot(\SI{,02}{}-x) = 0\Rightarrow x^2 -\SI{4,3e-7}{}x -\SI{8,6e-9}{} = 0\Rightarrow x= [\ce{H+}]_1 = \SI{9,29e-5}{\Molar}
                    $$
                            }
                \visible<4>{
                    {\small \structure{Resolver ecuación de 2"o grado con aproximación:}}
                    $$
                        x^2 -\SI{4,3e-7}{}\vdot(\underbrace{\SI{,02}{}-x}_{\approx\SI{,02}{}}) = 0\Rightarrow x^2 -\SI{8,6e-9}{} = 0\Rightarrow x=\SI{9,27e-5}{\Molar} = [\ce{H+}]
                    $$
                            }
            \onslide<5>
                $$[\ce{H+}]_1 = \SI{9,27e-5}{\Molar} = [\ce{HCO3-}]_1$$
                \structure{Segunda disociación:}
                \begin{center}
                    {\small \begin{tabular}{clll}
                            \toprule
                            Estado  & \ce{HCO3-}           & \ce{H+}         & \ce{CO3^{2-}} \\
                            \midrule
                            Inicial & $[\ce{HCO3-}]_{1}$   & 0               & 0             \\
                            Final   & $[\ce{HCO3-}]_{1}-y$ & $[\ce{H+}]_1+y$ & $y$           \\
                            \bottomrule
                    \end{tabular}}
                \end{center}
            \onslide<6>
                $$[\ce{H+}]_1 = \SI{9,27e-5}{\Molar} = [\ce{HCO3-}]_1$$
                \structure{Segunda disociación:}
                $$
                    K_{a_2} = \frac{[\ce{H+}][\ce{CO3^{2-}}]}{[\ce{HCO3-}]}
                    = \frac{([\ce{H+}]_1 + y)\vdot y}{[\ce{HCO3-}]_{1}-y}
                $$
            \onslide<7>
                $$[\ce{H+}]_1 = \SI{9,27e-5}{\Molar} = [\ce{HCO3-}]_1$$
                \structure{Segunda disociación:}
                $$
                    K_{a_2} = \frac{[\ce{H+}][\ce{CO3^{2-}}]}{[\ce{HCO3-}]}
                    = \frac{(\overbrace{[\ce{H+}]_1 + y)}^{\approx[\ce{H+}]_1}\vdot y}{\underbrace{[\ce{HCO3-}]_{1}-y}_{\approx[\ce{HCO3-}]_{1}}}
                $$
            \onslide<8>
                $$[\ce{H+}]_1 = \SI{9,27e-5}{\Molar} = [\ce{HCO3-}]_1$$
                \structure{Segunda disociación:}
                $$
                    K_{a_2} = \frac{[\ce{H+}][\ce{CO3^{2-}}]}{[\ce{HCO3-}]}
                    = \frac{(\overbrace{[\ce{H+}]_1 + y)}^{\approx[\ce{H+}]_1}\vdot y}{\underbrace{[\ce{HCO3-}]_{1}-y}_{\approx[\ce{HCO3-}]_{1}}}
                    = \frac{[\ce{H+}]_1\vdot y}{[\ce{HCO3-}]_{1}}
                $$
            \onslide<9>
                $$[\ce{H+}]_1 = \SI{9,27e-5}{\Molar} = [\ce{HCO3-}]_1$$
                \structure{Segunda disociación:}
                $$
                    K_{a_2} = \frac{[\ce{H+}][\ce{CO3^{2-}}]}{[\ce{HCO3-}]}
                    = \frac{[\ce{H+}]_1\vdot y}{\underbrace{[\ce{HCO3-}]_{1}}^{[\ce{H+}]_{1}}}
                    = \frac{\cancel{[\ce{H+}]_1}\vdot y}{\cancel{[\ce{H+}]_{1}}}
                    = y
                $$
            \onslide<10>
                $$[\ce{H+}]_1 = \SI{9,27e-5}{} = [\ce{HCO3-}]_1$$
                \structure{Segunda disociación:}
                $$
                    K_{a_2} = y = [\ce{H+}]_2 = \SI{5,6e-11}{\Molar}
                $$
                \structure{pH total:}
                $$
                    [\ce{H+}] = [\ce{H+}]_1 + [\ce{H+}]_2 = \SI{9,27e-5}{\Molar} + \SI{5,6e-11}{\Molar}\approx\SI{9,27e-5}{\Molar}
                $$
                $$
                    \tcbhighmath[boxrule=0.4pt,arc=4pt,colframe=blue,drop fuzzy shadow=red]{pH = -\log[\ce{H+}] = -\log\SI{9,27e-5}{} = \SI{4,03}{}}
                $$
        \end{overprint}
                }
\end{frame}

\begin{frame}
    \frametitle{\ejerciciocmd}
    \framesubtitle{Datos del apartado (b): [\ce{OH-}] y pH de disolución de \ce{Ba(OH)2*8H2O}}
    \begin{center}
        {\Large \textbf{¿pH?}}
    \end{center}
    $$
        \tcbhighmath[boxrule=0.4pt,arc=4pt,colframe=green,drop fuzzy shadow=red]{m(\ce{Ba(OH)2*8H2O}) = \SI{2,23}{\gram}}\quad
        \tcbhighmath[boxrule=0.4pt,arc=4pt,colframe=green,drop fuzzy shadow=red]{V(\ce{Ba(OH)2*8H2O}) = \SI{50}{\milli\liter}}
    $$
    $$
        \tcbhighmath[boxrule=0.4pt,arc=4pt,colframe=green,drop fuzzy shadow=red]{Mm(\ce{Ba(OH)2*8H2O}) = \SI{315,46}{\gram\per\mol}}
    $$
\end{frame}

\begin{frame}
    \frametitle{\ejerciciocmd}
    \framesubtitle{Resolución (\rom{1}): [\ce{OH-}] y pH de disolución de \ce{Ba(OH)2*8H2O}}
    \structure{Disociación de base fuerte:} \underline{TOTAL} según reacción: \ce{Ba(OH)2(ac) -> Ba^{2+}(ac) + 2OH-(ac)}
    $$
        [\ce{Ba(OH)2*8H2O}] = \frac{m(\ce{Ba(OH)2*8H2O})}{Mm(\ce{Ba(OH)2*8H2O})\vdot V(\ce{Ba(OH)2*8H2O})}\Rightarrow[\ce{Ba(OH)2*8H2O}] = \SI{,141}{\Molar}
    $$
    \visible<2->{
        \alert{\textbf{CUIDADO:}} \SI{1}{\mol} de \ce{Ba(OH)2 ->} \SI{2}{\mol} de \ce{OH-}
        \begin{center}
            \begin{tabular}{cccc}
                \toprule
                Estado  & [\ce{Ba(OH)2}] (\si{\Molar}) & [\ce{Ba^{2+}}] (\si{\Molar}) & [\ce{OH-}] (\si{\Molar}) \\
                \midrule
                Inicial & \SI{,141}{}                  & 0                            & 0                \\
                Final   &             0                & \SI{,141}{} & \tcbhighmath[boxrule=0.4pt,arc=4pt,colframe=green,drop fuzzy shadow=red]{\SI{,282}{}} \\
                \bottomrule
            \end{tabular}
        \end{center}
                }
    \visible<3->{
        $$
            pH = 14 -\overbrace{pOH}^{pOH = -\log[\ce{OH-}]} = 14+\log[\ce{OH-}]\Rightarrow\tcbhighmath[boxrule=0.4pt,arc=4pt,colframe=green,drop fuzzy shadow=red]{pH = 14+\log\SI{,282}{}=\SI{13,45}{}}
        $$
                }
\end{frame}

\begin{frame}
    \frametitle{\ejerciciocmd}
    \framesubtitle{Datos del apartado (c): $K_b(\ce{NH3})$}
    \begin{center}
        {\Large \textbf{¿$K_b(\ce{NH3})$?}}
    \end{center}
    $$
        \tcbhighmath[boxrule=0.4pt,arc=4pt,colframe=red,drop fuzzy shadow=blue]{m(\ce{NH3}) = \SI{1}{\gram}}\quad
        \tcbhighmath[boxrule=0.4pt,arc=4pt,colframe=red,drop fuzzy shadow=blue]{V(\ce{NH3}) = \SI{610}{\milli\liter}}
    $$
    $$
        \tcbhighmath[boxrule=0.4pt,arc=4pt,colframe=red,drop fuzzy shadow=blue]{pH = \SI{11,11}{}}\quad
        \tcbhighmath[boxrule=0.4pt,arc=4pt,colframe=red,drop fuzzy shadow=blue]{Mm(\ce{NH3}) = \SI{17,03}{\gram\per\mol}}
    $$
\end{frame}

\begin{frame}
    \frametitle{\ejerciciocmd}
    \framesubtitle{Resolución (\rom{3}): $K_b(\ce{NH3})$}
    \structure{Disociación de base débil:} \underline{PARCIAL} según reacción:
    \begin{center}
        \ce{NH3(ac) + H2O(l) -> NH4+(ac) + OH-(ac)}
    \end{center}
    $$
        [\ce{NH3}]_0 = \frac{m(\ce{NH3})}{Mm(\ce{NH3})\vdot V(\ce{NH3})}\Rightarrow[\ce{NH3}]_0 = \frac{\SI{1}{\cancel\gram}}{\SI{17,03}{\cancel\gram\per\mol}\vdot \SI{,610}{\liter}}
        = \SI{,0962}{\Molar}
    $$
    \visible<2->{
        $$
            [\ce{OH-}] = 10^{-(14-pH)}\Rightarrow[\ce{H+}] = \SI{1,288e-3}{\Molar} = x
        $$
                }
    \visible<3->{
        \begin{center}
            \begin{tabular}{cccc}
                \toprule
                Estado  & [\ce{NH3}] (\si{\Molar}) & [\ce{NH4+}] (\si{\Molar}) & [\ce{OH-}] (\si{\Molar}) \\
                \midrule
                Inicial & $[\ce{NH3}]_0$                  & 0                            & 0                \\
                Final   &             $[\ce{NH3}]_0-x$                & $x$ & $x$ \\
                \bottomrule
            \end{tabular}
        \end{center}
    }
        \visible<4->{
            \begin{overprint}
                \onslide<4>
                    $$
                        K_b = \frac{[\ce{OH-}][\ce{NH4+}]}{[\ce{NH3}]} = \frac{x^2}{[\ce{NH3}]_0-x}
                    $$
                \onslide<5>
                    $$
                        \tcbhighmath[boxrule=0.4pt,arc=4pt,colframe=red,drop fuzzy shadow=blue]{K_b = \frac{[\ce{OH-}][\ce{NH4+}]}{[\ce{NH3}]} = \frac{(\SI{1,288e-3}{})^2}{\SI{,0962}{}-\SI{1,288e-3}{}} = \SI{1,75e-5}{}}
                    $$
            \end{overprint}
                    }
\end{frame}

