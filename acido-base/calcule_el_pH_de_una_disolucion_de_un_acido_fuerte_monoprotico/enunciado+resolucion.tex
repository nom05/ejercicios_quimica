\begin{frame}
	\frametitle{\ejerciciocmd}
	\framesubtitle{Enunciado}
	\textbf{
		Una reacción tiene una constante de velocidad de \SI{,017}{\per\second} a \SI{298}{\kelvin} y una energía libre de activación del \SI{27,235}{\kilo\joule\per\mol}. La adición de un catalizador disminuye dicha energía de activación hasta un \SI{33}{\percent} de su valor inicial. Calcule la nueva constante de velocidad.
\resultadocmd{ \SI{26,86}{\per\second} }

		}
\end{frame}

\begin{frame}
	\frametitle{\ejerciciocmd}
	\framesubtitle{Datos del problema}
	\begin{center}
		{\huge ¿$\mathrm{pH}$?}\\[.5cm]
		\tcbhighmath[boxrule=0.4pt,arc=4pt,colframe=green,drop fuzzy shadow=blue]{\text{Ácido fuerte monoprótico: }\ce{HA}}
		\tcbhighmath[boxrule=0.4pt,arc=4pt,colframe=green,drop fuzzy shadow=blue]{[\ce{HA}]=\SI{5,0e-8}{\Molar}}\\[.8cm]
		Pero también ...
			\tcbhighmath[boxrule=0.4pt,arc=4pt,colframe=blue,drop fuzzy shadow=blue]{K_w=\num{1e-14}}
	\end{center}
\end{frame}

\begin{frame}
	\frametitle{\ejerciciocmd}
	\framesubtitle{Resolución (\rom{1}): $\mathrm{pH}$ de la disolución}
	\structure{Ácido fuerte monoprótico:} Disociación total\\
	\centering\ce{HA(ac) -> H+(ac) + A-(ac)}\quad$[\ce{HA}]= \SI{5,0e-8}{\Molar}=[\ce{H+}]$\\[.6cm]
	\begin{center}
		\alert{El \underline{ácido} fuerte está \underline{muy diluido}, la \underline{disociación del agua \textbf{no}} se puede \textbf{\underline{despreciar}}}
	\end{center}
	$$
		\ce{H2O(l) <=> H+(ac) + OH-(ac)}\quad K_w = [\ce{H+}][\ce{OH-}]=\num{1e-14}
	$$
	\begin{center}
		\begin{tabular}{cccc}
			\toprule
			Concentración(\si{\Molar}) & [\ce{H+}]       & [\ce{OH-}] \\
			{[Inicial]}                & \num{5,0e-8}     &       0	  \\
			{[Equilibrio]}             & $\num{5,0e-8}+x$ & $x$       \\
			\bottomrule
		\end{tabular}
	\end{center}
	$$
		\overbrace{K_w}^{\num{1e-14}} = \underbrace{[\ce{H+}]}_{\num{5,0e-8}+x}\cdot\overbrace{[\ce{OH-}]}^{x}\Rightarrow
		\num{1e-14} = (\num{5,0e-8}+x)x
	$$
	$$
		x^2+\num{5,0e-8}x -\num{1e-14}=0\Rightarrow x=[\ce{H+}]_{\text{agua}} = \SI{7,81e-8}{\Molar}
	$$
%			\onslide<6->
				$$
					[\ce{H+}]_{\text{total}} = \overbrace{[\ce{H+}]_{\ce{HA}}}^{\SI{5,0e-8}{\Molar}}+\underbrace{[\ce{H+}]_{\text{agua}}}_{\SI{7,81e-8}{\Molar}}
				$$
%		\end{overprint}
%		\visible<6->{
			\centering\tcbhighmath[boxrule=0.4pt,arc=4pt,colframe=green,drop fuzzy shadow=blue]{\mathrm{pH}=-\log(\num{5,0e-8}+\num{7,81e-8})=6,89}
%					}
\end{frame}
