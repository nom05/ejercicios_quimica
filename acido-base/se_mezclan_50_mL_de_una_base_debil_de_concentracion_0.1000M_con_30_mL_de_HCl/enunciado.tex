Se mezclan \SI{50}{\milli\liter} de una base débil ($K_b = \num{1,8e-5}$) de concentración \SI{,1000}{\Molar} con \SI{30}{\milli\liter} de ácido clorhídrico (\ce{HCl}) de concentración desconocida, obteniéndose una disolución cuyo pH es \num{1,903}. Calcule cuál sería el pH de la disolución formada entre el mismo número de \si{\milli\liter} de base y 
\begin{enumerate*}[label={\alph*)},font=\bfseries]
	\item\SI{15}{\milli\liter} del ácido clorhídrico,
	\item\SI{25}{\milli\liter} del ácido clorhídrico.
\end{enumerate*}
(Si es necesario suponga que los volúmenes de las disoluciones son aditivos).
\resultadocmd{\num{9,079}; \num{5,216}}