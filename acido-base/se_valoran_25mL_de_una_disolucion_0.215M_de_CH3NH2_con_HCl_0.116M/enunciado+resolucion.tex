\begin{frame}
    \frametitle{\ejerciciocmd}
    \framesubtitle{Enunciado}
    \textbf{
		Una reacción tiene una constante de velocidad de \SI{,017}{\per\second} a \SI{298}{\kelvin} y una energía libre de activación del \SI{27,235}{\kilo\joule\per\mol}. La adición de un catalizador disminuye dicha energía de activación hasta un \SI{33}{\percent} de su valor inicial. Calcule la nueva constante de velocidad.
\resultadocmd{ \SI{26,86}{\per\second} }

	}
\end{frame}

\begin{frame}
    \frametitle{\ejerciciocmd}
    \framesubtitle{Datos del apartado (a)}
    \begin{center}
        {\huge \textbf{¿pH?}}
    \end{center}
    $$
        \tcbhighmath[boxrule=0.4pt,arc=4pt,colframe=green,drop fuzzy shadow=yellow]{V(\ce{CH3NH2}) = \SI{25}{\milli\liter}}\quad
        \tcbhighmath[boxrule=0.4pt,arc=4pt,colframe=green,drop fuzzy shadow=yellow]{[\ce{CH3NH2}]_0 = \SI{,215}{\Molar}}\quad
        \tcbhighmath[boxrule=0.4pt,arc=4pt,colframe=green,drop fuzzy shadow=yellow]{K_b(\ce{CH3NH2})= \SI{5e-4}{}}
    $$
\end{frame}

\begin{frame}
    \frametitle{\ejerciciocmd}
    \framesubtitle{Resolución (\rom{1}): pH inicial de \ce{CH3NH2}}
    \structure{Reacción de disociación:}
    $$
        \ce{CH3NH2(ac) + H2O(l) <=> CH3NH3+(ac) + OH-(ac)}~~~K_b(\ce{CH3NH2})=\frac{[\ce{CH3NH3+}][\ce{OH-}]}{[\ce{CH3NH2}]}
    $$
    \visible<2->{
        \begin{center}
            \begin{tabular}{clll}
                \toprule
                Estado  & [\ce{CH3NH2}] (\si{\Molar})         & {[\ce{CH3NH3+}] (\si{\Molar})}          & {[\ce{OH-}] (\si{\Molar})}         \\
                \midrule
                Inicial & $[\ce{CH3NH2}]_0$ & 0                & 0                \\
                Final   & $[\ce{CH3NH2}]_0-x$                & $x$ & $x$ \\
                \bottomrule
            \end{tabular}
        \end{center}
                }
    \visible<3->{
        \structure{Si suponemos que $x=[\ce{CH3NH3+}]=[\ce{OH-}]$ es mucho menor que $[\ce{CH3NH2}]_0$}
        $$
            K_b(\ce{CH3NH2})=\frac{[\ce{CH3NH3+}][\ce{OH-}]}{[\ce{CH3NH2}]_0-x}\approx\frac{[\ce{CH3NH3+}][\ce{OH-}]}{[\ce{CH3NH2}]_0}
        $$
                }
    \visible<4->{
        \structure{Aplicando logaritmos y la relación $pH+pOH=14$:}
        \begin{overprint}
            \onslide<4>
                $$
                    K_b(\ce{CH3NH2})=\frac{\overbrace{[\ce{CH3NH3+}]}^{[\ce{CH3NH3+}]=[\ce{OH-}]}[\ce{OH-}]}{[\ce{CH3NH2}]_0-x}\approx\frac{[\ce{OH-}]^2}{[\ce{CH3NH2}]_0}
                $$
            \onslide<5>
                $$
                    [\ce{OH-}]^{-2} = K_b(\ce{CH3NH2})^{-1}\cdot\frac{1}{[\ce{CH3NH2}]_0}
                $$
            \onslide<6>
                $$
                    \log[\ce{OH-}]^{-2} = \log K_b(\ce{CH3NH2})^{-1} + \log\left(\frac{1}{[\ce{CH3NH2}]_0}\right)
                $$
            \onslide<7>
                $$
                    2\overbrace{pOH}^{pOH=14-pH} = pK_b(\ce{CH3NH2}) - \log[\ce{CH3NH2}]_0
                $$
            \onslide<8>
                $$
                    2\times(14-pH) = \left(pK_b(\ce{CH3NH2}) - \log[\ce{CH3NH2}]_0\right)
                $$
            \onslide<9>
                $$
                    pH = 14 - \frac{1}{2}\left(\overbrace{pK_b(\ce{CH3NH2})}^{\SI{3,301}{}} - \underbrace{\log[\ce{CH3NH2}]_0}_{\SI{-,668}{}}\right)
                $$
        \end{overprint}
                }
    \visible<9->{
        \begin{center}
            \tcbhighmath[boxrule=0.4pt,arc=4pt,colframe=green,drop fuzzy shadow=yellow]{pH = 12}
        \end{center}
                }
\end{frame}

\begin{frame}
    \frametitle{\ejerciciocmd}
    \framesubtitle{Datos del apartado (b)}
    \begin{center}
        {\huge \textbf{¿pH?}}
    \end{center}
    $$
        \tcbhighmath[boxrule=0.4pt,arc=4pt,colframe=green,drop fuzzy shadow=yellow]{V(\ce{CH3NH2}) = \SI{25}{\milli\liter}}\quad
        \tcbhighmath[boxrule=0.4pt,arc=4pt,colframe=green,drop fuzzy shadow=yellow]{[\ce{CH3NH2}]_0 = \SI{,215}{\Molar}}\quad
        \tcbhighmath[boxrule=0.4pt,arc=4pt,colframe=green,drop fuzzy shadow=yellow]{K_b(\ce{CH3NH2})= \SI{5e-4}{}}
    $$
    $$
        \tcbhighmath[boxrule=0.4pt,arc=4pt,colframe=blue,drop fuzzy shadow=green]{V(\ce{HCl}) = \SI{5}{\milli\liter}}
    $$
\end{frame}

\begin{frame}
    \frametitle{\ejerciciocmd}
    \framesubtitle{Resolución (\rom{2}): pH de disolución de \ce{CH3NH2} después de añadir \SI{5}{\milli\liter} de \ce{HCl}}
    \structure{Hay que trabajar con número de moles:}
    \visible<2->{
        $$
            n(\ce{CH3NH2}) = \SI{,215}{\mol\per\cancel\liter}\cdot\SI{,025}{\cancel\liter} = \SI{5,375e-3}{\mol}
        $$
        $$
            n(\ce{HCl}) = \SI{,116}{\mol\per\cancel\liter}\cdot\SI{,005}{\cancel\liter} = \SI{5,8e-4}{\mol}
        $$
                }
    \visible<3->{
        \textbf{El volumen cambia después de añadir \SI{5}{\milli\liter} de \ce{HCl(ac)}}: $V_{Total} = V_T = \SI{25}{\milli\liter} + \SI{5}{\milli\liter} = \SI{30}{\milli\liter}$
        \structure{Reacción de neutralización:}
        $$
            \ce{CH3NH2(ac) + HCl(ac) -> CH3NH3Cl(ac)}
        $$
        La reacción de neutralización es \underline{\textbf{TOTAL}}: reaccionaron \SI{5,8e-4}{\mol} de \ce{CH3NH2} y quedan: $\SI{5,375e-3}{\mol}-\SI{5,8e-4}{\mol} = \SI{4,795e-3}{\mol}$ de \ce{CH3NH2} sin neutralizar.
                }
    \visible<4->{
        $$
            \ce{CH3NH2(ac) + H2O(l) <=> CH3NH3+(ac) + OH-(ac)}~~~K_b(\ce{CH3NH2})=\frac{[\ce{CH3NH3+}][\ce{OH-}]}{[\ce{CH3NH2}]}
        $$
        \alert{\textbf{ATENCIÓN:} como los volúmenes cambian, se trabaja con número de moles. Para la concentración de \ce{OH-} dividir por ese volumen. \underline{$K_b$} se define a través de \underline{concentraciones}.}
        \begin{overprint}
            \onslide<4>
                \begin{center}
                    \begin{tabular}{clll}
                        \toprule
                        Estado  & [\ce{CH3NH2}] (\si{\mol})         & {[\ce{CH3NH3+}] (\si{\mol})}          & {[\ce{OH-}] (\si{\mol})}         \\
                        \midrule
                        Inicial & \SI{4,795e-3}{} & \SI{5,8e-4}{}                & 0                \\
                        Final   & $\SI{4,795e-3}{}-x$                & $\SI{5,8e-4}{}+x$ & $x$ \\
                        \bottomrule
                    \end{tabular}
                \end{center}
            \onslide<5>
                $$
                    K_b(\ce{CH3NH2}) = \frac{
                                \overbrace{\frac{x}{0,03}}^{[\ce{OH-}]}\cdot\frac{\overbrace{\SI{5,8e-4}{}+x}^{\SI{5,8e-4}{}}}{\cancel{0,03}}
                                            }{
                                \underbrace{\frac{\SI{4,795e-3}{}-x}{\cancel{0,03}}_{\approx\SI{4,795e-3}{}}}
                                             } = \SI{5e-4}{}
                $$
            \onslide<6>
                $$
                    \frac{
                        [\ce{OH-}]\cdot\SI{5,8e-4}{}
                    }{
                        \SI{4,795e-3}{}
                    } = \SI{5e-4}{}
                $$
            \onslide<7>
                $$
                    [\ce{OH-}] = \SI{4,13e-4}{\Molar}
                $$
            \onslide<8>
                $$
                    \overbrace{-\log[\ce{OH-}]}^{pOH} = \SI{2,38}{}
                $$
            \onslide<9>
                $$
                    \tcbhighmath[boxrule=0.4pt,arc=4pt,colframe=blue,drop fuzzy shadow=green]{pH = 11,62}
                $$
                $$
                    \tcbhighmath[boxrule=0.4pt,arc=4pt,colframe=blue,drop fuzzy shadow=green]{pH = 11,53\text{ (Sin aproximaciones matemáticas)}}
                $$
        \end{overprint}
                }
\end{frame}

\begin{frame}
    \frametitle{\ejerciciocmd}
    \framesubtitle{Datos del apartado (c)}
    \begin{center}
        {\huge \textbf{¿pH?}}
    \end{center}
    $$
        \tcbhighmath[boxrule=0.4pt,arc=4pt,colframe=green,drop fuzzy shadow=yellow]{V(\ce{CH3NH2}) = \SI{25}{\milli\liter}}\quad
        \tcbhighmath[boxrule=0.4pt,arc=4pt,colframe=green,drop fuzzy shadow=yellow]{[\ce{CH3NH2}]_0 = \SI{,215}{\Molar}}\quad
        \tcbhighmath[boxrule=0.4pt,arc=4pt,colframe=green,drop fuzzy shadow=yellow]{K_b(\ce{CH3NH2})= \SI{5e-4}{}}
    $$
    $$
        \tcbhighmath[boxrule=0.4pt,arc=4pt,colframe=blue,drop fuzzy shadow=green]{\text{Punto de equivalencia}}
    $$
\end{frame}

\begin{frame}
    \frametitle{\ejerciciocmd}
    \framesubtitle{Resolución (\rom{3}): pH de disolución de \ce{CH3NH2} en el punto de equivalencia}
    \structure{Hay que trabajar con número de moles:}
    \visible<2->{
        $$
            n(\ce{CH3NH2}) = n(\ce{HCl}) = \overbrace{\SI{5,375}{\mol}}^{\text{Anterior apartado}}
        $$
                }
    \visible<3->{
        $$
            V(\ce{HCl}) = \frac{n(\ce{HCl})}{[\ce{HCl}]}\Rightarrow V(\ce{HCl}) = \frac{\SI{5,375}{\mol}}{\SI{,116}{\cancel\mol\per\liter}} = \SI{,046}{\liter}
        $$
        $$
            V_T = \SI{,025}{\liter} + \SI{,046}{\liter} = \SI{,071}{\liter}
        $$
                }
    \visible<4->{
        \structure{En el punto de equivalencia únicamente hay el ácido conjugado:}
        $$
            [\ce{CH3NH3+}] = \frac{\SI{5,375e-3}{\mol}}{\SI{,071}{\liter}} = \SI{,0757}{\Molar}
        $$
        \begin{overprint}
            \onslide<4>
                $$
                    \ce{CH3NH3+(ac) <=> CH3NH2(ac) + H+(ac)}~~K_h(\ce{CH3NH3+}) = \frac{K_w}{K_b(\ce{CH3NH2})} = \frac{\overbrace{[\ce{CH3NH2}]}^{=[\ce{H+}]}[\ce{H+}]}{[\ce{CH3NH3+}]}       
                $$
            \onslide<5>
                \structure{Despreciamos disociación, despejamos y aplicamos logaritmos:}
                $$
                    K_h(\ce{CH3NH3+}) = \frac{[\ce{H+}]^2}{[\ce{CH3NH3+}]}       
                $$
            \onslide<6>
                $$
                    \frac{1}{[\ce{H+}]^2} = \frac{1}{K_h(\ce{CH3NH3+})}\cdot\frac{1}{[\ce{CH3NH3+}]}       
                $$
            \onslide<7>
                $$
                    \log[\ce{H+}]^{-2} = \log\frac{K_b(\ce{CH3NH3+})}{K_w}+\log[\ce{CH3NH3+}]^{-1}
                $$
            \onslide<8>
                $$
                    2pH = -pK_b(\ce{CH3NH3+})+14-\log[\ce{CH3NH3+}]
                $$
            \onslide<9->
                $$
                    pH = 7-\frac{1}{2}\left(\overbrace{pK_b(\ce{CH3NH3+})}^{\SI{3,301}{}}+\underbrace{\log[\ce{CH3NH3+}]}_{\SI{-1,121}{}}\right)
                $$
        \end{overprint}
                }
    \visible<9>{
        \begin{center}
            \tcbhighmath[boxrule=0.4pt,arc=4pt,colframe=blue,drop fuzzy shadow=green]{pH = \SI{5,91}{}}
        \end{center}
                }
\end{frame}

\begin{frame}
    \frametitle{\ejerciciocmd}
    \framesubtitle{Datos del apartado (d)}
    \begin{center}
        {\huge \textbf{¿pH?}}
    \end{center}
    $$
        \tcbhighmath[boxrule=0.4pt,arc=4pt,colframe=green,drop fuzzy shadow=yellow]{V(\ce{CH3NH2}) = \SI{25}{\milli\liter}}\quad
        \tcbhighmath[boxrule=0.4pt,arc=4pt,colframe=green,drop fuzzy shadow=yellow]{[\ce{CH3NH2}]_0 = \SI{,215}{\Molar}}\quad
        \tcbhighmath[boxrule=0.4pt,arc=4pt,colframe=green,drop fuzzy shadow=yellow]{K_b(\ce{CH3NH2})= \SI{5e-4}{}}
    $$
    $$
        \tcbhighmath[boxrule=0.4pt,arc=4pt,colframe=blue,drop fuzzy shadow=green]{\text{\SI{10}{\milli\liter} después del punto de equivalencia}}
    $$
\end{frame}

\begin{frame}
    \frametitle{\ejerciciocmd}
    \framesubtitle{Resolución (\rom{4}): pH de disolución \SI{10}{\milli\liter} después del punto de equivalencia}
    \structure{Del anterior apartado:} $V_T(\text{p.equiv.}) = \SI{71}{\milli\liter}$
    \visible<2->{
        \structure{Por tanto, el volumen en este apartado será:} $V_T = \SI{71}{\milli\liter} + \SI{10}{\milli\liter} = \SI{81}{\milli\liter}$
                }
    \visible<3->{
        $$
            n_0(\ce{CH3NH2}) = n(\ce{CH3NH3+}) = \SI{5,375e-3}{\mol}
        $$
                }
    \visible<4->{
        \begin{overprint}
            \onslide<4>
                Del apartado anterior (punto de equivalencia):
                $$
                    V(HCl) = \SI{46}{\milli\liter}
                $$
            \onslide<5>
                Pero hemos echado \SI{10}{\milli\liter} de \ce{HCl} más ahora:
                $$
                    V(HCl) = \SI{46}{\milli\liter} + \SI{10}{\milli\liter} = \SI{56}{\milli\liter}
                $$
            \onslide<6>
                Número de moles de \ce{HCl} totales:
                $$
                    [\ce{HCl}] = \frac{n(HCl)}{V(HCl)}\Rightarrow n(HCl) = \SI{,116}{\mol\per\cancel\liter}\cdot\SI{,056}{\cancel\liter} = \SI{6,49e-3}{\mol}
                $$
            \onslide<7->
                Número de moles de \ce{HCl} totales en exceso (no reaccionados):
                $$
                    n_{\text{exceso}}(HCl) = \SI{6,49e-3}{\mol} - \SI{5,375e-3}{\mol} = \SI{1,12e-3}{\mol}
                $$
        \end{overprint}
                }
    \visible<8>{
        \structure{Hay dos maneras para resolver el apartado:} suponer que pH viene dado por el exceso de \ce{HCl} o tener en cuenta también la disociación de la sal conjugada.
                }
\end{frame}

\begin{frame}
    \frametitle{\ejerciciocmd}
    \framesubtitle{Resolución (\rom{5}): pH de disolución \SI{10}{\milli\liter} después del punto de equivalencia}
    \begin{block}{Incluir únicamente pH del \ce{HCl}}
        $$
            [\ce{HCl}] = [\ce{H+}] = \frac{n(\ce{HCl})}{V_T}\Rightarrow[\ce{H+}] = \frac{\SI{1,12e-3}{\mol}}{\SI{,081}{\liter}} = \SI{,0138}{\Molar}
        $$
        $$
            \tcbhighmath[boxrule=0.4pt,arc=4pt,colframe=blue,drop fuzzy shadow=green]{pH = -\log[\ce{H+}] = \SI{1,86}{}}
        $$
    \end{block}
    \begin{block}<2->{Incluir disociación de sal conjugada}
        \structure{Reacción de disociación de \ce{CH3NH3+}:}
        $$
            \ce{CH3NH3+(ac) <=> CH3NH2(ac) + H+(ac)}\quad K_h(\ce{CH3NH3+}) = \frac{K_w}{K_b(\ce{CH3NH2})} = \frac{[\ce{CH3NH2}][\ce{H+}]}{[\ce{CH3NH3+}]}
        $$
        \begin{overprint}
            \onslide<2>
                \begin{center}
                    \begin{tabular}{clll}
                        \toprule
                        Estado  & [\ce{CH3NH3+}] (\si{\mol})         & {[\ce{CH3NH2}] (\si{\mol})}   & {[\ce{H+}] (\si{\mol})}   \\
                        \midrule
                        Inicial & \SI{5,375e-3}{}                    & \SI{0}{}                      & \SI{1,12e-3}{}            \\
                        Final   & $\SI{5,375e-3}{}-x$                & $x$                           & $\SI{1,12e-3}{}+x$        \\
                        \bottomrule
                    \end{tabular}
                \end{center}
            \onslide<3>
                $$
                    \frac{K_w}{K_b(\ce{CH3NH2})} = \frac{\frac{x}{\cancel{V_T}}\cdot\frac{\SI{1,12e-3}{}+x}{V_T}}{\frac{\SI{5,375e-3}{}-x}{\cancel{V_T}}}
                $$
            \onslide<4>
                $$
                    x = \SI{6,965e-12}{\mol}\Rightarrow [\ce{CH3NH2}] = \SI{8,598e-11}{\Molar}
                $$
            \onslide<5->
                $$
                    [\ce{H+}] = \SI{,0138}{\Molar} + \SI{8,598e-11}{\Molar}\Rightarrow\tcbhighmath[boxrule=0.4pt,arc=4pt,colframe=blue,drop fuzzy shadow=green]{pH = -\log[\ce{H+}] = \SI{1,86}{}}
                $$
        \end{overprint}
    \end{block}
\end{frame}
