Se valoran \SI{25}{\milli\liter} de una disolución \SI{,215}{\Molar} de metilamina ($K_b(\ce{CH3NH2})= \SI{5e-4}{}$) con ácido clorhídrico (ac) \SI{,116}{\Molar}. Calcular:
\begin{enumerate}[label={\alph*)},font={\color{red!50!black}\bfseries}]
    \item El pH inicial de la disolución de amina.
    \item El pH después de añadir \SI{5}{\milli\liter} de ácido.
    \item El pH en el punto de equivalencia.
    \item El pH después de añadir \SI{10}{\milli\liter} de ácido más que los necesarios para alcanzar el punto de equivalencia.
\end{enumerate}
\resultadocmd{ \num{12,00}; \num{11,53}; \num{5,91}; \num{1,86} }
