\begin{frame}
	\frametitle{\ejerciciocmd}
	\framesubtitle{Enunciado}
	\textbf{
		Dadas las siguientes reacciones:
\begin{itemize}
    \item \ce{I2(g) + H2(g) -> 2 HI(g)}~~~$\Delta H_1 = \SI{-0,8}{\kilo\calorie}$
    \item \ce{I2(s) + H2(g) -> 2 HI(g)}~~~$\Delta H_2 = \SI{12}{\kilo\calorie}$
    \item \ce{I2(g) + H2(g) -> 2 HI(ac)}~~~$\Delta H_3 = \SI{-26,8}{\kilo\calorie}$
\end{itemize}
Calcular los parámetros que se indican a continuación:
\begin{description}%[label={\alph*)},font={\color{red!50!black}\bfseries}]
    \item[\texttt{a)}] Calor molar latente de sublimación del yodo.
    \item[\texttt{b)}] Calor molar de disolución del ácido yodhídrico.
    \item[\texttt{c)}] Número de calorías que hay que aportar para disociar en sus componentes el yoduro de hidrógeno gas contenido en un matraz de \SI{750}{\cubic\centi\meter} a \SI{25}{\celsius} y \SI{800}{\torr} de presión.
\end{description}
\resultadocmd{\SI{12,8}{\kilo\calorie}; \SI{-13,0}{\kilo\calorie}; \SI{12,9}{\calorie}}

				}
\end{frame}

\begin{frame}
	\frametitle{\ejerciciocmd}
	\framesubtitle{Datos del problema}
	\begin{center}
		\structure{Reacción:} 	\ce{HgO(s) <=> Hg(g) + 1/2 O2(g)}\\[.6cm]
		{\huge ¿$\Delta G^0$?, ¿$K_p$?, ¿$P_{\text{total}}$?}\\[.6cm]
		{\large \begin{tabular}{cSSS}
			\toprule
			& {\ce{HgO(s)}} & {\ce{Hg(g)}} 	& {\ce{O2(g)}}	\\
			\midrule
			$\Delta H^0_f(\si{\kilo\joule\per\mol})$	& 	-90,7		&	61,32		&				\\
			$S^0_f(\si{\joule\per\kelvin\per\mol})$		&	 72			&  175			& 	205,0		\\
			\bottomrule
		\end{tabular}}
		\tcbhighmath[boxrule=0.4pt,arc=4pt,colframe=red,drop fuzzy shadow=blue]{T = \SI{700}{\kelvin}}
	\end{center}
\end{frame}

\begin{frame}
	\frametitle{\ejerciciocmd}
	\framesubtitle{Resolución (\rom{1}): energía libre Gibbs en condiciones estándar}
	\structure{Usamos ley de Hess:}
	$$
		\tcbhighmath[boxrule=0.4pt,arc=4pt,colframe=black,drop fuzzy shadow=black]{\Delta H^0_{\text{R}} = \sum_{i=1}^{\text{n"o prod}} n_i\vdot\Delta H^0_{\text{f},i} - \sum_{j=1}^{\text{n"o react}} n_j\vdot\Delta H^0_{\text{f},j}}\quad
		\tcbhighmath[boxrule=0.4pt,arc=4pt,colframe=black,drop fuzzy shadow=black]{\Delta S^0_{\text{R}} = \sum_{i=1}^{\text{n"o prod}} n_i\vdot S^0_{\text{f},i} - \sum_{j=1}^{\text{n"o react}} n_j\vdot S^0_{\text{f},j}}
	$$
	$$
		\Delta H^0_{\text{R}} = \Delta H^0_{\text{f}}(\ce{Hg(g)}) - \Delta H^0_{\text{f}}(\ce{HgO(g)})
	$$
	$$
		\Delta H^0_{\text{R}} = \SI{61,32}{\kilo\joule\per\mol} - (\SI{-90,7}{\kilo\joule\per\mol}) = \SI{152,02}{\kilo\joule\per\mol}
	$$
	$$
		\Delta S^0_{\text{R}} = S^0_{\text{f}}(\ce{Hg(g)}) + \frac{1}{2}S^0_{\text{f}}(\ce{Hg(g)}) - S^0_{\text{f}}(\ce{HgO(g)})
	$$
	$$
		\Delta S^0_{\text{R}} = \SI{175}{\joule\per\kelvin\per\mol} + \frac{1}{2}\SI{205,0}{\joule\per\kelvin\per\mol} - (\SI{-72}{\joule\per\kelvin\per\mol})
	$$
	\begin{center}
		\myovalbox{\textcolor{yellow}{$\Delta H^0_{\text{R}} = \SI{152,02}{\kilo\joule\per\mol}$}}
		\myovalbox{\textcolor{yellow}{$\Delta S^0_{\text{R}} = \SI{205,50}{\joule\per\kelvin\per\mol}$}}
	\end{center}
	\structure{Usamos:} $\Delta G = \Delta H - T\Delta S$
	$$
		\Delta G^0_{\text{R}} = \SI{152,02}{\kilo\joule\per\mol} - \SI{700}{\cancel\kelvin}\vdot\SI{205,50}{\cancel\joule\per\cancel\kelvin\per\mol}\vdot\SI{e-3}{\kilo\joule\per\cancel\joule}
	$$
	$$
		\tcbhighmath[boxrule=0.4pt,arc=4pt,colframe=blue,drop fuzzy shadow=green]{\Delta G^0_{\text{R}} = \SI{8,17}{\kilo\joule\per\mol}}
	$$
\end{frame}

\begin{frame}
	\frametitle{\ejerciciocmd}
	\framesubtitle{Resolución (\rom{2}): constante de equilibrio $K_p$}
	\structure{Recordemos la expresión que pasa de $\Delta G^0$ (en condiciones estándar) a otras condiciones:}
	$$
		\Delta G = \Delta G^0 + RT\ln Q_p
	$$
	\structure{Como en situación de equilibrio $\Delta G = 0$ y $Q_p = K_p$, obtenemos:} 
	$$
		\Delta G^0 + RT\ln K_p = 0\Rightarrow
		\Delta G^0 = -RT\ln K_p\Rightarrow
		K_p = \exp(-\frac{\Delta G^0}{RT})
	$$
	$$
		\tcbhighmath[boxrule=0.4pt,arc=4pt,colframe=green,drop fuzzy shadow=blue]{
			K_p =
				\exp(-\frac{\SI{8,17}{\cancel\kilo\joule\per\cancel\mol}\vdot\SI{e3}{\cancel\joule\per\cancel\kilo\joule}}{\SI{8,314}{\cancel\joule\per\cancel\mol\per\cancel\kelvin}\vdot\SI{700}{\cancel\kelvin}}) = \num{,245}
		}
	$$
\end{frame}

\begin{frame}
	\frametitle{\ejerciciocmd}
	\framesubtitle{Resolución (\rom{3}): presión total de la mezcla}
	Como desconocemos el n"o de moles de las especies suponemos un valor genérico ``$n$'' por ejemplo para los compuestos gaseosos:
	\begin{center}
		\begin{tabular}{lcc}
			\toprule
										&	\ce{Hg(g)}	&	\ce{1/2O2(g)}	\\
			\midrule
				$n_{\ce{reaccionan}}$	&	$+n$		&	$+\frac{1}{2}n$	\\
				$n_{\ce{equilibrio}}$	&	$n$			&	$\frac{1}{2}n$	\\
			\bottomrule
		\end{tabular}
	\end{center}
	\structure{N"o de moles totales ($n_T$):} $n_T = n + \frac{1}{2}n = \frac{3}{2}n$
	\structure{Fracciones molares:}
	$$
		x_{\ce{Hg}} = \frac{\cancel{n}}{\frac{3}{2}\cancel{n}} = \frac{2}{3};\quad
		x_{\ce{O2}} = \frac{\frac{1}{2}\cancel{n}}{\frac{3}{2}\cancel{n}} = \frac{1}{3}
	$$
	\structure{Ley de Dalton y $K_p$:} $P_i = x_i\vdot P_T$; $K_p = P_{\ce{Hg}}\vdot (P_{\ce{O2}})^{\frac{1}{2}}$
	$$
		P_{\ce{Hg}} = x_{\ce{Hg}}\vdot P_T;\quad
		P_{\ce{O2}} = x_{\ce{O2}}\vdot P_T
	$$
	$$
		K_p = x_{\ce{Hg}}\vdot P_T\vdot (x_{\ce{O2}}\vdot P_T)^{\frac{1}{2}} = x_{\ce{Hg}}\vdot (x_{\ce{O2}})^{\frac{1}{2}}\vdot  (P_T)^{\frac{3}{2}}\Rightarrow
		P_T = \left\{\frac{K_p}{x_{\ce{Hg}}\vdot (x_{\ce{O2}})^{\frac{1}{2}}}\right\}^{\frac{2}{3}}\Rightarrow
	$$
	$$
		\tcbhighmath[boxrule=0.4pt,arc=4pt,colframe=green,drop fuzzy shadow=blue]{
			P_T = \left\{\frac{\num{,245}}{\frac{2}{3}\vdot\left(\frac{1}{3}\right)^{\frac{1}{2}}}\right\}^{\frac{2}{3}} = \SI{,74}{\atm}
		}
	$$
\end{frame}