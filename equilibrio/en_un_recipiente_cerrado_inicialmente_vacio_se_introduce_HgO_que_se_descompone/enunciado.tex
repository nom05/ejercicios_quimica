En un recipiente cerrado inicialmente vacío se introduce óxido de mercurio~(\rom{2}) sólido que se descompone estableciendo el equilibrio:
$$
	\ce{HgO(s) <=> Hg(g) + 1/2 O2(g)}
$$
\begin{enumerate}[label={\alph*)},font=\bfseries]
	\item Calcula $\Delta G^0$ para la reacción a \SI{700}{\kelvin}.
	\item Calcula el valor de $K_p$ a dicha temperatura.
	\item Calcula la presión que ejercen los gases una vez alcanzado el equilibrio.
\end{enumerate}
Datos:
\begin{center}
	\begin{tabular}{cSSS}
		\toprule
		& {\ce{HgO(s)}} & {\ce{Hg(g)}} 	& {\ce{O2(g)}}	\\
		\midrule
		$\Delta H^0_f(\si{\kilo\joule\per\mol})$	& 	-90,7		&	61,32		&				\\
		$S^0_f(\si{\joule\per\kelvin\per\mol})$		&	 72			&  175			& 	205,0		\\
		\bottomrule
	\end{tabular}
\end{center}
\resultadocmd{\SI{8,17}{\kilo\joule\per\mol}; \num{,245}; \SI{,74}{\atm}}