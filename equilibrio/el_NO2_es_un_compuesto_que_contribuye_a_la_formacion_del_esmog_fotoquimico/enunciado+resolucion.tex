\begin{frame}
	\frametitle{\ejerciciocmd}
	\framesubtitle{Enunciado}
	\textbf{
		Una reacción tiene una constante de velocidad de \SI{,017}{\per\second} a \SI{298}{\kelvin} y una energía libre de activación del \SI{27,235}{\kilo\joule\per\mol}. La adición de un catalizador disminuye dicha energía de activación hasta un \SI{33}{\percent} de su valor inicial. Calcule la nueva constante de velocidad.
\resultadocmd{ \SI{26,86}{\per\second} }

		}
\end{frame}

\begin{frame}
	\frametitle{\ejerciciocmd}
	\framesubtitle{Datos del problema}
    {\large $$
        \ce{2NO2(g) <=> 2NO(g)+O2(g)}
    $$}
    \begin{center}
        {\Large ¿grado de disociación, $\alpha$? ¿$K_c$? ¿$K_p$?}\\[.5cm]
        \tcbhighmath[boxrule=0.4pt,arc=4pt,colframe=blue,drop fuzzy shadow=green]{T_{\text{inicial}}\equiv T_i = \SI{25}{\celsius} = \SI{298,15}{\kelvin}}\\[.3cm]
        \tcbhighmath[boxrule=0.4pt,arc=4pt,colframe=blue,drop fuzzy shadow=green]{P_{\text{inicial}}\equiv P_i = \SI{21,1}{\atm}}\quad
        \tcbhighmath[boxrule=0.4pt,arc=4pt,colframe=blue,drop fuzzy shadow=green]{T_{\text{final}}\equiv T_f = \SI{300}{\celsius} = \SI{573,15}{\kelvin}}\\[.3cm]
        \tcbhighmath[boxrule=0.4pt,arc=4pt,colframe=black,drop fuzzy shadow=blue]{P_{\text{total,final}}\equiv P_T = \SI{50}{\atm} = \SI{573,15}{\kelvin}}
    \end{center}
\end{frame}

\begin{frame}
	\frametitle{\ejerciciocmd}
	\framesubtitle{Resolución (\rom{1}): presiones en el equilibrio}
    \structure{Usamos ecuación de los gases ideales:} para obtener la presión de \ce{NO2} en las condiciones de temperatura del equilibrio (\SI{300}{\celsius}). Como $n$ (n"o de moles) se mantiene constante, podemos escribir una expresión que relacione dos estados.
    $$
        P\vdot V = n\vdot R\vdot T\Rightarrow n\vdot R = \frac{P\vdot V}{T}\Rightarrow\frac{P_i\vdot\cancel{V}}{T_i} = \frac{P_f\vdot\cancel{V}}{T_f} = n\vdot R\Rightarrow
        P_f = P_i\vdot\frac{T_f}{T_i}
    $$
    $$
        P(\ce{NO2}) = \SI{21,1}{\atm}\vdot\frac{\SI{573,15}{\kelvin}}{\SI{298,15}{\kelvin}} = \SI{40,56}{\atm} = P_0
    $$
    \begin{center}
        \begin{tabular}{lccc}
            \toprule
            \multicolumn{4}{c}{\ce{2NO2(g) <=> 2NO(g) + O2(g)}}	\\
            \midrule
                                &	$P(\ce{NO2})$	&	$P(\ce{NO})$	&	$P(\ce{O2})$	\\
            Inicial				&	$P_0$			&	\num{0}			&	\num{0}			\\
            Reaccionan			&	$-2x$			&	$2x$			&	$x$				\\
            Equilibrio			&	$P_0-2x$		&	$2x$			&	$x$				\\
            \cmidrule{2-4}
            $P_{\text{total}}$	&	\multicolumn{3}{c}{$P_0 \cancel{-2x} + \cancel{2x} + x = P_0 + x = \SI{50}{\atm}$}	\\
            \bottomrule
        \end{tabular}
    \end{center}
    \structure{Despejamos $x$:} $x = \SI{50}{\atm} - \SI{40,56}{\atm} = \SI{9,44}{\atm}$
    \structure{Presiones parciales en el equilibrio:}
    \begin{itemize}
        \item $P(\ce{NO2}) = \SI{40,56}{\atm} - 2\vdot\SI{9,44}{\atm} = \SI{21,68}{\atm}$
        \item $P(\ce{NO})  = 2\vdot\SI{9,44}{\atm} = \SI{18,88}{\atm}$
        \item $P(\ce{O2})  = \SI{9,44}{\atm}$
    \end{itemize}
\end{frame}

\begin{frame}
	\frametitle{\ejerciciocmd}
	\framesubtitle{Resolución (\rom{2}): grado de disociación, $K_p$, $K_c$}
    \begin{block}{Grado de disociación, $\alpha$}
        $$
            \tcbhighmath[boxrule=0.4pt,arc=4pt,colframe=blue,drop fuzzy shadow=green]{\alpha = \frac{x}{P_0}\Rightarrow\alpha(\ce{NO2}) = \frac{\SI{9,44}{\atm}}{\SI{40,56}{\atm}} = \num{,2327}}
        $$
    \end{block}
    \begin{exampleblock}{$K_p$}
        $$
            \tcbhighmath[boxrule=0.4pt,arc=4pt,colframe=blue,drop fuzzy shadow=red]{K_p = \frac{P(\ce{NO})^2\vdot P(\ce{O2})}{P(\ce{NO2})^2}\Rightarrow K_p = \frac{(\SI{18,88}{\atm})^2\vdot \SI{9,44}{\atm}}{(\SI{21,68}{\atm})^2} = \num{7,16}}
        $$
    \end{exampleblock}
    \begin{alertblock}{$K_c$}
        \begin{overprint}
            \onslide<1>
                $$
                    P\vdot V = n\vdot R\vdot T\Rightarrow P = \frac{n}{V}\vdot R\vdot T\Rightarrow
                    P = M\vdot R\vdot T\Rightarrow M = \frac{P}{R\vdot T}
                $$
                $$
                    K_c = \frac{[\ce{NO}]^2\vdot[\ce{O2}]}{[\ce{NO2}]^2}\Rightarrow
                    K_c = \frac{\left\{\frac{P(\ce{NO})}{\cancel{RT}}\right\}^2\vdot\left\{\frac{P(\ce{O2})}{RT}\right\}}{\left\{\frac{P(\ce{NO2})}{\cancel{RT}}\right\}^2}\Rightarrow
                    K_c = \frac{P(\ce{NO})^2\vdot P(\ce{O2})}{P(\ce{NO2})^2}\vdot \frac{1}{RT}\Rightarrow K_c = \frac{K_p}{R\vdot T}
                $$
            \onslide<2->
                $$
                    \tcbhighmath[boxrule=0.4pt,arc=4pt,colframe=blue,drop fuzzy shadow=yellow]{K_c = \frac{K_p}{R\vdot T}\Rightarrow
                      K_p = \frac{7,16}{\SI{8,314}{\joule\per\mol\per\kelvin}\vdot\SI{573,15}{\kelvin}} = \num{,15}}
                $$
        \end{overprint}
    \end{alertblock}
\end{frame}