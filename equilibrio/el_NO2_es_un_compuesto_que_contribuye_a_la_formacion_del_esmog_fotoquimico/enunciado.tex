El dióxido de nitrógeno es un compuesto que contribuye a la formación del esmog fotoquímico en los procesos de contaminación urbana debido a que a temperaturas elevadas se descompone según la reacción:
$$
    \ce{2NO2(g) <=> 2NO(g)+O2(g)}
$$
Si en un recipiente de \SI{2}{\liter} se introduce \ce{NO2} a \SI{25}{\celsius} y \SI{21,1}{\atm} de presión y se calienta hasta \SI{300}{\celsius} (a volumen constante) se observa que la presión una vez que se alcanza el equilibrio es de \SI{50}{\atm}.Calcular a \SI{300}{\celsius}
\begin{enumerate}[label={\alph*)},font=\bfseries]
    \item El grado de disociación del dióxido de nitrógeno.
    \item El valor de $K_c$ y $K_p$.
\end{enumerate}
Datos: $R = \SI{,082}{\atm\liter\per\mol\per\kelvin}$.
\resultadocmd{\num{,2427}; \num{7,16}, \num{,15}}