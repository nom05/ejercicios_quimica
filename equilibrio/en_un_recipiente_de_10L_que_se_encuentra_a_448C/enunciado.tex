En un recipiente de \SI{10}{\liter} que se encuentra a \SI{448}{\celsius} se introducen \SI{1}{\gram} de hidrógeno (\ce{H2(g)}) y \SI{126,9}{\gram} de yodo (\ce{I2(g)}), que reaccionan parcialmente produciendo yoduro de hidrógeno (\ce{HI(g)}). A esa temperatura la constante de equilibrio $K_c$ vale \num{50}.
\begin{enumerate}[label={\alph*)},font=\bfseries]
	\item ¿Qué presión total en el equilibrio hay en el recipiente?\label{part:eq-presion_total}
	\item ¿Cuántos moles de yodo están en el equilibrio?\label{part:eq-I2_sin_reaccionar}
	\item ¿Cuánto vale $K_p$ a la misma temperatura?\label{part:eq-Kp}
\end{enumerate}
DATOS: $Mm(\ce{H2}) = \SI{2,016}{\gram\per\mol}$, $Mm(\ce{I2}) = \SI{253,809}{\gram\per\mol}$ (La masa molecular de \ce{HI} se puede calcular con estos datos)
\resultadocmd{ \SI{5,913}{\atm}; \SI{,110}{\mol}; \num{50} }
