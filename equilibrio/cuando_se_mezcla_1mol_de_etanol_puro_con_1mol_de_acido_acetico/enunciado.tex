Cuando se mezcla \SI{1}{\mol} de etanol puro, \ce{CH3CH2OH(g)}, con \SI{1}{\mol} de ácido acético, \ce{CH3COOH(g)}, a cierta temperatura la mezcla en el equilibrio contiene $\rfrac{2}{3}$ de \si{\mol} de producto (acetato de etilo, \ce{CH3COOCH2CH3(g)}). Si todos los compuestos están en fase gas, calcule:
\begin{enumerate}[label={\alph*)},font=\bfseries]
	\item la constante de equilibrio,
	\item ¿cuántos moles de éster se forman en el equilibrio cuando se mezclan \SI{3}{\mol} de alcohol y uno de ácido?
\end{enumerate}
\resultadocmd{\num{4}; \SI{,903}{\mol}}
