\begin{frame}
	\frametitle{\ejerciciocmd}
	\framesubtitle{Enunciado}
	\textbf{
		Dadas las siguientes reacciones:
\begin{itemize}
    \item \ce{I2(g) + H2(g) -> 2 HI(g)}~~~$\Delta H_1 = \SI{-0,8}{\kilo\calorie}$
    \item \ce{I2(s) + H2(g) -> 2 HI(g)}~~~$\Delta H_2 = \SI{12}{\kilo\calorie}$
    \item \ce{I2(g) + H2(g) -> 2 HI(ac)}~~~$\Delta H_3 = \SI{-26,8}{\kilo\calorie}$
\end{itemize}
Calcular los parámetros que se indican a continuación:
\begin{description}%[label={\alph*)},font={\color{red!50!black}\bfseries}]
    \item[\texttt{a)}] Calor molar latente de sublimación del yodo.
    \item[\texttt{b)}] Calor molar de disolución del ácido yodhídrico.
    \item[\texttt{c)}] Número de calorías que hay que aportar para disociar en sus componentes el yoduro de hidrógeno gas contenido en un matraz de \SI{750}{\cubic\centi\meter} a \SI{25}{\celsius} y \SI{800}{\torr} de presión.
\end{description}
\resultadocmd{\SI{12,8}{\kilo\calorie}; \SI{-13,0}{\kilo\calorie}; \SI{12,9}{\calorie}}

		}
\end{frame}

\begin{frame}
	\frametitle{\ejerciciocmd}
	\framesubtitle{Datos del problema}
	\begin{center}
		{\huge ¿$K_x$?, ¿$n(\ce{CH3COOCH2CH3})$?}\\[.3cm]
		\tcbhighmath[boxrule=0.4pt,arc=4pt,colframe=blue,drop fuzzy shadow=green]{n_0(\ce{CH3CH2OH})=\SI{1}{\mol}}\quad
		\tcbhighmath[boxrule=0.4pt,arc=4pt,colframe=green,drop fuzzy shadow=blue]{n_0(\ce{CH3COOH})=\SI{1}{\mol}}\\[.3cm]
		\tcbhighmath[boxrule=0.4pt,arc=4pt,colframe=red,drop fuzzy shadow=orange]{n(\ce{CH3COOH})=\rfrac{2}{3}~\si{\mol}}
	\end{center}
\end{frame}

\begin{frame}
	\frametitle{\ejerciciocmd}
	\framesubtitle{Resolución (\rom{1}): determinación de $K_x$}
	\begin{overprint}
		\onslide<1>
			\structure{Reacción incompleta:} \ce{CH3COOH(l) + CH3CH2OH(l) <=> CH3COOCH2CH3(l)}\\
			(los carbonos están ajustados pero los hidrógenos y oxígenos no. No hay forma matemática salvo que falte un compuesto)
		\onslide<2->
			\structure{Reacción:} \ce{CH3COOH(l) + CH3CH2OH(l) <=> CH3COOCH2CH3(l) + H2O(l)}\\
			\centering(faltaba una molécula de agua)
	\end{overprint}
	\visible<2->{
%		\begin{overprint}
%			\onslide<3>
				\begin{center}
					\begin{tabular}{ccccc}
						\toprule
							$n(\si{\mol})$ &\ce{CH3COOH} & \ce{CH3CH2OH} & \ce{CH3COOCH2CH3} & \ce{H2O}\\
						\midrule
							Inicio:        &    1        &    1          & 0                 & 0\\
							Cambio:        &  $-y$       &  $-y$         & y                 & y\\
							Equilibrio:    & $1-y$       & $1-y$         & y                 & y\\
						\bottomrule
					\end{tabular}
				\end{center}
%			\onslide<4>
				\centering Pero $y=\rfrac{2}{3}~\si{\mol}$ según el enunciado {\footnotesize (reservamos $x$ para representar la fracción molar)}.
%			\onslide<5->
				\begin{center}
					\begin{tabular}{ccccc}
						\toprule
							$n(\si{\mol})$ &\ce{CH3COOH}     & \ce{CH3CH2OH}   & \ce{CH3COOCH2CH3} & \ce{H2O}\\
						\midrule
							Inicio:        &        1        &      1          & 0              & 0\\
							Cambio:        & $-\rfrac{2}{3}$ & $-\rfrac{2}{3}$ & $\rfrac{2}{3}$ & $\rfrac{2}{3}$\\
							Equilibrio:    &  $\rfrac{1}{3}$ &  $\rfrac{1}{3}$ & $\rfrac{2}{3}$ & $\rfrac{2}{3}$\\
						\bottomrule
					\end{tabular}
				\end{center}
%		\end{overprint}
				}
	\visible<2->{
		\structure{Por la definición de fracción molar:} $x_i=\frac{n_i}{n_{\text{total}}}$
				$$
					K_x = \frac{x_{\ce{CH3COOCH2CH3}}\vdot x_{\ce{H2O}}}{x_{\ce{CH3COOH}}\vdot x_{\ce{CH3CH2OH}}}\Rightarrow
					K_x = \frac{n(\ce{CH3COOCH2CH3})\vdot\cancel{\frac{1}{n_{\text{total}}}}\vdot n(\ce{H2O})\vdot\cancel{\frac{1}{n_{\text{total}}}}}{n(\ce{CH3COOH})\vdot\cancel{\frac{1}{n_{\text{total}}}}\vdot n(\ce{CH3CH2OH})\vdot\cancel{\frac{1}{n_{\text{total}}}}}
				$$
				$$
					K_x = \frac{n(\ce{CH3COOCH2CH3})\vdot n(\ce{H2O})}{n(\ce{CH3COOH})\vdot n(\ce{CH3CH2OH})}\Rightarrow
					K_x = \frac{\frac{2}{\cancel{3}}\cdot\frac{2}{\cancel{3}}}{\frac{1}{\cancel{3}}\cdot\frac{1}{\cancel{3}}}\Rightarrow
					\tcbhighmath[boxrule=0.4pt,arc=4pt,colframe=blue,drop fuzzy shadow=black]{K_x=4}
				$$
				}
\end{frame}

\begin{frame}
	\frametitle{\ejerciciocmd}
	\framesubtitle{Resolución (\rom{2}): $n(\ce{C4H8O2})$ si $n_0(\ce{CH3COOH})=1$ y $n_0(\ce{CH3CH2OH})=3$}
	\structure{Reacción:} \ce{CH3COOH(l) + CH3CH2OH(l) <=> CH3COOCH2CH3(l) + H2O(l)}\\
	\begin{center}
		\begin{tabular}{ccccc}
			\toprule
			$n(\si{\mol})$ &\ce{CH3COOH}     & \ce{CH3CH2OH}   & \ce{CH3COOCH2CH3} & \ce{H2O}\\
			\midrule
			Inicio:        &  1    &  3    &  0  &  0 \\
			Cambio:        &  $-z$ &  $-z$ & $z$ & $z$\\
			Equilibrio:    & $1-z$ & $3-z$ & $z$ & $z$\\
			\bottomrule
		\end{tabular}
	\end{center}
	\structure{La constante de equilibrio será:}
			$$
				K_x = \frac{x_{\ce{CH3COOCH2CH3}}\vdot x_{\ce{H2O}}}{x_{\ce{CH3COOH}}\vdot x_{\ce{CH3CH2OH}}}
				\overbrace{K_x}^4 = \frac{\overbrace{n(\ce{CH3COOCH2CH3})}^{z}\cdot\overbrace{n(\ce{H2O})}^{z}}{\underbrace{n(\ce{CH3COOH})}_{1-z}\cdot\underbrace{n(\ce{CH3CH2OH})}_{3-z}}
			$$
			$$
				4 = \frac{z^2}{(1-z)\cdot(3-z)}\Rightarrow
				4 = \frac{z^2}{z^2-4z+3}\Rightarrow
				3z^2-16z+12 = 0
			$$
			\structure{Soluciones de la ecuación de 2"o grado:}
			$$
				z_1 = \SI{4,43}{\mol}>n_0(\ce{CH3COOH})\text{ (solución no válida)}
			$$
			\begin{center}
				\tcbhighmath[boxrule=0.4pt,arc=4pt,colframe=red,drop fuzzy shadow=orange]{z_2 = \SI{,903}{\mol}$\text{ de \ce{CH3COOCH2CH3}}$}
			\end{center}
\end{frame}
