Se calentó una cierta cantidad de pentacloruro de fósforo gas, \ce{PCl5}, en un matraz de \SI{12}{\liter} y \SI{250}{\celsius}. El pentacloruro se descompone en tricloruro de fósforo, \ce{PCl3}, y cloro (gases ambos), de forma que se alcanza el equilibrio cuando en el sistema hay \SI{0,21}{\mol} de pentacloruro, \SI{,32}{\mol} de tricloruro y \SI{,32}{\mol} de cloro. Calcule la constante de equilibrio de la reacción en esas condiciones.
\resultadocmd{\num{1,74}, \num{176740}; \num{0,040}, \num{40,63}}
