\begin{frame}
	\frametitle{\ejerciciocmd}
	\framesubtitle{Enunciado}
	\textbf{
		Dadas las siguientes reacciones:
\begin{itemize}
    \item \ce{I2(g) + H2(g) -> 2 HI(g)}~~~$\Delta H_1 = \SI{-0,8}{\kilo\calorie}$
    \item \ce{I2(s) + H2(g) -> 2 HI(g)}~~~$\Delta H_2 = \SI{12}{\kilo\calorie}$
    \item \ce{I2(g) + H2(g) -> 2 HI(ac)}~~~$\Delta H_3 = \SI{-26,8}{\kilo\calorie}$
\end{itemize}
Calcular los parámetros que se indican a continuación:
\begin{description}%[label={\alph*)},font={\color{red!50!black}\bfseries}]
    \item[\texttt{a)}] Calor molar latente de sublimación del yodo.
    \item[\texttt{b)}] Calor molar de disolución del ácido yodhídrico.
    \item[\texttt{c)}] Número de calorías que hay que aportar para disociar en sus componentes el yoduro de hidrógeno gas contenido en un matraz de \SI{750}{\cubic\centi\meter} a \SI{25}{\celsius} y \SI{800}{\torr} de presión.
\end{description}
\resultadocmd{\SI{12,8}{\kilo\calorie}; \SI{-13,0}{\kilo\calorie}; \SI{12,9}{\calorie}}

		}
\end{frame}

\begin{frame}
	\frametitle{\ejerciciocmd}
	\framesubtitle{Datos del problema}
	\begin{center}
		\tcbhighmath[boxrule=0.4pt,arc=4pt,colframe=black,drop fuzzy shadow=black]{\text{Reacción: }\ce{PCl5(g) <=> PCl3(g) + Cl2(g)}}\quad
		\tcbhighmath[boxrule=0.4pt,arc=4pt,colframe=black,drop fuzzy shadow=black]{V=\SI{12}{\liter}}\quad
		\tcbhighmath[boxrule=0.4pt,arc=4pt,colframe=black,drop fuzzy shadow=black]{T=\SI{523,15}{\kelvin}}\\[.4cm]
		\tcbhighmath[boxrule=0.4pt,arc=4pt,colframe=blue,drop fuzzy shadow=green]{n(\ce{PCl5})=\SI{,21}{\mol}}\quad
		\tcbhighmath[boxrule=0.4pt,arc=4pt,colframe=green,drop fuzzy shadow=blue]{n(\ce{PCl3})=\SI{,32}{\mol}}\quad
		\tcbhighmath[boxrule=0.4pt,arc=4pt,colframe=orange,drop fuzzy shadow=blue]{n(\ce{Cl2})=\SI{,32}{\mol}}
	\end{center}
\end{frame}

\begin{frame}
	\frametitle{\ejerciciocmd}
	\framesubtitle{Resolución (\rom{1}): determinación de $K_p$}
	\structure{Teniendo en cuenta la reacción:} \ce{PCl5(g) <=> PCl3(g) + Cl2(g)}
	\structure{y usando la ecuación de los gases ideales:} $P\cdot V=n\cdot R\cdot T\Rightarrow P=\frac{n\cdot R\cdot T}{V}$
	\structure{Las presiones parciales serán:}
	$$
		P(\ce{PCl3}) = \frac{\SI{,32}{\mol}\vdot\SI{,082}{\atm\cancel\liter\per\cancel\mol\per\cancel\kelvin}\vdot\SI{523,15}{\kelvin}}{\SI{12}{\liter}} = \SI{1,14}{\atm} = P(\ce{Cl2})
	$$
	$$
		P(\ce{PCl5}) = \frac{\SI{,21}{\mol}\vdot\SI{,082}{\atm\cancel\liter\per\cancel\mol\per\cancel\kelvin}\vdot\SI{523,15}{\kelvin}}{\SI{12}{\liter}} = \SI{,75}{\atm}
	$$
	\structure{La presión total será en consecuencia:} $P_{\text{total}} = \num{2}\vdot\SI{1,14}{\atm} + \SI{,75}{\atm} = \SI{3,03}{\atm}$
	\structure{La constante de equilibrio $K_p$ será:}
	$$
		K_p = \frac{P(\ce{PCl3})\vdot P(\ce{Cl2})}{P(\ce{PCl5})}\Rightarrow
		K_p = \frac{\num{1,14}\vdot\num{1,14}}{\num{,75}}
	$$
	\begin{center}
		\tcbhighmath[boxrule=0.4pt,arc=4pt,colframe=black,drop fuzzy shadow=black]{K_p=\num{1,74}}
	\end{center}
\end{frame}

\begin{frame}
	\frametitle{\ejerciciocmd}
	\framesubtitle{Resolución (\rom{2}): determinación de $K_c$}
	\structure{Teniendo en cuenta la reacción:} \ce{PCl5(g) <=> PCl3(g) + Cl2(g)}
	\structure{y usando la ecuación de los gases ideales:} $P\cdot V=n\cdot R\cdot T\Rightarrow \frac{n}{V}=M=\frac{P}{R\cdot T}$
	\structure{La constante de equilibrio $K_c$ será:}
	$$
		K_c = \frac{[\ce{PCl3}]\vdot[\ce{Cl2}]}{[\ce{PCl5}]}\Rightarrow
		K_c = \frac{\frac{P(\ce{PCl3})}{\cancel{RT}}\vdot\frac{P(\ce{Cl2})}{RT}}{\frac{P(\ce{PCl5})}{\cancel{RT}}}\Rightarrow
		K_c = \overbrace{\frac{P(\ce{PCl3})\vdot P(\ce{Cl2})}{P(\ce{PCl5})}}^{K_p}\vdot\frac{1}{RT}
	$$
	$$
		K_c = K_p\vdot\frac{1}{R\vdot T}\Rightarrow
		K_c = \num{1,74}\vdot\frac{1}{\num{,082}\vdot\num{523,15}}
	$$
	\begin{center}
		\tcbhighmath[boxrule=0.4pt,arc=4pt,colframe=black,drop fuzzy shadow=black]{K_c=\num{,04063}}
	\end{center}
\end{frame}

\begin{frame}
	\frametitle{\ejerciciocmd}
	\framesubtitle{Resolución (\rom{3}): determinación de $K_x$}
	\structure{Teniendo en cuenta la reacción:} \ce{PCl5(g) <=> PCl3(g) + Cl2(g)}
	\structure{y usando la consecuencia de la ley de Dalton:} $P_i = x_i\vdot P_{\text{total}}\Rightarrow x_i = \rfrac{P_i}{P_{\text{total}}}$
	\structure{La constante de equilibrio $K_x$ será:}
	$$
		K_x = \frac{x_{\ce{PCl3}}\vdot x_{\ce{Cl2}}}{x_{\ce{PCl5}}}\Rightarrow
		K_x = \frac{\frac{P(\ce{PCl3})}{\cancel{P_{\text{total}}}}\vdot\frac{P(\ce{Cl2})}{P_{\text{total}}}}{\frac{P(\ce{PCl5})}{\cancel{P_{\text{total}}}}}\Rightarrow
		K_x = \overbrace{\frac{P(\ce{PCl3})\vdot P(\ce{Cl2})}{P(\ce{PCl5})}}^{K_p}\vdot\frac{1}{P_{\text{total}}}
	$$
	$$
		K_x = \num{1,74}\vdot\underbrace{\frac{1}{\num{3,03}}}_{\text{calculado en Resolución~\rom{1}}}
	$$
	\begin{center}
		\tcbhighmath[boxrule=0.4pt,arc=4pt,colframe=black,drop fuzzy shadow=black]{K_x=\num{,58}}
	\end{center}
\end{frame}
