\begin{frame}
	\frametitle{\ejerciciocmd}
	\framesubtitle{Enunciado}
	\textbf{
		Dadas las siguientes reacciones:
\begin{itemize}
    \item \ce{I2(g) + H2(g) -> 2 HI(g)}~~~$\Delta H_1 = \SI{-0,8}{\kilo\calorie}$
    \item \ce{I2(s) + H2(g) -> 2 HI(g)}~~~$\Delta H_2 = \SI{12}{\kilo\calorie}$
    \item \ce{I2(g) + H2(g) -> 2 HI(ac)}~~~$\Delta H_3 = \SI{-26,8}{\kilo\calorie}$
\end{itemize}
Calcular los parámetros que se indican a continuación:
\begin{description}%[label={\alph*)},font={\color{red!50!black}\bfseries}]
    \item[\texttt{a)}] Calor molar latente de sublimación del yodo.
    \item[\texttt{b)}] Calor molar de disolución del ácido yodhídrico.
    \item[\texttt{c)}] Número de calorías que hay que aportar para disociar en sus componentes el yoduro de hidrógeno gas contenido en un matraz de \SI{750}{\cubic\centi\meter} a \SI{25}{\celsius} y \SI{800}{\torr} de presión.
\end{description}
\resultadocmd{\SI{12,8}{\kilo\calorie}; \SI{-13,0}{\kilo\calorie}; \SI{12,9}{\calorie}}

		}
\end{frame}

\begin{frame}
	\frametitle{\ejerciciocmd}
	\framesubtitle{Datos del problema}
	{\huge $¿n_\text{eq}(\ce{H2})$, $n_\text{eq}(\ce{I2})$ y $n_\text{eq}(\ce{HI})$?}
	\structure{Reacción (ajustada):} \ce{H2(g) + I2(g) <=> 2 HI(g)}\qquad$K_x = 10$
	\structure{Condiciones iniciales:}
	\begin{center}
		\tcbhighmath[boxrule=0.4pt,arc=4pt,colframe=red,drop fuzzy shadow=blue]{n_\text{iniciales}(\ce{H2})=\SI{1}{\mol}}\qquad
		\tcbhighmath[boxrule=0.4pt,arc=4pt,colframe=blue,drop fuzzy shadow=red]{n_\text{iniciales}(\ce{I2})=\SI{e-3}{\mol}}
	\end{center}
\end{frame}

\begin{frame}
	\frametitle{\ejerciciocmd}
	\framesubtitle{Resolucion (\rom{1}): condiciones de equilibrio}
	\begin{center}
		\begin{tabular}{lccc}
				& \multicolumn{3}{c}{\ce{H2(g) + I2(s) <=> 2HI(g)}}		\\
			\midrule
				(\si{\mol}) & \ce{H2} 		& \ce{I2} 		& \ce{HI}	\\
				Inicial		& \num{1}		& \num{e-3}		& \num{0}	\\
				Equilibrio	& $\num{1}-x$	& $\num{e-3}-x$	& $2x$
		\end{tabular}
	\end{center}
	\structure{N"o de moles totales en fase gas:} $n_T = \num{1}-x + 2x = \num{1} + x$
	\structure{Fracción molar ($x$) de sustancias gaseosas:}
		$$
			x_{\ce{H2}} = \frac{\num{1}-x}{\num{1}+x};\qquad x_{\ce{HI}} = \frac{2x}{\num{1}+x}
		$$
	\structure{Constante de equilibrio:}
		$$
			K_x = 	\frac{x^2_{\ce{HI}}}{x_{\ce{H2}}} = 
						\frac{
							\frac{4x^2}{(\num{1}+x)^{\cancel{2}}}
						}{
							\frac{\num{1}-x}{\cancel{\num{1}+x}}
						}=
					\frac{4x^2}{(\num{1}+x)\vdot(\num{1}-x)} =
					\frac{4x^2}{\num{1}-x^2} = 10
		$$
		$$
			x=\SI{,845}{\mol}
		$$
		\begin{center}
			\tcbhighmath[boxrule=0.4pt,arc=4pt,colframe=blue,drop fuzzy shadow=red]{\text{Imposible porque }x>>\SI{e-3}{\mol}\text{. Debe de haber exceso de reactivo sólido}}
		\end{center}

\end{frame}
