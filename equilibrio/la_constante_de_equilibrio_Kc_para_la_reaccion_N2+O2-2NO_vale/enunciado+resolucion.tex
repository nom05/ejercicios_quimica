\begin{frame}
	\frametitle{\ejerciciocmd}
	\framesubtitle{Enunciado}
	\textbf{
		Dadas las siguientes reacciones:
\begin{itemize}
    \item \ce{I2(g) + H2(g) -> 2 HI(g)}~~~$\Delta H_1 = \SI{-0,8}{\kilo\calorie}$
    \item \ce{I2(s) + H2(g) -> 2 HI(g)}~~~$\Delta H_2 = \SI{12}{\kilo\calorie}$
    \item \ce{I2(g) + H2(g) -> 2 HI(ac)}~~~$\Delta H_3 = \SI{-26,8}{\kilo\calorie}$
\end{itemize}
Calcular los parámetros que se indican a continuación:
\begin{description}%[label={\alph*)},font={\color{red!50!black}\bfseries}]
    \item[\texttt{a)}] Calor molar latente de sublimación del yodo.
    \item[\texttt{b)}] Calor molar de disolución del ácido yodhídrico.
    \item[\texttt{c)}] Número de calorías que hay que aportar para disociar en sus componentes el yoduro de hidrógeno gas contenido en un matraz de \SI{750}{\cubic\centi\meter} a \SI{25}{\celsius} y \SI{800}{\torr} de presión.
\end{description}
\resultadocmd{\SI{12,8}{\kilo\calorie}; \SI{-13,0}{\kilo\calorie}; \SI{12,9}{\calorie}}

	}
\end{frame}

\begin{frame}
	\frametitle{\ejerciciocmd}
	\framesubtitle{Datos del problema}
	\begin{center}
		{\Large\textbf{¿N"o de moles de cada sustancia en el equilibrio?}}\\[.2cm]
		\tcbhighmath[boxrule=0.4pt,arc=4pt,colframe=black,drop fuzzy shadow=red]{K_c = \num{8,8e-4}}\quad
		\tcbhighmath[boxrule=0.4pt,arc=4pt,colframe=black,drop fuzzy shadow=red]{V = \SI{2}{\liter}}\\[.2cm]
		\ce{N2(g) + O2(g) <=> NO(g)} {\tiny (sin ajustar)}
		\structure{Condiciones iniciales:}\\[.2cm]
		\tcbhighmath[boxrule=0.4pt,arc=4pt,colframe=red,drop fuzzy shadow=blue]{n_0(\ce{N2}) = \SI{2,0}{\mol}}\quad
		\tcbhighmath[boxrule=0.4pt,arc=4pt,colframe=blue,drop fuzzy shadow=red]{n_0(\ce{H2}) = \SI{1,0}{\mol}}
	\end{center}
\end{frame}

\begin{frame}
	\frametitle{\ejerciciocmd}
	\framesubtitle{Resolución (\rom{1}): presión total en el equilibrio}
	\structure{Reacción de equilibrio y expresión de $K_c$:}\quad\ce{N2(g) + H2(g) <=> 2NO(g)}
	\begin{center}
		\begin{tabular}{lSSS}
			\toprule
				Estado		& {$n(\ce{N2})~(\si{\mol})$}	&  {$n(\ce{H2})~(\si{\mol})$}	&  {$n(\ce{NO})~(\si{\mol})$}	\\
			\midrule
				Inicial		&            2,0				&	1,0 						&	0							\\
				Reaccionan	&			{$-x$}				&	{$-x$}						&	{$2x$}						\\
				Equilibrio	&			{$\num{2,0}-x$}		&	{$\num{1,0}-x$}				&	{$2x$}						\\
			\bottomrule
		\end{tabular}
	\end{center}
	$$
		K_c=\underbrace{\frac{[\ce{NO}]^2}{[\ce{N2}][\ce{H2}]}}_{M=\frac{n}{V}}=\frac{\rfrac{n(\ce{NO})^2}{\cancel{V^2}}}{\rfrac{n(\ce{N2})}{\cancel{V}}\vdot \rfrac{n(\ce{H2})}{\cancel{V}}}=\num{8,8e-4}\Rightarrow
		K_c = \frac{(2x)^2}{\left(2-x\right)\left(1-x\right)}=\num{8,8e-4}\Rightarrow
	$$
	$$
		 4544,45x^2 + 3x -2 =0
		\begin{cases}
			x_1 = \SI{,02065}{\mol}\\
			\cancel{x_2 < 0}\quad\text{(Químicamente no posible)}
		\end{cases}
	$$
	$$
		\tcbhighmath[boxrule=0.4pt,arc=4pt,colframe=red,drop fuzzy shadow=blue]{n(\ce{N2}) = \SI{2,0}{\mol} - \SI{,02065}{\mol} = \SI{1,9794}{\mol}}
	$$
	$$
		\tcbhighmath[boxrule=0.4pt,arc=4pt,colframe=blue,drop fuzzy shadow=red]{n(\ce{H2}) = \SI{1,0}{\mol} - \SI{,02065}{\mol} = \SI{,9794}{\mol}}
	$$
	$$
		\tcbhighmath[boxrule=0.4pt,arc=4pt,colframe=green,drop fuzzy shadow=black]{n(\ce{NO}) = 2\times\SI{,02065}{\mol} = \SI{,0430}{\mol}}
	$$
\end{frame}