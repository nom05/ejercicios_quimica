\begin{frame}
	\frametitle{\ejerciciocmd}
	\framesubtitle{Enunciado}
	\textbf{
		Dadas las siguientes reacciones:
\begin{itemize}
    \item \ce{I2(g) + H2(g) -> 2 HI(g)}~~~$\Delta H_1 = \SI{-0,8}{\kilo\calorie}$
    \item \ce{I2(s) + H2(g) -> 2 HI(g)}~~~$\Delta H_2 = \SI{12}{\kilo\calorie}$
    \item \ce{I2(g) + H2(g) -> 2 HI(ac)}~~~$\Delta H_3 = \SI{-26,8}{\kilo\calorie}$
\end{itemize}
Calcular los parámetros que se indican a continuación:
\begin{description}%[label={\alph*)},font={\color{red!50!black}\bfseries}]
    \item[\texttt{a)}] Calor molar latente de sublimación del yodo.
    \item[\texttt{b)}] Calor molar de disolución del ácido yodhídrico.
    \item[\texttt{c)}] Número de calorías que hay que aportar para disociar en sus componentes el yoduro de hidrógeno gas contenido en un matraz de \SI{750}{\cubic\centi\meter} a \SI{25}{\celsius} y \SI{800}{\torr} de presión.
\end{description}
\resultadocmd{\SI{12,8}{\kilo\calorie}; \SI{-13,0}{\kilo\calorie}; \SI{12,9}{\calorie}}

		}
\end{frame}

\begin{frame}
	\frametitle{\ejerciciocmd}
	\framesubtitle{Datos del problema}
	{\huge\begin{enumerate*}[label={\alph*)},font=\bfseries]
		\item ¿$\Delta G^0$ y $K_p$?
		\item ¿$\Delta G_{\text{inv}}^0$ y $K_p^{\text{inv}}$?
		\item ¿$\Delta G^0_{\rfrac{1}{2}}$ y $K_p^{\rfrac{1}{2}}$?
	\end{enumerate*}}
	\begin{center}
		\tcbhighmath[boxrule=0.4pt,arc=4pt,colframe=black,drop fuzzy shadow=black]{T=\SI{298,15}{\kelvin}}\\[.3cm]
		\tcbhighmath[boxrule=0.4pt,arc=4pt,colframe=blue,drop fuzzy shadow=blue]{\ce{2NO2 <=> N2O4}}\\[.3cm]
		\tcbhighmath[boxrule=0.4pt,arc=4pt,colframe=green,drop fuzzy shadow=green]{\ce{NO2 <=> 1/2N2O4}}\\[.3cm]
		\tcbhighmath[boxrule=0.4pt,arc=4pt,colframe=blue,drop fuzzy shadow=green]{\Delta G_{\text{formación}}^0(\ce{NO2})=\Delta G_f^0(\ce{NO2})=\SI{51,30}{\kilo\joule\per\mol}}\\[.3cm]
		\tcbhighmath[boxrule=0.4pt,arc=4pt,colframe=green,drop fuzzy shadow=blue]{\Delta G_{\text{formación}}^0(\ce{N2O4})=\Delta G_f^0(\ce{N2O4})=\SI{97,82}{\kilo\joule\per\mol}}
	\end{center}
\end{frame}

\begin{frame}
	\frametitle{\ejerciciocmd}
	\framesubtitle{Resolución (\rom{1}): $\Delta G^0$ y $K_p$ en condiciones estándar de \ce{2NO2 <=> N2O4}}
	\begin{block}{$\Delta G^0$}
		\structure{Ley de Hess:} $\Delta G^0 = \sum_{i=1}^{n\text{ prod}}n_i\cdot\Delta G_{f,i}^0 - \sum_{j=1}^{n\text{ react}}n_j\cdot\Delta G_{f,j}^0$
		$$
			\Delta G^0 = 1\cdot\SI{97,82}{\kilo\joule\per\mol}-2\cdot\SI{51,30}{\kilo\joule\per\mol}
		$$
		\centering\tcbhighmath[boxrule=0.4pt,arc=4pt,colframe=blue,drop fuzzy shadow=blue]{\Delta G^0 =\SI{-4,78}{\kilo\joule\per\mol}<0}
		\centering\myovalbox{\textcolor{yellow}{Espontánea}}
	\end{block}
	\begin{alertblock}<1->{$K_p$}
		\structure{Aplicando:} $\Delta G = \Delta G^0 + RT\ln{K_p}$, si $\Delta G=0$ entonces $\Delta G^0= -RT\ln{K_p}$
				$$
					\overbrace{\Delta G^0}^{\SI{-4,78}{\kilo\joule\per\mol}} = \SI{-8,314}{\cancel\joule\per\mol\per\cancel\kelvin}\cdot\frac{\SI{1}{\kilo\joule}}{\SI{1e3}{\cancel\joule}}\cdot\SI{298,15}{\cancel\kelvin}\cdot\ln{K_p}
				$$
				$$
					\ln{K_p} = \frac{\SI{-4,78}{\cancel\kilo\joule\per\cancel\mol}}{\num{-8,314e-3}\cdot\SI{298,15}{\cancel\kilo\joule\per\cancel\mol}}=\num{1,928}\Rightarrow
					\tcbhighmath[boxrule=0.4pt,arc=4pt,colframe=black,drop fuzzy shadow=black]{K_p=\num{6,88}=\frac{P(\ce{N2O4})}{P(\ce{NO2})^2}}
				$$
				\centering\myovalbox{\textcolor{yellow}{Reacción favorecida hacia productos}}
	\end{alertblock}
\end{frame}

\begin{frame}
	\frametitle{\ejerciciocmd}
	\framesubtitle{Resolución (\rom{2}): $\Delta G^0$ y $K_p$ en condiciones estándar de \ce{N2O4 <=> 2NO2}}
	\begin{block}{$\Delta G^0_{\text{inv}}$}
		\structure{Reacción inversa, cambio de signo:} $\Delta G^0_{\text{inv}} = -\Delta G^0$\\[.4cm]
		\centering\tcbhighmath[boxrule=0.4pt,arc=4pt,colframe=blue,drop fuzzy shadow=blue]{\Delta G^0_{\text{inv}} =\SI{4,78}{\kilo\joule\per\mol}>0}
	\end{block}
	\begin{alertblock}<1->{$K_p^{\text{inv}}$}
		\structure{Reacción inversa, invertimos $K_p^{\text{inv}}=\rfrac{1}{K_p}$:}\\[.4cm]
		\centering\tcbhighmath[boxrule=0.4pt,arc=4pt,colframe=black,drop fuzzy shadow=black]{K_p^{\text{inv}}=\frac{1}{\num{6,88}}=\num{,145}}
	\end{alertblock}
\end{frame}

\begin{frame}
	\frametitle{\ejerciciocmd}
	\framesubtitle{Resolución (\rom{3}): $\Delta G^0$ y $K_p$ en condiciones estándar de \ce{NO2 <=> 1/2N2O4}}
	\begin{block}{$\Delta G^0_{\rfrac{1}{2}}$}
		$$
			\Delta G^0_{\rfrac{1}{2}} = \frac{1}{2}\cdot\Delta G^0\Rightarrow\Delta G^0_{\rfrac{1}{2}} = \frac{1}{2}\cdot\left(\SI{97,82}{\kilo\joule\per\mol}-2\cdot\SI{51,30}{\kilo\joule\per\mol}\right)
		$$
		\centering\tcbhighmath[boxrule=0.4pt,arc=4pt,colframe=blue,drop fuzzy shadow=blue]{\Delta G^0_{\rfrac{1}{2}} =\SI{-2,39}{\kilo\joule\per\mol}<0}
		\centering\myovalbox{\textcolor{yellow}{Espontánea}}
	\end{block}
	\begin{alertblock}<1->{$K_p^{\rfrac{1}{2}}$}
		\structure{Aplicando:} $\Delta G = \Delta G^0 + RT\ln{K_p}$, si $\Delta G=0$ entonces $\Delta G^0= -RT\ln{K_p}$
		$$
			\ln K_p^{\rfrac{1}{2}} = -\frac{\overbrace{\Delta G^0_{\rfrac{1}{2}}}^{\Delta G^0_{\rfrac{1}{2}}=\frac{1}{2}\cdot\Delta G^0}}{RT}\Rightarrow
			\ln K_p^{\rfrac{1}{2}} = \frac{1}{2}\cdot\overbrace{-\frac{\Delta G^0}{RT}}^{\ln K_p}\Rightarrow
			\ln K_p^{\rfrac{1}{2}} = \frac{1}{2}\ln K_p\Rightarrow
		$$
		$$
			\Rightarrow\ln K_p^{\rfrac{1}{2}} = \ln\left(\sqrt{K_p}\right)\Rightarrow
			K_p^{\rfrac{1}{2}} = \sqrt{K_p}\Rightarrow
			\tcbhighmath[boxrule=0.4pt,arc=4pt,colframe=black,drop fuzzy shadow=black]{K_p^{\rfrac{1}{2}}=\num{2,62}}
		$$
		\centering\myovalbox{\textcolor{yellow}{Reacción favorecida hacia productos}}
	\end{alertblock}
\end{frame}
