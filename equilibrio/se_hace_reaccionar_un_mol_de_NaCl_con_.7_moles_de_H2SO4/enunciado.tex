Se hace reaccionar un mol de cloruro de sodio, \ce{NaCl}, con \SI{,7}{\mol} de ácido sulfúrico, \ce{H2SO4}, y \SI{,26}{\mol} de sulfato de sodio, \ce{Na2SO4}. Cuando se alcanza el equilibrio la cantidad de moles de \ce{NaCl} presente es de \num{,34}. Calcule cuál es la cantidad de ácido clorhídrico, \ce{HCl}, que se ha formado.
\resultadocmd{\SI{,66}{\mol}}