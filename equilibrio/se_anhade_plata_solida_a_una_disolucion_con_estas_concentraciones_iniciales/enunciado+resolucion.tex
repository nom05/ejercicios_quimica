\begin{frame}
	\frametitle{\ejerciciocmd}
	\framesubtitle{Enunciado}
	\textbf{
		Una reacción tiene una constante de velocidad de \SI{,017}{\per\second} a \SI{298}{\kelvin} y una energía libre de activación del \SI{27,235}{\kilo\joule\per\mol}. La adición de un catalizador disminuye dicha energía de activación hasta un \SI{33}{\percent} de su valor inicial. Calcule la nueva constante de velocidad.
\resultadocmd{ \SI{26,86}{\per\second} }

		}
\end{frame}

\begin{frame}
	\frametitle{\ejerciciocmd}
	\framesubtitle{Datos del ejercicio}
	\begin{center}
		{\LARGE ¿$[\ce{Ag+}]_{eq}$, $[\ce{Fe^2+}]_{eq}$ y $[\ce{Fe^3+}]_{eq}$?}
		\structure{Reacción (ajustada):}\quad\ce{Ag+(ac) + Fe^2+(ac) <=> Ag(s) + Fe^3+(ac)}\quad
		\tcbhighmath[boxrule=0.4pt,arc=4pt,colframe=black,drop fuzzy shadow=black]{K_c=\num{2,98}}\\[.4cm]
		\tcbhighmath[boxrule=0.4pt,arc=4pt,colframe=blue,drop fuzzy shadow=green]{[\ce{Ag+}]=\SI{,200}{\Molar}}\quad
		\tcbhighmath[boxrule=0.4pt,arc=4pt,colframe=green,drop fuzzy shadow=orange]{[\ce{Fe^2+}]=\SI{,100}{\Molar}}\quad
		\tcbhighmath[boxrule=0.4pt,arc=4pt,colframe=yellow,drop fuzzy shadow=gray]{[\ce{Fe^3+}]=\SI{,300}{\Molar}}
	\end{center}
\end{frame}

\begin{frame}
	\frametitle{Resolución (\rom{1}): sentido del desplazamiento del equilibrio}
	\structure{Reacción:} \ce{Ag+(ac) + Fe^2+(ac) <=> Ag(s) + Fe^3+(ac)}\\[.3cm]
	\structure{Determinar en qué sentido ocurre el cambio neto del equilibrio (hacia reactivos o productos):} Como están presentes todos los reactivos y productos inicialmente, necesitamos usar el producto de reacción $Q_C$ (con las concentraciones de no equilibrio) y compararlo con $K_C$.
	$$
		Q_C = \frac{\overbrace{[\ce{Fe^3+}]}^{\num{,300}}}{\underbrace{[\ce{Ag+}]}_{\num{,200}}\vdot\underbrace{[\ce{Fe^2+}]}_{\num{,100}}}\Rightarrow
		Q_C = \frac{\num{,300}}{\num{,200}\vdot\num{,100}} = \num{15} > \num{2,98}=K_c
	$$
	\textbf{Como $Q_c$ es mayor que $K_c$}, la proporción de productos (numerador) es mayor que en el equilibrio, por lo que el \textbf{equilibrio se desplaza hacia la izquierda}. Sin embargo, puede ocurrir que el sólido no sea suficiente (\textit{reactivo limitante}) y la reacción pare antes de alcanzar el equilibrio (comprobarlo). En este caso no nos dicen nada así que suponemos que hay suficiente sólido.
\end{frame}

\begin{frame}
	\frametitle{Resolución (\rom{3}): concentraciones en el equilibrio}
	\structure{Reacción:} \ce{Ag+(ac) + Fe^2+(ac) <=> Ag(s) + Fe^3+(ac)}\\[.3cm]
	\begin{center}
		\begin{tabular}{cccc}
			\toprule
				(\si{\Molar}) & [\ce{Ag+}]     & [\ce{Fe^2+}]   & [\ce{Fe^3+}]   \\
			\midrule
				Inicial       & \num{,200}     & \num{,100}     & \num{,300}     \\
				Reacciona     &    $+x$        &    $+x$        &    $-x$        \\
				Equilibrio    & $\num{,200}+x$ & $\num{,100}+x$ & $\num{,300}-x$ \\
			\bottomrule
		\end{tabular}
	\end{center}
	$$
		K_c = \frac{[\ce{Fe^3+}]}{[\ce{Ag+}]\vdot[\ce{Fe^2+}]}\Rightarrow K_c = \frac{\num{,300}-x}{(\num{,200}+x)(\num{,100}+x)} = \num{2,98}
	$$
	$$
		x^2 + \num{,6356}x - \num{,0807} = 0 \Rightarrow
		\begin{dcases}
			x_1 = \SI{,1084}{\Molar}\\
			\cancel{x_2 < 0}
		\end{dcases}
	$$
	\begin{multicols}{2}
			Usando $x_1$ y la última fila de la tabla donde vemos las concentraciones de equilibrio:
			$$
				\tcbhighmath[boxrule=0.4pt,arc=4pt,colframe=blue,drop fuzzy shadow=green]{[\ce{Ag+}]=\SI{,200}{\Molar}+\SI{,108}{\Molar}=\SI{,308}{\Molar}}
			$$
			$$
				\tcbhighmath[boxrule=0.4pt,arc=4pt,colframe=green,drop fuzzy shadow=orange]{[\ce{Fe^2+}]=\SI{,100}{\Molar}+\SI{,108}{\Molar}=\SI{,208}{\Molar}}
			$$
			$$
				\tcbhighmath[boxrule=0.4pt,arc=4pt,colframe=yellow,drop fuzzy shadow=gray]{[\ce{Fe^3+}]=\SI{,300}{\Molar}-\SI{,108}{\Molar}=\SI{,192}{\Molar}}
			$$
			Comprobamos nuestro resultado:
			$$
				\frac{\SI{,192}{\Molar}}{\SI{,308}{\Molar}\times\SI{,208}{\Molar}} \approx\num{3}
			$$
			{\footnotesize Añadiendo una cifra significativa más obtenemos el valor de \num{2,98}. Sin embargo, nuestros datos parten de tres.}
	\end{multicols}
\end{frame}
