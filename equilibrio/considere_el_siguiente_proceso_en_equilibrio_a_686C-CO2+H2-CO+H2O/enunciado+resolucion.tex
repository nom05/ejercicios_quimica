\begin{frame}
	\frametitle{\ejerciciocmd}
	\framesubtitle{Enunciado}
	\textbf{
		Dadas las siguientes reacciones:
\begin{itemize}
    \item \ce{I2(g) + H2(g) -> 2 HI(g)}~~~$\Delta H_1 = \SI{-0,8}{\kilo\calorie}$
    \item \ce{I2(s) + H2(g) -> 2 HI(g)}~~~$\Delta H_2 = \SI{12}{\kilo\calorie}$
    \item \ce{I2(g) + H2(g) -> 2 HI(ac)}~~~$\Delta H_3 = \SI{-26,8}{\kilo\calorie}$
\end{itemize}
Calcular los parámetros que se indican a continuación:
\begin{description}%[label={\alph*)},font={\color{red!50!black}\bfseries}]
    \item[\texttt{a)}] Calor molar latente de sublimación del yodo.
    \item[\texttt{b)}] Calor molar de disolución del ácido yodhídrico.
    \item[\texttt{c)}] Número de calorías que hay que aportar para disociar en sus componentes el yoduro de hidrógeno gas contenido en un matraz de \SI{750}{\cubic\centi\meter} a \SI{25}{\celsius} y \SI{800}{\torr} de presión.
\end{description}
\resultadocmd{\SI{12,8}{\kilo\calorie}; \SI{-13,0}{\kilo\calorie}; \SI{12,9}{\calorie}}

	}
\end{frame}

\begin{frame}
	\frametitle{\ejerciciocmd}
	\framesubtitle{Datos del problema}
	\begin{center}
		{\huge ¿$K_c$?\quad¿[ ] en eq. al cambiar [\ce{CO2}]?\quad¿$K_p$ y $K_x$?}\\
		\tcbhighmath[boxrule=0.2pt,arc=2pt,colframe=black,drop fuzzy shadow=red]{\ce{CO2 (g) + H2 (g) <=> CO(g) + H2O(g)}}
	\end{center}
	\begin{enumerate}[label={\alph*)},font=\bfseries]
		\item	\tcbhighmath[boxrule=0.4pt,arc=4pt,colframe=blue,drop fuzzy shadow=green]{[\ce{CO}] = \SI{,050}{\Molar}}\quad
				\tcbhighmath[boxrule=0.4pt,arc=4pt,colframe=blue,drop fuzzy shadow=green]{[\ce{H2}] = \SI{,045}{\Molar}}\quad
				\tcbhighmath[boxrule=0.4pt,arc=4pt,colframe=blue,drop fuzzy shadow=green]{[\ce{CO2}] = \SI{,086}{\Molar}}\\[.2cm]
				\tcbhighmath[boxrule=0.4pt,arc=4pt,colframe=blue,drop fuzzy shadow=green]{[\ce{H2O}] = \SI{,040}{\Molar}}
		\item	\tcbhighmath[boxrule=0.4pt,arc=4pt,colframe=orange,drop fuzzy shadow=blue]{[\ce{CO2}] = \SI{,50}{\Molar}}
	\end{enumerate}
\end{frame}

\begin{frame}
	\frametitle{\ejerciciocmd}
	\framesubtitle{Resolución (\rom{1}): calcular la constante de equilibrio}
	\structure{Reacción de equilibrio:}
	$$
		\ce{CO2(g) + H2(g) <=> CO(g) + H2O(g)}\qquad K_c = \ce{\frac{[CO][H2O]}{[CO2][H2]}}
	$$
	\structure{Como datos del problema dan las concentraciones en equilibrio:}
	$$
		K_c = \frac{
						\overbrace{[\ce{CO}]}^{\SI{,050}{\Molar}}
							\vdot
						\overbrace{[\ce{H2O}]}^{\SI{,040}{\Molar}}
				}{
						\underbrace{[\ce{CO2}]}_{\SI{,086}{\Molar}}
							\vdot
						\underbrace{[\ce{H2}]}_{\SI{,045}{\Molar}}
				}\Rightarrow
		K_c = \frac{\num{,050}\vdot\num{,040}}{\num{,086}\vdot\num{,045}}\Rightarrow
		\tcbhighmath[boxrule=0.4pt,arc=4pt,colframe=blue,drop fuzzy shadow=green]{K_c = \num{,6541}}
	$$
\end{frame}

\begin{frame}
	\frametitle{\ejerciciocmd}
	\framesubtitle{Resolución (\rom{2}): calcular concentraciones en eq. al cambiar [\ce{CO2}]}
	\structure{Reacción de equilibrio:}
	$$
		\ce{CO2(g) + H2(g) <=> CO(g) + H2O(g)}\qquad K_c = \ce{\frac{[CO][H2O]}{[CO2][H2]}} = \num{,6541}
	$$
	\structure{Actualizar concentraciones:}
	\begin{center}
		\begin{tabular}{lcccc}
			\toprule
							&	[\ce{CO2}]~(\si{\Molar})	&	[\ce{H2}]~(\si{\Molar})	&	[\ce{CO}]~(\si{\Molar})	&	[\ce{H2O}]~(\si{\Molar})	\\
			\midrule
				Inicial		&	\num{,500}					&	\num{,045}				&	\num{,050}				&	\num{,040}					\\
				Reacciona	&	$-x$						&	$-x$					&	$+x$					&	$+x$						\\
				Equilibrio	&	$\num{,500}-x$				&	$\num{,045}-x$			&	$\num{,050}+x$			&	$\num{,040}+x$				\\
			\bottomrule
		\end{tabular}
	\end{center}
	\begin{overprint}
		\onslide<1>
			\structure{Sustituirlas en la expresión del equilibrio:}
			$$
				K_c = \frac{
					(\num{,050}+x)
						\vdot
					(\num{,040}+x)
				}{
					(\num{,500}-x)
						\vdot
					(\num{,045}-x)
				}\Rightarrow
				\num{,6541} = \frac{
					x^2 + \num{,090}x + \num{,0020}
				}{
					x^2 -\num{,545}x + \num{,0225}
				}\Rightarrow
			$$
			$$
				x^2 + \num{,090}x + \num{,0020} = \num{,6541}x^2 -\num{,3565}x + \num{,01472}\Rightarrow
				\num{,3459}x^2 + \num{0,4465}x - \num{,01272} = 0\Rightarrow
			$$
			$$
				x = \frac{-\num{,4465}\pm\sqrt{\num{,4465}^2 +4\vdot\num{,3459}\vdot\num{,01272}}}{2\vdot\num{,3459}} =
				\begin{cases}
					x_1 = \SI{,02789}{\Molar}\\
					\cancel{x_2 < 0}&
				\end{cases}
			$$
		\onslide<2->
			\structure{Sustituimos el valor de la x (\num{,02789}) en la tabla de concentraciones:}
			\begin{center}
				\begin{tabular}{lcccc}
					\toprule
								&	[\ce{CO2}]~(\si{\Molar})	&	[\ce{H2}]~(\si{\Molar})	&	[\ce{CO}]~(\si{\Molar})	&	[\ce{H2O}]~(\si{\Molar})	\\
					\midrule
					Equilibrio	&	$\num{,500}-\num{,02789}$	&
									$\num{,040}-\num{,02789}$	&
									$\num{,050}+\num{,02789}$	&
									$\num{,045}+\num{,02789}$	\\
					Resultado	&	$\tcbhighmath[boxrule=0.4pt,arc=4pt,colframe=orange,drop fuzzy shadow=blue]{\num{,4721}}$	&
									$\tcbhighmath[boxrule=0.4pt,arc=4pt,colframe=orange,drop fuzzy shadow=blue]{\num{,0121}}$	&
									$\tcbhighmath[boxrule=0.4pt,arc=4pt,colframe=orange,drop fuzzy shadow=blue]{\num{,0779}}$	&
									$\tcbhighmath[boxrule=0.4pt,arc=4pt,colframe=orange,drop fuzzy shadow=blue]{\num{,0729}}$	\\
					\bottomrule
				\end{tabular}
			\end{center}
	\end{overprint}
\end{frame}

\begin{frame}
	\frametitle{\ejerciciocmd}
	\framesubtitle{Resolución (\rom{3}): $K_p$ y $K_x$}
	Primero de todo vamos a destacar los coeficientes estequiométricos de nuestra reacción:
	\begin{center}
		\ce{$\colorbox{blue}{\color{yellow}\textbf{1}}$CO2(g)
			 	+ 
			$\colorbox{green}{\color{yellow}\textbf{1}}$H2(g)
			 	<=>
		 	$\colorbox{red}{\color{yellow}\textbf{1}}$CO(g)
		 		+
	 		$\colorbox{black}{\color{yellow}\textbf{1}}$H2O(g)
		}
	\end{center}
	\structure{Calculamos la variación de los coeficientes estequiométricos:} los productos los representamos con valor positivo, los reactivos con valor negativo.
	$$
		\Delta n = 1+1-1-1 = 0
	$$
	\structure{Por tanto:} \tcbhighmath[boxrule=0.4pt,arc=4pt,colframe=blue,drop fuzzy shadow=yellow]{K_c = K_p = K_x}. Veremos cómo obtener la relación entre $K_c$, $K_p$ y $K_c$.
	$$
		PV = nRT\Rightarrow P = \overbrace{\frac{n}{V}}^MRT\qquad\qquad P_i = x_i\vdot P_T
	$$
	$$
		K_p = 	\frac{P_{\ce{CO}}\vdot P_{\ce{H2O}}}{P_{\ce{CO2}}\vdot P_{\ce{H2}}} =
				\frac{[\ce{CO}]\vdot\cancel{RT}\vdot[\ce{H2O}]\vdot\cancel{RT}}{[\ce{CO2}]\vdot\cancel{RT}\vdot[\ce{H2}]\vdot\cancel{RT}} = K_c\Rightarrow
		K_p = K_c\vdot (RT)^{\Delta n}
	$$
	\newline
	$$
		K_p = 	\frac{P_{\ce{CO}}\vdot P_{\ce{H2O}}}{P_{\ce{CO2}}\vdot P_{\ce{H2}}} =
				\frac{x_{\ce{CO}}\vdot\cancel{P_T}\vdot x_{\ce{H2O}}\vdot\cancel{P_T}}{x_{\ce{CO2}}\vdot\cancel{P_T}\vdot x_{\ce{H2}}\vdot\cancel{P_T}} = K_x\Rightarrow
		K_p = K_x\vdot(P_T)^{\Delta n}
	$$
\end{frame}