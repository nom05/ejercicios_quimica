Considere el siguiente proceso en equilibrio a \SI{686}{\celsius}:
\begin{center}
	\ce{CO2 (g) + H2 (g) <=> CO(g) + H2O(g)}
\end{center}
Las concentraciones en el equilibrio de las especies reactivas son: $[\ce{CO}] = \SI{,050}{\Molar}$, $[\ce{H2}] = \SI{,045}{\Molar}$, $[\ce{CO2}] = \SI{,086}{\Molar}$ y $[\ce{H2O}] = \SI{,040}{\Molar}$. 
\begin{enumerate}[label={\alph*)},font=\bfseries]
	\item Calcule $K_c$ para la reacción a \SI{686}{\celsius}.
	\item Si se añadiera \ce{CO2} para aumentar su concentración a \SI{,50}{\mol\per\liter}, ¿cuáles serían las concentraciones de todos los gases una vez que se hubiera restablecido el equilibrio? 
	\item Calcule $K_p$ y $K_x$. 
\end{enumerate}
\resultadocmd{
				\num{,6541};
				\num{,4762}, \num{,0162}, \num{,0738},\num{,0688};
				$K_p=K_c=K_x$
			}
