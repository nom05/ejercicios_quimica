\begin{frame}
	\frametitle{\ejerciciocmd}
	\framesubtitle{Enunciado}
	\textbf{
		Dadas las siguientes reacciones:
\begin{itemize}
    \item \ce{I2(g) + H2(g) -> 2 HI(g)}~~~$\Delta H_1 = \SI{-0,8}{\kilo\calorie}$
    \item \ce{I2(s) + H2(g) -> 2 HI(g)}~~~$\Delta H_2 = \SI{12}{\kilo\calorie}$
    \item \ce{I2(g) + H2(g) -> 2 HI(ac)}~~~$\Delta H_3 = \SI{-26,8}{\kilo\calorie}$
\end{itemize}
Calcular los parámetros que se indican a continuación:
\begin{description}%[label={\alph*)},font={\color{red!50!black}\bfseries}]
    \item[\texttt{a)}] Calor molar latente de sublimación del yodo.
    \item[\texttt{b)}] Calor molar de disolución del ácido yodhídrico.
    \item[\texttt{c)}] Número de calorías que hay que aportar para disociar en sus componentes el yoduro de hidrógeno gas contenido en un matraz de \SI{750}{\cubic\centi\meter} a \SI{25}{\celsius} y \SI{800}{\torr} de presión.
\end{description}
\resultadocmd{\SI{12,8}{\kilo\calorie}; \SI{-13,0}{\kilo\calorie}; \SI{12,9}{\calorie}}

		}
\end{frame}

\begin{frame}
	\frametitle{\ejerciciocmd}
	\framesubtitle{Datos del problema}
	\begin{center}
		{\huge ¿pH en que aumenta solubilidad?}\\[.3cm]
		\tcbhighmath[boxrule=0.4pt,arc=4pt,colframe=black,drop fuzzy shadow=red]{K_{ps}(\ce{Mg(OH)2})=\num{1,8e-11}}
	\end{center}
\end{frame}

\begin{frame}
	\frametitle{\ejerciciocmd}
	\framesubtitle{Resolución (\rom{1}): pH en que aumenta solubilidad de disolución \ce{Mg(OH)2}}
	\structure{Reacción de solubilidad:}
	$$
		\ce{Mg(OH)2(s) <=> Mg^{2+}(ac) + 2OH-}\quad K_{ps}(\ce{Mg(OH)2}) = [\ce{Mg^{2+}}]\vdot[\ce{OH-}]^2
	$$
	\begin{center}
		\begin{tabular}{ccc}
			\toprule
				Concentración(\si{\Molar}) & $[\ce{Mg^{2+}}]$ &     [\ce{OH-}] \\
				Solubilidad                &    $s$    	    &       $2s$    \\
			\bottomrule
		\end{tabular}
	\end{center}
	\visible<2->{
		\structure{La solubilidad en el equilibrio será:}
		\begin{overprint}
			\onslide<2>
				$$
					K_{ps}(\ce{Mg(OH)2}) = \overbrace{[\ce{Mg^{2+}}]}^{s}\vdot\underbrace{[\ce{OH-}]^2}_{(2s)^2}
				$$
			\onslide<3>
				$$
					K_{ps}(\ce{Mg(OH)2}) = s\vdot(2s)^2 = s\vdot{4s^2} = 4s^3
				$$
			\onslide<4->
				Despejamos $s\equiv s(\ce{Mg(OH)2})$. Recordad que es una raíz cúbica:
				$$
					s(\ce{Mg(OH)2}) = \sqrt[3]{\frac{K_{ps}(\ce{Mg(OH)2})}{4}}\Rightarrow s(\ce{Mg(OH)2}) = \SI{1,65e-4}{\Molar}
				$$
		\end{overprint}
				}
	\visible<4->{
		\structure{Concentración de iones hidroxilo:} Fijaos en la relación entre $s$ y [\ce{OH-}] en la tabla de arriba.
		$$
			[\ce{OH-}] = 2s\Rightarrow[\ce{OH-}]=\num{2}\vdot\SI{1,65e-4}{\Molar}=\SI{3,30e-4}{\Molar}
		$$
				}
	\visible<5->{
		\structure{Determinación del pH:}
		\begin{overprint}
			\onslide<5>
				Recordad:
				$$
					K_w = [\ce{H+}][\ce{OH-}]=\num{1e-14}\Rightarrow-\log([\ce{H+}][\ce{OH-}])=-\log(\num{1e-14})
				$$
			\onslide<6>
				Operando:
				$$
					\overbrace{-\log([\ce{H+}])}^{\text{pH}=-\log([\ce{H+}])}\underbrace{-\log([\ce{OH-}])}_{\text{pOH}=-\log([\ce{OH-}])}=-(\num{-14})
				$$
			\onslide<7>
				Operando:
				$$
					\text{pH}-\log([\ce{OH-}])=\num{14}
				$$
			\onslide<8->
				Operando:
				$$
					\text{pH}=\num{14}+\log(\overbrace{[\ce{OH-}]}^{\SI{3,30e-4}{\Molar}})=\num{14}-\num{3,49}\Rightarrow
					\tcbhighmath[boxrule=0.4pt,arc=4pt,colframe=black,drop fuzzy shadow=red]{\text{pH}<\num{10,51}}
				$$
		\end{overprint}
				}
\end{frame}
