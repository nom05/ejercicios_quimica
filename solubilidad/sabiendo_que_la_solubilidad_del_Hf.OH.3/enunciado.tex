.\begin{enumerate}[label={\alph*)},font={\color{red!50!black}\bfseries}]
	\item Sabiendo que la solubilidad del trihidróxido de hafnio (\ce{Hf(OH)3}) es \SI{4,503e-3}{\gram\per\liter}, calcula el pH y $K_{ps}$ de una disolución acuosa saturada de este compuesto.
	\item ¿Precipitará hidróxido de hafnio si a \SI{5}{\liter} de una disolución que contiene ion hafnio (\rom{3}) (\ce{Hf^3+} disuelto completamente en agua) con la concentración del apartado anterior añadimos \ce{NaOH} sólido, sin cambiar el volumen de disolución, hasta alcanzar pH \num{11}? Si la respuesta es afirmativa, ¿cuántos gramos precipitarán?
\end{enumerate}
\resultadocmd{
				\num{9,77}, \num{4,000e-18};
				\SI{,023}{\gram}
			}
