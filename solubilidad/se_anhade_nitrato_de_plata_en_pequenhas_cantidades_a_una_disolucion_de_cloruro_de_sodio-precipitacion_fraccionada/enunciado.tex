Se añade nitrato de plata en pequeñas cantidades a una disolución que contiene cloruro de sodio \SI{0,01}{\Molar}, bromuro de sodio \SI{0,01}{\Molar} y yoduro de sodio \SI{0,01}{\Molar}. Suponiendo que la variación de volumen de la disolución debida a la adición del nitrato de plata es despreciable, calcular:
\begin{enumerate}[label={\alph*)},font={\color{red!50!black}\bfseries}]
	\item El orden en que precipitarán.
	\item La concentración de ion plata a la que comienza a precipitar la especie menos soluble.
	\item La concentración de ion yoduro cuando empieza a precipitar el bromuro de plata.
	\item Las concentraciones de los iones plata y bromuro cuando empieza a precipitar el cloruro de plata.
\end{enumerate}
Datos del ejercicio: $K_{ps}(\ce{AgCl}) = \num{1,6e-10}$; $K_{ps}(\ce{AgBr}) = \num{5,2e-13}$; $K_{ps}(\ce{AgI}) = \num{8,3e-17}$
\resultadocmd{\SI{8,3e-15}{\Molar}; \SI{1,596e-6}{\Molar}; \SI{3,25e-5}{\Molar}}
