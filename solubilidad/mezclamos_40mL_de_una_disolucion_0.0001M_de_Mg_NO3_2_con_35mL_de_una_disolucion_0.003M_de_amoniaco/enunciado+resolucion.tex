\begin{frame}
    \frametitle{\ejerciciocmd}
    \framesubtitle{Enunciado}
    \textbf{
	Dadas las siguientes reacciones:
\begin{itemize}
    \item \ce{I2(g) + H2(g) -> 2 HI(g)}~~~$\Delta H_1 = \SI{-0,8}{\kilo\calorie}$
    \item \ce{I2(s) + H2(g) -> 2 HI(g)}~~~$\Delta H_2 = \SI{12}{\kilo\calorie}$
    \item \ce{I2(g) + H2(g) -> 2 HI(ac)}~~~$\Delta H_3 = \SI{-26,8}{\kilo\calorie}$
\end{itemize}
Calcular los parámetros que se indican a continuación:
\begin{description}%[label={\alph*)},font={\color{red!50!black}\bfseries}]
    \item[\texttt{a)}] Calor molar latente de sublimación del yodo.
    \item[\texttt{b)}] Calor molar de disolución del ácido yodhídrico.
    \item[\texttt{c)}] Número de calorías que hay que aportar para disociar en sus componentes el yoduro de hidrógeno gas contenido en un matraz de \SI{750}{\cubic\centi\meter} a \SI{25}{\celsius} y \SI{800}{\torr} de presión.
\end{description}
\resultadocmd{\SI{12,8}{\kilo\calorie}; \SI{-13,0}{\kilo\calorie}; \SI{12,9}{\calorie}}

            }
\end{frame}

\begin{frame}
    \frametitle{\ejerciciocmd}
    \framesubtitle{Datos del problema}
    \begin{center}
        {\Large \textbf{¿Precipita el \ce{Mg(OH)2}?}}
    \end{center}
    $$
        \tcbhighmath[boxrule=0.4pt,arc=4pt,colframe=green,drop fuzzy shadow=yellow]{[\ce{Mg(NO3)2}]_0 = \SI{1e-4}{\Molar}}\quad
        \tcbhighmath[boxrule=0.4pt,arc=4pt,colframe=green,drop fuzzy shadow=yellow]{V(\ce{Mg(NO3)2})_0 = \SI{,04}{\liter}}\quad
    $$
    $$
        \tcbhighmath[boxrule=0.4pt,arc=4pt,colframe=blue,drop fuzzy shadow=red]{K_{ps}(\ce{Mg(OH)2}) = \SI{1,8e-11}{}}
    $$
    $$
        \tcbhighmath[boxrule=0.4pt,arc=4pt,colframe=red,drop fuzzy shadow=green]{K_b(\ce{NH3}) = \SI{1,8e-5}{}}\quad
        \tcbhighmath[boxrule=0.4pt,arc=4pt,colframe=red,drop fuzzy shadow=green]{V_0(\ce{NH3}) = \SI{,035}{\liter}}\quad
        \tcbhighmath[boxrule=0.4pt,arc=4pt,colframe=red,drop fuzzy shadow=green]{[\ce{NH3}]_0 = \SI{3e-3}{\Molar}}
    $$
\end{frame}

\begin{frame}
    \frametitle{\ejerciciocmd}
    \framesubtitle{Resolución (\rom{1}): determinar si hay precipitado de \ce{Mg(OH)2}}
    \structure{Recalcular concentraciones al mezclar dos disoluciones:}
    \begin{overprint}
        \onslide<1>
            $$
                V_T = \SI{,040}{\liter} + \SI{,035}{\liter} = \SI{,075}{\liter}
            $$
            $$
                [\ce{NH3}] = \frac{\overbrace{n(\ce{NH3})}^{n=M\cdot V}}{V_T}
            $$
        \onslide<2>
            $$
                V_T = \SI{,040}{\liter} + \SI{,035}{\liter} = \SI{,075}{\liter}
            $$
            $$
                [\ce{NH3}] = \frac{[\ce{NH3}]_0\cdot V_0(\ce{NH3})}{V_T}
            $$
        \onslide<3-4>
            $$
                [\ce{NH3}] = \frac{[\ce{NH3}]_0\cdot V_0(\ce{NH3})}{V_T}\Rightarrow[\ce{NH3}] = \frac{\SI{3e-3}{\Molar}\cdot\SI{,035}{\cancel\liter}}{\SI{,075}{\cancel\liter}} = \SI{1,4e-3}{\Molar}
            $$
        \onslide<5->
            $$
                [\ce{NH3}] = \SI{1,4e-3}{\Molar}
            $$
    \end{overprint}
    \begin{overprint}
        \onslide<4>
            $$
                [\ce{Mg(NO3)2}] = \frac{[\ce{Mg(NO3)2}]_0\cdot V_0(\ce{Mg(NO3)2})}{V_T}\Rightarrow[\ce{Mg(NO3)2}] = \frac{\SI{1e-4}{\Molar}\cdot\SI{,040}{\cancel\liter}}{\SI{,075}{\cancel\liter}}
            $$
        \onslide<5->
            $$
                [\ce{Mg(NO3)2}] = [\ce{Mg^2+}] = \SI{5,333e-5}{\Molar}
            $$
    \end{overprint}
    \visible<6->{
        \structure{Habrá precipitado si producto de solubilidad ($Q(\ce{Mg(OH)2})$) es mayor que $K_{ps}(\ce{Mg(OH)2})$}
        $$
            Q(\ce{Mg(OH)2}) = [\ce{Mg^2+}][\ce{OH-}]^2 \text{(producto de concentraciones que no están en equilibrio)}
        $$
                }
    \visible<7->{
        \structure{Concentración de aniones hidróxido [\ce{OH-}]:}
        \begin{overprint}
            \onslide<7>
                $$
                    \ce{
                        $\underset{\SI{1,4e-3}{\Molar}-x}{\ce{NH3(ac)}}$
                        + 
                        H2O(l)
                        <=>
                        $\underset{x}{\ce{NH4+(ac)}}$
                        +
                        $\underset{x}{\ce{OH-(ac)}}$
                       }\quad K_b(\ce{NH3}) = \frac{[\ce{NH4+}][\ce{OH-}]}{[\ce{NH3}]}
                $$
            \onslide<8>
                $$
                    K_b(\ce{NH3}) = \frac{[\ce{NH4+}][\ce{OH-}]}{[\ce{NH3}]} = \frac{x^2}{\SI{1,4e-3}{}-x}\approx\frac{x^2}{\SI{1,4e-3}{}}
                $$
            \onslide<9->
                $$
                    x=[\ce{OH-}] = +\sqrt{K_b(\ce{NH3})\cdot\SI{1,4e-3}{}} = \SI{1,587e-4}{\Molar}
                $$
        \end{overprint}
                }
    \visible<10->{
        \structure{Comparación de $Q(\ce{Mg(OH)2})$ con $K_{ps}(\ce{Mg(OH)2})$:}
        \begin{overprint}
            \onslide<10>
                $$
                    Q(\ce{Mg(OH)2}) = [\ce{Mg^2+}][\ce{OH-}]^2\Rightarrow Q(\ce{Mg(OH)2})=  [\ce{Mg^2+}] = \SI{5,333e-5}{}\cdot(\SI{1,587e-4}{})^2
                $$
            \onslide<11>
                $$
                    \tcbhighmath[boxrule=0.4pt,arc=4pt,colframe=blue,drop fuzzy shadow=red]{Q(\ce{Mg(OH)2}) = \SI{1,344e-12}{} < K_{ps}(\ce{Mg(OH)2}) = \SI{1,8e-11}{}\quad\text{\Large\textbf{No precipita}}}
                $$
        \end{overprint}
                 }
\end{frame}
