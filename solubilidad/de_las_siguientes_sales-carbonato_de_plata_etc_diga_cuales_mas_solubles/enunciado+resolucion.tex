\begin{frame}
    \frametitle{\ejerciciocmd}
    \framesubtitle{Enunciado}
    \textbf{
	Una reacción tiene una constante de velocidad de \SI{,017}{\per\second} a \SI{298}{\kelvin} y una energía libre de activación del \SI{27,235}{\kilo\joule\per\mol}. La adición de un catalizador disminuye dicha energía de activación hasta un \SI{33}{\percent} de su valor inicial. Calcule la nueva constante de velocidad.
\resultadocmd{ \SI{26,86}{\per\second} }

            }
\end{frame}

\begin{frame}
    \frametitle{\ejerciciocmd}
    \framesubtitle{Datos del problema}
    \begin{center}
        {\Large \textbf{¿Compuestos más solubles en medio ácido que en agua?}}\\[.5cm]
        $$
            \tcbhighmath[boxrule=0.4pt,arc=4pt,colframe=green,drop fuzzy shadow=yellow]{\ce{Ag2CO3}}~
            \tcbhighmath[boxrule=0.4pt,arc=4pt,colframe=red,drop fuzzy shadow=black]{\ce{CaF2}}~
            \tcbhighmath[boxrule=0.4pt,arc=4pt,colframe=orange,drop fuzzy shadow=green]{\ce{BaSO4}}~
            \tcbhighmath[boxrule=0.4pt,arc=4pt,colframe=black,drop fuzzy shadow=red]{\ce{PbI2}}~
            \tcbhighmath[boxrule=0.4pt,arc=4pt,colframe=blue,drop fuzzy shadow=yellow]{\ce{Mg(OH)2}}
        $$\\[.5cm]
        {\Large \textbf{¿Compuestos más solubles en disolución de \ce{NH3} en agua que en agua únicamente?}}
    \end{center}
\end{frame}

\begin{frame}
    \frametitle{\ejerciciocmd}
    \framesubtitle{Resolución: aumento de solubilidad}
    \begin{block}{Compuestos con carácter básico}
        \structure{Hidróxidos:}
            $$
                \ce{Mg(OH)2(s) <=> Mg^2+(ac) +
                     $\underset{\text{se consume en m. ácido}}{\ce{2OH-(ac)}}$
                   }
            $$\\[.5cm]
        \visible<2->{
            \structure{Sales conjugadas de ácido débil:}
            $$
                \ce{CO3^2-(ac) + 2H2O(l) <=> H2CO3(ac) + 
                    $\underset{\text{se consume en m. ácido}}{\ce{2OH-(ac)}}$
                }
            $$
            Como son 
            \tcbhighmath[boxrule=0.4pt,arc=4pt,colframe=green,drop fuzzy shadow=yellow]{\ce{Ag2CO3}},
            \tcbhighmath[boxrule=0.4pt,arc=4pt,colframe=red,drop fuzzy shadow=black]{\ce{CaF2}} y 
            \tcbhighmath[boxrule=0.4pt,arc=4pt,colframe=blue,drop fuzzy shadow=yellow]{\ce{Mg(OH)2}}
                    }
    \end{block}
    \begin{block}<3->{En medio acuoso con \ce{NH3}}
        \structure{Como base, cualquier sal con carácter ácido, que \underline{NO ES ESTE EL CASO}}
        \visible<4->{
            \structure{Como ligando en compuestos de coordinación:} Las sales complejas tienen constantes de formación muy elevadas. Este es el caso de:
            $$
                \tcbhighmath[boxrule=0.4pt,arc=4pt,colframe=green,drop fuzzy shadow=yellow]{\ce{Ag2CO3(s) v + NH3(ac) -> Ag(NH3)2^+(ac) + CO3^2-(ac)}}
            $$
                    }
    \end{block}
\end{frame}

