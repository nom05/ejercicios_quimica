Resolver las siguientes cuestiones:
\begin{enumerate}[label={\alph*)},font={\color{red!50!black}\bfseries}]
	\item Si se disuelven \SI{0,05}{\mol} de carbonato de potasio en \SI{1}{\liter} de una disolución de cloruro de estroncio \SI{e-3}{\Molar}, ¿qué precipitado se forma?
 	\item Se añaden \SI{0,04}{\gram} de cloruro de magnesio y \SI{0,16}{\gram} de hidróxido de potasio a \SI{500}{\milli\liter} de agua. Determinar si se forma algún precipitado y la naturaleza de este.
\end{enumerate}
Datos del ejercicio: $K_{ps}(\ce{SrCO3}) = \num{9,4e-10}$; $K_{ps}(\ce{Mg(OH)2}) = \num{1,8e-11}$
\resultadocmd{\SI{,143}{\gram}; \SI{,020}{\gram}}
