\begin{frame}
    \frametitle{\ejerciciocmd}
    \framesubtitle{Enunciado}
    \textbf{
		Una reacción tiene una constante de velocidad de \SI{,017}{\per\second} a \SI{298}{\kelvin} y una energía libre de activación del \SI{27,235}{\kilo\joule\per\mol}. La adición de un catalizador disminuye dicha energía de activación hasta un \SI{33}{\percent} de su valor inicial. Calcule la nueva constante de velocidad.
\resultadocmd{ \SI{26,86}{\per\second} }

	}
\end{frame}

\begin{frame}
    \frametitle{\ejerciciocmd}
    \framesubtitle{Datos del apartado (a) del problema}
    \begin{center}
        {\Large \textbf{¿$m(\text{precipitado})$}?}
    \end{center}
    $$
        \tcbhighmath[boxrule=0.4pt,arc=4pt,colframe=orange,drop fuzzy shadow=yellow]{[\ce{SrCl2}] = \SI{e-3}{\Molar}}\quad
        \tcbhighmath[boxrule=0.4pt,arc=4pt,colframe=orange,drop fuzzy shadow=yellow]{V = \SI{1}{\liter}}
    $$
    $$
        \tcbhighmath[boxrule=0.4pt,arc=4pt,colframe=blue,drop fuzzy shadow=yellow]{n(\ce{K2CO3}) = \SI{,05}{\mol}}
    $$
    $$
        \tcbhighmath[boxrule=0.4pt,arc=4pt,colframe=green,drop fuzzy shadow=yellow]{K_{ps}(\ce{SrCO3}) = \SI{9,4e-10}{}}\quad
        \tcbhighmath[boxrule=0.4pt,arc=4pt,colframe=green,drop fuzzy shadow=yellow]{Mm(\ce{SrCO3}) = \SI{147,63}{\gram\per\mol}}
    $$
\end{frame}

\begin{frame}
    \frametitle{\ejerciciocmd}
    \framesubtitle{Resolución (\rom{1}): determinación de la masa de precipitado}
    \structure{Reacciones de solubilidad:}
    \begin{overprint}
        \onslide<1>
            $$
                \ce{K2CO3(s) -> 2K+(ac) + CO3^2-(ac)}
            $$
            $$
                \ce{SrCl2(s) -> Sr^2+(ac) + 2Cl-(ac)}
            $$
        \onslide<2->
            $$
                \ce{K2CO3(s) -> 2K+(ac) + }\tcbhighmath[boxrule=0.4pt,arc=4pt,colframe=blue,drop fuzzy shadow=yellow]{\ce{CO3^2-(ac)}}
            $$
            $$
                \ce{SrCl2(s) -> } \tcbhighmath[boxrule=0.4pt,arc=4pt,colframe=green,drop fuzzy shadow=yellow]{\ce{Sr^2+(ac)}} \ce{ + 2Cl-(ac)}
            $$
    \end{overprint}
    \visible<2->{
        $$
            \ce{SrCO3(s) <=> Sr^2+(ac) + CO3^2-(ac)}
        $$
        \structure{$K_{ps}(\ce{SrCO3}) = \SI{9,4e-10}{}$ muy bajo.}
        
        Como $V = \SI{1}{\liter}$, $n(\ce{SrCO3}) = \SI{1e-3}{\mol}$ y $n(\ce{CO3^2-}) = \SI{5e-2}{\mol}$\\[.5cm]
                }
    \visible<3->{
        \begin{overprint}
            \onslide<3>
                \begin{center}
                    \textbf{¿Reactivo limitante?}
                \end{center}
            \onslide<4->
                El reactivo limitante será \ce{Sr^2+} porque se necesitarían \SI{1e-3}{\mol} de \ce{CO3^2-} (reacción 1:1) y tenemos \SI{5e-2}{\mol}.
        \end{overprint}        
                }
    \visible<5->{
        \structure{Solubilidad (s) de los iones:} $K_{ps}(\ce{SrCO3}) = s^2\Rightarrow s = \SI{3,066e-5}{\Molar} = \frac{n_{\text{solubles}}(\ce{SrCO3})}{\SI{1}{\liter}}$
                }
    \visible<6->{
        \begin{overprint}
            \onslide<6>
                \structure{Número de moles precipitados:} $n_{\text{precipitado}}(\ce{SrCO3}) = \SI{1e-3}{\mol} - \SI{3,066e-5}{\mol} = \SI{9,69e-4}{\mol}$
            \onslide<7>
                \structure{Masa de precipitado:} $\tcbhighmath[boxrule=0.4pt,arc=4pt,colframe=green,drop fuzzy shadow=yellow]{m_{\text{precipitado}}(\ce{SrCO3}) = \SI{9,69e-4}{\cancel\mol}\cdot\SI{147,63}{\gram\per\cancel\mol} = \SI{,143}{\gram}}$
        \end{overprint}
                }
\end{frame}

\begin{frame}
    \frametitle{\ejerciciocmd}
    \framesubtitle{Datos del apartado (b) del problema}
    \begin{center}
        {\Large \textbf{¿$m(\text{precipitado})$}?}
    \end{center}
    $$
        \tcbhighmath[boxrule=0.4pt,arc=4pt,colframe=orange,drop fuzzy shadow=yellow]{m(\ce{MgCl2}) = \SI{,04}{\gram}}\quad
        \tcbhighmath[boxrule=0.4pt,arc=4pt,colframe=orange,drop fuzzy shadow=yellow]{Mm(\ce{MgCl2}) = \SI{95,21}{\gram\per\mol}}\quad
        \tcbhighmath[boxrule=0.4pt,arc=4pt,colframe=orange,drop fuzzy shadow=yellow]{V = \SI{,500}{\liter}}
    $$
    $$
        \tcbhighmath[boxrule=0.4pt,arc=4pt,colframe=blue,drop fuzzy shadow=yellow]{m(\ce{KOH}) = \SI{,16}{\gram}}\quad
        \tcbhighmath[boxrule=0.4pt,arc=4pt,colframe=blue,drop fuzzy shadow=yellow]{Mm(\ce{KOH}) = \SI{56,11}{\gram\per\mol}}
    $$
    $$
        \tcbhighmath[boxrule=0.4pt,arc=4pt,colframe=green,drop fuzzy shadow=yellow]{K_{ps}(\ce{Mg(OH)2}) = \SI{1,8e-11}{}}\quad
        \tcbhighmath[boxrule=0.4pt,arc=4pt,colframe=green,drop fuzzy shadow=yellow]{Mm(\ce{Mg(OH)2}) = \SI{58,32}{\gram\per\mol}}
    $$
\end{frame}

\begin{frame}
    \frametitle{\ejerciciocmd}
    \framesubtitle{Resolución (\rom{2}): determinación de la masa de precipitado (b)}
    \structure{Reacción de solubilidad:}
        $$
            \ce{
                    Mg(OH)2(s) v
                     <=> 
                    $\underset{s}{\ce{Mg^2+(ac)}}$
                     +
                    $\underset{2s}{\ce{2OH-(ac)}}$
               }
        $$
        $$
            K_{ps}(\ce{Mg(OH)2}) = [\ce{Mg^2+}][\ce{OH-}]^2 = 4s^3 =\SI{1,8e-11}{}
        $$
    \visible<2->{
        \structure{Reactivo limitante:}
        $$
            n(\ce{MgCl2}) = n(\ce{Mg^2+}) = \frac{\SI{,04}{\gram}}{\SI{95,21}{\gram\per\mol}} = \SI{4,201e-4}{\mol}
        $$
        $$
            n(\ce{KOH}) = n(\ce{OH-}) = \frac{\SI{,16}{\gram}}{\SI{56,11}{\gram\per\mol}} = \SI{2,852e-3}{\mol}
        $$
                }
    \visible<3->{
            Para formar el \ce{Mg(OH)2} consumiendo todos los iones se necesitarán \SI{4,201e-4}{\mol} de \ce{Mg^2+} y \SI{8,402e-4}{\mol} de \ce{OH-} (tenemos \SI{2,852e-3}{\mol}). Por tanto, el \textbf{\alert{reactivo limitante es el \ce{Mg^2+}}}.
                }
    \visible<4->{
        \begin{overprint}
            \onslide<4>
                \structure{Obtenemos s y $n_{\text{solubles}}(\ce{Mg(OH)2})$:}
                $$
                    s = \left(\frac{K_{ps}(\ce{Mg(OH)2})}{4}\right)^{\frac{1}{3}} \Rightarrow s = \SI{1,650e-4}{\Molar}\Rightarrow n_{\text{solubles}}(\ce{Mg(OH)2}) = \SI{1,650e-4}{\mol\per\cancel\liter}\cdot\SI{,5}{\cancel\liter}
                $$
            \onslide<5>
                \structure{Obtenemos s y $n_{\text{solubles}}(\ce{Mg(OH)2})$:}
                $$
                    n_{\text{solubles}}(\ce{Mg(OH)2}) = \SI{8,255e-5}{\mol}
                $$
            \onslide<6>
                \structure{$n_{\text{precipitan}}(\ce{Mg(OH)2})$:}
                $$
                    n_{\text{precipitan}}(\ce{Mg(OH)2}) = \SI{4,201e-4}{\mol}- \SI{8,255e-5}{\mol} = \SI{3,376e-4}{\mol}
                $$
            \onslide<7>
                \structure{$m_{\text{precipitan}}(\ce{Mg(OH)2})$:}
                $$
                    \tcbhighmath[boxrule=0.4pt,arc=4pt,colframe=green,drop fuzzy shadow=yellow]{m_{\text{precipitan}}(\ce{Mg(OH)2}) = \SI{3,376e-4}{\cancel\mol}\cdot\SI{58,32}{\gram\per\cancel\mol} = \SI{,020}{\gram}}
                $$
        \end{overprint}
                }
\end{frame}
