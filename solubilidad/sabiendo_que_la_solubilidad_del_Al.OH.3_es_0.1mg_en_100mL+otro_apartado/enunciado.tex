.\begin{enumerate}
	\item Sabiendo que la solubilidad del hidróxido de aluminio (\ce{Al(OH)3}) es \SI{,1}{\milli\gram} en \SI{100}{\milli\liter}, calcula el pH de una disolución acuosa saturada de este compuesto.
	\item ¿Precipitará \ce{Al(OH)3} si a \SI{5}{\liter} de una disolución de sulfato de aluminio (sal soluble en agua, \ce{Al2(SO4)3}) \SI{1e-6}{\Molar}, añadimos \ce{NaOH} sólido, sin cambiar el volumen de disolución, hasta alcanzar pH \num{12}? Justifica la respuesta.
\end{enumerate}
\resultadocmd{
				\num{9,59};
				Sí
			}
