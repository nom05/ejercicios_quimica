\begin{frame}
	\frametitle{\ejerciciocmd}
	\framesubtitle{Enunciado}
	\textbf{
		Dadas las siguientes reacciones:
\begin{itemize}
    \item \ce{I2(g) + H2(g) -> 2 HI(g)}~~~$\Delta H_1 = \SI{-0,8}{\kilo\calorie}$
    \item \ce{I2(s) + H2(g) -> 2 HI(g)}~~~$\Delta H_2 = \SI{12}{\kilo\calorie}$
    \item \ce{I2(g) + H2(g) -> 2 HI(ac)}~~~$\Delta H_3 = \SI{-26,8}{\kilo\calorie}$
\end{itemize}
Calcular los parámetros que se indican a continuación:
\begin{description}%[label={\alph*)},font={\color{red!50!black}\bfseries}]
    \item[\texttt{a)}] Calor molar latente de sublimación del yodo.
    \item[\texttt{b)}] Calor molar de disolución del ácido yodhídrico.
    \item[\texttt{c)}] Número de calorías que hay que aportar para disociar en sus componentes el yoduro de hidrógeno gas contenido en un matraz de \SI{750}{\cubic\centi\meter} a \SI{25}{\celsius} y \SI{800}{\torr} de presión.
\end{description}
\resultadocmd{\SI{12,8}{\kilo\calorie}; \SI{-13,0}{\kilo\calorie}; \SI{12,9}{\calorie}}

	}
\end{frame}

\begin{frame}
	\frametitle{\ejerciciocmd}
	\framesubtitle{Datos del problema}
	\begin{center}
		{\LARGE ¿pH disolución saturada?}\\[.2cm]
		\tcbhighmath[boxrule=0.4pt,arc=4pt,colframe=blue,drop fuzzy shadow=red]{\text{solubilidad másica:} \SI{,1}{\gram}\text{ en }\SI{100}{\gram}}\quad
		\tcbhighmath[boxrule=0.4pt,arc=4pt,colframe=blue,drop fuzzy shadow=red]{Mm(\ce{Al(OH)3}) = \SI{78,00}{\gram\per\mol}}\\[1.cm]
		{\LARGE ¿precipitará \ce{Al(OH)3}?}\\[.2cm]
		\tcbhighmath[boxrule=0.4pt,arc=4pt,colframe=orange,drop fuzzy shadow=red]{V(\ce{Al2(SO4)3}) = \SI{5}{\liter}}\quad
		\tcbhighmath[boxrule=0.4pt,arc=4pt,colframe=orange,drop fuzzy shadow=red]{[\ce{Al2(SO4)3}]  = \SI{1e-6}{\liter}}\quad
		\tcbhighmath[boxrule=0.4pt,arc=4pt,colframe=orange,drop fuzzy shadow=red]{\text{pH}         = \num{12}}
	\end{center}
\end{frame}

\newcommand{\pH}{\text{pH}}
\newcommand{\pOH}{\text{pOH}}

\begin{frame}
	\frametitle{\ejerciciocmd}
	\framesubtitle{Resolución (\rom{1}): pH de disolución acuosa saturada}
	\structure{Convertimos la solubilidad másica a unas unidades más convenientes (\unit{\gram\per\liter}):}
	$$
		\frac{\SI{,1}{\cancel\gram}}{\SI{100}{\cancel\milli\liter}}\vdot\frac{\SI{100}{\cancel\milli\liter}}{\SI{0,1}{\liter}} = \SI{1e-3}{\gram\per\liter}
	$$
	\structure{Obtenemos la solubilidad \underline{molar}:}
	$$
		\SI[per-mode = fraction]{1e-3}{\cancel\gram\per\liter}\vdot\frac{\SI{1}{\mol}}{\SI{78,00}{\cancel\gram}} = \SI{1,282e-5}{\Molar} = s(\ce{Al(OH)3})
	$$
	\structure{Equilibrio de solubilidad y solubilidad molar:} usamos la estequiometría del equilibrio de solubilidad para averiguar la relación entre $s$ y [\ce{OH-}] en \ce{Al(OH)3}:
	\begin{center}
		\begin{tabular}{lcc}
			\multicolumn{3}{r}{$\ce{Al(OH)3(s) <=> Al^3+(ac) + 3OH-(ac)}\quad K_{ps} = \ce{[Al^3+][OH^-]^3}$}	\\
			\midrule
										&	[\ce{Al^3+}]	&	[\ce{OH-}]	\\
			Solubilidad~(\si{\Molar})	&	$s$				&	$3s$
		\end{tabular}
	\end{center}
	\textbf{Según la estequiometría:} $[\ce{OH-}] = 3s\Rightarrow\pH = 14 - \pOH = 14 + \log[\ce{OH-}]$
	$$
		\tcbhighmath[boxrule=0.4pt,arc=4pt,colframe=blue,drop fuzzy shadow=red]{\pH = 14 + \log(\underbrace{3\vdot\overbrace{s}^{\num{1,282e-5}}}_{\num{3,85e-5}})\Rightarrow\pH = \num{9,59}}
	$$
\end{frame}

\begin{frame}
	\frametitle{\ejerciciocmd}
	\framesubtitle{Resolución (\rom{2}): ¿precipitará \ce{Al(OH)3}?}
	\structure{Equilibrio de solubilidad y solubilidad molar:} usamos la estequiometría del equilibrio de solubilidad para averiguar la relación entre $s$ y $K_{ps}$ para \ce{Al(OH)3}
	\begin{center}
		\begin{tabular}{lcc}
			\multicolumn{3}{r}{$\ce{Al(OH)3(s) <=> Al^3+(ac) + 3OH-(ac)}\quad K_{ps} = \ce{[Al^3+][OH^-]^3}$}	\\
			\midrule
			&	[\ce{Al^3+}]	&	[\ce{OH-}]	\\
			Solubilidad~(\si{\Molar})	&	$s$				&	$3s$
		\end{tabular}
	\end{center}
	\structure{Calculamos $K_{ps}$:} ($s(\ce{Al(OH)3}) = \SI{1,282e-5}{\Molar}$, previamente calculada)
	$$
		 K_{ps}(\ce{Al(OH)3}) = \overbrace{[\ce{Al^3+}]}^s\overbrace{[\ce{OH-}]^3}^{(3s)^3 = 27s^3} = 27s^4 = 27\vdot(\num{1,282e-5})^4 = \num{7,29e-19}
	$$
	\textbf{\alert{Tenemos que calcular $Q_{ps} = \overbrace{[\ce{Al^3+}][\ce{OH-}]^3}^{\text{no equilibrio (n.e.)}}$ y compararlo con $K_{ps}$ para saber si precipita:}}
	\structure{Calculamos [\ce{OH-}]:} $\pH = 12\Rightarrow[\ce{OH-}] = 10^{-14+\pH}\Rightarrow[\ce{OH-}] = \SI{,01}{\Molar}$
	\structure{Sal soluble:}
	$$
		\ce{$\overbrace{\ce{Al2(SO4)3(s)}}^{\SI{1e-6}{\Molar}}$ -> $\overbrace{\ce{2Al^3+(ac)}}^{2\vdot\SI{1e-6}{\Molar}}$ + 3SO4^2-(ac)}
	$$
	$$
		Q_{ps} = \overbrace{\num{2e-6}}^{[\ce{Al^3+}]_{\text{n.e.}}}\vdot(\overbrace{\num{,01}}^{[\ce{OH-}]_{\text{n.e.}}})^3 = \num{2e-12}\Rightarrow
		Q_{ps} = \num{2e-12} > \num{7,29e-19} = K_{ps}
	$$
	$$
		\tcbhighmath[boxrule=0.4pt,arc=4pt,colframe=black,drop fuzzy shadow=yellow]{\text{\textbf{\large SÍ PRECIPITA}}}
	$$
\end{frame}
