Una disolución es \SI{,001}{\Molar} en estroncio(\rom{2}) y \SI{2}{\Molar} en Ca(\rom{2}). Si los productos de solubilidad del sulfato de estroncio (\ce{SrSO4}) y del sulfato de calcio (\ce{CaSO4}) son, respectivamente, \num{1,0e-7} y \num{1,0e-5}, calcule:
	\begin{enumerate*}[label={\alph*)},font=\bfseries]
		\item ¿qué ion precipitará antes al añadir lentamente sulfato de sodio (\ce{Na2SO4}) \SI{,1}{\Molar}?,
		\item ¿qué concentración del primer ion quedará en la disolución cuando comience a precipitar el segundo?
	\end{enumerate*}
\resultadocmd{\ce{CaSO4}; \SI{,1}{\Molar}}
