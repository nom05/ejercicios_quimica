\begin{frame}
    \frametitle{\ejerciciocmd}
    \framesubtitle{Enunciado}
    \textbf{
		Dadas las siguientes reacciones:
\begin{itemize}
    \item \ce{I2(g) + H2(g) -> 2 HI(g)}~~~$\Delta H_1 = \SI{-0,8}{\kilo\calorie}$
    \item \ce{I2(s) + H2(g) -> 2 HI(g)}~~~$\Delta H_2 = \SI{12}{\kilo\calorie}$
    \item \ce{I2(g) + H2(g) -> 2 HI(ac)}~~~$\Delta H_3 = \SI{-26,8}{\kilo\calorie}$
\end{itemize}
Calcular los parámetros que se indican a continuación:
\begin{description}%[label={\alph*)},font={\color{red!50!black}\bfseries}]
    \item[\texttt{a)}] Calor molar latente de sublimación del yodo.
    \item[\texttt{b)}] Calor molar de disolución del ácido yodhídrico.
    \item[\texttt{c)}] Número de calorías que hay que aportar para disociar en sus componentes el yoduro de hidrógeno gas contenido en un matraz de \SI{750}{\cubic\centi\meter} a \SI{25}{\celsius} y \SI{800}{\torr} de presión.
\end{description}
\resultadocmd{\SI{12,8}{\kilo\calorie}; \SI{-13,0}{\kilo\calorie}; \SI{12,9}{\calorie}}

            }
\end{frame}

\begin{frame}
    \frametitle{\ejerciciocmd}
    \framesubtitle{Datos del problema}
    {\Large
        \textbf{
            \begin{enumerate}[label={\alph*)},font={\color{red!50!black}\bfseries}]
                \item ¿$pH$ de disolución saturada?
                \item ¿$m(\ce{Ca(OH)2})$ disuelta en \SI{150}{\milli\liter} de $pH = 14$?
            \end{enumerate}        
    }}
    $$
        \tcbhighmath[boxrule=0.4pt,arc=4pt,colframe=blue,drop fuzzy shadow=red]{K_{ps}(\ce{Ca(OH)2})= \SI{7,9e-6}{}}\quad
        \tcbhighmath[boxrule=0.4pt,arc=4pt,colframe=blue,drop fuzzy shadow=red]{Mm(\ce{Ca(OH)2})= \SI{74,10}{\gram\per\mol}}
    $$
\end{frame}

\begin{frame}
    \frametitle{\ejerciciocmd}
    \framesubtitle{Resolución (\rom{1}): $pH$ de disolución saturada}
    \structure{Reacción:}
    $$
        \ce{
            Ca(OH)2(s) v
               <=>
            $\underset{s}{\ce{Ca^2+(ac)}}$
               +
            $\underset{2s}{\ce{2OH-(ac)}}$
           }
    $$
    \visible<2->{
        \structure{La solubilidad en el equilibrio:}
        \begin{overprint}
            \onslide<2>
                $$
                    K_{ps}(\ce{Ca(OH)2}) = [\ce{Ca^2+}][\ce{OH-}]^2 = 4s^3
                $$
            \onslide<3->
                $$
                    s = \left(\frac{K_{ps}(\ce{Ca(OH)2})}{4}\right)^{\frac{1}{3}}\Rightarrow s=\SI{,0125}{\Molar}
                $$
        \end{overprint}
                }
    \visible<4->{
        \structure{El $pH$ será:}
        \begin{overprint}
            \onslide<4>
                $$
                    [\ce{OH-}] = 2s = 2\cdot\SI{,0125}{\Molar} = \SI{,02509}{\Molar}
                $$
            \onslide<5>
                $$
                    pOH = -\log[\ce{OH-}]\Rightarrow pOH = \SI{1,60}{}
                $$
            \onslide<6->
                $$
                    \tcbhighmath[boxrule=0.4pt,arc=4pt,colframe=blue,drop fuzzy shadow=red]{pH = 14-pOH = \SI{12,40}{}}
                $$
        \end{overprint}
                }
\end{frame}

\begin{frame}
    \frametitle{\ejerciciocmd}
    \framesubtitle{Resolución (\rom{2}): $m(\ce{Ca(OH)2})$ disuelta en \SI{150}{\milli\liter} de $pH = 14$}
    \structure{Concentración de anión hidróxido debido al pH:}
        $$
            pH = 14 \Rightarrow[\ce{OH-}] = \SI{1}{\Molar}
        $$
    \visible<2->{
        \structure{Nueva solubilidad:}
        \begin{overprint}
            \onslide<2>
                $$
                    \ce{Ca(OH)2(s) v <=>
                        $\underset{s}{\ce{Ca^2+(ac)}}$
                           +
                        $\underset{2s+1\approx 1}{\ce{2OH-(ac)}}$
                    }
                $$
            \onslide<3>
                $$
                    K_{ps}(\ce{Ca(OH)2}) = [\ce{Ca^2+}][\ce{OH-}]^2 = s(\underbrace{2s+1}_{\approx 1})^2\approx s=K_{ps}(\ce{Ca(OH)2})= \SI{7,9e-6}{\Molar}
                $$
            \onslide<4>
                $$
                    s = \overbrace{[\ce{Ca(OH)2}]}^{M=\frac{n}{V}}_{\text{que se disuelve}} = \SI{7,9e-6}{\Molar}
                $$
            \onslide<5>
                $$
                    \frac{\overbrace{n(\ce{Ca(OH)2})}^{n=\frac{m}{Mm}}}{V} = \SI{7,9e-6}{\Molar}
                $$
            \onslide<6>
                $$
                    \frac{m(\ce{Ca(OH)2})}{Mm(\ce{Ca(OH)2})\cdot V} = \SI{7,9e-6}{\Molar}
                $$
            \onslide<7>
                $$
                    m(\ce{Ca(OH)2}) = \SI{7,9e-6}{\Molar}\cdot Mm(\ce{Ca(OH)2})\cdot V
                $$
            \onslide<8>
                $$
                    \tcbhighmath[boxrule=0.4pt,arc=4pt,colframe=blue,drop fuzzy shadow=red]{m(\ce{Ca(OH)2}) = \SI{7,9e-6}{\cancel\mol\per\cancel\liter}\cdot\SI{74,10}{\gram\per\cancel\mol}\cdot\SI{,150}{\cancel\liter}=\SI{8,78e-5}{\gram}}
                $$
        \end{overprint}
                }
\end{frame}

