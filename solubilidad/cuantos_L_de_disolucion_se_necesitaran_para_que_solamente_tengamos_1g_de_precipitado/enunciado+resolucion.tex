\begin{frame}
	\frametitle{\ejerciciocmd}
	\framesubtitle{Enunciado}
	\textbf{
		Dadas las siguientes reacciones:
\begin{itemize}
    \item \ce{I2(g) + H2(g) -> 2 HI(g)}~~~$\Delta H_1 = \SI{-0,8}{\kilo\calorie}$
    \item \ce{I2(s) + H2(g) -> 2 HI(g)}~~~$\Delta H_2 = \SI{12}{\kilo\calorie}$
    \item \ce{I2(g) + H2(g) -> 2 HI(ac)}~~~$\Delta H_3 = \SI{-26,8}{\kilo\calorie}$
\end{itemize}
Calcular los parámetros que se indican a continuación:
\begin{description}%[label={\alph*)},font={\color{red!50!black}\bfseries}]
    \item[\texttt{a)}] Calor molar latente de sublimación del yodo.
    \item[\texttt{b)}] Calor molar de disolución del ácido yodhídrico.
    \item[\texttt{c)}] Número de calorías que hay que aportar para disociar en sus componentes el yoduro de hidrógeno gas contenido en un matraz de \SI{750}{\cubic\centi\meter} a \SI{25}{\celsius} y \SI{800}{\torr} de presión.
\end{description}
\resultadocmd{\SI{12,8}{\kilo\calorie}; \SI{-13,0}{\kilo\calorie}; \SI{12,9}{\calorie}}

	}
\end{frame}

\begin{frame}
	\frametitle{\ejerciciocmd}
	\framesubtitle{Datos del problema}
	\begin{center}
		{\LARGE ¿$V$ disolución? ¿$K_{ps}(\ce{Ac(OH)3})$? ¿pH?}\\[.2cm]
		\tcbhighmath[boxrule=0.4pt,arc=4pt,colframe=blue,drop fuzzy shadow=red]{\text{solubilidad: } \SI{2,1e-3}{\gram}\text{ en }\SI{100}{\milli\liter}}\quad
		\tcbhighmath[boxrule=0.4pt,arc=4pt,colframe=blue,drop fuzzy shadow=red]{Mm(\ce{Ac(OH)3}) = \SI{278,022}{\gram\per\mol}}\\[.2cm]
		\tcbhighmath[boxrule=0.4pt,arc=4pt,colframe=orange,drop fuzzy shadow=red]{m_{\text{total}}(\ce{Ac(OH)3})\equiv m_T = \SI{3,1}{\gram}}\quad
		\tcbhighmath[boxrule=0.4pt,arc=4pt,colframe=orange,drop fuzzy shadow=red]{m_{\text{precipitado}}(\ce{Ac(OH)3})\equiv m_{prec} = \SI{1,0}{\gram}}\\[1.cm]
		{\LARGE se disuelve \SI{3,1}{\gram} si:}\\[.2cm]
		\tcbhighmath[boxrule=0.4pt,arc=4pt,colframe=black,drop fuzzy shadow=yellow]{\text{pH} = \num{8,5}}
	\end{center}
\end{frame}

\begin{frame}
	\frametitle{\ejerciciocmd}
	\framesubtitle{Resolución (\rom{1}): $V$ de disolución y solubilidad molar}
	\structure{Masa disuelta de \ce{Ac(OH)3}:} $m_{\text{disuelta}}\equiv m_{dis} = m_T - m_{prec}\Rightarrow m_{dis} = \SI{3,1}{\gram} - \SI{1,0}{\gram} = \SI{2,1}{\gram}$
	\structure{Convertimos la solubilidad másica a unas unidades más convenientes (\unit{\gram\per\liter}):}
	$$
		\frac{\SI{2,1e-3}{\cancel\gram}}{\SI{100}{\cancel\milli\liter}}\vdot\frac{\SI{100}{\cancel\milli\liter}}{\SI{0,1}{\liter}} = \SI{2,1e-2}{\gram\per\liter}
	$$
	\structure{Volumen disuelto:}
	$$
		s\text{ másica} = \frac{m}{V}\Rightarrow V=\frac{m}{s\text{ másica}}\Rightarrow V=\frac{\SI{2,1}{\cancel\gram}}{\SI{2,1e-2}{\cancel\gram\per\liter}}\Rightarrow
		\tcbhighmath[boxrule=0.4pt,arc=4pt,colframe=blue,drop fuzzy shadow=red]{V = \SI{100}{\liter}}
	$$
	\structure{Solubilidad molar:}
	$$
		s = \SI[per-mode = fraction]{2,1e-2}{\cancel\gram\per\liter}\vdot\frac{\SI{1}{\mol}}{\SI{278,022}{\cancel\gram}} = \SI{7,55e-5}{\mol\per\liter}
	$$
\end{frame}

\begin{frame}
	\frametitle{\ejerciciocmd}
	\framesubtitle{Resolución (\rom{2}): $K_{ps}(\ce{Ac(OH)3})$ y pH de disolución}
	\structure{Equilibrio de solubilidad y solubilidad molar:} usamos la estequiometría del equilibrio de solubilidad para averiguar la relación entre $s$ y $K_{ps}$ para \ce{Ac(OH)3}
	\begin{center}
		\begin{tabular}{lcc}
			\multicolumn{3}{r}{$\ce{Ac(OH)3(s) <=> Ac^3+(ac) + 3OH-(ac)}\quad K_{ps} = \ce{[Ac^3+][OH^-]^3}$}	\\
			\midrule
			&	[\ce{Ac^3+}]	&	[\ce{OH-}]	\\
			\midrule
			Solubilidad~(\si{\Molar})	&	$s$				&	$3s$
		\end{tabular}
	\end{center}
	\textbf{Según la estequiometría:} $[\ce{OH-}] = 3s\Rightarrow\text{pH} = 14 - \text{pOH} = 14 + \log[\ce{OH-}]$
	$$
		\tcbhighmath[boxrule=0.4pt,arc=4pt,colframe=blue,drop fuzzy shadow=red]{\text{pH} = 14 + \log(3s)\Rightarrow\text{pH} = \num{10,36}}
	$$
	$$
		K_{ps} = \overbrace{[\ce{Ac^3+}]}^s\overbrace{[\ce{OH-}]^3}^{(3s)^3 = 27s^3} = 27s^4\Rightarrow\tcbhighmath[boxrule=0.4pt,arc=4pt,colframe=blue,drop fuzzy shadow=red]{K_{ps}(\ce{Ac(OH)3}) = \num{8,79e-16}}
	$$
	\alert{\textbf{¡AVISO!}} $s(3s)^3 = 27s^4$. Ni es $s(3s)^3\neq 27s^3\neq 3s^4\neq 9s^4$. El ejercicio está \textbf{inmediatamente mal} si no se hace correctamente esta operación.
\end{frame}

\begin{frame}
	\frametitle{\ejerciciocmd}
	\framesubtitle{Resolución (\rom{3}): ¿se conseguirá disolver \SI{3,1}{\gram} a pH \num{8,5}?}
	\structure{Calculamos [\ce{OH-}]:} $\text{pH} = \num{8,5}\Rightarrow[\ce{OH-}] = 10^{-14+\text{pH}}\Rightarrow[\ce{OH-}] = \SI{3,16e-6}{\Molar}$\\
	Esta es la concentración de \ce{OH-}, pero no es $3s$ ni $s$. Además, no se debe relacionar estequiométricamente con el catión. Este valor proviene de un pH-metro y no distingue su origen.
	\structure{Concentración de \ce{Ac^3+}:} La concentración de \ce{OH-} se ha cambiado de alguna forma que no nos dice el enunciado (añadiendo un sólido ácido sin cambiar de volumen, p.ej.), pero la cantidad de \ce{Ac^3+} suponemos que proviene de los \SI{3,1}{\gram}. Cuando comparemos $Q_{ps}$ y $K_{ps}$ sabremos si es verdad nuestra suposición.
	$$
		n = \frac{m}{Mm}\Rightarrow n(\ce{Ac(OH)3}) = \frac{\SI{3,1}{\cancel\gram}}{\SI{278,022}{\cancel\gram\per\mol}} = \SI{1,11e-2}{\mol}
	$$
	$$
		n(\ce{Ac(OH)3}) = n(\ce{Ac^3+}) = \SI{1,11e-2}{\mol}\Rightarrow
		[\ce{Ac^3+}] = \frac{\SI{1,11e-2}{\mol}}{\SI{100}{\liter}} = \SI{1,11e-4}{\Molar}
	$$
	\structure{Producto iónico ($Q_{ps}$) \emph{vs.} producto de solubilidad ($K_{ps}$):} {\small (n.e. significa de ``no equilibrio'')}
	$$
		Q_{ps} = [\ce{Ac^3+}]_{\text{n.e.}}\vdot[\ce{OH-}]_{\text{n.e.}}^3\Rightarrow
		Q_{ps} = \num{1,11e-4}\vdot(\num{3,16e-6})^3
	$$
	$$
		\tcbhighmath[boxrule=0.4pt,arc=4pt,colframe=blue,drop fuzzy shadow=yellow]{
			Q_{ps} = \num{3,53e-21} < \num{8,79e-16} = K_{ps}(\ce{Ac(OH)3})\qquad\text{\textbf{SÍ SE DISUELVE}}
		}
	$$
\end{frame}