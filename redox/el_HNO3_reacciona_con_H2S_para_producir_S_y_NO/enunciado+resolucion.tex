\begin{frame}
	\frametitle{\ejerciciocmd}
	\framesubtitle{Enunciado}
	\textbf{
		Dadas las siguientes reacciones:
\begin{itemize}
    \item \ce{I2(g) + H2(g) -> 2 HI(g)}~~~$\Delta H_1 = \SI{-0,8}{\kilo\calorie}$
    \item \ce{I2(s) + H2(g) -> 2 HI(g)}~~~$\Delta H_2 = \SI{12}{\kilo\calorie}$
    \item \ce{I2(g) + H2(g) -> 2 HI(ac)}~~~$\Delta H_3 = \SI{-26,8}{\kilo\calorie}$
\end{itemize}
Calcular los parámetros que se indican a continuación:
\begin{description}%[label={\alph*)},font={\color{red!50!black}\bfseries}]
    \item[\texttt{a)}] Calor molar latente de sublimación del yodo.
    \item[\texttt{b)}] Calor molar de disolución del ácido yodhídrico.
    \item[\texttt{c)}] Número de calorías que hay que aportar para disociar en sus componentes el yoduro de hidrógeno gas contenido en un matraz de \SI{750}{\cubic\centi\meter} a \SI{25}{\celsius} y \SI{800}{\torr} de presión.
\end{description}
\resultadocmd{\SI{12,8}{\kilo\calorie}; \SI{-13,0}{\kilo\calorie}; \SI{12,9}{\calorie}}

		}
\end{frame}

\begin{frame}
	\frametitle{\ejerciciocmd}
	\framesubtitle{Datos del problema}
	\begin{center}
		{\huge¿V(\ce{H2S})?}\\[.3cm]
		\tcbhighmath[boxrule=0.4pt,arc=4pt,colframe=green,drop fuzzy shadow=blue]{[\ce{HNO3}]=\SI{,2}{\Molar}}\quad
		\tcbhighmath[boxrule=0.4pt,arc=4pt,colframe=green,drop fuzzy shadow=blue]{V(\ce{HNO3})=\SI{500}{\milli\liter}=\SI{,5}{\liter}}\\[.3cm]
		\tcbhighmath[boxrule=0.4pt,arc=4pt,colframe=blue,drop fuzzy shadow=orange]{T(\ce{H2S})=\SI{60}{\celsius}=\SI{333,15}{\kelvin}}\quad
		\tcbhighmath[boxrule=0.4pt,arc=4pt,colframe=blue,drop fuzzy shadow=orange]{P(\ce{H2S})=\SI{760}{\torr}=\SI{1}{\atm}}\\[.3cm]
		\structure{Reacción sin ajustar:}\quad\ce{HNO3 + H2S -> S + NO}
	\end{center}
\end{frame}

\begin{frame}
	\frametitle{\ejerciciocmd}
	\framesubtitle{Resolución (\rom{1}): ajuste de la reacción en medio ácido}
	\structure{Ajuste en medio ``\textbf{\underline{ácido}}''}
	\begin{overprint}
		\onslide<1>
			$$
				\ce{NO3-(ac) + S^2-(ac) -> S(s) + NO(g)^}
			$$
		\onslide<2>
			$$
				\ce{
					$\overset{+1}{\ce{H}}$
					$\overset{+5}{\ce{N}}$
					$\overset{-2}{\ce{O}}$3-
					+
					$\overset{-2}{\ce{S}}$^2-
					->
					$\overset{0}{\ce{S}}$
					+
					$\overset{+2}{\ce{N}}$
					$\overset{-2}{\ce{O}}$
				}
			$$
		\onslide<3>
			\structure{Semirreacción de oxidación:}\quad\ce{S^2- -> S}
			\structure{Semirreacción de reducción:}\quad\ce{NO3- -> NO}
		\onslide<4>
			\structure{Semirreacción de oxidación:}\quad\ce{S^2- -> S}
			\structure{Semirreacción de reducción:}\quad\ce{N\textbf{\ce{O3}}- -> N\textbf{O} + \textbf{\color{red}\ce{2H2O}}}\quad
					{\footnotesize (3 \ce{O} en reactivos / 1 \ce{O} en productos $\rightarrow$ \ce{2H2O})}
		\onslide<5>
		\structure{Semirreacción de oxidación:}\quad\ce{S^2- -> S}
		\structure{Semirreacción de reducción:}\quad\ce{NO3- + \textbf{\color{red}\ce{4H+}} -> NO + \textbf{\ce{2H2}}O}
					{\footnotesize (el agua ha añadido 4 hidrógenos)}
		\onslide<6>
			\structure{Semirreacción de oxidación:}\quad\ce{S^2- -> S \textbf{\color{red}\ce{+ 2e-}}}\quad
					{\footnotesize (2 cargas negativas en reactivos frente a 0 en productos $\rightarrow$ 2 electrones en productos)}
			\structure{Semirreacción de reducción:}\quad\ce{NO3- + 4H+ \textbf{\color{red}\ce{+ 3e-}} -> NO + 2H2O}\quad
					{\footnotesize (3 cargas positivas en reactivos frente a 0 en productos $\rightarrow$ 3 electrones en reactivos)}
		\onslide<7>
			\structure{Semirreacción de oxidación:} $3\times\left(\ce{S^2- -> S + 2e-}\right)$
			\structure{Semirreacción de reducción:} $2\times\left(\ce{NO3- + 4H+ + 3e- -> NO + 2H2O}\right)$
		\onslide<8>
			\structure{Semirreacción de oxidación:}\quad\ce{3S^2- -> 3S + 6e-}
			\structure{Semirreacción de reducción:}\quad\ce{2NO3- + 8H+ + 6e- -> 2NO + 4H2O}
		\onslide<9-10>
			\begin{center}
				\begin{tabular}{c}
						\ce{3S^2- -> 3S + \cancel{6e-}} \\
						\ce{2NO3- + 8H+ + \cancel{6e-} -> 2NO + 4H2O} \\
					\midrule
						\ce{2NO3- + 3S^2- + 8H+ -> 3S + 2NO + 4H2O}
				\end{tabular}				
			\end{center}
		\onslide<11->
			\ce{2HNO3 + 3H2S -> 3S + 2NO + 4H2O} \\
	\end{overprint}
	\begin{enumerate}[label={\alph*)},font={\color{red!50!black}\bfseries}]
		\item<1-> Escribimos la reacción sin ajustar.
		\item<2-> Asignamos números de oxidación omitiendo los iones que no entran en la reacción.
		\item<3-> Escribimos las ecuaciones de las semirreacciones de oxidación y reducción.
		\item<3-> Ajustamos los átomos diferentes al \ce{O} y al \ce{H}.
		\item<4-> Ajustamos los átomos de \ce{O} sumando \ce{H2O}.
		\item<5-> Ajustamos los átomos de \ce{H} sumando \ce{H+}.
		\item<6-> Ajustamos las cargas de las semirreacciones \underline{añadiendo} electrones.
		\item<7-> Hacemos que el número de electrones sea igual en las dos semirreacciones. Para ello multiplicamos por un factor.
		\item<9-> Sumamos ambas reacciones.
		\item<10-> Simplificamos si procede (en este caso no).
		\item<11-> Añadimos los iones que no entraban en la reacción vigilando que esté todo ajustado.
	\end{enumerate}
\end{frame}

\begin{frame}
	\frametitle{\ejerciciocmd}
	\framesubtitle{Resolución (\rom{2}): volumen de \ce{H2S}}
	\structure{Según la estequiometría:}\quad$3n(\ce{HNO3}) = 2n(\ce{H2S})$
	\structure{Aplicando la definición de concentración y la ecuación de los gases ideales:}
	$$
		3\overbrace{n(\ce{HNO3})}^{n=M\vdot V} = 2\underbrace{n(\ce{H2S})}_{n=\frac{P\vdot V}{R\vdot T}}\Rightarrow
		3\vdot[\ce{HNO3}]\vdot V(\ce{HNO3}) = 2\vdot \frac{P(\ce{H2S})\vdot V(\ce{H2S})}{R\vdot T(\ce{H2S})}
	$$
	$$
		V(\ce{H2S})=\frac{3\vdot[\ce{HNO3}]\vdot V(\ce{HNO3})\vdot R\vdot T(\ce{H2S})}{2\vdot P(\ce{H2S})}
	$$
	$$
		V(\ce{H2S})=\frac{3\vdot\SI{,2}{\cancel\mol\per\cancel\liter}\vdot\SI{,5}{\liter}\vdot\SI{,082}{\cancel\atm\cancel\liter\per\cancel\mol\per\cancel\kelvin}\vdot\SI{333,15}{\cancel\kelvin}}{2\vdot\SI{1}{\cancel\atm}}
	$$
	\begin{center}
		\tcbhighmath[boxrule=0.4pt,arc=4pt,colframe=blue,drop fuzzy shadow=orange]{V(\ce{H2S})=\SI{4,10}{\liter}}
	\end{center}
\end{frame}
