\begin{frame}
	\frametitle{\ejerciciocmd}
	\framesubtitle{Síntesis de lejía a partir de disolución de \ce{NaCl}}
	\structure{Primera fase -- proceso electrolítico:}
	{\small
		 \begin{center}
			\begin{tabular}{ccc}
					\ce{2Cl- -> Cl2 + 2e-}										& ox. (ánodo)		& $\varepsilon^0_{\text{ox}}=\SI{-1,36}{\volt}$	\\
					\ce{2H2O + 2e- -> H2 ^ + 2OH-}								& red. (cátodo)		& $\varepsilon^0_{\text{re}}=\SI{-,83}{\volt}$	\\
				\midrule
					\ce{2Cl- + 2H2O -> Cl2 + H2 + 2OH-}							&	Global			& $\varepsilon^0_{\text{T}}=\SI{-2,19}{\volt}$	\\[.3cm]
					{\normalsize\ce{2NaCl(ac) + 2H2O(l) -> Cl2(g) + H2(g) ^ + 2NaOH(ac)}}	& 	& 
			\end{tabular}
		\end{center}
	}
	\structure{Segunda fase -- dismutación espontánea del \ce{Cl2} en medio básico:} {\footnotesize (Si celda no particionada)}
	\begin{overprint}
		\onslide<1>
			$$
				\ce{Cl2(g) + OH-(ac) -> Cl-(ac) + ClO-(ac)}
			$$
		\onslide<2>
		$$
			\ce{
				$\overset{0}{\ce{Cl}}$2(g)
					+
				$\overset{-2}{\ce{O}}$
				$\overset{+1}{\ce{H}}$-(ac)
					->
				$\overset{-1}{\ce{Cl}}$-(ac)
					+
				$\overset{+1}{\ce{Cl}}
				$\overset{-2}{\ce{O}}-(ac)
			}
		$$
		\onslide<3>
			\structure{Semirreacción de oxidación:} \ce{Cl2(g) -> ClO-(ac)}
			\structure{Semirreacción de reducción:} \ce{Cl2(g) -> Cl-(ac)}
		\onslide<4>
			\structure{Semirreacción de oxidación:} \ce{\textbf{\textcolor{red}{\ce{Cl2}}}(g)   -> \textbf{\textcolor{red}{2}}ClO-(ac)}
			\structure{Semirreacción de reducción:} \ce{\textbf{\textcolor{green}{\ce{Cl2}}}(g) -> \textbf{\textcolor{green}{2}}Cl-(ac)}
		\onslide<5>
			\structure{Semirreacción de oxidación:} \ce{Cl2(g) -> \textbf{\textcolor{orange}{2}}Cl\textbf{\textcolor{orange}{\ce{O}}}-(ac) + \textbf{\textcolor{orange}{\ce{2H2O}}}(l)}
			\structure{Semirreacción de reducción:} \ce{Cl2(g) -> 2Cl-(ac)}
		\onslide<6>
			\structure{Semirreacción de oxidación:} \ce{Cl2(g) + \textbf{\textcolor{blue}{\ce{4OH-}}}(ac) -> \textbf{\textcolor{blue}{2}}Cl\textbf{\textcolor{blue}{\ce{O}}}-(ac) + \textbf{\textcolor{blue}{\ce{2H2O}}}(l)}
			\structure{Semirreacción de reducción:} \ce{Cl2(g) -> 2Cl-(ac)}
		\onslide<7>
			\structure{Semirreacción de oxidación:} \ce{Cl2(g) + 4OH-(ac) -> 2ClO-(ac) + 2H2O(l) \textbf{\textcolor{red}{+ 2e-}}}
			\structure{Semirreacción de reducción:} \ce{Cl2(g) \textbf{\textcolor{red}{+ 2e-}} -> 2Cl-(ac)}
		\onslide<8>
			\structure{Semirreacción de oxidación:} $\boldsymbol{1}\times\left(\ce{Cl2(g) + 4OH-(ac) -> 2ClO-(ac) + 2H2O(l) + 2e-}\right)$
			\structure{Semirreacción de reducción:} $\boldsymbol{1}\times\left(\ce{Cl2(g) + 2e- -> 2Cl-(ac)}\right)$
		\onslide<9>
			\structure{Semirreacción de oxidación:} \ce{Cl2(g) + 4OH-(ac) -> 2ClO-(ac) + 2H2O(l) + 2e-}
			\structure{Semirreacción de reducción:} \ce{Cl2(g) + 2e- -> 2Cl-(ac)}
		\onslide<10>
			\begin{center}
				{\small
					\begin{tabular}{c}
						\ce{Cl2(g) + 4OH-(ac) -> 2ClO-(ac) + 2H2O(l) + \cancel{\ce{2e-}}} \\
						\ce{Cl2(g) + \cancel{\ce{2e-}} -> 2Cl-(ac)} \\
						\midrule
						\ce{2Cl2(g) + 4OH-(ac) -> 2ClO-(ac) + 2H2O(l) + 2Cl-(ac)}
					\end{tabular}
				}
			\end{center}
		\onslide<11>
			$$
				\ce{Cl2(g) + 2OH-(ac) -> ClO-(ac) + H2O(l) + Cl-(ac)}
			$$
		\onslide<12>
			$$
				\ce{Cl2(g) + 2NaOH(ac) -> NaClO(ac) + H2O(l) + NaCl(ac)}
			$$
	\end{overprint}
	{\small
		\begin{enumerate}[label={\alph*)},font={\color{red!50!black}\bfseries}]
			\item<1-> Escribimos la reacción sin ajustar.
			\item<2-> Asignamos números de oxidación.
			\item<3-> Escribimos las ecuaciones de las semirreacciones de oxidación y reducción.
			\item<4-> Ajustamos los átomos diferentes al \ce{O} y al \ce{H}.
			\item<5-> En el lado de la semirreacción que presente exceso de oxígenos, añadiremos tantas moléculas de agua como oxígenos hay de más.
			\item<6-> Ajustamos en el otro lado el número de oxígenos e hidrógenos añadiendo \ce{OH-}.
			\item<7-> Ajustamos número de electrones en cada reacción.
			\item<8-> Hacemos que el número de electrones sea igual en las dos ecuaciones. Para ello multiplicamos por un factor en cada una.
			\item<10-> Sumamos ambas reacciones.
			\item<11-> Simplificamos si procede.
		\end{enumerate}
	}
\end{frame}
