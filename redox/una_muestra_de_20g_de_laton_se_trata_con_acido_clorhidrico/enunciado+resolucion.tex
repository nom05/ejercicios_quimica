\begin{frame}
	\frametitle{\ejerciciocmd}
	\framesubtitle{Enunciado}
	\textbf{
		Dadas las siguientes reacciones:
\begin{itemize}
    \item \ce{I2(g) + H2(g) -> 2 HI(g)}~~~$\Delta H_1 = \SI{-0,8}{\kilo\calorie}$
    \item \ce{I2(s) + H2(g) -> 2 HI(g)}~~~$\Delta H_2 = \SI{12}{\kilo\calorie}$
    \item \ce{I2(g) + H2(g) -> 2 HI(ac)}~~~$\Delta H_3 = \SI{-26,8}{\kilo\calorie}$
\end{itemize}
Calcular los parámetros que se indican a continuación:
\begin{description}%[label={\alph*)},font={\color{red!50!black}\bfseries}]
    \item[\texttt{a)}] Calor molar latente de sublimación del yodo.
    \item[\texttt{b)}] Calor molar de disolución del ácido yodhídrico.
    \item[\texttt{c)}] Número de calorías que hay que aportar para disociar en sus componentes el yoduro de hidrógeno gas contenido en un matraz de \SI{750}{\cubic\centi\meter} a \SI{25}{\celsius} y \SI{800}{\torr} de presión.
\end{description}
\resultadocmd{\SI{12,8}{\kilo\calorie}; \SI{-13,0}{\kilo\calorie}; \SI{12,9}{\calorie}}

		}
\end{frame}

\begin{frame}
	\frametitle{\ejerciciocmd}
	\framesubtitle{Datos del ejercicio}
	\begin{center}
		{\huge ¿$\% m/m(\ce{Zn})$?}\\[.3cm]
		\tcbhighmath[boxrule=0.4pt,arc=4pt,colframe=green,drop fuzzy shadow=blue]{m(\text{latón})=m(\ce{Cu})+m(\ce{Zn})=\SI{20}{\gram}}\quad
		\tcbhighmath[boxrule=0.4pt,arc=4pt,colframe=green,drop fuzzy shadow=blue]{\varepsilon^0(\ce{Zn^2+/Zn})=\SI{-,76}{\volt}}\\[.3cm]
		\tcbhighmath[boxrule=0.4pt,arc=4pt,colframe=blue,drop fuzzy shadow=green]{\varepsilon^0(\ce{Cu^2+/Cu})=\SI{,34}{\volt}}\\[.3cm]
		\tcbhighmath[boxrule=0.4pt,arc=4pt,colframe=black,drop fuzzy shadow=red]{\varepsilon^0(\ce{H+/H2})=\SI{,00}{\volt}}\quad
		\tcbhighmath[boxrule=0.4pt,arc=4pt,colframe=black,drop fuzzy shadow=red]{V(\ce{H2})=\SI{2,8}{\liter}}\quad
		\tcbhighmath[boxrule=0.4pt,arc=4pt,colframe=black,drop fuzzy shadow=red]{P(\ce{H2})=\SI{1}{\atm}}\\[.3cm]
		\tcbhighmath[boxrule=0.4pt,arc=4pt,colframe=black,drop fuzzy shadow=red]{T(\ce{H2})=\SI{25}{\celsius}=\SI{298,15}{\kelvin}}\\[.6cm]
	\end{center}
	\structure{\Large Pero también:}
	\begin{center}
		\tcbhighmath[boxrule=0.4pt,arc=4pt,colframe=green,drop fuzzy shadow=blue]{Mm(\ce{Zn})=\SI{65,38}{\gram\per\mol}}\quad
		\tcbhighmath[boxrule=0.4pt,arc=4pt,colframe=blue,drop fuzzy shadow=green]{Mm(\ce{Cu})=\SI{63,54}{\gram\per\mol}}\\[.3cm]
		\tcbhighmath[boxrule=0.4pt,arc=4pt,colframe=black,drop fuzzy shadow=red]{$\text{pH ácido}$}
	\end{center}
\end{frame}

\begin{frame}
	\frametitle{\ejerciciocmd}
	\framesubtitle{Resolución (\rom{1}): ajuste de reacción y composición de la aleación}
	\structure{Punto de partida:} {\small El proceso es ``$\text{metal}^{(0)} \rightarrow \text{ion}^{2+}$'', se van a \textbf{oxidar}. \underline{Los potenciales estándar, por convenio de la IUPAC, siempre los damos de reducción}. Para obtener los de oxidación, cambiamos el signo.}
	\begin{center}
		\begin{tabular}{cc}
			\ce{Zn -> Zn^2+ + 2e-} & $\varepsilon^0_{\text{ox}}=\SI{,76}{\volt}$\\
			\ce{Cu -> Cu^2+ + 2e-} & $\varepsilon^0_{\text{ox}}=\SI{-,34}{\volt}$
		\end{tabular}
	\end{center}
	\structure{Obtención del potencial estándar total:} {\small hay que combinar el potencial estándar de oxidación con el de reducción de la otra reacción (\ce{2H+ +2e- -> H2}, $\varepsilon^0=\SI{0}{\volt}$). El proceso ocurre sin aportar energía eléctrica, entonces $\Delta G = -n\vdot F\vdot\varepsilon<0\Rightarrow\varepsilon>0$. Por tanto, estamos ante la oxidación del cinc: \textcolor{red}{$\varepsilon^0_{\text{total}}=\SI{,76}{\volt}$}.}
	\structure{Reacción global:}
	\begin{center}
		\begin{tabular}{rl}
				S. oxidación: 	& \ce{Zn -> Zn^2+ + 2e-} 	\\
				S. reducción: 	& \ce{2H+ + 2e- -> H2}		\\
			\midrule
								& \ce{2H+ + Zn -> Zn^2+ + H2}$\Rightarrow$\ce{2HCl(ac) + Zn(s) -> ZnCl2(ac) + H2(g) ^}
		\end{tabular}
	\end{center}
	\structure{Según estequiometría y recordando la definición de $\% m/m$:} $\% m/m=100\vdot\frac{m_i}{m_T}$
	\begin{overprint}
		\onslide<1>
			$$
				\overbrace{n(\ce{Zn})}^{n=\frac{m}{Mm}} = \overbrace{n(\ce{H2})}^{n=\frac{PV}{RT}}\Rightarrow
				\frac{m(\ce{Zn})}{M_{at}(\ce{Zn})} = \frac{P(\ce{H2})\vdot V(\ce{H2})}{R\vdot T(\ce{H2})}\Rightarrow
				m(\ce{Zn}) = M_{at}(\ce{Zn})\vdot\frac{P(\ce{H2})\vdot V(\ce{H2})}{R\vdot T(\ce{H2})}\Rightarrow
			$$
			$$
				\% m/m(\ce{Zn})=\frac{100}{m_T}\vdot M_{at}(\ce{Zn})\vdot\frac{P(\ce{H2})\vdot V(\ce{H2})}{R\vdot T(\ce{H2})}\Rightarrow
			$$
		\onslide<2>
			$$
				\% m/m(\ce{Zn})=\num{100}\vdot\frac{\SI{65,38}{\cancel\gram\per\cancel\mol}}{\SI{20}{\cancel\gram}}\vdot\frac{\SI{1}{\cancel\atm}\vdot\SI{2,8}{\cancel\liter}}{\SI{,082}{\cancel\atm\cancel\liter\per\cancel\mol\per\cancel\kelvin}\vdot\SI{298,15}{\cancel\kelvin}}
			$$
			$$
				\tcbhighmath[boxrule=0.4pt,arc=4pt,colframe=green,drop fuzzy shadow=blue]{\% m/m(\ce{Zn})=\SI{37,44}{\percent}}\qquad
				\tcbhighmath[boxrule=0.4pt,arc=4pt,colframe=blue,drop fuzzy shadow=green]{\% m/m(\ce{Cu})=\SI{62,56}{\percent}}
			$$
	\end{overprint}

\end{frame}
