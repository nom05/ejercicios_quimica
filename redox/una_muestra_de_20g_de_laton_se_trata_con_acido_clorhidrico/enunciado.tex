Una muestra de \SI{20}{\gram} de latón (aleación de cinc y cobre) se trata con ácido clorhídrico (\ce{HCl}), desprendiéndose \SI{2,8}{\liter} de hidrógeno gas (\ce{H2}) medidos a \SI{1}{\atm} y \SI{25}{\celsius}.
\begin{enumerate}[label={\alph*)},font=\bfseries]
	\item Formula y ajusta la reacción o reacciones que tienen lugar.
	\item Calcula la composición de la aleación, expresándola como porcentaje en peso.
\end{enumerate}
Datos: $R = \SI{,082}{\atm\liter\per\mol\per\kelvin}$;
$\varepsilon^0(\ce{Zn^2+/Zn})= \SI{-,76}{\volt}$;
$\varepsilon^0(\ce{Cu^2+/Cu})=\SI{+,34}{\volt}$;
$\varepsilon(\ce{H+/H2})=\SI{,00}{\volt}$
\resultadocmd{\SI{37,44}{\percent} de \ce{Zn}}
\resultadocmd{\SI{37,44}{\percent}}