\begin{frame}
	\frametitle{\ejerciciocmd}
	\framesubtitle{Enunciado}
	\textbf{
		Una reacción tiene una constante de velocidad de \SI{,017}{\per\second} a \SI{298}{\kelvin} y una energía libre de activación del \SI{27,235}{\kilo\joule\per\mol}. La adición de un catalizador disminuye dicha energía de activación hasta un \SI{33}{\percent} de su valor inicial. Calcule la nueva constante de velocidad.
\resultadocmd{ \SI{26,86}{\per\second} }

		}
\end{frame}

\begin{frame}
	\frametitle{\ejerciciocmd}
	\framesubtitle{Datos del problema}
	\begin{center}
		{\huge¿$\varepsilon_{\text{total}}$ si $[\ce{Zn^{2+}}]=\SI{1,5}{\Molar}$ y $[\ce{Cu^2+}]=\SI{,75}{\Molar}$?}\\[.3cm]
		\tcbhighmath[boxrule=0.4pt,arc=4pt,colframe=green,drop fuzzy shadow=blue]{\varepsilon^0_{\text{total}}=\SI{1,1}{\volt}}
	\end{center}
\end{frame}

\begin{frame}
	\frametitle{\ejerciciocmd}
	\framesubtitle{Resolución (\rom{1}): $\varepsilon_{\text{total}}$ en otras condiciones que no son estándar}
	\structure{Reacción de oxidación-reducción:}
	\begin{center}
		\begin{tabular}{lcr}
			S. oxidación (ánodo): & \ce{Zn -> Zn^{2+} + \cancel{2e-}} & $\varepsilon^0_{\text{ox}}=\SI{,76}{\volt}$\\
			S. reducción (cátodo): & \ce{Cu^{2+} + \cancel{2e-} -> Cu} & $\varepsilon^0_{\text{red}}=\SI{,34}{\volt}$\\
			\midrule
			Reacción global: & \ce{Cu^{2+}(ac) + Zn(s) -> Cu(s) + Zn^{2+}(ac))}  & $\varepsilon^0_{\text{total}}=\SI{1,10}{\volt}$\\
		\end{tabular}				
	\end{center}
	\begin{overprint}
		\onslide<2>
			\structure{Recordemos dos expresiones estudiadas anteriormente:}
			$$
				\overbrace{\Delta G}^{\Delta G = -nF\varepsilon} = \underbrace{\Delta G^0}_{\Delta G^0 = -nF\varepsilon^0} + RT\ln{Q}
			$$
		\onslide<3>
			\structure{Operamos:}
			$$
				-nF\varepsilon = -nF\varepsilon^0 + RT\ln{Q}
			$$
		\onslide<4>
			\structure{Ecuación de \textbf{\underline{NERNST}} general expresada con logaritmo natural:}
			$$
				\varepsilon = \varepsilon^0 - \frac{\overbrace{R}^{R=\SI{8,314}{\joule\per\mol\per\kelvin}}\cdot T}{n\cdot\underbrace{F}_{F=\SI{96485}{\coulomb\per\mol}}}\ln{Q}
			$$
		\onslide<5>
			\structure{Ecuación de \textbf{\underline{NERNST}} expresada con logaritmo decimal (si temperatura en condiciones estándar):}
			$$
				\varepsilon = \varepsilon^0 - \frac{\num{8,314}\cdot\num{298,15}}{n\cdot\num{96485}}\overbrace{\ln{10}}^{\num{2,303}}\cdot\log{Q}
			$$
		\onslide<6->
			\structure{Ecuación de \textbf{\underline{NERNST}} expresada con logaritmo decimal (si temperatura en condiciones estándar):}
			$$
				\varepsilon = \varepsilon^0 - \frac{\num{,0592}}{n}\log{Q}
			$$
	\end{overprint}
	\visible<6->{
		siendo $n$ el número de electrones implicados en el proceso ($n=\num{2}$) y $Q$ el cociente de las concentraciones (acuosos) y presiones (gases) de productos entre las de reactivos:
		$$
			Q=\frac{[\ce{Zn^{2+}}]}{[\ce{Cu^{2+}}]}
		$$
				}
	\visible<7>{
		\structure{Sustituimos por los valores en la ecuación de Nernst:}
		$$
			\varepsilon=\num{1,10}-\frac{\num{,0592}}{2}\log\left(\frac{\num{1,5}}{\num{,75}}\right)
		$$
		\centering\tcbhighmath[boxrule=0.4pt,arc=4pt,colframe=green,drop fuzzy shadow=blue]{\varepsilon=\SI{1,09}{\volt}}
				}
\end{frame}
