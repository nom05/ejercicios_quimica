\begin{frame}
    \frametitle{\ejerciciocmd}
    \framesubtitle{Enunciado}
    \textbf{
		Dadas las siguientes reacciones:
\begin{itemize}
    \item \ce{I2(g) + H2(g) -> 2 HI(g)}~~~$\Delta H_1 = \SI{-0,8}{\kilo\calorie}$
    \item \ce{I2(s) + H2(g) -> 2 HI(g)}~~~$\Delta H_2 = \SI{12}{\kilo\calorie}$
    \item \ce{I2(g) + H2(g) -> 2 HI(ac)}~~~$\Delta H_3 = \SI{-26,8}{\kilo\calorie}$
\end{itemize}
Calcular los parámetros que se indican a continuación:
\begin{description}%[label={\alph*)},font={\color{red!50!black}\bfseries}]
    \item[\texttt{a)}] Calor molar latente de sublimación del yodo.
    \item[\texttt{b)}] Calor molar de disolución del ácido yodhídrico.
    \item[\texttt{c)}] Número de calorías que hay que aportar para disociar en sus componentes el yoduro de hidrógeno gas contenido en un matraz de \SI{750}{\cubic\centi\meter} a \SI{25}{\celsius} y \SI{800}{\torr} de presión.
\end{description}
\resultadocmd{\SI{12,8}{\kilo\calorie}; \SI{-13,0}{\kilo\calorie}; \SI{12,9}{\calorie}}

	}
\end{frame}

\begin{frame}
	\frametitle{\ejerciciocmd}
	\framesubtitle{Datos del enunciado}
	\begin{center}
		{\Large\textbf{¿Reacciones espontáneas?}}\\
		\tcbhighmath[boxrule=0.4pt,arc=4pt,colframe=red,drop fuzzy shadow=blue]{\varepsilon^0(\ce{Pb^2+/Pb})=\SI{-,126}{\volt}}\quad
		\tcbhighmath[boxrule=0.4pt,arc=4pt,colframe=blue,drop fuzzy shadow=green]{\varepsilon^0(\ce{Cu^2+/Cu})=\SI{,337}{\volt}}\quad
		\tcbhighmath[boxrule=0.4pt,arc=4pt,colframe=blue,drop fuzzy shadow=green]{\varepsilon^0(\ce{Ni^2+/Ni})=\SI{-,250}{\volt}}
	\end{center}
	\begin{enumerate}[label={\alph*)},font={\color{red!50!black}\bfseries}]
		\item\tcbhighmath[boxrule=0.4pt,arc=4pt,colframe=red,drop fuzzy shadow=blue]{\ce{Pb(II) (\SI{1}{\Molar}) + Cu -> Cu(II) (\SI{1}{\Molar}) + Pb}}
		\item\tcbhighmath[boxrule=0.4pt,arc=4pt,colframe=blue,drop fuzzy shadow=yellow]{\ce{Pb(II) (\SI{1}{\Molar}) + Ni -> Ni(II) (\SI{,1}{\Molar}) + Pb}}
		\item\tcbhighmath[boxrule=0.4pt,arc=4pt,colframe=yellow,drop fuzzy shadow=blue]{\ce{Pb(II) (\SI{1e-7}{\Molar}) + Ni -> Ni(II) (\SI{,1}{\Molar}) + Pb}}
		\item\tcbhighmath[boxrule=0.4pt,arc=4pt,colframe=red,drop fuzzy shadow=blue]{\ce{Pb(II) (\SI{,1}{\Molar}) + Cu -> Cu(II) (\SI{,1}{\Molar}) + Pb}}
		\item\tcbhighmath[boxrule=0.4pt,arc=4pt,colframe=blue,drop fuzzy shadow=yellow]{\ce{Pb(II) (\SI{,1}{\Molar}) + Ni -> Ni(II) (\SI{,01}{\Molar}) + Pb}}
		\item\tcbhighmath[boxrule=0.4pt,arc=4pt,colframe=yellow,drop fuzzy shadow=blue]{\ce{Pb(II) (\SI{1e-8}{\Molar}) + Ni -> Ni(II) (\SI{,01}{\Molar}) + Pb}}
	\end{enumerate}
\end{frame}

\begin{frame}
	\frametitle{\ejerciciocmd}
	\framesubtitle{Resolución (\rom{1}): teoría a tener en cuenta}
	Una reacción es espontánea o no en función de $\Delta G$. La Termodinámica nos dice:
	\begin{itemize}
		\item<1->$\Delta G>0$: proceso no espontáneo.
		\item<2->$\Delta G=0$: proceso en equilibrio.
		\item<3->$\Delta G<0$: proceso espontáneo.
	\end{itemize}
	\visible<3->{
		La expresión que relaciona los potenciales de oxidación y reducción ($\varepsilon$) con $\Delta G$ es la siguiente: $\Delta G = -n\vdot F\vdot\varepsilon$, siendo $F$ la constante de Faraday (\SI{96485}{\coulomb\per\mol}) y $n$ el número de electrones implicados en el proceso.
				}
	\visible<4->{
		Fuera de condiciones estándar, además, tenemos que tener en cuenta:
		$$
			\Delta G = \overbrace{\Delta G^0}^{\text{estándar}} + R\vdot T\ln(Q)\Rightarrow\varepsilon=\varepsilon^0-\frac{R\vdot T}{n\vdot F}\ln(Q)
		$$
		La última expresión es la conocida como \textbf{ecuación de Nernst.}
				}
	\visible<5->{
		Vamos a hacer una tabla calculando $\Delta G$ de la reacción.
				}
\end{frame}

\begin{frame}
	\frametitle{\ejerciciocmd}
	\framesubtitle{Resolución (\rom{2}): reacciones del ejercicio}
    \structure{Reacción de oxidación-reducción plomo/cobre:}
	\begin{center}
		\begin{tabular}{lcr}
			Oxidación, reductor (ánodo):  & \ce{Cu(s) + -> Cu^2+(ac) + \cancel{2e-}} & $\varepsilon^0_{\text{ox}}=\SI{-,337}{\volt}$\\
			Reducción, oxidante (cátodo): & \ce{Pb^2+(ac) + \cancel{2e-} -> Pb(s)} & $\varepsilon^0_{\text{red}}=\SI{-,126}{\volt}$\\
			\midrule
			Reacción global: & \ce{Pb^2+(ac) + Cu(s) -> Pb(s) + Cu^2+(ac)}  & $\varepsilon^0_{\text{total}}=\SI{-,463}{\volt}$\\
		\end{tabular}
	\end{center}
	\textbf{2 electrones implicados}
	$$
		Q=\frac{[\ce{Cu^2+}]}{[\ce{Pb^2+}]}
	$$
	\visible<2->{
		\structure{Reacción de oxidación-reducción plomo/níquel:}
		\begin{center}
			\begin{tabular}{lcr}
				Oxidación, reductor (ánodo):  & \ce{Ni(s) + -> Ni^2+(ac) + \cancel{2e-}} & $\varepsilon^0_{\text{ox}}=\SI{,250}{\volt}$\\
				Reducción, oxidante (cátodo): & \ce{Pb^2+(ac) + \cancel{2e-} -> Pb(s)} & $\varepsilon^0_{\text{red}}=\SI{-,126}{\volt}$\\
				\midrule
				Reacción global: & \ce{Pb^2+(ac) + Ni(s) -> Pb(s) + Ni^2+(ac)}  & $\varepsilon^0_{\text{total}}=\SI{,124}{\volt}$\\
			\end{tabular}
		\end{center}
		\textbf{2 electrones implicados}
		$$
			Q=\frac{[\ce{Ni^2+}]}{[\ce{Pb^2+}]}
		$$
				}
\end{frame}

\begin{frame}
	\frametitle{\ejerciciocmd}
	\framesubtitle{Resolución (\rom{3}): análisis de la espontaneidad de las reacciones REDOX}
	\structure{Reacciones:}
	\begin{enumerate}[label={\alph*)},font={\color{red!50!black}\bfseries}]
		\item\label{Pb21MCu}  \ce{Pb^2+(\SI{1}{\Molar})    + Cu(s) -> Cu^2+(\SI{1}{\Molar})   + Pb(s)}
		\item\label{Pb21MNi}  \ce{Pb^2+(\SI{1}{\Molar})    + Ni(s) -> Ni^2+(\SI{,1}{\Molar}   + Pb(s)}
		\item\label{Pb2107MNi}\ce{Pb^2+(\SI{1e-7}{\Molar}) + Ni(s) -> Ni^2+(\SI{,1}{\Molar})  + Pb(s)}
		\item\label{Pb201MCu} \ce{Pb^2+(\SI{,1}{\Molar})   + Cu(s) -> Cu^2+(\SI{,1}{\Molar})  + Pb(s)}
		\item\label{Pb201MNi} \ce{Pb^2+(\SI{,1}{\Molar})   + Ni(s) -> Ni^2+(\SI{,01}{\Molar}) + Pb(s)}
		\item\label{Pb2108MNi}\ce{Pb^2+(\SI{1e-8}{\Molar}) + Ni(s) -> Ni^2+(\SI{,01}{\Molar}) + Pb(s)}
	\end{enumerate}
	\structure{Condiciones:} $T=\SI{293,15}{\kelvin}$, $n=2$
	\begin{center}
		\begin{tabular}{lSSSSSSSc}
			\toprule
				R. &
			    {$\varepsilon^0_T(\si{\volt})$} &
			    {$\Delta G^0$\footnote[1]{\si{\kilo\joule\per\mol}}} &
			    {$[ox](\si{\Molar})$\footnote[2]{Concentración de oxidante}} &
			    {$[re](\si{\Molar})$\footnote[3]{Concentración de reductor}} &
			    {$Q$} &
			    {$\frac{RT\ln(Q)}{1000}$} &
			    {$\Delta G$\footnotemark[1]} &
			    {\footnotesize ¿Espontánea?} \\
			\midrule
				\ref{Pb21MCu}   & -,463 &  89,3 & 1   & 1    & 1   &     0    &  89,3 & NO \\
				\ref{Pb21MNi}   &  ,124 & -23,9 & 1   &  ,1  &  ,1 &    -5,6  & -29,5 & SÍ \\
				\ref{Pb2107MNi} &  ,124 & -23,9 & e-7 &  ,1  &  e6 &    33,7  &   9,7 & NO \\
				\ref{Pb201MCu}  & -,463 &  89,3 &  ,1 &  ,1  & 1   &     0    &  89,3 & NO \\
				\ref{Pb201MNi}  &  ,124 & -23,9 &  ,1 &  ,01 &  ,1 &    -5,6  & -29,5 & SÍ \\
				\ref{Pb2108MNi} &  ,124 & -23,9 & e-8 &  ,01 & e6  &    33,7  &   9,7 & NO \\
			\bottomrule
		\end{tabular}
	\end{center}
\end{frame}
