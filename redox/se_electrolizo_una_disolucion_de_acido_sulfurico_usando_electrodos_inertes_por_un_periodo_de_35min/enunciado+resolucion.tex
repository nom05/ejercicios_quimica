\begin{frame}
    \frametitle{\ejerciciocmd}
    \framesubtitle{Enunciado}
    \textbf{
		Dadas las siguientes reacciones:
\begin{itemize}
    \item \ce{I2(g) + H2(g) -> 2 HI(g)}~~~$\Delta H_1 = \SI{-0,8}{\kilo\calorie}$
    \item \ce{I2(s) + H2(g) -> 2 HI(g)}~~~$\Delta H_2 = \SI{12}{\kilo\calorie}$
    \item \ce{I2(g) + H2(g) -> 2 HI(ac)}~~~$\Delta H_3 = \SI{-26,8}{\kilo\calorie}$
\end{itemize}
Calcular los parámetros que se indican a continuación:
\begin{description}%[label={\alph*)},font={\color{red!50!black}\bfseries}]
    \item[\texttt{a)}] Calor molar latente de sublimación del yodo.
    \item[\texttt{b)}] Calor molar de disolución del ácido yodhídrico.
    \item[\texttt{c)}] Número de calorías que hay que aportar para disociar en sus componentes el yoduro de hidrógeno gas contenido en un matraz de \SI{750}{\cubic\centi\meter} a \SI{25}{\celsius} y \SI{800}{\torr} de presión.
\end{description}
\resultadocmd{\SI{12,8}{\kilo\calorie}; \SI{-13,0}{\kilo\calorie}; \SI{12,9}{\calorie}}

           }
\end{frame}

\begin{frame}
    \frametitle{\ejerciciocmd}
    \framesubtitle{Datos del problema}
    \begin{center}
        \textbf{\Large ¿Intensidad?}
    \end{center}
    $$
        \tcbhighmath[boxrule=0.4pt,arc=4pt,colframe=yellow,drop fuzzy shadow=blue]{P(\ce{H2}) = \SI{752}{\torr}-\SI{28}{\torr}}~
        \tcbhighmath[boxrule=0.4pt,arc=4pt,colframe=yellow,drop fuzzy shadow=blue]{V(\ce{H2}) = \SI{145}{\milli\liter}}~
        \tcbhighmath[boxrule=0.4pt,arc=4pt,colframe=yellow,drop fuzzy shadow=blue]{T(\ce{H2}) = \SI{301,15}{\kelvin}}
    $$
    $$
        \tcbhighmath[boxrule=0.4pt,arc=4pt,colframe=green,drop fuzzy shadow=red]{t = \SI{35}{\minute}}\quad
        \tcbhighmath[boxrule=0.4pt,arc=4pt,colframe=green,drop fuzzy shadow=red]{\text{ánodo:} (\ce{H2O/H+,O2})}\quad
        \tcbhighmath[boxrule=0.4pt,arc=4pt,colframe=green,drop fuzzy shadow=red]{\text{cátodo:} (\ce{H+/H2})}
    $$
\end{frame}

\begin{frame}
    \frametitle{\ejerciciocmd}
    \framesubtitle{Resolución (\rom{1}): intensidad de la corriente}
    \begin{center}
        \begin{tabular}{lc}
            \textbf{\textit{Ánodo} (oxidación):}  & \ce{2H2O -> O2 + 4H+ + 4e-}\\
            \textbf{\textit{Cátodo} (reducción):} & $2\times\left(\ce{2H+ + 2e- -> H2}\right)$\\
            \midrule
            \textbf{\textit{Total:}}  & \ce{2H2O -> O2 + 2H2}
        \end{tabular}
    \end{center}
    \structure{Número de moles de \ce{H2} (ecuación de gases ideales):}
    $$
        n(\ce{H2}) = \frac{\SI{,953}{\cancel\atm}\cdot\SI{,145}{\cancel\liter}}{\SI{,082}{\cancel\atm\cancel\liter\per\mol\per\cancel\kelvin}\cdot\SI{301,15}{\cancel\kelvin}} = \SI{5,593e-3}{\mol}
    $$
    \structure{Número de moles de electrones:} hay \SI{2}{\mol} de \ce{e-} por cada \SI{1}{\mol} de \ce{H2}
    $$
        n(\ce{e-}) = 2\times\SI{5,593e-3}{\mol} = \SI{1,119e-2}{\mol}
    $$
    $$
        n(\ce{e-}) = \frac{I\cdot t}{96500}\Rightarrow I = \frac{n(\ce{e-})\cdot 96500}{t}\Rightarrow I = \frac{\SI{1,119e-2}{\cancel\mol}\cdot\SI{96500}{\coulomb\per\cancel\mol}}{\SI{35}{\cancel\minute}\cdot\SI{60}{\second\per\cancel\minute}}
    $$
    $$
        \tcbhighmath[boxrule=0.4pt,arc=4pt,colframe=yellow,drop fuzzy shadow=blue]{I = \SI{,514}{\ampere}}
    $$
\end{frame}
