Se electrolizó una disolución de ácido sulfúrico usando electrodos inertes por un periodo de \SI{35}{\minute}. El hidrógeno producido en el cátodo se recoge sobre agua a una presión total de \SI{752}{\torr} y a una temperatura de \SI{28}{\celsius}. Si el volumen de \ce{H2} fue de \SI{145}{\milli\liter}, ¿cuál fue la corriente que circuló durante la electrólisis? $P_v(\ce{H2O},~\SI{28}{\celsius}) = \SI{28}{\torr}$. Reacciones en los electrodos: ánodo (\ce{H2O}/\ce{H+},\ce{O2}); cátodo (\ce{H+}/\ce{H2}).
\resultadocmd{ \SI{,514}{\ampere} }
