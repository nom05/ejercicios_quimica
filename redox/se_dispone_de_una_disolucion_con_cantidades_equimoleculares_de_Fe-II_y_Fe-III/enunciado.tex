Se dispone de una disolución con cantidades equimoleculares de \mbox{Fe(\rom{2})} y \mbox{Fe(\rom{3})} a la que se añade \mbox{Ag(\rom{1})} produciéndose la oxidación del hierro y la reducción de la plata. Calcular:
\begin{enumerate}[label={\alph*)},font={\bfseries}]
	\item la concentración de \mbox{Ag(\rom{1})} necesaria para que el potencial del sistema sea a \SI{0,0}{\volt} a \SI{25}{\celsius},
	\item la constante de equilibrio de la reacción.
\end{enumerate}
Datos: $\varepsilon^0(\ce{Fe^2+/Fe})=\SI{-,44}{\volt}$, $\varepsilon^0(\ce{Fe^3+/Fe})=\SI{-,036}{\volt}$, $\varepsilon^0(\ce{Ag+/Ag})=\SI{,7996}{\volt}$
\resultadocmd{\SI{,342}{\Molar}; \num{2,93}}