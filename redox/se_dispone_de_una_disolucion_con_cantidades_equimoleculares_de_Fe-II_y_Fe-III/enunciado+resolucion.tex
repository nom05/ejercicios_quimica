\begin{frame}
	\frametitle{\ejerciciocmd}
	\framesubtitle{Enunciado}
	\textbf{
		Una reacción tiene una constante de velocidad de \SI{,017}{\per\second} a \SI{298}{\kelvin} y una energía libre de activación del \SI{27,235}{\kilo\joule\per\mol}. La adición de un catalizador disminuye dicha energía de activación hasta un \SI{33}{\percent} de su valor inicial. Calcule la nueva constante de velocidad.
\resultadocmd{ \SI{26,86}{\per\second} }

		}
\end{frame}

\begin{frame}
	\frametitle{\ejerciciocmd}
	\framesubtitle{Datos del problema}
	\begin{center}
		{\huge 
			¿$[\ce{Ag+}]$ si $\varepsilon_{\text{total}}=\SI{0}{\volt}$ y $K_{\text{eq}}$?
		}\\[.3cm]
		\tcbhighmath[boxrule=0.4pt,arc=4pt,colframe=black,drop fuzzy shadow=blue]{T=\SI{25}{\celsius}}\\[.3cm]
		\tcbhighmath[boxrule=0.4pt,arc=4pt,colframe=green,drop fuzzy shadow=blue]{\varepsilon^0(\ce{Fe^2+/Fe})=\SI{-,44}{\volt}}\quad
		\tcbhighmath[boxrule=0.4pt,arc=4pt,colframe=blue,drop fuzzy shadow=green]{\varepsilon^0(\ce{Fe^3+/Fe})=\SI{-,036}{\volt}}\\[.3cm]
		\tcbhighmath[boxrule=0.4pt,arc=4pt,colframe=red,drop fuzzy shadow=orange]{\varepsilon^0(\ce{Ag+/Ag})=\SI{,7996}{\volt}}\\[.3cm]
		\structure{Equimoleculares o equimolares:}
		\tcbhighmath[boxrule=0.4pt,arc=4pt,colframe=black,drop fuzzy shadow=blue]{[\ce{Fe^3+}]=[\ce{Fe^2+}]}
	\end{center}
\end{frame}

\begin{frame}
	\frametitle{\ejerciciocmd}
	\framesubtitle{Resolución (\rom{1}): determinación de $\varepsilon(\ce{Fe^3+/Fe^2+})$}
	Si se dispone de una disolución formada por \ce{Fe^2+} y \ce{Fe^3+} los potenciales de reducción no nos sirven directamente. El proceso que ocurre es la oxidación del \ce{Fe(II)} a \ce{Fe(III)}.
	\begin{center}
		{\small \begin{tabular}{ccl}
				\ce{Fe^2+ + 2e- -> Fe} 				 & $\varepsilon^0(\ce{Fe^2+/Fe})=\SI{-,44}{\volt}$				& $\Delta G^0_{\text{red}} = -2\vdot F\vdot\varepsilon(\ce{Fe^2+/Fe})$	\\[.3cm]
			\midrule
			\midrule
				\ce{Fe^3+ + 3e- -> \cancel{\ce{Fe}}} & $\varepsilon^0(\ce{Fe^3+/Fe})=\SI{-,036}{\volt}$				& $\Delta G^0_{\text{red}} = -3\vdot F\vdot\varepsilon(\ce{Fe^3+/Fe})$	\\
				\ce{\cancel{\ce{Fe}} -> Fe^2+ + 2e-} & $\varepsilon^0(\ce{Fe/Fe^2+})=\SI{,44}{\volt}$				& $\Delta G^0_{\text{ox}} = -2\vdot F\vdot\varepsilon(\ce{Fe/Fe^2+})$	\\
			\midrule
				\ce{Fe^3+ +1e- -> Fe^2+} 			 & $\cancel{\varepsilon^0(\ce{Fe^3+/Fe^2+})=\SI{,404}{\volt}}$	& $\Delta G^0 = -3\vdot F\vdot\varepsilon(\ce{Fe^3+/Fe})-2\vdot F\vdot\varepsilon(\ce{Fe/Fe^2+})$	\\
		\end{tabular}}
	\end{center}
	{\small $$
		\overbrace{\Delta G^0}^{-1\vdot\cancel{F}\vdot\varepsilon^0(\ce{Fe^3+/Fe^2+})} =
			 -3\vdot\cancel{F}\vdot\varepsilon(\ce{Fe^3+/Fe})-2\vdot\cancel{F}\vdot\varepsilon(\underbrace{\ce{Fe/Fe^2+}}_{\text{oxidación}})\Rightarrow
		\varepsilon^0(\ce{Fe^3+/Fe^2+}) = 3\vdot\varepsilon(\ce{Fe^3+/Fe})-2\vdot\varepsilon(\underbrace{\ce{Fe^2+/Fe}}_{\text{reducción}})
	$$}
	$$
		\varepsilon^0(\ce{Fe^3+/Fe^2+}) = 3\vdot\varepsilon(\ce{Fe^3+/Fe})-2\vdot\varepsilon(\ce{Fe^2+/Fe})\Rightarrow
		\varepsilon^0(\ce{Fe^3+/Fe^2+}) = 3\vdot(\SI{-,036}{\volt})-2\vdot(\SI{-,44}{\volt})
	$$
	\begin{center}
		{\Large\myovalbox{\textbf{\textcolor{yellow}{$\varepsilon^0(\ce{Fe^3+/Fe^2+}) = \SI{,772}{\volt}$}}}}
	\end{center}
\end{frame}

\begin{frame}
	\frametitle{\ejerciciocmd}
	\framesubtitle{Resolución (\rom{2}): cálculo de [\ce{Ag+}] y $K_{\text{eq}}$}
%	{\small
		\begin{center}
			\begin{tabular}{lcr}
					S. oxidación (ánodo): 	& \ce{Fe^2+ -> Fe^{3+} + \cancel{1e-}} 			& $\varepsilon^0_{\text{ox}}=\SI{-,7720}{\volt}$	\\
					S. reducción (cátodo): 	& \ce{Ag+ + \cancel{1e-} -> Ag} 				& $\varepsilon^0_{\text{red}}=\SI{,7996}{\volt}$	\\
				\midrule
					Reacción global:		& \ce{Fe^2+(ac) + Ag+(ac) -> Ag + Fe^3+(ac)}	& $\varepsilon^0_{\text{total}}=\SI{,0276}{\volt}$	\\
			\end{tabular}				
		\end{center}
%	}
	\structure{Datos:} $n = 1$,\quad $Q = \frac{[\ce{Fe^3+}]}{[\ce{Ag+}]\vdot[\ce{Fe^2+}]}$,\quad $\varepsilon=0$,\quad $[\ce{Fe^2+}] = [\ce{Fe^3+}]$\\
	\structure{Si $\varepsilon=0$:}
	$$
		\overbrace{\varepsilon}^0 =\varepsilon^0 - \frac{\num{,0592}}{n}\log Q\Rightarrow
		\overbrace{\varepsilon^0}^{\SI{,0276}{\volt}} = \frac{\num{,0592}}{\underbrace{n}_1}\log K_{\text{eq}}\Rightarrow
		K_{\text{eq}} = 10^{\frac{1\vdot\num{,0276}}{\num{,0592}}}\Rightarrow
		\tcbhighmath[boxrule=0.4pt,arc=4pt,colframe=black,drop fuzzy shadow=blue]{K_{\text{eq}}=\num{2,926}}
	$$
	$$
		K_{\text{eq}}=\frac{1}{[\ce{Ag+}]}\vdot\underbrace{\cancelto{1}{\frac{[\ce{Fe^3+}]}{[\ce{Fe^2+}]}}}_{[\ce{Fe^2+}] = [\ce{Fe^3+}]}\Rightarrow
		[\ce{Ag+}] = \frac{1}{K_{\text{eq}}}\Rightarrow
		\tcbhighmath[boxrule=0.4pt,arc=4pt,colframe=red,drop fuzzy shadow=orange]{[\ce{Ag+}]=\SI{,342}{\Molar}}
	$$
\end{frame}
