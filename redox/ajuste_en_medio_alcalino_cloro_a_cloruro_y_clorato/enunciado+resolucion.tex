\begin{frame}
	\frametitle{\ejerciciocmd}
	\framesubtitle{Enunciado}
	\textbf{
			Dadas las siguientes reacciones:
\begin{itemize}
    \item \ce{I2(g) + H2(g) -> 2 HI(g)}~~~$\Delta H_1 = \SI{-0,8}{\kilo\calorie}$
    \item \ce{I2(s) + H2(g) -> 2 HI(g)}~~~$\Delta H_2 = \SI{12}{\kilo\calorie}$
    \item \ce{I2(g) + H2(g) -> 2 HI(ac)}~~~$\Delta H_3 = \SI{-26,8}{\kilo\calorie}$
\end{itemize}
Calcular los parámetros que se indican a continuación:
\begin{description}%[label={\alph*)},font={\color{red!50!black}\bfseries}]
    \item[\texttt{a)}] Calor molar latente de sublimación del yodo.
    \item[\texttt{b)}] Calor molar de disolución del ácido yodhídrico.
    \item[\texttt{c)}] Número de calorías que hay que aportar para disociar en sus componentes el yoduro de hidrógeno gas contenido en un matraz de \SI{750}{\cubic\centi\meter} a \SI{25}{\celsius} y \SI{800}{\torr} de presión.
\end{description}
\resultadocmd{\SI{12,8}{\kilo\calorie}; \SI{-13,0}{\kilo\calorie}; \SI{12,9}{\calorie}}

	  	}
\end{frame}
 
\begin{frame}
	\frametitle{\ejerciciocmd}
	\framesubtitle{Resolución (\rom{1}): cloro molecular a cloruro y clorato}
	\structure{Medio alcalino:} ajuste en medio ``\textbf{\underline{básico}}''.
	\begin{overprint}
		\onslide<1>
			$$
				\ce{Cl2(g) + OH-(ac) -> Cl-(ac) + ClO3-(ac)}
			$$
		\onslide<2>
			$$
				\ce{
					$\overset{0}{\ce{Cl}}$2(g)
						+
					$\overset{-2}{\ce{O}}$
					$\overset{+1}{\ce{H}}$-(ac)
						->
					$\overset{-1}{\ce{Cl}}$-(ac)
						+
					$\overset{+5}{\ce{Cl}}
					$\overset{-2}{\ce{O}}3-(ac)
				}
			$$
		\onslide<3>
			\structure{Semirreacción de oxidación:}\quad\ce{Cl2(g) -> ClO3-(ac)}
			\structure{Semirreacción de reducción:}\quad\ce{Cl2(g) -> Cl-(ac)}
		\onslide<4>
			\structure{Semirreacción de oxidación:}\quad\ce{Cl2(g) -> 2ClO3-(ac)}
			\structure{Semirreacción de reducción:}\quad\ce{Cl2(g) -> 2Cl-(ac)}
		\onslide<5>
			\structure{Semirreacción de oxidación:} \ce{Cl2(g) -> 2ClO3-(ac) + 6H2O(l)}
			\structure{Semirreacción de reducción:} \ce{Cl2(g) -> 2Cl-(ac)}
		\onslide<6>
			\structure{Semirreacción de oxidación:} \ce{Cl2(g) + 12OH-(ac) -> 2ClO3-(ac) + 6H2O(l)}
			\structure{Semirreacción de reducción:} \ce{Cl2(g) -> 2Cl-(ac)}
		\onslide<7>
			\structure{Semirreacción de oxidación:} \ce{Cl2(g) + 12OH-(ac) -> 2ClO3-(ac) + 6H2O(l) + 10e-}
			\structure{Semirreacción de reducción:} \ce{Cl2(g) + 2e- -> 2Cl-(ac)}
		\onslide<8>
			\structure{Semirreacción de oxidación:} \ce{Cl2(g) + 12OH-(ac) -> 2ClO3-(ac) + 6H2O(l) + 10e-}
			\structure{Semirreacción de reducción:} $5\times\left(\ce{Cl2(g) + 2e- -> 2Cl-(ac)}\right)$
		\onslide<9>
			\structure{Semirreacción de oxidación:} \ce{Cl2(g) + 12OH-(ac) -> 2ClO3-(ac) + 6H2O(l) + 10e-}
			\structure{Semirreacción de reducción:} \ce{5Cl2(g) + 10e- -> 10Cl-(ac)}
		\onslide<10>
			\begin{tabular}{c}
				\ce{Cl2(g) + 12OH-(ac) -> 2ClO3-(ac) + 6H2O(l) + \cancel{\ce{10e-}}} \\
				\ce{5Cl2(g) + \cancel{\ce{10e-}} -> 10Cl-(ac)} \\
				\midrule
				\ce{6Cl2(g) + 12OH-(ac) -> 2ClO3-(ac) + 6H2O(l) + 10Cl-(ac)}
			\end{tabular}
		\onslide<11>
			$$
				\ce{3Cl2(g) + 6OH-(ac) -> ClO3-(ac) + 3H2O(l) + 5Cl-(ac)}
			$$
	\end{overprint}                 
	\begin{enumerate}[label={\alph*)},font={\color{red!50!black}\bfseries}]
		\item<1-> Escribimos la reacción sin ajustar.
		\item<2-> Asignamos números de oxidación.
		\item<3-> Escribimos las ecuaciones de las semirreacciones de oxidación y reducción.
		\item<4-> Ajustamos los átomos diferentes al \ce{O} y al \ce{H}.
		\item<5-> En el lado de la semirreacción que presente exceso de oxígenos, añadiremos tantas moléculas de agua como oxígenos hay de más.
		\item<6-> Ajustamos en el otro lado el número de oxígenos e hidrógenos añadiendo \ce{OH-}.
		\item<7-> Ajustamos número de electrones en cada reacción.
		\item<8-> Hacemos que el número de electrones sea igual en las dos ecuaciones. Para ello multiplicamos por un factor.
		\item<10-> Sumamos ambas reacciones.
		\item<11-> Simplificamos si procede.
	\end{enumerate}
\end{frame}
