Se desea crear una pila galvánica en la que se oxide litio en medio ácido y que trabaje con el mayor voltaje posible en condiciones estándar. En la siguiente tabla se recogen los esquemas de las semirreacciones disponibles y sus potenciales estándar de reducción:
\begin{center}
	{\footnotesize
	\begin{tabular}{l|SSSSSS}
		\toprule
			{Semirr.}					&
			{\ce{Li+(ac)/Li(s)}}		&
			{\ce{Na+(ac)/Na(s)}}		&
			{\ce{TiO2+(ac)/Ti(s)}}		&
			{\ce{H+(ac)/H2(g)}}			&
			{\ce{H2MoO4(ac)/Mo(s)}}		&
			{\ce{VO2+(ac)/V^3+(ac)}}	\\
		{$\varepsilon^0$~(\si{\volt})}	&
			-3,05						&
			-2,71						&
			 -,86						&
			 +,00						&
			 +,11						&
			 +,36						\\
		\bottomrule
	\end{tabular}
	}
\end{center}
Contesta a las siguientes cuestiones:
\begin{enumerate}[label={\alph*)},font={\bfseries}]
	\item Selecciona el electrodo adecuado que cree el voltaje mayor en combinación con el del litio. Escribe las dos semirreacciones y la reacción global ajustadas.
	\item Calcula el potencial estándar total de la pila.
	\item Escribe cuántos electrones están implicados en el proceso global.
	\item Expresa el cociente de reacción ($Q$).
	\item Mediante la ecuación de Nernst calcula la constante de equilibrio.
	\item Con las leyes de Faraday, calcula cuánto tiempo necesitará la pila para pasar de \SI{2}{\gram} a \SI{1,5}{\gram} de \ce{Li} sólido a una intensidad de \SI{1,2}{\ampere}. ¿Cuántos moles se consumirán del resto de los reactivos?
\end{enumerate}
Datos: $F = \SI{96485}{\coulomb\per\mol}$; $M_{\text{at}}(\ce{Li}) = \SI{6,941}{\gram\per\mol}$.
\resultadocmd{\ce{VO2+(ac)/V^3+(ac)}; \SI{3,41}{\volt}; \num{2}; $10^{\num{115,20}}$; \SI{1}{\hour} \SI{36}{\minute} \SI{32}{\second}.}