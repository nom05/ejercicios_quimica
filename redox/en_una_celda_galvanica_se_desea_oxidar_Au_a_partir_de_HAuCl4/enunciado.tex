En una celda galvánica se desea oxidar oro metálico a su hidrocloruro (\ce{HAuCl4}), el cual es muy soluble en agua. Debéis seleccionar de la tabla una de las semirreacciones como el otro electrodo teniendo en cuenta que el potencial estándar total de la celda debe ser máximo (voltaje más alto).
\begin{center}
	\begin{tabular}{cS}
		\toprule
			Electrodo	&	
			{$\varepsilon^0_{\text{reducción}} (\unit{\volt})$}				\\
		\midrule
			\ce{Mg^2+(ac)|Mg(s)}						&	-2,38	\\
			\ce{Al^3+(ac)|Al(s)}						&	-1,66	\\
			\ce{Ti2O3(s)|TiO(s)}						&	-1,23	\\
			\ce{Cr^3+(ac)|Cr^2+(ac)}					&	 -,42	\\
			\ce{CO2(g)|CO(g)}							&	 -,11	\\
			\ce{AuCl4^-(ac) + 3e- -> Au(s) + 4Cl-(ac)}	&	 1,00	\\
			\ce{Cr2O7^2-(ac)|Cr^3+(ac)}					&	 1,33	\\
			\ce{MnO4^-(ac)|Mn^2+(ac)}					&	 1,51	\\
		\bottomrule
	\end{tabular}
\end{center}
\begin{enumerate}
	\item ¿Cuáles son las semirreacciones de reducción y oxidación? ¿Cuáles son sus correspondientes potenciales estándar de reducción y de oxidación? ¿Cuál es el ajuste de las semirreacciones? ¿Cuál es la reacción global y su potencial estándar total?
	\item Si inicialmente todas las concentraciones acuosas de reactivos y productos son \SI{1,5}{\Molar}, ¿qué potencial tendrá la reacción a \SI{25}{\celsius}? 
	\item En las condiciones del apartado anterior, ¿qué volumen mínimo necesitaremos para oxidar \SI{2,00}{\gram} de oro sólido? ¿Cuánto tiempo se necesitará si pasa una corriente de \SI{1,00}{\ampere}?
\end{enumerate}
DATOS: $F = \SI{96485}{\coulomb\per\mol}$, $R = \SI{8,314}{\joule\per\mol\per\kelvin}$, masa atómica de \ce{Au}: \SI{196,967}{\atomicmassunit}.
