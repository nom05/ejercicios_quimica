\begin{frame}
    \frametitle{\ejerciciocmd}
    \framesubtitle{Enunciado}
    \textbf{
		Dadas las siguientes reacciones:
\begin{itemize}
    \item \ce{I2(g) + H2(g) -> 2 HI(g)}~~~$\Delta H_1 = \SI{-0,8}{\kilo\calorie}$
    \item \ce{I2(s) + H2(g) -> 2 HI(g)}~~~$\Delta H_2 = \SI{12}{\kilo\calorie}$
    \item \ce{I2(g) + H2(g) -> 2 HI(ac)}~~~$\Delta H_3 = \SI{-26,8}{\kilo\calorie}$
\end{itemize}
Calcular los parámetros que se indican a continuación:
\begin{description}%[label={\alph*)},font={\color{red!50!black}\bfseries}]
    \item[\texttt{a)}] Calor molar latente de sublimación del yodo.
    \item[\texttt{b)}] Calor molar de disolución del ácido yodhídrico.
    \item[\texttt{c)}] Número de calorías que hay que aportar para disociar en sus componentes el yoduro de hidrógeno gas contenido en un matraz de \SI{750}{\cubic\centi\meter} a \SI{25}{\celsius} y \SI{800}{\torr} de presión.
\end{description}
\resultadocmd{\SI{12,8}{\kilo\calorie}; \SI{-13,0}{\kilo\calorie}; \SI{12,9}{\calorie}}

	}
\end{frame}

\begin{frame}
	\frametitle{\ejerciciocmd}
	\framesubtitle{Datos del enunciado}
	\begin{center}
		{\Large\textbf{¿$K_{ps}(\ce{AgCl})$?}}\\
	\end{center}
	Potenciales estándar de reducción:\\[.2cm]
	\begin{enumerate}[label={\alph*)},font={\color{red!50!black}\bfseries}]
		\item\tcbhighmath[boxrule=0.4pt,arc=4pt,colframe=red,drop fuzzy shadow=blue]{\ce{Ag+(ac) + 1e- -> Ag(s)}\quad\varepsilon^0=\SI{,80}{\volt}}
		\item\tcbhighmath[boxrule=0.4pt,arc=4pt,colframe=blue,drop fuzzy shadow=yellow]{\ce{AgCl(s) + 1e- -> Ag(s) + Cl-(ac)}\quad\varepsilon^0=\SI{,22}{\volt}}
	\end{enumerate}
\end{frame}

\begin{frame}
	\frametitle{\ejerciciocmd}
	\framesubtitle{Resolución (\rom{1}): determinación de $K_{ps}(\ce{AgCl})$}
	\structure{Reacción de solubilidad:}
	$$
		\ce{AgCl(s) <=> Ag+(ac) + Cl-(ac)}\quad K_{ps}(\ce{AgCl})=[\ce{Ag+}]\vdot[\ce{Cl-}]
	$$
	Tendremos que combinar las dos semirreacciones de reducción de la forma que se obtenga la reacción de solubilidad.\\[.6cm]
	\begin{overprint}
		\onslide<1>
			\begin{center}
				\begin{tabular}{c}
					\ce{Ag+(ac) + 1e- -> Ag(s)}\quad$\varepsilon^0_{\text{reducción}}=\SI{,80}{\volt}$\\[.2cm]
					\ce{AgCl(s) + 1e- -> Ag(s) + Cl-(ac)}\quad$\varepsilon^0_{\text{reducción}}=\SI{,22}{\volt}$
				\end{tabular}
			\end{center}
		\onslide<2->
			\begin{center}
				\begin{tabular}{cc}
					\ce{\cancel{Ag(s)} -> Ag+(ac) + \cancel{1e-}} & $\varepsilon^0_{\text{oxidación}}=\SI{-,80}{\volt}$\footnote{Si cambiamos el sentido de la reacción, se cambia el signo del potencial. Ahora tenemos una semirreacción de oxidación.}\\[.2cm]
					\ce{AgCl(s) + \cancel{1e-} -> \cancel{Ag(s)} + Cl-(ac)} & $\varepsilon^0_{\text{reducción}}=\SI{,22}{\volt}$\\
					\midrule
					\ce{AgCl(s) -> Ag+(ac) + Cl-(ac)} & $\varepsilon^0_{\text{total}}=\SI{-,58}{\volt}$\\[.2cm]
				\end{tabular}
			\end{center}
	\end{overprint}
	\visible<2->{
		Hay un electrón implicado en el proceso ($n=1$) y el proceso no es espontáneo ($\varepsilon^0_{\text{total}}<0$). $Q$ se expresa como:
		$$
			Q=[\ce{Ag+}]\vdot[\ce{Cl-}]
		$$
				}
	\visible<3->{
		\structure{Ecuación de Nernst en situación de equilibrio $\Delta G = -n\vdot F\vdot\varepsilon=0$:}
		\begin{overprint}
			\onslide<3>
				$$
					\overbrace{\varepsilon}^{\varepsilon=0} = \varepsilon^0 - \frac{R\vdot T}{n\vdot F}\ln(Q)\Rightarrow
					\overbrace{\varepsilon}^{\varepsilon=0} = \varepsilon^0 - \frac{\num{,0592}}{n}\log(Q)\Rightarrow
					\varepsilon^0 = \frac{\num{,0592}}{n}\log(K_{ps})
				$$
			\onslide<4>
				$$
					\log(K_{ps})=  \frac{\varepsilon^0\vdot n}{\num{,0592}}
				$$
			\onslide<5>
				$$
					K_{ps}(\ce{AgCl}) =  10^{\frac{\num{-,58}\vdot\num{1}}{\num{,0592}}}
				$$
			\onslide<6->
				$$
					\tcbhighmath[boxrule=0.4pt,arc=4pt,colframe=red,drop fuzzy shadow=blue]{K_{ps}(\ce{AgCl}) =  \num{1,59e-10}}
				$$
		\end{overprint}
				}
\end{frame}
