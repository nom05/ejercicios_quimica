\begin{frame}
    \frametitle{\ejerciciocmd}
    \framesubtitle{Enunciado}
    \textbf{
	Dadas las siguientes reacciones:
\begin{itemize}
    \item \ce{I2(g) + H2(g) -> 2 HI(g)}~~~$\Delta H_1 = \SI{-0,8}{\kilo\calorie}$
    \item \ce{I2(s) + H2(g) -> 2 HI(g)}~~~$\Delta H_2 = \SI{12}{\kilo\calorie}$
    \item \ce{I2(g) + H2(g) -> 2 HI(ac)}~~~$\Delta H_3 = \SI{-26,8}{\kilo\calorie}$
\end{itemize}
Calcular los parámetros que se indican a continuación:
\begin{description}%[label={\alph*)},font={\color{red!50!black}\bfseries}]
    \item[\texttt{a)}] Calor molar latente de sublimación del yodo.
    \item[\texttt{b)}] Calor molar de disolución del ácido yodhídrico.
    \item[\texttt{c)}] Número de calorías que hay que aportar para disociar en sus componentes el yoduro de hidrógeno gas contenido en un matraz de \SI{750}{\cubic\centi\meter} a \SI{25}{\celsius} y \SI{800}{\torr} de presión.
\end{description}
\resultadocmd{\SI{12,8}{\kilo\calorie}; \SI{-13,0}{\kilo\calorie}; \SI{12,9}{\calorie}}

           }
\end{frame}

\begin{frame}
    \frametitle{\ejerciciocmd}
    \framesubtitle{Datos del apartado (a)}
    \textbf{\Large $K$ y $\Delta G^0$}
    $$
        \tcbhighmath[boxrule=0.4pt,arc=4pt,colframe=red,drop fuzzy shadow=orange]{\varepsilon^0(\ce{Sn^4+/Sn^2+}) = \SI{,15}{\volt}}\quad
        \tcbhighmath[boxrule=0.4pt,arc=4pt,colframe=red,drop fuzzy shadow=orange]{\varepsilon^0(\ce{Cr^3+/Cr^2+}) = \SI{-,41}{\volt}}
    $$
\end{frame}

\begin{frame}
    \frametitle{\ejerciciocmd}
    \framesubtitle{Resolución (\rom{1}): constante de equilibrio y $\Delta G^0$}
    \begin{tabular}{lcr}
	        \textbf{Semirreacción de oxidación (ánodo):}  & \ce{2Cr^2+ -> 2Cr^3+ + \cancel{2e-}}  & $\varepsilon^0_{ox} = \SI{,41}{\volt}$\\
	        \textbf{Semirreacción de reducción (cátodo):} & \ce{Sn^4+ \cancel{+2e-} -> Sn^2+}     & $\varepsilon^0_{re} = \SI{,15}{\volt}$\\
        \midrule
	        \textbf{Total:}  & \ce{2Cr^2+ + Sn^4+ -> 2Cr^3+ + Sn^2+} & $\varepsilon^0_{T} = \SI{,56}{\volt}$
    \end{tabular}
    \structure{En el equilibrio:} $\Delta G = 0$, $\varepsilon_T = 0$ y $Q=K$ (Ec. de Nernst)
    \structure{Ecuación de Nernst:}
    $$
        \cancelto{0}{\varepsilon_T} = \varepsilon^0_T -\frac{\SI{,0592}{}}{n}\log{\overbrace{Q}^{Q=K}}\Rightarrow
        \overbrace{\varepsilon^0_T}^{\varepsilon^0_{T} = \SI{,56}{\volt}} = \frac{\SI{,0592}{}}{\underbrace{n}_{n=2}}\log{K}\Rightarrow
        \tcbhighmath[boxrule=0.4pt,arc=4pt,colframe=red,drop fuzzy shadow=orange]{K=\num{9,62e18}}
    $$
    \structure{Relación entre $\varepsilon^0$ y $\Delta G^0$:}\quad$\Delta G^0 = -n\vdot F\vdot\varepsilon^0$\quad{\footnotesize (podríamos usar también $\Delta G^0 = -RT\ln K$ con resultados similares)}
    $$
        \Delta G^0 = -\overbrace{n}^{n=2}\vdot \underbrace{F}_{F=\num{96485}}\vdot\overbrace{\varepsilon^0_T}^{\SI{,56}{\volt}}\Rightarrow\tcbhighmath[boxrule=0.4pt,arc=4pt,colframe=red,drop fuzzy shadow=orange]{\Delta G^0=\SI{-108,06}{\kilo\joule\per\mol}}
    $$
\end{frame}

\begin{frame}
    \frametitle{\ejerciciocmd}
    \framesubtitle{Datos del apartado (b)}
    \textbf{\Large ¿$\varepsilon$?}
    $$
        \tcbhighmath[boxrule=0.4pt,arc=4pt,colframe=blue,drop fuzzy shadow=yellow]{[\ce{Sn^4+}] = \SI{e-2}{\Molar}}\quad
        \tcbhighmath[boxrule=0.4pt,arc=4pt,colframe=blue,drop fuzzy shadow=yellow]{[\ce{Sn^2+}] = \SI{5e-3}{\Molar}}
    $$
    $$
        \tcbhighmath[boxrule=0.4pt,arc=4pt,colframe=green,drop fuzzy shadow=blue]{[\ce{Cr^3+}] = \SI{2,5e-3}{\Molar}}\quad
        \tcbhighmath[boxrule=0.4pt,arc=4pt,colframe=green,drop fuzzy shadow=blue]{[\ce{Cr^2+}] = \SI{5,5e-3}{\Molar}}
    $$
\end{frame}
%
\begin{frame}
    \frametitle{\ejerciciocmd}
    \framesubtitle{Resolución (\rom{2}): potencial de la pila}
    \structure{Reacción química:}
    $$
        \ce{2Cr^2+ + Sn^4+ -> 2Cr^3+ + Sn^2+}\quad\varepsilon^0_{T} = \SI{,56}{\volt}
    $$
    \structure{Ecuación de Nernst:}
    $$
        \varepsilon_T = \varepsilon^0_T -\frac{\SI{,0592}{}}{n}\log{Q}\Rightarrow
        \varepsilon_T = \varepsilon^0_T -\frac{\SI{,0592}{}}{\underbrace{n}_{n=2}}\log(\frac{[\ce{Cr^3+}]^2[\ce{Sn^2+}]}{[\ce{Cr^2+}]^2[\ce{Sn^4+}]})
    $$
    $$
        \varepsilon_T = \num{,56} -\frac{\num{,0592}}{2}\log(\frac{(\num{2,5e-3})^2\vdot\num{5e-3}}{(\num{5,5e-3})^2\vdot\num{e-2}})
    $$
    \begin{center}
        \tcbhighmath[boxrule=0.4pt,arc=4pt,colframe=blue,drop fuzzy shadow=yellow]{\varepsilon_T = \SI{,56}{\volt} + \SI{,029}{\volt} = \SI{,589}{\volt}}
    \end{center}
\end{frame}

\begin{frame}
    \frametitle{\ejerciciocmd}
    \framesubtitle{Datos del apartado (c)}
    \textbf{\Large ¿$\varepsilon_T$?}
    $$
        \tcbhighmath[boxrule=0.4pt,arc=4pt,colframe=orange,drop fuzzy shadow=red]{I = \SI{1,00505}{\ampere}}\quad
        \tcbhighmath[boxrule=0.4pt,arc=4pt,colframe=orange,drop fuzzy shadow=red]{t = \SI{8}{\minute}}
    $$
\end{frame}

\begin{frame}
    \frametitle{\ejerciciocmd}
    \framesubtitle{Resolución (\rom{3}): potencial de la pila después de pasar \SI{1,00505}{\ampere} en \SI{8}{\minute}}
    \structure{Reacción química:}
%	{\small
		\begin{tabular}{lcr}
				\textbf{Semirreacción de oxidación (ánodo):}  & \ce{2Cr^2+ -> 2Cr^3+ + \cancel{2e-}}  & $\varepsilon^0_{ox} = \SI{,41}{\volt}$\\
				\textbf{Semirreacción de reducción (cátodo):} & \ce{Sn^4+ \cancel{+2e-} -> Sn^2+}     & $\varepsilon^0_{re} = \SI{,15}{\volt}$\\
			\midrule
				\textbf{Total:}  & \ce{2Cr^2+ + Sn^4+ -> 2Cr^3+ + Sn^2+} & $\varepsilon^0_{T} = \SI{,56}{\volt}$
		\end{tabular}
%	}
    \structure{Ley de Faraday:}\quad$n(\ce{e-})\vdot F = I\vdot t\Rightarrow n(\ce{e-}) = \frac{I\vdot t}{F}\Rightarrow n(\ce{e-})=\SI{5e-3}{\mol}$
    \structure{Según la estequiometría:}
%    {\small
    	\begin{enumerate}[label={\alph*)},font={\color{red!50!black}\bfseries}]
	        \item en el \textbf{ánodo}: se oxidan \SI{5e-3}{\mol} de \ce{Cr^2+} y se forman \SI{5e-3}{\mol} de \ce{Cr^3+}.
	        \item en el \textbf{cátodo}: se reducen $\frac{\SI{5e-3}{}}{2}~\si{\mol}$ de \ce{Sn^4+} y se forman $\frac{\SI{5e-3}{}}{2}~\si{\mol}$ de \ce{Sn^2+}
	    \end{enumerate}
%    }
    \structure{Las nuevas concentraciones son (volumen constante):}
%	{\small
		$$
	        [\ce{Cr^2+}] = \SI{5,5e-3}{\Molar}-\SI{5e-3}{\Molar} = \SI{,5e-3}{\Molar}
	    $$
	    $$
	        [\ce{Cr^3+}] = \SI{2,5e-3}{\Molar}+\SI{5e-3}{\Molar} = \SI{7,5e-3}{\Molar}
	    $$
	    $$
	        [\ce{Sn^2+}] = \SI{5e-3}{\Molar}+\SI{2,5e-3}{\Molar} = \SI{7,5e-3}{\Molar}
	    $$
	    $$
	        [\ce{Sn^4+}] = \SI{e-2}{\Molar}-\SI{2,5e-3}{\Molar} = \SI{7,5e-3}{\Molar}
	    $$
%    }
    \structure{Sustituyendo en la ecuación de Nernst:}
	$$
	    \varepsilon_T = \num{,56} -\frac{\num{,0592}}{2}\log(\frac{(\num{7,5e-3})^2\vdot\cancel{\num{7,5e-3}}}{(\num{,5e-3}{})^2\vdot\cancel{\num{7,5e-3}}})\Rightarrow
	    \tcbhighmath[boxrule=0.4pt,arc=4pt,colframe=orange,drop fuzzy shadow=red]{\varepsilon_T = \SI{,49}{\volt}}
	$$
\end{frame}
