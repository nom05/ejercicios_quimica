Se construye una pila empleando disoluciones de los iones \ce{Sn^4+/Sn^2+} y \ce{Cr^3+/Cr^2+}. Calcular:
\begin{enumerate}[label={\alph*)},font={\color{red!50!black}\bfseries}]
    \item La constante de equilibrio y la energía libre estándar.
    \item El potencial de la pila cuando:
        $[\ce{Sn^4+}] = \SI{e-2}{\Molar}$, $[\ce{Sn^2+}] = \SI{5e-3}{\Molar}$, $[\ce{Cr^3+}] = \SI{2,5e-3}{\Molar}$ y $[\ce{Cr^2+}] = \SI{5,5e-3}{\Molar}$.
    \item El potencial de la pila una vez han pasado \SI{5e-3}{\farad} por el conductor que une las dos semipilas.
\end{enumerate}
\resultadocmd{ \num{9,62e18}; \SI{-108,35}{\kilo\joule\per\mol}; \SI{,589}{\volt}; \SI{,49}{\volt} }
