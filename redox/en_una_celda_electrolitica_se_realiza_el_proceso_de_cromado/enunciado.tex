En una celda electrolítica se realiza el proceso de cromado de una pieza (recubrimiento con \ce{Cr(s)}). Debéis seleccionar de la tabla adjunta el otro electrodo teniendo en cuenta que el potencial de la celda debe ser lo más cercano a cero posible para aportar el mínimo voltaje. El ajuste debe realizarse en medio básico.
\begin{enumerate}
	\item ¿Cuáles son las semirreacciones de reducción y oxidación y dónde se producen? ¿Cuáles son sus correspondientes potenciales estándar de reducción y de oxidación? ¿Cuál es el ajuste de las semirreacciones? ¿Cuál es la reacción global y su potencial estándar total?
	\item Si inicialmente todas las especies en medio acuoso tienen una concentración de \SI{,5}{\Molar}, ¿qué potencial tendrá la reacción a \SI{25}{\celsius}?
	\item En las condiciones del apartado anterior, ¿qué volumen mínimo necesitaremos para formar \SI{100}{\gram} de cromo? ¿Cuánto tiempo se necesitará si pasa una corriente de \SI{20}{\ampere}?
\end{enumerate}
{\normalsize
	\begin{center}
		\begin{tabular}{cS}
			\toprule
				Electrodo						&	{Potencial estándar de reducción (\si{\volt})}	\\
			\midrule
				\ce{Mg^2+(ac)|Mg(s)}			&		-2,38 										\\
				\ce{SiO(s)|Si(s)} 				&	 	 -,80										\\
				\ce{Cr^3+(ac)|Cr(s)}			&	 	 -,74 										\\
				\ce{AsO4^3-(ac)|AsO2^-(ac)}		&	 	 -,71										\\
				\ce{Cr^3+(ac)|Cr^2+(ac)}		&	 	 -,42 										\\
				\ce{AuCl4^-(ac)|Au(s),Cl-(ac)}	&	 	 1,00										\\
				\ce{Cr2O7^2-(ac)|Cr^3+(ac)}		&	 	 1,33										\\
				\ce{MnO4^-(ac)|Mn^2+(ac)}			&	 	 1,51										\\
			\bottomrule
		\end{tabular}
	\end{center}
}
DATOS: $F = \SI{96485}{\coulomb\per\mol}$, $R = \SI{8,314}{\joule\per\mol\per\kelvin}$.
\resultadocmd{
				\SI{-,030}{\volt};
				\SI{-,072}{\volt};
				\SI{3,85}{\liter}, \SI{7}{\hour}~\SI{43}{\minute}~\SI{54,30}{\second}
			}
