En una celda electrolítica en medio ácido se necesita obtener aluminio metálico (\ce{Al}) a partir de una sal soluble de este. Para ello, seleccione una de las semirreacciones siguientes como el otro electrodo teniendo en cuenta que el potencial total de la celda tiene que ser lo más cercano a cero posible.
{\normalsize
	\begin{center}
		\begin{tabular}{cS}
			\toprule
				Electrodo					&	{Potencial estándar de reducción (\si{\volt})}	\\
			\midrule
				\ce{Mg^2+(ac)|Mg(s)}		&	-2,38 											\\
				\ce{Al^3+(ac)|Al(s)} 		&	-1,66											\\
				\ce{Ti2O3(s)|TiO(s)} 		&	-1,23											\\
				\ce{Cr^3+(ac)|Cr^2+(ac)}	&	-0,42 											\\
				\ce{CO2(g)|CO(g)}			&	-0,11											\\
				\ce{SnO(s)|Sn(s)}			&	-0,10											\\
				\ce{Cr2O7^2-(ac)|Cr^3+(ac)}	&	 1,33											\\
				\ce{MnO4-(ac)|Mn2+(ac)}		&	 1,51											\\
			\bottomrule
		\end{tabular}
	\end{center}
}
\begin{enumerate}[label={\alph*)},font={\color{red!50!black}\bfseries}]
	\item ¿Cuáles son las semirreacciones de reducción y oxidación y sus correspondientes potenciales estándar de reducción y de oxidación? 
	\item ¿Cuál es el ajuste de las semirreacciones? 
	\item ¿Cuál es la reacción global y su potencial estándar total?
	\item Si inicialmente tenemos una concentración de \ce{Al^3+} de \SI{2,5}{\Molar} y un pH de \num{2}, ¿qué potencial tendrá la reacción a \SI{25}{\celsius}?
	\item En las condiciones del apartado anterior, ¿qué volumen mínimo necesitaremos para formar \SI{2}{\gram} de aluminio sólido? ¿Cuánto tiempo se necesitará si pasa una corriente de \SI{5}{\ampere}?
\end{enumerate}
DATOS: $F = \SI{96485}{\coulomb\per\mol}$, $R = \SI{8,314}{\joule\per\mol\per\kelvin}$.
\resultadocmd{
				\SI{-,430}{\volt};
				\SI{-,304}{\volt};
				\SI{4291}{\second}
			}
