\begin{frame}
    \frametitle{\ejerciciocmd}
    \framesubtitle{Enunciado}
    \textbf{
		Una reacción tiene una constante de velocidad de \SI{,017}{\per\second} a \SI{298}{\kelvin} y una energía libre de activación del \SI{27,235}{\kilo\joule\per\mol}. La adición de un catalizador disminuye dicha energía de activación hasta un \SI{33}{\percent} de su valor inicial. Calcule la nueva constante de velocidad.
\resultadocmd{ \SI{26,86}{\per\second} }

           }
\end{frame}

\begin{frame}
    \frametitle{\ejerciciocmd}
    \framesubtitle{Resolución (\rom{1}): cloro molecular a cloruro y clorato}
    \structure{Medio alcalino:} ajuste en medio ``\textbf{\underline{básico}}''.
    \begin{overprint}
        \onslide<1>
            $$
                \ce{Cl2(g) + OH-(ac) -> Cl-(ac) + ClO3-(ac)}
            $$
        \onslide<2>
            $$
                \ce{
                      $\overset{0}{\ce{Cl}}$2(g)
                        +
                      $\overset{-2}{\ce{O}}$
                      $\overset{+1}{\ce{H}}$-(ac)
                        ->
                      $\overset{-1}{\ce{Cl}}$-(ac)
                        +
                      $\overset{+5}{\ce{Cl}}
                      $\overset{-2}{\ce{O}}3-(ac)
                    }
            $$
        \onslide<3>
            \structure{Semirreacción de oxidación:} \ce{Cl2(g) -> ClO3-(ac)}
            \structure{Semirreacción de reducción:} \ce{Cl2(g) -> Cl-(ac)}
        \onslide<4>
            \structure{Semirreacción de oxidación:} \ce{Cl2(g) -> 2ClO3-(ac)}
            \structure{Semirreacción de reducción:} \ce{Cl2(g) -> 2Cl-(ac)}
        \onslide<5>
            \structure{Semirreacción de oxidación:} \ce{Cl2(g) -> 2ClO3-(ac) + 6H2O(l)}
            \structure{Semirreacción de reducción:} \ce{Cl2(g) -> 2Cl-(ac)}
        \onslide<6>
            \structure{Semirreacción de oxidación:} \ce{Cl2(g) + 12OH-(ac) -> 2ClO3-(ac) + 6H2O(l)}
            \structure{Semirreacción de reducción:} \ce{Cl2(g) -> 2Cl-(ac)}
        \onslide<7>
            \structure{Semirreacción de oxidación:} \ce{Cl2(g) + 12OH-(ac) -> 2ClO3-(ac) + 6H2O(l) + 10e-}
            \structure{Semirreacción de reducción:} \ce{Cl2(g) + 2e- -> 2Cl-(ac)}
        \onslide<8>
            \structure{Semirreacción de oxidación:} \ce{Cl2(g) + 12OH-(ac) -> 2ClO3-(ac) + 6H2O(l) + 10e-}
            \structure{Semirreacción de reducción:} $5\times\left(\ce{Cl2(g) + 2e- -> 2Cl-(ac)}\right)$
        \onslide<9>
            \structure{Semirreacción de oxidación:} \ce{Cl2(g) + 12OH-(ac) -> 2ClO3-(ac) + 6H2O(l) + 10e-}
            \structure{Semirreacción de reducción:} \ce{5Cl2(g) + 10e- -> 10Cl-(ac)}
        \onslide<10>
            \begin{tabular}{c}
                \ce{Cl2(g) + 12OH-(ac) -> 2ClO3-(ac) + 6H2O(l) + \cancel{\ce{10e-}}} \\
                \ce{5Cl2(g) + \cancel{\ce{10e-}} -> 10Cl-(ac)} \\
                \midrule
                \ce{6Cl2(g) + 12OH-(ac) -> 2ClO3-(ac) + 6H2O(l) + 10Cl-(ac)}
            \end{tabular}
        \onslide<11>
            $$
                \ce{3Cl2(g) + 6OH-(ac) -> ClO3-(ac) + 3H2O(l) + 5Cl-(ac)}
            $$
    \end{overprint}
    \begin{enumerate}[label={\alph*)},font={\color{red!50!black}\bfseries}]
        \item<1-> Escribimos la reacción sin ajustar.
        \item<2-> Asignamos números de oxidación.
        \item<3-> Escribimos las ecuaciones de las semirreacciones de oxidación y reducción.
        \item<4-> Ajustamos los átomos diferentes al \ce{O} y al \ce{H}.
        \item<5-> En el lado de la semirreacción que presente exceso de oxígenos, añadiremos tantas moléculas de agua como oxígenos hay de más.
        \item<6-> Ajustamos en el otro lado el número de oxígenos e hidrógenos añadiendo \ce{OH-}.
        \item<7-> Ajustamos número de electrones en cada reacción.
        \item<8-> Hacemos que el número de electrones sea igual en las dos ecuaciones. Para ello multiplicamos por un factor.
        \item<10-> Sumamos ambas reacciones.
        \item<11-> Simplificamos si procede.
    \end{enumerate}
\end{frame}

\begin{frame}
    \frametitle{\ejerciciocmd}
    \framesubtitle{Resolución (\rom{2}): de \ce{HI} a \ce{I2} en medio ácido con \ce{HNO3}}
    \structure{Ácido nítrico:} ajuste en medio ``\textbf{\underline{ácido}}''.
    \begin{overprint}
        \onslide<1>
            $$
                \ce{HI(ac) + HNO3(ac) -> I2(g) + NO(g)}
            $$
        \onslide<2>
            $$
                \ce{
                       $\overset{+1}{\ce{H}}$
                       $\overset{-1}{\ce{I}}$(ac)
                        +
                       $\overset{+1}{\ce{H}}$
                       $\overset{+5}{\ce{N}}$
                       $\overset{-2}{\ce{O}}$3(ac)
                        ->
                       $\overset{0}{\ce{I}}$2(g)
                        +
                       $\overset{+2}{\ce{N}}$
                       $\overset{-2}{\ce{O}}$(g)
                    }
            $$
        \onslide<3>
            \structure{Semirreacción de oxidación:} \ce{HI(ac) -> I2(g)}
            \structure{Semirreacción de reducción:} \ce{HNO3(ac) -> NO(g)}
        \onslide<4>
            \structure{Semirreacción de oxidación:} \ce{2HI(ac) -> I2(g)}
            \structure{Semirreacción de reducción:} \ce{HNO3(ac) -> NO(g)}
        \onslide<5>
            \structure{Semirreacción de oxidación:} \ce{2HI(ac) -> I2(g)}
            \structure{Semirreacción de reducción:} \ce{HNO3(ac) -> 2H2O(l) + NO(g)}
        \onslide<6>
            \structure{Semirreacción de oxidación:} \ce{2HI(ac) -> I2(g) + 2H+(ac)}
            \structure{Semirreacción de reducción:} \ce{HNO3(ac) + 3H+(ac) -> 2H2O(l) + NO(g)}
        \onslide<7>
            \structure{Semirreacción de oxidación:} \ce{2HI(ac) -> I2(g) + 2H+(ac) + 2e-}
            \structure{Semirreacción de reducción:} \ce{HNO3(ac) + 3H+(ac) + 3e- -> 2H2O(l) + NO(g)}
        \onslide<8>
            \structure{Semirreacción de oxidación:} $3\times\left(\ce{2HI(ac) -> I2(g) + 2H+(ac) + 2e-}\right)$
            \structure{Semirreacción de reducción:} $2\times\left(\ce{HNO3(ac) + 3H+(ac) + 3e- -> 2H2O(l) + NO(g)}\right)$
        \onslide<9>
            \structure{Semirreacción de oxidación:} \ce{6HI(ac) -> 3I2(g) + 6H+(ac) + 6e-}
            \structure{Semirreacción de reducción:} \ce{2HNO3(ac) + 6H+(ac) + 6e- -> 4H2O(l) + 2NO(g)}
        \onslide<10->
            \begin{tabular}{c}
                \ce{6HI(ac) -> 3I2(g) + \cancel{\ce{6H+(ac)}} + \cancel{\ce{6e-}}} \\
                \ce{2HNO3(ac) + \cancel{\ce{6H+(ac)}} + \cancel{\ce{6e-}} -> 4H2O(l) + 2NO(g)} \\
                \midrule
                \ce{6HI(ac) + 2HNO3(ac) -> 3I2(g) + 4H2O(l) + 2NO(g)} \\
            \end{tabular}
    \end{overprint}
    \begin{enumerate}[label={\alph*)},font={\color{red!50!black}\bfseries}]
        \item<1-> Escribimos la reacción sin ajustar.
        \item<2-> Asignamos números de oxidación.
        \item<3-> Escribimos las ecuaciones de las semirreacciones de oxidación y reducción.
        \item<4-> Ajustamos los átomos diferentes al \ce{O} y al \ce{H}.
        \item<5-> Ajustamos los átomos de \ce{O} sumando \ce{H2O}.
        \item<6-> Ajustamos los átomos de \ce{H} sumando \ce{H+}.
        \item<7-> Ajustamos número de electrones en cada reacción.
        \item<8-> Hacemos que el número de electrones sea igual en las dos ecuaciones. Para ello multiplicamos por un factor.
        \item<10-> Sumamos ambas reacciones.
        \item<11-> Simplificamos si procede.
    \end{enumerate}
\end{frame}
