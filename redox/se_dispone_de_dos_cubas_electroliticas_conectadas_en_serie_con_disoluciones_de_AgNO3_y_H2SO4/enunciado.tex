Se dispone de dos cubas electrolíticas conectadas en serie con disoluciones de nitrato de plata (\ce{AgNO3}) y ácido sulfúrico (\ce{H2SO4}), respectivamente. Se pasa corriente a través de ellas, de forma que en la primera se depositan \SI{,2325}{\gram} de plata. Calcule el volumen de hidrógeno (\ce{H2}) que se depositará en la segunda a presión de \SI{1}{\atm} y temperatura \SI{25}{\celsius}.
\resultadocmd{\SI{,024}{\liter}}