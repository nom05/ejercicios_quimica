\begin{frame}
	\frametitle{\ejerciciocmd}
	\framesubtitle{Enunciado}
	\textbf{
		Una reacción tiene una constante de velocidad de \SI{,017}{\per\second} a \SI{298}{\kelvin} y una energía libre de activación del \SI{27,235}{\kilo\joule\per\mol}. La adición de un catalizador disminuye dicha energía de activación hasta un \SI{33}{\percent} de su valor inicial. Calcule la nueva constante de velocidad.
\resultadocmd{ \SI{26,86}{\per\second} }

		}
\end{frame}

\begin{frame}
	\frametitle{\ejerciciocmd}
	\framesubtitle{Datos del problema}
	\begin{center}
		{\huge ¿$V(\ce{H2})$?}\\[.3cm]
		\textbf{circuito en serie}\quad
		\tcbhighmath[boxrule=0.4pt,arc=4pt,colframe=green,drop fuzzy shadow=blue]{m(\ce{Ag(s) v})=\SI{,2325}{\gram}}\\[.3cm]
		\tcbhighmath[boxrule=0.4pt,arc=4pt,colframe=black,drop fuzzy shadow=blue]{P(\ce{H2(g) ^})=\SI{1}{\atm}}\quad
		\tcbhighmath[boxrule=0.4pt,arc=4pt,colframe=black,drop fuzzy shadow=blue]{T(\ce{H2(g) ^})=\SI{25}{\celsius}=\SI{298,15}{\kelvin}}
	\end{center}
\end{frame}

\begin{frame}
	\frametitle{\ejerciciocmd}
	\framesubtitle{Resolución (\rom{1}): volumen de hidrógeno gas}
	\structure{Las dos cubas están conectadas \underline{en serie}:} pasa la misma cantidad de corriente a través de ellas durante el mismo tiempo.
	$$
		I(\ce{AgNO3})\vdot\cancel{t} = I(\ce{H2SO4})\vdot\cancel{t}
	$$
	\structure{Ley de Faraday:} $n(\ce{e-})\vdot F = I\vdot t$
	$$
		n_{\ce{H2SO4}}(\ce{e-})\vdot\cancel{F} = n_{\ce{AgNO3}}(\ce{e-})\vdot\cancel{F}
	$$
	\structure{Semirreacciones de reducción:}
	$$
		\left.
			\begin{array}{ll}
				\ce{Ag+ + 1e- -> Ag} & n(\ce{Ag}) = n_{\ce{AgNO3}}(\ce{e-})\\
				\ce{2H+ + 2e- -> H2} & 2\vdot n(\ce{H2}) = n_{\ce{H2SO4}}(\ce{e-})
			\end{array}
		\right\}\Rightarrow
		\overbrace{n(\ce{Ag})}^{n=\frac{m}{M_{at}}}=2\vdot\underbrace{n(\ce{H2})}_{n=\frac{PV}{RT}}\Rightarrow
	$$
	$$
		\frac{m(\ce{Ag})}{M_{at}(\ce{Ag})} = 2\vdot\frac{P(\ce{H2})\vdot V(\ce{H2})}{R\vdot T(\ce{H2})}\Rightarrow
		V(\ce{H2})=\frac{m(\ce{Ag})\vdot R\vdot T(\ce{H2})}{2\vdot M_{at}(\ce{Ag})\vdot P(\ce{H2})}\Rightarrow
	$$
	$$
		V(\ce{H2})=\frac{\SI{,2325}{\cancel\gram}\vdot\SI{,082}{\cancel\atm\liter\per\cancel\mol\per\cancel\kelvin}\vdot\SI{298,15}{\cancel\kelvin}}{2\vdot\SI{107,868}{\cancel\gram\per\cancel\mol}\vdot\SI{1}{\cancel\atm}}\Rightarrow
		\tcbhighmath[boxrule=0.4pt,arc=4pt,colframe=black,drop fuzzy shadow=blue]{V(\ce{H2(g) ^})=\SI{,026}{\liter}=\SI{26}{\milli\liter}}
	$$
\end{frame}
