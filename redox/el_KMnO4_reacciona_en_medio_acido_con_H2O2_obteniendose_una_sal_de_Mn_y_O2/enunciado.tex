El permanganato potásico reacciona en medio ácido con peróxido de hidrógeno obteniéndose una sal de
manganeso (\rom{2}) y oxígeno gas.
\begin{enumerate}[label={\alph*)},font={\color{red!50!black}\bfseries}]
    \item Ajustar la reacción
    \item ¿Cuál es la normalidad de la disolución de peróxido si \SI{10}{\milli\liter} de la misma necesitan \SI{18,6}{\milli\liter} de
    permanganato de potasio \SI{0,02}{\Molar} para reaccionar completamente?
    \item ¿Qué volumen de oxígeno (medido en c.n.) se desprende?
    \item ¿Qué volumen de ácido clorhídrido de una riqueza del \SI{12}{\percent} y densidad \SI{1,06}{\gram\per\milli\liter} es necesario para
    reaccionar totalmente con el peróxido y el permanganato empleados?
\end{enumerate}
\resultadocmd{
            \SI{,186}{N};
            \SI{20,16}{\milli\liter};
            \SI{,32}{\milli\liter}
}
