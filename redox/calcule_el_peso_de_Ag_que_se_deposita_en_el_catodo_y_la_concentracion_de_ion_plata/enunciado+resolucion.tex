\begin{frame}
	\frametitle{\ejerciciocmd}
	\framesubtitle{Enunciado}
	\textbf{
		Dadas las siguientes reacciones:
\begin{itemize}
    \item \ce{I2(g) + H2(g) -> 2 HI(g)}~~~$\Delta H_1 = \SI{-0,8}{\kilo\calorie}$
    \item \ce{I2(s) + H2(g) -> 2 HI(g)}~~~$\Delta H_2 = \SI{12}{\kilo\calorie}$
    \item \ce{I2(g) + H2(g) -> 2 HI(ac)}~~~$\Delta H_3 = \SI{-26,8}{\kilo\calorie}$
\end{itemize}
Calcular los parámetros que se indican a continuación:
\begin{description}%[label={\alph*)},font={\color{red!50!black}\bfseries}]
    \item[\texttt{a)}] Calor molar latente de sublimación del yodo.
    \item[\texttt{b)}] Calor molar de disolución del ácido yodhídrico.
    \item[\texttt{c)}] Número de calorías que hay que aportar para disociar en sus componentes el yoduro de hidrógeno gas contenido en un matraz de \SI{750}{\cubic\centi\meter} a \SI{25}{\celsius} y \SI{800}{\torr} de presión.
\end{description}
\resultadocmd{\SI{12,8}{\kilo\calorie}; \SI{-13,0}{\kilo\calorie}; \SI{12,9}{\calorie}}

		}
\end{frame}

\begin{frame}
	\frametitle{\ejerciciocmd}
	\framesubtitle{Datos del problema}
	\begin{center}
		{\huge¿$m(\ce{Ag})$ y $[\ce{Ag+}]$? en cátodo}\\[.3cm]
		\tcbhighmath[boxrule=0.4pt,arc=4pt,colframe=green,drop fuzzy shadow=blue]{[\ce{AgNO3}]_0=[\ce{Ag+}]_0=\SI{,1}{\Molar}~\text{(concentración inicial)}}\quad
		\tcbhighmath[boxrule=0.4pt,arc=4pt,colframe=green,drop fuzzy shadow=blue]{V(\ce{AgNO3})=\SI{1}{\liter}}\\[.3cm]
		\tcbhighmath[boxrule=0.4pt,arc=4pt,colframe=green,drop fuzzy shadow=blue]{I=\SI{,5}{\ampere}}\quad
		\tcbhighmath[boxrule=0.4pt,arc=4pt,colframe=green,drop fuzzy shadow=blue]{t=\SI{2}{\hour}=\SI{7200}{\second}}\quad
		\tcbhighmath[boxrule=0.4pt,arc=4pt,colframe=green,drop fuzzy shadow=blue]{M_{at}(\ce{Ag})=\SI{107,868}{\gram\per\mol}}
	\end{center}
\end{frame}

\begin{frame}
	\frametitle{\ejerciciocmd}
	\framesubtitle{Resolución (\rom{1}): masa de plata y concentración molar de ion plata}
	\structure{Aplicamos la ley de Faraday para el número de moles de electrones del proceso:}
	\begin{overprint}
		\onslide<1>
			$$
				n(\ce{e-})\cdot\overbrace{F}^{\SI{96485}{\coulomb\per\mol}}=\underbrace{I}_{\SI{,5}{\ampere}}\cdot\overbrace{t}^{\SI{7200}{\second}}
			$$
		\onslide<2->
			$$
				n(\ce{e-})=\frac{\SI{,5}{\ampere}\cdot\SI{7200}{\second}}{\SI{96485}{\coulomb\per\mol}}=\SI{,037}{\mol}
			$$
	\end{overprint}
	\visible<2->{
		\structure{Semirreacción de reducción:}\\[.2cm]
		\begin{center}
			\ce{Ag+(ac) + 1e- -> Ag(s)}\quad Según estequiometría:~$n(\ce{e-})=n(\ce{Ag+})=n(\ce{Ag})$
		\end{center}
				}
	\visible<3->{
		\structure{Masa de \ce{Ag} sólida:}
		$$
			n=\frac{m}{M_{at}}\Rightarrow m=n\cdot M_{at}\Rightarrow m(\ce{Ag})=\SI{,037}{\mol}\cdot\SI{107,868}{\gram\per\cancel\mol}
			\Rightarrow\tcbhighmath[boxrule=0.4pt,arc=4pt,colframe=blue,drop fuzzy shadow=green]{m(\ce{Ag})=\SI{4,02}{\gram}}
		$$
				}
	\visible<4->{
		\structure{Concentración molar de ion plata:}
		$$
			M=\frac{n}{V}\Rightarrow n=M\cdot V\Rightarrow\overbrace{n_0(\ce{Ag+})}^{\text{iniciales}}=\SI{,1}{\mol\per\cancel\liter}\cdot\SI{1}{\cancel\liter}=\SI{,1}{\mol}
		$$
				}
	\visible<5>{
		$$
			\underbrace{n(\ce{Ag+})}_{\text{quedan disueltos}}=\SI{,1}{\mol}-\SI{,037}{\mol}=\SI{,063}{\mol}
		$$
		$$
			[\ce{Ag+}]=\frac{\SI{,063}{\mol}}{\SI{1}{\liter}}\Rightarrow\tcbhighmath[boxrule=0.4pt,arc=4pt,colframe=green,drop fuzzy shadow=blue]{[\ce{Ag+}]=[\ce{AgNO3}]=\SI{,063}{\Molar}~\text{(concentración final)}}
		$$
				}
\end{frame}
