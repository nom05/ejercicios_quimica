A la vista de la tabla de potenciales estándar de reducción adjunta imagine una celda galvánica maximizando la fuerza electromotriz y conteste a las siguientes preguntas. En ella tendrá que tener como cátodo \ce{Cr2O7^2-}/\ce{Cr^3+} en medio ácido. DATO: $F = \SI{96485}{\coulomb\per\mol}$
{\normalsize
	\begin{center}
		\begin{tabular}{cS}
			\toprule
			Electrodo			&	{Potencial estándar de reducción (\si{\volt})}	\\
			\midrule
			\ce{Ni^2+|Ni}		&	 -,250											\\
			\ce{Sn^2+|Sn}		&	 -,136											\\
			\ce{Pb^2+|Pb}		&	 -,126											\\
			\ce{H+|H2}			&	  ,000											\\
			\ce{Cr2O7^2-|Cr^3+}	&	+1,333											\\
			\ce{MnO4^-|Mn^2+}	&	+1,507											\\
			\ce{Pb^4+|Pb^2+}	&	+1,693											\\
			\ce{F^-|F2}			&	+2,865											\\
			\bottomrule
		\end{tabular}
	\end{center}
				}
\begin{enumerate}[label={\alph*)},font={\color{red!50!black}\bfseries}]
	\item ¿Cuáles son las reacciones de reducción y oxidación y sus correspondientes potenciales estándar de reducción y de oxidación?
	\item ¿Cuál es el ajuste de las semirreacciones?
	\item ¿Cuál es la reacción global?
	\item ¿Cuál es el potencial estándar total de la pila?
	\item ¿Cuál es su constante de equilibrio a \SI{25}{\celsius}?
	\item Si inicialmente tenemos una concentración de dicromato potásico (\ce{K2Cr2O7}) de \SI{,20}{\Molar}, un pH de \num{2} y el resto de los cationes concentraciones de \SI{,15}{\Molar}, ¿qué potencial tendrá la reacción a \SI{25}{\celsius}?
	\item Usando las concentraciones y pH del anterior apartado y usando un volumen de \SI{1}{\liter}, calcula el potencial de la pila una vez pasados \SI{,5}{\ampere} durante \SI{5}{\minute}.
\end{enumerate}
\resultadocmd{
				\SI{1,583}{\volt};
				\num{2,75e160};
				\SI{1,340}{\volt};
				\SI{1,313}{\volt}
			}
