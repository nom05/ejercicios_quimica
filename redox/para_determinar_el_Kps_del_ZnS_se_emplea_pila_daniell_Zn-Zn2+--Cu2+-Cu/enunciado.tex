Para determinar el producto de solubilidad del sulfuro de zinc a \SI{25}{\celsius} se emplea una pila Daniell:
\ce{Zn/Zn^2+//Cu^2+/Cu}. En el cátodo la concentración del cobre (\rom{2}) es \SI{1}{\Molar} y en el ánodo se añade sulfuro de
sodio hasta que la concentración de sulfuro es \SI{1}{\Molar} y ha precipitado casi todo el zinc. En estas condiciones el
potencial de la celda es \SI{1,78}{\volt}. Calcular:
\begin{enumerate}[label={\alph*)},font={\color{red!50!black}\bfseries}]
    \item El potencial normal de la pila.
    \item La concentración del zinc (\rom{2}) en las condiciones de trabajo.
    \item El valor del $K_{ps}$ del sulfuro de zinc a \SI{25}{\celsius}.
    \item El valor de la fuerza electromotriz de la pila cuando la concentración del ion \ce{Zn^2+} es igual a la solubilidad del sulfuro de zinc.
\end{enumerate}
\resultadocmd{ \SI{1,10}{\volt}; $\approx\SI{e-23}{\Molar}$; $\approx\num{e-23}$; \SI{1,44}{\volt} }
