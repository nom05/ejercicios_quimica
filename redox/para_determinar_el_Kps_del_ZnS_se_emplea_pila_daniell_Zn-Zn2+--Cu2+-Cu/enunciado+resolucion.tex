\begin{frame}
    \frametitle{\ejerciciocmd}
    \framesubtitle{Enunciado}
    \textbf{
		Dadas las siguientes reacciones:
\begin{itemize}
    \item \ce{I2(g) + H2(g) -> 2 HI(g)}~~~$\Delta H_1 = \SI{-0,8}{\kilo\calorie}$
    \item \ce{I2(s) + H2(g) -> 2 HI(g)}~~~$\Delta H_2 = \SI{12}{\kilo\calorie}$
    \item \ce{I2(g) + H2(g) -> 2 HI(ac)}~~~$\Delta H_3 = \SI{-26,8}{\kilo\calorie}$
\end{itemize}
Calcular los parámetros que se indican a continuación:
\begin{description}%[label={\alph*)},font={\color{red!50!black}\bfseries}]
    \item[\texttt{a)}] Calor molar latente de sublimación del yodo.
    \item[\texttt{b)}] Calor molar de disolución del ácido yodhídrico.
    \item[\texttt{c)}] Número de calorías que hay que aportar para disociar en sus componentes el yoduro de hidrógeno gas contenido en un matraz de \SI{750}{\cubic\centi\meter} a \SI{25}{\celsius} y \SI{800}{\torr} de presión.
\end{description}
\resultadocmd{\SI{12,8}{\kilo\calorie}; \SI{-13,0}{\kilo\calorie}; \SI{12,9}{\calorie}}

           }
\end{frame}

\begin{frame}
    \frametitle{\ejerciciocmd}
    \framesubtitle{Datos del problema}
    \textbf{\Large \begin{enumerate}[label={\alph*)},font={\color{red!50!black}\bfseries}]
            \item ¿$\varepsilon_T^0$?
            \item ¿[\ce{Zn^2+}]?
            \item ¿$K_{ps}(\ce{ZnS})$?
            \item ¿$\varepsilon_T$?
    \end{enumerate}}
    $$
        \tcbhighmath[boxrule=0.4pt,arc=4pt,colframe=yellow,drop fuzzy shadow=blue]{\text{Pila Daniell: }\ce{Zn/Zn^2+//Cu^2+/Cu}}\quad
        \tcbhighmath[boxrule=0.4pt,arc=4pt,colframe=yellow,drop fuzzy shadow=blue]{\text{Cátodo: }\ce{[Cu^2+] = \SI{1}{\Molar}}}
    $$
    $$
        \tcbhighmath[boxrule=0.4pt,arc=4pt,colframe=yellow,drop fuzzy shadow=blue]{\text{Ánodo: }\ce{[Na2S] = \SI{1}{\Molar}}}\quad
        \tcbhighmath[boxrule=0.4pt,arc=4pt,colframe=yellow,drop fuzzy shadow=blue]{\varepsilon = \SI{1,78}{\volt}}
    $$
\end{frame}

\begin{frame}
    \frametitle{\ejerciciocmd}
    \framesubtitle{Resolución (\rom{1}): potencial normal}
    \begin{center}
        \begin{tabular}{lcr}
            \textbf{\textit{Ánodo} (oxidación):}  & \ce{Zn -> Zn^2+ + 2e-} & $\varepsilon^0_{ox} = \SI{,76}{\volt}$\\
            \textbf{\textit{Cátodo} (reducción):} & \ce{Cu^2+ + 2e- -> Cu} & $\varepsilon^0_{re} = \SI{,34}{\volt}$\\
            \midrule
            \textbf{\textit{Total:}}  & \ce{Zn + Cu^2+ -> Zn^2+ + Cu} & $\varepsilon^0_{to} = \SI{1,10}{\volt}$
        \end{tabular}
    \end{center}
    $$
        \tcbhighmath[boxrule=0.4pt,arc=4pt,colframe=yellow,drop fuzzy shadow=blue]{\varepsilon^0_{to} = \SI{1,10}{\volt}}
    $$
    $$
        \text{\textbf{\underline{2 electrones en el proceso}}}
    $$
\end{frame}

\begin{frame}
    \frametitle{\ejerciciocmd}
    \framesubtitle{Resolución (\rom{2}) y (\rom{3}): concentración de Zn (\rom{2})}
    \structure{Ecuación de Nernst:}
    \begin{overprint}
        \onslide<1>
            $$
                \varepsilon_T = \varepsilon^0_T -\frac{\SI{,0592}{}}{n}\log{Q}
            $$
        \onslide<2>
            $$
                \varepsilon_T = \varepsilon^0_T -\frac{\SI{,0592}{}}{\underbrace{n}_{n=2}}\log{\frac{[\ce{Zn^2+}]}{[\ce{Cu^2+}]}}
            $$
        \onslide<3->
            $$
                \log{[\ce{Zn^2+}]} - \cancelto{0}{\log{\underbrace{[\ce{Cu^2+}]}_{[\ce{Cu^2+}]=\SI{1}{\Molar}}}} =
                -\frac{2\cdot(\varepsilon_T - \varepsilon^0_T)}{\SI{,0592}{}}
            $$
    \end{overprint}
    \visible<3->{
        Si usamos \SI{,059}{}, \tcbhighmath[boxrule=0.4pt,arc=4pt,colframe=yellow,drop fuzzy shadow=blue]{[\ce{Zn^2+}] = \SI{8,895e-24}{\Molar}}\\[.3cm]
        Si usamos \SI{,0592}{}, \tcbhighmath[boxrule=0.4pt,arc=4pt,colframe=yellow,drop fuzzy shadow=blue]{[\ce{Zn^2+}] = \SI{1,06e-23}{\Molar}}\\[.3cm]
        Dependiendo del redondeo: \tcbhighmath[boxrule=0.4pt,arc=4pt,colframe=yellow,drop fuzzy shadow=blue]{[\ce{Zn^2+}] \approx\SI{e-23}{\Molar}}
                }
    \visible<4->{
        $$
            \tcbhighmath[boxrule=0.4pt,arc=4pt,colframe=yellow,drop fuzzy shadow=blue]{K_{ps}(\ce{ZnS}) = [\ce{Zn^2+}]\underbrace{[\ce{S^2-}]}_{[\ce{S^2-}]=\SI{1}{\Molar}}\approx\num{e-23}}
        $$
                }
\end{frame}

\begin{frame}
    \frametitle{\ejerciciocmd}
    \framesubtitle{Resolución (\rom{4}): potencial del electrodo cuando $[\ce{Zn^2+}] = s$}
    \structure{Del seminario de solubilidad:}
    $$
        K_{ps}(ZnS) = [\ce{Zn^2+}][\ce{S^2-}] = s^2\Rightarrow s = \sqrt{K_{ps}(\ce{ZnS})} = K_{ps}(\ce{ZnS})^{\frac{1}{2}}\Rightarrow s = 10^{-\frac{23}{2}}~\si{\Molar}=10^{-11,5}\si{\Molar}
    $$
    \structure{Ecuación de Nernst:}
    \begin{overprint}
        \onslide<1>
            $$
                \varepsilon_T = \varepsilon^0_T -\frac{\SI{,0592}{}}{n}\log{Q}
            $$
        \onslide<2>
            $$
                \varepsilon_T = \varepsilon^0_T -\frac{\SI{,0592}{}}{\underbrace{n}_{n=2}}\log{\frac{[\ce{Zn^2+}]}{[\ce{Cu^2+}]}}
            $$
        \onslide<3->
            $$
                \varepsilon_T = \SI{1,10}{} -\frac{\SI{,0592}{}}{2}
                             \left(\log{[\ce{Zn^2+}]}-                    
                \cancelto{0}{\log{\underbrace{[\ce{Cu^2+}]}_{[\ce{Cu^2+}]=\SI{1}{\Molar}}}}
                             \right)
            $$
    \end{overprint}
    \visible<3->{
        $$
            \tcbhighmath[boxrule=0.4pt,arc=4pt,colframe=yellow,drop fuzzy shadow=blue]{\varepsilon_T = \SI{1,44}{\volt}}
        $$
                }
\end{frame}
