\begin{frame}
	\frametitle{\ejerciciocmd}
	\framesubtitle{Enunciado}
	\textbf{
		Una reacción tiene una constante de velocidad de \SI{,017}{\per\second} a \SI{298}{\kelvin} y una energía libre de activación del \SI{27,235}{\kilo\joule\per\mol}. La adición de un catalizador disminuye dicha energía de activación hasta un \SI{33}{\percent} de su valor inicial. Calcule la nueva constante de velocidad.
\resultadocmd{ \SI{26,86}{\per\second} }

		}
\end{frame}

\begin{frame}
	\frametitle{\ejerciciocmd}
	\framesubtitle{Datos del problema}
	\begin{center}
		{\huge¿[\ce{Cu^{2+}}] y [\ce{Zn^{2+}}] en el equilibrio?}\\[.3cm]
		\tcbhighmath[boxrule=0.4pt,arc=4pt,colframe=green,drop fuzzy shadow=blue]{[\ce{CuCl2}]=[\ce{Cu^{2+}}]=\SI{,05}{\Molar}}\\[.3cm]
		\tcbhighmath[boxrule=0.4pt,arc=4pt,colframe=green,drop fuzzy shadow=blue]{\varepsilon^0(\ce{Cu^{2+}}/\ce{Cu})=\SI{,34}{\volt}}\quad
		\tcbhighmath[boxrule=0.4pt,arc=4pt,colframe=blue,drop fuzzy shadow=green]{\varepsilon^0(\ce{Zn^{2+}}/\ce{Zn})=\SI{-,76}{\volt}}
	\end{center}
\end{frame}

\begin{frame}
	\frametitle{\ejerciciocmd}
	\framesubtitle{Resolución (\rom{1}): determinación de concentraciones en equilibrio}
	\structure{Reacción de oxidación-reducción:}
	{\small\begin{center}
		\begin{tabular}{lcr}
			S. oxidación (ánodo): & \ce{Zn -> Zn^{2+} + \cancel{2e-}} & $\varepsilon^0_{\text{ox}}=\SI{,76}{\volt}$\\
			S. reducción (cátodo): & \ce{Cu^{2+} + \cancel{2e-} -> Cu} & $\varepsilon^0_{\text{red}}=\SI{,34}{\volt}$\\
			\midrule
			Reacción global: & \ce{Cu^{2+}(ac) + Zn(s) -> Cu(s) + Zn^{2+}(ac)}  & $\varepsilon^0_{\text{total}}=\SI{1,10}{\volt}$\\
		\end{tabular}				
	\end{center}}
	\structure{Ecuación de \textbf{\underline{NERNST}} y cómo determinar $K$:}
	{\small $$
		\text{Si }\Delta G = 0\text{ (equilibrio), entonces }\Delta G = -n\vdot F\vdot\varepsilon=0\Rightarrow\varepsilon=0
	$$
	$$
		\overbrace{\varepsilon}^{\varepsilon=0} = \varepsilon^0 - \frac{\num{,0592}}{n}\log{Q}\Rightarrow
		\varepsilon^0 = \frac{\num{,0592}}{n}\log{K}
	$$}
	{\small siendo $n$ el número de electrones implicados en el proceso ($n=\num{2}$). $Q$ el cociente de las concentraciones (acuosos) y presiones (gases) de productos entre las de reactivos. $K$ es la constante de equilibrio:}
	{\small $$
		Q=\frac{[\ce{Zn^{2+}}]}{[\ce{Cu^{2+}}]};\qquad
		K = \frac{[\ce{Zn^{2+}}]_{\text{eq}}}{[\ce{Cu^{2+}}]_{\text{eq}}}\Rightarrow
		\SI{1,10}{\volt} = \frac{\num{,0592}}{2}\log(\frac{[\ce{Zn^2+}]_{\text{eq}}}{[\ce{Cu^2+}]_{\text{eq}}})
	$$
	$$
		\frac{\overbrace{[\ce{Zn^{2+}}]}^{\cancelto{0}{[\ce{Zn^{2+}}]_0}+x}}{\underbrace{[\ce{Cu^{2+}}]}_{[\ce{Cu^{2+}}]_0-x}}=10^{\frac{\num{1,10}\cdot\num{2}}{\num{,0592}}}=\num{1,45e37}\Rightarrow
		\frac{x}{\num{,05}-x}=\num{1,45e37}\Rightarrow x=\tcbhighmath[boxrule=0.4pt,arc=4pt,colframe=green,drop fuzzy shadow=blue]{[\ce{Zn^{2+}}]\approx\SI{,05}{\Molar}}
	$$}
	\begin{multicols}{2}
		{\small 
			\begin{center}
				\begin{tabular}{lcc}
						(\si{\Molar})	& [\ce{Cu^2+}] 		& [\ce{Zn^2+}]		\\
					\midrule
						Inicial			& \num{,05}			& \num{0}			\\
						Reaccionan		& $-x$				& $+x$				\\
						Equilibrio		& $\num{,05}-x$		& $x$
				\end{tabular}
				\tcbhighmath[boxrule=0.4pt,arc=4pt,colframe=blue,drop fuzzy shadow=green]{[\ce{Cu^{2+}}]=\SI{3,44e-39}{\Molar}}
			\end{center}
		}
	\end{multicols}
\end{frame}
