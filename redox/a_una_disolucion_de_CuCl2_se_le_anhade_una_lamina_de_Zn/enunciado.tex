A una disolución de cloruro de cobre (\rom{2}), \ce{CuCl2}, \SI{,05}{\Molar} se le añade una lámina de cinc metálico. Calcular las concentraciones de los iones \ce{Cu^2+} y \ce{Zn^2+} en el equilibrio. $\varepsilon(\ce{Zn^2+}/\ce{Zn})=\SI{-,76}{\volt}$, $\varepsilon(\ce{Cu^2+}/\ce{Cu}) = \SI{,34}{\volt}$.
\resultadocmd{\SI{,05}{\Molar}; $\sim\SI{e-39}{\Molar}$}
