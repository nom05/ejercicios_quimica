El ácido nítrico (\ce{HNO3}) en disolución \SI{1}{\Molar} reacciona con cadmio metálico, produciendo nitrato de cadmio (\ce{Cd(NO3)2}) y monóxido de nitrógeno (\ce{NO}). Calcule el potencial estándar de la reacción y si se produciría esta reacción con cobre metal. Datos: $\varepsilon^0(\ce{NO3-}/\ce{NO}) = \SI{,96}{\volt}$, $\varepsilon^0(\ce{Cd^{2+}}/\ce{Cd}) = \SI{-,40}{\volt}$, $\varepsilon^0(\ce{Cu^{2+}}/\ce{Cu}) = \SI{,34}{\volt}$.
\resultadocmd{\SI{1,36}{\volt}; \SI{,62}{\volt}}
