\begin{frame}
    \frametitle{\ejerciciocmd}
    \framesubtitle{Enunciado}
    \textbf{
	Dadas las siguientes reacciones:
\begin{itemize}
    \item \ce{I2(g) + H2(g) -> 2 HI(g)}~~~$\Delta H_1 = \SI{-0,8}{\kilo\calorie}$
    \item \ce{I2(s) + H2(g) -> 2 HI(g)}~~~$\Delta H_2 = \SI{12}{\kilo\calorie}$
    \item \ce{I2(g) + H2(g) -> 2 HI(ac)}~~~$\Delta H_3 = \SI{-26,8}{\kilo\calorie}$
\end{itemize}
Calcular los parámetros que se indican a continuación:
\begin{description}%[label={\alph*)},font={\color{red!50!black}\bfseries}]
    \item[\texttt{a)}] Calor molar latente de sublimación del yodo.
    \item[\texttt{b)}] Calor molar de disolución del ácido yodhídrico.
    \item[\texttt{c)}] Número de calorías que hay que aportar para disociar en sus componentes el yoduro de hidrógeno gas contenido en un matraz de \SI{750}{\cubic\centi\meter} a \SI{25}{\celsius} y \SI{800}{\torr} de presión.
\end{description}
\resultadocmd{\SI{12,8}{\kilo\calorie}; \SI{-13,0}{\kilo\calorie}; \SI{12,9}{\calorie}}

           }
\end{frame}

\begin{frame}
    \frametitle{\ejerciciocmd}
    \framesubtitle{Datos del problema}
    \begin{center}
        {\Large\textbf{pH}}
    \end{center}
    $$
        \tcbhighmath[boxrule=0.4pt,arc=4pt,colframe=red,drop fuzzy shadow=orange]{\text{PILA}}\quad
        \tcbhighmath[boxrule=0.4pt,arc=4pt,colframe=red,drop fuzzy shadow=orange]{\varepsilon_T = \SI{1,24}{\volt}}
    $$
    $$
        \tcbhighmath[boxrule=0.4pt,arc=4pt,colframe=blue,drop fuzzy shadow=black]{P(\ce{H2}) = \SI{1}{\atm}}\quad
        \tcbhighmath[boxrule=0.4pt,arc=4pt,colframe=blue,drop fuzzy shadow=black]{\varepsilon^0_{\ce{H+/H2}} = \SI{0}{\volt}}\quad
        \tcbhighmath[boxrule=0.4pt,arc=4pt,colframe=blue,drop fuzzy shadow=black]{\text{electrodo de \ce{H2}+disolución}}
    $$
    $$
        \tcbhighmath[boxrule=0.4pt,arc=4pt,colframe=green,drop fuzzy shadow=blue]{\varepsilon^0_{\ce{Ag+/Ag}} = \SI{,80}{\volt}}\quad
        \tcbhighmath[boxrule=0.4pt,arc=4pt,colframe=green,drop fuzzy shadow=blue]{\text{electrodo de \ce{Ag}+dis. \ce{Ag+}}}\quad
        \tcbhighmath[boxrule=0.4pt,arc=4pt,colframe=green,drop fuzzy shadow=blue]{[\ce{Ag+}] = \SI{,01}{\Molar}}
    $$
\end{frame}

\begin{frame}
    \frametitle{\ejerciciocmd}
    \framesubtitle{Resolución (\rom{1}): determinación del pH de una disolución}
    \begin{tabular}{llcr}
        \textbf{\textit{Oxidación}} & Ánodo  & $\ce{H2(g) -> 2H+ + \cancel{\ce{2e-}}}$ & $\varepsilon^0_{\text{ánodo}} = \SI{,00}{\volt}$\\
        \textbf{\textit{Reducción}} & Cátodo & $2\times\left(\ce{Ag+(ac) + \cancel{\ce{1e-}} -> Ag(s)}\right)$ & $\varepsilon^0_{\text{cátodo}} = \SI{,80}{\volt}$\\
        \midrule
        \textbf{\textit{Total}} & & $\ce{H2(g) + 2Ag+(ac) -> 2H+ + 2Ag(s)}$ & $\varepsilon^0_T = \SI{,80}{\volt}$\\
    \end{tabular}
    \visible<2->{
        \structure{\textbf{\underline{2 electrones}} en el proceso:} $n=2$
        $$
            Q = \frac{[\ce{H+}]^2}{P(\ce{H2})\vdot[\ce{Ag+}]^2}
        $$
        \structure{Ecuación de Nernst:}
        \begin{overprint}
            \onslide<2>
                $$
                    \varepsilon_T = \varepsilon^0_T -\frac{\SI{,0592}{}}{n}\log Q
                $$
            \onslide<3>
                $$
                    \varepsilon_T-\varepsilon^0_T =  -\frac{\SI{,0592}{}}{n}\log Q
                $$
            \onslide<4>
                $$
                    \frac{n\left(\varepsilon_T-\varepsilon^0_T\right)}{\SI{,0592}{}} =  -\log\left(\frac{[\ce{H+}]^2}{P(\ce{H2})\vdot[\ce{Ag+}]^2}\right)
                $$
            \onslide<5>
                $$
                    \frac{n\left(\varepsilon_T-\varepsilon^0_T\right)}{\SI{,0592}{}} =  -\log[\ce{H+}]^2 + \cancelto{0}{\log\overbrace{P(\ce{H2})}^{P(\ce{H2}) = \SI{1}{\atm}}} +
                    \cancelto{-4}{\log\overbrace{[\ce{Ag+}]^2}^{[\ce{Ag+}]=\SI{e-2}{\Molar}}}
                $$
            \onslide<6>
                $$
                    \frac{\overbrace{n}^{n=2}\left(\varepsilon_T-\varepsilon^0_T\right)}{\SI{,0592}{}} =  2\vdot\overbrace{-\log[\ce{H+}]}^{pH} - \underbrace{4}_{2\vdot 2}
                $$
            \onslide<7>
                $$
                    \frac{\cancel{2}\vdot(\overbrace{\varepsilon_T}^{\SI{1,24}{\volt}}-\overbrace{\varepsilon^0_T}^{\SI{,80}{\volt}})}{\SI{,0592}{}} =  \cancel{2}pH - 2\vdot\cancel{2}
                $$
            \onslide<8>
                $$
                    \tcbhighmath[boxrule=0.4pt,arc=4pt,colframe=blue,drop fuzzy shadow=black]{pH = \frac{\left(\SI{1,24}{}-\SI{,80}{}\right)}{\SI{,0592}{}} + 2 = \SI{9,43}{}}
                $$
        \end{overprint}

                }
\end{frame}

